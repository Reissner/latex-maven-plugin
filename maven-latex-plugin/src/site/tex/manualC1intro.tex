\chapter{Introduction}%\label{chap:intro}

This document is created with \lualatex{} or that like 
with output format 
\ifpdf%
pdf%
\else
dvi%
\fi.
The package \pkg{tex4ht} 
is \ifthenelse{\boolean{texFhtLoaded}}{}{not} loaded. 

\LaTeX{} is a beautiful way to create printable documents, 
in our days preferably as \gls{pdf}-files, 
with a particular strength in typesetting formulae like
% pandoc invocation given in README.md
% TBD: check why pandoc can create this formula whereas it fails for align.
% TBD: pandoc seems to have problems with links to tables
% and tables dont look optimal.
% TBD: link to figures seem not to work, to be honest i cannot see any figure
% whereas toc is present, other tables are not. 
% also index, bibliography and that like seems to miss. 
%
% \begin{equation*}
%   \pi  = \sqrt{12}\;\sum^\infty_{k=0} \frac{(-3)^{-k}}{2k+1}. %chktex 3
% \end{equation*}
%
\begin{align}
\pi & = \sqrt{12}\;\sum^\infty_{k=0} \frac{(-3)^{-k}}{2k+1}. %chktex 3
\end{align}
%
Here, portability of the format \gls{pdf} is a vital feature. 
In the past, normally \gls{dvi} (device independent)
described in~\cite{DviF} has been used 
and still creation of external formats like \gls{html}, 
\gls{odt} and \gls{docx} are based on an intermediate \gls{dvi}-file.
It is much more lightweight than pdf specified in~\cite{Pdf1}, in~\cite{Pdf2}
and in~\cite{Pdf3}.
% strange: no mention of other formats
% epub for example, htm, xhtml and office formats also rtf
% manpages, javahelp

This piece of software implements both an ant-task and a maven-plugin 
generating documentation of various formats from \LaTeX-files 
in a uniform way. 
Chapter~\ref{chap:install} shows how to install both the maven-plugin 
and the ant-task 
and Chapter~\ref{chap:usage} describes the usage. 
Note that the maven-plugin is both easier to install 
and more versatile to be used. 

From the \LaTeX-files, the latex main files must be extracted, 
only these must be compiled. 
It is very usual to endow \LaTeX-files with figures. 
On the other hand, there are many graphic formats 
which cannot be included directly in a \LaTeX-file 
but must be preprocessed. 
If there is some format needed but not yet provided, 
please write an email to the author. 

Graphic files must be preprocessed before processing latex main files, 
as described in Chapter~\ref{chap:GraphConversions}. 
Then follows the proper processing of latex main files 
including creation of index and glossaries 
as described in Chapter~\ref{chap:latexMainConversions}. 
Besides \gls{pdf}, these formats include the web-formats \gls{html} 
and \gls{xhtml}, 
open offices format \gls{odt}, Microsoft's word formats like \gls{docx} 
and finally plain text. 

Uniformity of ant-task and a maven-plugin means in particular, 
that the settings which may be passed to the task 
and those allowed for the plugin are in a one-to-one relation. 
They are both described in Chapter~\ref{chap:settings}. 
It is a design goal, that the auxiliary programs 
used by this software are fully configurable via parameters, 
that aspects not completely specified can be handled flexibly, 
there are parameters supporting information development 
and that for the parameters are default values 
which allow doing without explicit parametrization in most of the cases.\index{ant-task}%
Both, the ant-task and the maven-plugin rely on the same code base 
which form the package \texttt{org.m2latex.core}. 
The code specific for the ant-task is in \texttt{org.m2latex.antTask} 
and that specific for the maven-plugin is in \texttt{org.m2latex.mojo}. 


The creation process supports an index, a glossary and a bibliography. 
In addition, code written in python and other languages can be included and executed 
during creation of the document. 
Again, further functionality can be added by demand. 

The present manual is created by the maven-plugin or the ant-task 
described here. 
There should be no difference in the result. 
This manual is designed in a way that it covers the most important features 
but also to demand the most important features. 
That way, creating this manual is a top level test 
for the underlying software. 
The maven-plugin is somehow superior 
because it better supports the design process for the \LaTeX{} sources. 

If something goes wrong in the build process, 
or there is an indication 
of some deficiency in the result of the build process, 
processing must be aborted if going on does not make sense 
and there must be some error or warning logging 
as described in Chapter~\ref{chap:exceptionLogging}. 

The author found some gaps, i.e.~desirable features 
which are not yet implemented. 
To prioritize further work, 
all these gaps are collected in Chapter~\ref{chap:gaps}. 
Accordingly, the most important bugs are collected in
Chapter~\ref{chap:bugs}. 
The user is encouraged to contribute with feature requests 
and bug reports and to vote for realization of features 
and on fixing bugs. 
Software quality is ensured mainly through tests 
which are described in Chapter~\ref{chap:tests}. 

