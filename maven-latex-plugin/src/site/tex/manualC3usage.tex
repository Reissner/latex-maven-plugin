% !TEX root = manualLMP.tex

\chapter{Usage of Plugin and Task}\label{chap:usage}

% TBD: in the long run: plugin, task and standalone, or as a dependency
This software offers both, a maven plugin and an according ant task,
but the emphasis is on the maven plugin.
Thus, all sections of this chapter are either general
or apply to the maven plugin.
Only Section~\ref{sec:usageAntTask} specifically refer to the ant task. 
Usage presupposes installation as described in Chapter~\ref{chap:install}
including settings in \texttt{pom.xml}
as described in Section~\ref{sec:sgml}.

This plugin may be used both if the \LaTeX-sources are finished 
to create the output described by them 
and also to support development of the \LaTeX{} sources. 
Accordingly, this chapter has Section~\ref{sec:sources}
devoted to the form of the sources, including directory structure,
\LaTeX-files and others, mainly graphic files included
and a Section~\ref{sec:stableUsage} on exporting into various formats.

There is a very special usage, called development of documents,
which means while the document is under construction.
The features and goals tied to this phase
are collected in Section~\ref{sec:devel}.

In contrast, Section~\ref{sec:usageLifecycle}
is on usage of the maven plugin within the lifecycles.
This can be used during development of documents
but is more appropriate for small changes
or when development finished at a stage. 



\section{The source files}\label{sec:sources}

The \LaTeX-files and also files included via \cmd{input}{} 
are in the \emph{tex-source directory}, 
\index{tex-source directory}% chktex 24
which is by default \texttt{./src/site/tex}, 
where ``\texttt{.}'' is the \emph{base directory} of this maven-project. 
\index{base directory}% chktex 24
The \LaTeX-files to be compiled top level, 
typically not inputted anywhere via \cmd{input}, 
are called \emph{\LaTeX{} main files}. 
As an example, 
in the \emph{tex-source directory} of this software, 
\texttt{manualLMP.tex} is a \LaTeX{} main file, 
whereas the file \texttt{header.tex} is not although also a \LaTeX-file. 
% identification of the latex main file?
\LaTeX{} main files are detected automatically. 
\index{latex main file}% chktex 24

The included files may be again \LaTeX-files, but also bibliography files, 
listings included by package \pkg{listings}, verbatim text included with \pkg{verbatim} 
and maybe soon even code files to be executed via the package \pkg{pythontex}. 
\index{pythontex}% chktex 24
The great bulk of input files however, are graphic files in various formats. 
As regards the way their files are included in \LaTeX-files, 
there are the following kinds of graphic formats, 
all included in the tex-source directory. 
%
\begin{enumerate}
\item
The first can be included into \LaTeX-files directly via \cmd{input}. 
These formats are essentially \LaTeX{}
and are defined in an according package. 
Examples are \pkg{eepic} described in~\cite{EEpic}
and above all \pkg{tikz} described in~\cite{TikzPGF}. 
\item
the second one via the command \cmd{includegraphics}{} 
defined by the package \pkg{graphicx} 
which is described in~\cite{GraX}. 
Chapter 2 therein mentions the supported drivers, 
among these are also \texttt{dvipdfm} and \texttt{dvipdfmx}. 
It is not the package but the driver 
which decides on the support of graphic formats. 
The dvipdfm user manual,~\cite{DviPdfMx} lists the allowed formats 
metapost-output (i.e.~\gls{mps}), postscript, 
\gls{pdf}, \gls{jpg} and \gls{png}. 
\item\label{it:transExp}
the third one must be transformed into a graphics format 
of one of the former two kinds using an external tool for transformation. 
Here, of course, only a limited support is possible, 
because there is a broad variety of formats. 
We have chosen
%
\begin{itemize}
\item
the \gls{fig}-format described in~\cite{XFigF}
because of its simplicity, 
\item
the gnuplot format, described in~\cite{GnuPlot}, 
because it allows computation of function plots, 
\item
  scalable vector graphics svg format specified in~\cite{Svg11} 
  as it is important for construction and the counterpart of pixel oriented
  formats.
  As the specification is hard to digest,
  we refer to the tutorial~\cite{SvgTut}. 
\item
  likewise, metapost (\gls{mp}-format),
  described in~\cite{MPost} because it is native to \LaTeX{} 
  and quite versatile. 
\end{itemize}
\item\label{it:transImp}
the fourth kind of graphics formats 
has to be transformed into one of the kinds one or two 
but unlike in type three, this is not done explicitly 
by an external tool but by a latex-package during the \LaTeX-run. 
Note that, although not required to be explicitly transformed, 
those graphics files induce additional files 
by running \LaTeX.
Essentially, each of the abovementioned type of format
can be included that way but currently,
this is done for the \gls{svg}-format only
included by the package \pkg{svg} (see~\cite{SvgP}).
The author personally refrains from using packages like that
because of the lack of flexibility and further drawbacks. 
\item 
Finally, there is a way to include graphics which is not really a graphic format: 
In the course of running code, e.g.~by package \pkg{pythontex} in Python, 
as described in Section~\ref{sec:pythontex}, 
it is also possible to create computed graphics. 
It may be advisable to separate code into special files to be included via \cmd{input}, 
but it is not strictly required. 
In the long run it seems a good idea, to extend \pkg{pythontex} 
to read in code files, e.g.~in python directly. 
\end{enumerate}

The \LaTeX-files and the graphic files belonging to a \LaTeX{} main file 
are assumed to be in one single folder. 
If one file is included by two different main files, 
a link shall be used.
%TBC: True???

Note that unlike former version, 
the current version of this software does not create a working directory 
by cloning the tex-source directory. 
Instead, it operates directly on the tex-source directory 
also creating intermediate files.
The advantage of processing that way is,
that this allows cooperation between this software
and other toolchains which are better suited for developping latex files.
Details are described in Section~\ref{sec:devel}.

The downside is that a file residing in the tex-source directory 
risks being overwritten by this software, 
if it does not stick to the rules. 
The rules are simple: 
For each graphic file, being transformed, 
i.e.~of types~\ref{it:transExp} or~\ref{it:transImp} above, 
additional files are created with the same name up to the suffix. 
Thus for these graphic files no file with the same name 
up to the ending is allowed. 
The same is true for the \LaTeX{} main files.

Besides the \LaTeX-files and the graphics files
there is a third kind of file supported:
Bibliographies in bib-files.
This software treats them automatically. 


\section{Exporting in various formats}\label{sec:stableUsage}


After having added the configuration of the plugin to the \texttt{pom.xml},
minimally the one given in Section~\ref{sec:sgml},
it can be used directly invoking maven through 
\texttt{mvn latex:cfg}. 
Here \texttt{latex} is the (short) name of the plugin 
and \texttt{cfg} is the goal. 
It can also be interpreted as \texttt{mvn $<$source$>$:$<$target$>$}: 
The source files are in \texttt{latex}-format and the target-formats 
are read from the \emph{configuration} in the pom 
(\emph{configuration} is what \texttt{cfg} stands for). 

By default, the target formats are \texttt{pdf} and \texttt{html}. 
The following Listing~\ref{lst:targetsAll} shows a configuration 
with the full range of output formats including in addition 
the OpenOffice document format \texttt{odt}, 
the MS word-formats \texttt{doc(x)} and \texttt{rtf} % chktex 36
and also plain text format \texttt{txt} in utf8 encoding. 

Note that the target \texttt{docx} converts by default into \gls{docx} 
but may also be configured to produce the old-fashioned \gls{doc} format. 

Be aware that the target \texttt{dvi} creates output in DVI format 
only for latex processors \lualatex{} and \pdflatex{}, 
whereas \xelatex{} creates the XDV (extended DVI) format for target \texttt{dvi}. 

%\lstset{language=xml, basicstyle=\small}
\begin{lstlisting}[language=xml, basicstyle=\small,
escapechar=|,
float, captionpos=b, label={lst:targetsAll}, 
caption={Configuration with full range output formats}]
<!-- create html and pdf and other formats from latex -->
<plugin>
  <groupId>|\groupId|/groupId>
  <artifactId>|\artifactId|</artifactId>
  <version>|\strippedVersionID|</version>
	
  <configuration>
    <settings>
      <targets>pdf, dvi, html, odt, docx, rtf, txt</targets>
    </settings>
  </configuration>
</plugin>
\end{lstlisting}

The resulting files in the given output formats 
are copied to the site directory, 
which is \texttt{./target/site} in a default maven project. 

Sometimes it is more convenient 
to specify the output formats not via the pom 
but directly as a goal on the command line. 
In particular, one may write \texttt{mvn latex:pdf} to create documentation 
in PDF-format only.
Likewise, command \texttt{mvn latex:dvi} to get good old dvi/xdv files
or even \texttt{mvn latex:txt} for plain text, just as examples. 

Note that the \texttt{-X} switch activates debugging 
which results in a more verbose output. 
Example: \texttt{mvn -X latex:cfg}. 

% The code shows that allowed is only what is defined in enum Target. 
% Thus chk, which is a goal, is not output format. 
% This makes sense because Target specifies the files to be copied to the target filder. 

% in a sense this section mixes up output formats and goals. 
% For goal latex:cfg the output formats are as listed under config target. 
% Other goals, like latex:pdf specify a single output format. 

% Still other goals have no output format. These are 
% - latex:chk, although a check file is created. 
%   TBD: currently, there is no config to make chk run on latex main files. 
% - latex:grp, although graphic files are created but they are not exported. 
%   This may be done if source distributions are defined. 
% - latex:vrs, creating a version output or emitting warning if a version does not fit. 
% - latex:clr, for cleanup. 

% TBD: latex:grp does not fit: it is not an output format. 
% maybe it becomes output format if introducing source distributions 
% 
% Same is true for target clr. 

For creating the graphic files in the TEX source directory, 
there is a goal \emph{graphics}, invoked by \texttt{mvn latex:grp}. 
This goal does not create any output in the site directory. 
Instead, it populates the source directories 
with graphic files which can be directly included into the \LaTeX-file 
and so it allows to run the \LaTeX-compiler on the latex main files 
from within a development environment. 
Thus, the goal \emph{graphics} is thus a vital feature 
for development of documents. 


Finally, there is another target for clearing the tex-source directory 
recursively, invoked by \texttt{mvn latex:clr}. 
For more details on the last three goals, see Section~\ref{sec:devel}. 

In a standard maven project, 
the above minimal configuration should be sufficient. 
Only if the folder structure deviates from the standard 
or if the \LaTeX{} sources require special configuration, 
parameters have to be given explicitly, 
because they deviate from the default values. 
Chapter~\ref{chap:settings} summarizes all available parameters, 
giving the default value and a description. 


For sake of uniformity, 
the name of the ant-task is \texttt{latex:cfg}, 
and it can be invoked via \texttt{ant latex:cfg}. 
Unlike the maven-plugin, the ant-task 
does not allow to specify a target on the command line. 
The \texttt{-d} switch activates debugging 
which results in a more verbose output. 
Example: \texttt{ant -d latex:cfg}. 

Whereas by default the target directory and in particular 
the target site directory with all output of this plugin is deleted 
in maven's \texttt{clean} life-cycle. 
As is described in more detail in Section~\ref{sec:devel}, 
this software creates target documents and also intermediate files 
in the tex source directory, at least with cleanup disabled. 
To eliminate the created files from the source directory, 
just type \texttt{mvn latex:clr}. 
By default, the goal \texttt{clr} 
is also executed in maven's \texttt{clear} life-cycle. 

There is an according ant-task \texttt{latex:cfg} 
which can be invoked via \texttt{ant -d latex:cfg}. 
FIXME\@: ant  \texttt{latex:clr} has duplicate parameters. 
This can be fixed only by properties. 
Another problem is, to provide a complete subset of parameters 
which apply to \texttt{latex:cfg} and to \texttt{latex:cfg}, respectively. 

If this ant-task is used in an ant project 
with folder structure conforming with a maven project 
and if the \LaTeX{} sources do not require a special configuration, 
the above configuration is sufficient. 
Otherwise, parameters have to be given explicitly 
overwriting the default values. 

\section{Development of Documents}\label{sec:devel}

During development, it is comfortable, 
to have the log-file in the same directory as the \LaTeX{} main file. 
Also, if pdf- and tex-files are synchronized, 
% FIXME: reference to package 
also the pdf-file should be in the same directory. 
Likewise, files in graphic formats 
which cannot be included into a \LaTeX-file without conversion, 
that converted file shall be in the same directory as the original one. 
So, all files, manually created files 
and files arising from automatic conversions 
shall be in the same folder, at least during development. 
Also, typically, one wants to mix creation by this maven-plugin or ant-task 
with at least partial creation through external tools. 
For example, if writing \LaTeX-files with Emacs, 
it is much more convenient, to compile the \LaTeX{} main file 
via \pdflatex{} from within Emacs 
or to create a pdf-file from a \gls{fig}-file 
through \texttt{xfig}'s export dialog, 
than using this maven-plugin or this ant-task. 
Also these tools work best, if all is in one folder. 

On the other hand, 
conventionally, in a maven project, 
sources are held in folder \texttt{src}, 
whereas created files occur in the folder \texttt{target}. 
Likewise for ant. 
The compromise, this maven-plugin and this ant-task take, 
is, that at the end of a run, 
at most the files present at the beginning of the run 
may be present in the source directory. 
So, this software builds in the following steps: 
%
\begin{itemize}
\item
Store a list of all files present at the beginning of a run.
\item
Process all graphics files of the formats requiring preprocessing.
\item
Determine the \LaTeX{} main files.
\item
Run the \LaTeX{} converter, e.g.~the one creating pdf-output or docx-output.
This may include running auxiliary programs like \texttt{bibtex} or \texttt{pythontex} 
and also rerunning the \LaTeX{} converter several times. 
\item
Copy the result files (if any) into the target folder.
\item
Remove all files not present at the beginning of a run, by default. 
% FIXME: maybe different for goal chk. 
\end{itemize}

To keep e.g.~the resulting pdf, 
just create it via compilation through Emacs, 
even if not all graphic files to be included are present 
or just by a \texttt{touch}-command. 
Then in the next run of this plugin, 
this pdf will be re-created, 
that time complete with the graphics output. 
That way, synchronization between \LaTeX- and pdf-files is possible. 
Likewise, to keep the log-file or the aux-file, just touch it. 
This technique is really valuable for debugging. 

To keep all created files after a run of this maven-plugin, 
set the parameter \texttt{cleanUp} in the pom 
to \texttt{false} as illustrated in Listing~\ref{lst:noCleanup}. 
For the ant-task likewise. 

%\lstset{language=xml, basicstyle=\small}
\begin{lstlisting}[language=xml, basicstyle=\small,
escapechar=|,
float=b, captionpos=b, label={lst:noCleanup},
caption={Configuration without cleanup}]
<!-- create html and pdf and other formats from latex -->
<plugin>
  <groupId>|\groupId|</groupId>
  <artifactId>|\artifactId|</artifactId>
  <version>|\strippedVersionID|</version>
	
  <configuration>
    <settings>
      <targets>pdf</targets>
      <cleanUp>false</cleanUp>
    </settings>
  </configuration>
</plugin>
\end{lstlisting}


But how can one get rid of all these newly created files? 
That is what is the goal \texttt{latex:clr} is for: 
% 
\texttt{mvn latex:clr}
%
removes all created graphic files 
and for each latex main file, it removes all files with ``similar'' names
including log files, index files and that like.
Typically, this suffices, to remove all files created. 
If not, 
try to modify parameter \texttt{\$patternCreatedFromLatexMain} 
in the pom accordingly. 
If this does not help either, please inform the developer of this software. 
Of course, if further software is used which creates additional files, 
like Emacs creates a folder \texttt{auto}, 
these files cannot be removed by this maven-plugin or this ant-task.
Note that \texttt{latex:clr}
also removes exported files as listed in Section~\ref{sec:stableUsage}
from the target folder. 

During development of a \LaTeX-main file, 
it is often more convenient to compile from within an editor like Emacs. 
The problem is, that compilation fails if the graphic files are missing. 
This is what the goal \emph{graphics} accessible via 
% 
\begin{Verbatim}
mvn latex:grp
\end{Verbatim}
%
is for: 
It creates all graphic files required to compile the \LaTeX-main files. 

Still this does not create a bibliography, an index or a glossary. 
With \emph{auctex}\index{auctex}, an Emacs-package for editing \LaTeX, 
bibliography and index are well supported. 
To create a glossary, auctex has to be modified a little. 

%FIXME\@: include this into auctex. 

That way also the log-files required are created: 
In case of this manual, 
the files \texttt{manualLMP.xxx} are created 
where \texttt{xxx} is 
%
\begin{itemize}
\item
\texttt{log} for \LaTeX, 
\item
\texttt{blg} for \texttt{BibTeX}, 
\item
\texttt{glg} for \texttt{makeglossaries} and 
\item
\texttt{ilg} for \texttt{makeindex}. 
\end{itemize}

The last goal regularly used for development of documentation is \emph{check}. 
It is invoked via 
% 
\begin{Verbatim}
mvn latex:chk
\end{Verbatim}
%
and runs \texttt{chktex}, described in~\cite{ChkTeX}, 
on each latex main file 
after having created graphic files as for goal \emph{graphics}. 
As a result, a log-file with suffix \texttt{.clg} is created 
but not copied to the target folder. 
If the log-file contains an entry, 
an according message is logged. 
% FIXME: there is a lot more to do here. 
Note that, with default configuration, 
\texttt{chktex} requires the \LaTeX-package \pkg{booktabs} 
described in~\cite{BooktP}. 

% rsvg-convert -f pdf -o t.pdf t.svg
% inkscape t.svg --export-pdf=t.pdf
% convert file.svgz file.pdf 
% rasterizer -m application/pdf file.svgz -d file.pdf
% cairosvg in.svg -o out.pdf
% yyy

Finally, we have the goal \texttt{latex:vrs}
to display meta information, above all version information:
% 
\begin{Verbatim}
mvn latex:vrs
\end{Verbatim}
%
displays something like what is displayed in Listing~\ref{lst:vrsOut}.
As one can see, there may be converters with inappropriate versions
which is indicated by a warning WMI02 
and there may be converters which are excluded
according to the configuration \texttt{convertersExcluded}.

Note that in the given version and in the installation of the author,
of course, all converters are installed and are up-to-date
to be able to check validity.
The according messages are for illustration only. 
% TBC: what is an interface of a converter?
% Note that \texttt{makeindex} is not in the list.
% This is because it is not possible
% to regularly check the version of that application.
% It may be an option to use \texttt{upmendex} instead,
% although in beta state at time of this writing
% and not completely compatible.
% According to \cite{UpMendex}, \cite{MkIdxMoe}
% and further research,
% \texttt{upmendex} does not support options \texttt{-T} and \texttt{-L}
% and does interprete \texttt{-g} differently,
% namely as japanese instead of german. 


% TBD: make this listing dynamic 
% TBD: this is almost duplicate of lst:version.properites
% maybe one shall be eliminated. 
\begin{lstlisting}[basicstyle=\tiny,
float, captionpos=b, label={lst:vrsOut}, 
caption={Output of goal \texttt{latex:vrs}}]
[INFO] --- latex-maven-plugin:1.6-SNAPSHOT:vrs (default-cli) @ latex-maven-plugin ---
[INFO] Manifest properties: 
[INFO] MANIFEST: (1.0)
[INFO]        Implementation-Version: '1.6-SNAPSHOT'
[INFO] PackageImplementation-Version: '1.6-SNAPSHOT'
[INFO] pom properties:
[INFO] coordinate.groupId:    'eu.simuline.m2latex'
[INFO] coordinate.artifactId: 'latex-maven-plugin'
[INFO] coordinate.version:    '1.6-SNAPSHOT'
[INFO] git properties: 
[INFO] build version:  '1.6-SNAPSHOT'
[INFO] commit id desc: 'latex-maven-plugin-1.5-1-g17f5fad-dirty'
[INFO] buildTime:      '2022-06-07T16:51:25+0200'
[INFO] tool versions: 
[INFO] ?warn?    command             'actual version'(not)in[expected version interval]
[INFO]           pdflatex:           '1.40.24'in[1.40.21;1.40.24]
[INFO]           lualatex:           '1.15.0'in[1.12.0;1.15.0]
[INFO]           xelatex:            '0.999994'in[0.999992;0.999994]
[INFO]           latex2rtf:          '2.3.18 r1267'in[2.3.16 r1254;2.3.18 r1267]
[INFO]           odt2doc:            '0.9.0'in[0.9.0]
[INFO]           pdftotext:          '22.05.0'in[21.04.0;22.05.0]
[INFO]           dvips:              '2022.1'in[2020.1;2022.1]
[INFO]           dvipdfm:            '20211117'in[20210318;20211117]
[INFO]           dvipdfmx:           '20211117'in[20200315;20211117]
[INFO]           xdvipdfmx:          '20211117'in[20200315;20211117]
[INFO]           dvipdft:            '20090604.0046'in[20090604.0046]
[INFO]           gs:                 '9.54.0'in[9.52.0;9.54.0]
[INFO]           chktex:             '1.7.6'in[1.7.6]
[INFO]           diff-pdf-visually:  '0'in[0]
[INFO]           diff-pdf:           '300'in[300]
[INFO]           diff:               '3.8'in[3.8]
[INFO]           pdfinfo:            '22.05.0'in[22.01.0;22.05.0]
[INFO]           exiftool:           '12.41'in[12.39;12.41]
[INFO]           bibtex:             '0.99d'in[0.99d]
[INFO]           bibtexu:            '3.72'in[3.71;3.72]
[INFO]           bibtex8:            '3.72in[3.71;3.72]
[INFO]           makeindex:          '2.16'in[2.15;2.16]
[INFO]           splitindex:         '0.1'in[0.1]
[INFO]           makeglossaries:     '4.49'in[4.45;4.49]
[INFO]           pythontex:          '0.18'in[0.17;0.18]
[INFO]           depythontex:        '0.18'in[0.17;0.18]
[INFO]           latexmk:            '4.77'in[4.70b;4.77]
[INFO]           mpost:              '2.02'in[2.00;2.02]
[INFO]           ebb:                '20211117'in[20200315;20211117]
[INFO]           gnuplot:            '5.4 patchlevel 3'in[5.4 patchlevel 0;5.4 patchlevel 3]
[INFO]           inkscape:           '1.1.2'in[1.0.2;1.1.2]
[INFO]           fig2dev:            '3.2.8b'in[3.2.7b;3.2.8b]
[INFO] excluded tools: 
[INFO] upmendex, xindy
[INFO] ------------------------------------------------------------------------
\end{lstlisting}
%[INFO]           upmendex:           '1.00'in[0.54;1.00]
%[INFO]           xindy:              '2.5.1'in[2.5.1]



Another aspect of document development is integration with other tools. 

Document development starts with the editor. 
Above the emacs editor enhanced with auctex was mentioned. 
We recommend VS code in conjunction with several extensions. 
If VS code itself is already installed 
the script in Listing~\ref{lst:instVScode} installs and updates all extensions 
the author used to develop this manual. 
The core extension is \texttt{latex workshop}, 
the others are mainly used for editing graphic files. 

\lstinputlisting[
language=bash, basicstyle=\scriptsize,
float, captionpos=b, label={lst:instVScode}, 
caption={Install script for extensions of VS code. }]%
{instVScode4tex.sh}


A valuable standalone build tool is \texttt{latexmk}. 
It can even run in background. 
This manual can also be compiled with \texttt{latexmk} 
but only with adapted configuration file \texttt{.latexmkrc}; 
the one depicted in Listing~\ref{lst:latexmkrc}. 

\lstinputlisting[firstline=94, lastline=112,
language=Perl, basicstyle=\scriptsize,
float, captionpos=b, label={lst:latexmkrc}, 
caption={The config file \texttt{.latexmkrc} to compile this manual. }]%
{header.tex}


% to be included. 
This software and \texttt{latexmk} follow a different philosphy in finding dependencies: 
Whereas this software creates image files in advance before invoking a latex converter, 
\texttt{latexmk} first calls the latex converter in nonstopmode 
to avoid a stop because of a missing file. 
Then the file is created using the appropriate rule (hopefully unique) 
and the converter is run again, 
this time passing the inclusion of the first created files 
failing at the next one. 
This goes on that way until the last included file is created. 
Then the latex converter runs through without failure caused by missing files. 

There are two problems with this: 
This yields a huge number of runs for the converter 
which is time consuming 
and there is at least one kind of inclusion which does not work that way: 
inclusion with \texttt{\textbackslash lstinputlisting} 
provided by \pkg{listings}. 
In fact, i have an email from J. Hoffmann, author of listings 
telling that there are more packages with the same problem. 
To be checked: \pkg{fancyvrb} and \pkg{moreverb}. 
Nevertheless, all other ways of inclusion \emph{used by this manual} 
like the one with \texttt{\textbackslash import} 
seem to work fine. 

The current workaround is illustrated in Listing~\ref{lst:fileIfExists} 
modifying \pkg{listings}' output to make it digestable for \texttt{latexmk}. 

\lstinputlisting[firstline=94, lastline=112,
language=tex, basicstyle=\scriptsize,
float, captionpos=b, label={lst:fileIfExists}, 
caption={Patch of the \texttt{listings} package given by the header file. }]%
{header.tex}

Still some generalization in \texttt{latexmk} could spare this modification. 

Another point is, that currently for each file \texttt{latexmk} creates with a separate rule, 
another run of the latex processor is required: 
The initial run is interrupted with the first missing file. 
Then that file is created by an appropriate rule and the latex processor is rerun 
failing with the next missing file. 
That way the process goes on until the last file is created with a rule. 
Of course this proceedure is quite time consuming, so an alternative is required. 

Both, the install script \texttt{instVScode4tex.sh} for VS code extensions 
given in Listing~\ref{lst:instVScode} 
and the config file \texttt{.latexmkrc} depicted in Listing~\ref{lst:latexmkrc} 
are packaged in the JAR file representing this software. 
% TBD: in the long run these listings shall come from the proper pom 
% but with different versions. 
That way these resources can be made available via the maven dependency plugin 
by unpacking from the JAR. 
Listing~\ref{lst:depUnpack} shows the according configuration in the pom. 
In addition, it is available on the web as 
\href{\urlSite fromMain/.latexmkrc}{\texttt{.latexmkrc}}. 
Pasting the into vscode one can see the highlighting, 
and a preview, 
of course provided the extensions Section~\ref{sec:devel} 
and given by Listing~\ref{lst:instVScode} are installed. 


\begin{lstlisting}[language=XML, basicstyle=\scriptsize,
  float, captionpos=b, label={lst:depUnpack}, 
  caption={Part of a pom describing installing files }]
  <plugin>
  <groupId>org.apache.maven.plugins</groupId>
  <artifactId>maven-dependency-plugin</artifactId>
  <version>3.3.0</version>
  <executions>
    <execution>
      <?m2e execute onConfiguration,onIncremental?>
      <id>unpack-latex</id>
      <goals>
        <goal>unpack</goal>
      </goals>
      <phase>initialize</phase>
    </execution>
  </execution>

  <configuration>
  <artifactItems>
    <artifactItem>
      <groupId>eu.simuline.m2latex</groupId>
      <artifactId>latex-maven-plugin</artifactId>
      <version>2.0</version>
      <type>jar</type>
      <overWrite>true</overWrite>
      <outputDirectory>...</outputDirectory>
      <includes>**/.latexmkrc</includes>
      <fileMappers>
        <org.codehaus.plexus.components.io.filemappers.FlattenFileMapper />
      </fileMappers>
    </artifactItem>
    <artifactItem>
      <groupId>eu.simuline.m2latex</groupId>
      <artifactId>latex-maven-plugin</artifactId>
      <version>2.0</version>
      <type>jar</type>
      <overWrite>true</overWrite>
      <outputDirectory>...</outputDirectory>
      <includes>**/instVScode4tex.sh</includes>
      <fileMappers>
        <org.codehaus.plexus.components.io.filemappers.FlattenFileMapper />
      </fileMappers>
    </artifactItem>
  </artifactItems>
</configuration>
</plugin>
\end{lstlisting}

Accordingly, Listing~\ref{lst:depClean} shows a configuration for the pom 
that makes the clean plugin cleanup the unpacked files. 


\begin{lstlisting}[language=XML, basicstyle=\scriptsize,
  float, captionpos=b, label={lst:depClean}, 
  caption={Part of a pom describing cleaning installed files }]
  <!-- configuration for clean plugin -->
  <plugin>
    <artifactId>maven-clean-plugin</artifactId>
    <version>3.1.0</version>
    <configuration>
      <filesets>
        <fileset>
          <directory>...</directory>
          <includes>
            <include>instVScode4tex.sh</include>
          </includes>
        </fileset>
        <fileset>
          <directory>...</directory>
          <includes>
            <include>.latexmkrc</include>
          </includes>
        </fileset>
      </filesets>
    </configuration>
  </plugin>
\end{lstlisting}

\section{Goals in the maven lifecycle}\label{sec:usageLifecycle}

The goal \texttt{latex:cfg} exporting in the formats configured
is tied to the lifecycle phase \texttt{site} so is invoked
when commanding
%
\begin{Verbatim}[fontsize=\scriptsize]
mvn site
\end{Verbatim}
%
or subsequent phase.

Also, the goal \texttt{latex:clr} cleaning created files
both from source directory and from target directory
is tied to phase \texttt{clean} so is invoked
when commanding
%
\begin{Verbatim}[fontsize=\scriptsize]
mvn clean
\end{Verbatim}

Finally, the goal \texttt{latex:vrs} displaying versions of converters
is tied to the phase \texttt{validate}. 
Thus it is thus invoked when commanding
%
\begin{Verbatim}[fontsize=\scriptsize]
mvn validate
\end{Verbatim}
%
which is invoked not only in installation, but also by the site plugin.
This ensures, that the converters are checked for correct version
before being used. 
Note that by default, \texttt{mvn latex:vrs} displays complete version info,
whereas \texttt{mvn validate} only displays warnings if appropriate. 




\section{The ant-tasks}\label{sec:usageAntTask}

This section is missing.
What has to be described are the two tasks. 
