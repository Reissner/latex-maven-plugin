\chapter{Preferred usage, Test Concepts and Tests}\label{chap:tests}

This software may be used in different environments, 
is highly configurable and 
also there is a huge amount of packages potentially in use. 

In order not to get lost in extensive tests for covering all and everything, 
the author applies the notion of\index{preferred usage} \emph{preferred usage}. 
This is essentially the way the author uses this software. 
This is also what is tested extensively. 
Other ways of usage are supported insofar as reported bugs are fixed in general, 
but since explicit tests lack, the quality is lower for these cases. 

The preferred usage is defined as follows: 
%
\begin{itemize}
  \item Linux, to be more precise, SuSE tumbleweed. 
  The author used this software frequently and always with success on Windows also. 
  As a shell use git shell. 
  \item \LaTeX{} Distribution \texlive, to be more precise the SuSE specific variant. 
  In the long run \miktex{} must be at least tried also. 
  As \miktex{} is available for Linux also, test will be under Linux. 
  \item 
  VS Code with the extensions installed by \texttt{instVScode4tex.sh} 
  which is described in Section~\ref{subsec:instExtVsCode}, 
  and the viewer \texttt{okular}. 
  This is defined here, although not going into the artifacts. 
  \item The VS Code extension \texttt{james-yu.latex-workshop}, 
  which is installed by \texttt{instVScode4tex.sh} is used only with build recipe 
  \texttt{latexmk (latexmkrc)}. 
  \item If \texttt{latexmk} is used, then with config file \texttt{.latexmkrc}, 
  whereas \texttt{chktex} is definitely used, 
  and it is configured with config file \texttt{.chktexrc}. 
  Both config files are injected by \texttt{inj} described in Section~\ref{subsec:latChkRc}. 
  \item Neither \texttt{latexmk} nor \texttt{chktex} is used with options. 
  \item The maven plugin, rather than the ant task. 
  \item Configuration is the default setting of this plugin. 
  In particular, the latex processor is \lualatex. 
  \item Document classes are \texttt{book}, \texttt{article}, 
  \texttt{beamer}, \texttt{leaflet} and \texttt{scrlttr2}. 
  For beamer documents, preferred usage are both, the presentation and the handout 
  as described in Section~\ref{subsec:sourcesLatex}. 
  Note that, unlike most other aspects of preferred usage, 
  this is not tested through this manual 
  but with the beamer presentation and handout given by~\cite{PresBeamer}. 
  Also, the test for \texttt{leaflet} class is tested by document~\cite{LeafLeaf} 
  and the test for the letter class is tested in~\cite{LetScrlttr2}. 

  Caution: Currently, the classes \texttt{letter} and \texttt{report} 
  are not preferred usage, 
  but the latter is accepted with default configuration without warning. 
  In the long run, the manual shall be a report, whereas there shall be a user guide 
  which shall be a book. 
  
  Among the document classes provided by KOMA script 
  described in~\cite{KomaAnl23}, 
  only \texttt{scrlttr2} is preferred usage. 
  It replaces \texttt{letter}. 
  \item \LaTeX{} packages are those given by \texttt{header.tex} 
  described in Section~\ref{subsec:header}. 
  \item Graphics are in the formats described and used in the manual. 
  \item Using tools pythontex, makeindex, splitindex, makeglossaries 
  belongs to the preferred usage. 
  This does not mean that these tools must be used, 
  but it means that the usage is restricted to these tools, not taking graphics into account. 
  \item Tools must be used with accepted version, 
  in the sense that \texttt{mvn latex:vrs} does not emit a warning. 
  \item The output format is PDF\@. 
  \item The support of reproducible PDF files and here of the first kind: 
  Reproducibility does not only refer to the visible aspects of the PDF file, 
  but also to meta-data. 
  \item Reproducibility is reached by manipulating the system time, 
  although as an alternative method also suppressing writing meta-info into the PDF file is supported. 
  \item 
  Security issues are addressed by including a header described in Section~\ref{subsec:headerSuppressMetaPDF}. 
\end{itemize}


It is the ``founded conviction'' of the author, that in most of the use cases, 
restriction to the preferred usage is possible 
but when deviating, there is some increased risk 
that there is a bug in this \LaTeX{} builder. 

The set of documents coming with this software 
are compiled sticking to the preferred usage. 
Above all, this manual~\cite{LatexPlugin} not only describes all vital features, 
but also uses them, with one big exception: Its document class is \texttt{book}, 
and it cannot have other document classes of course. 
Most other documents are \texttt{article}s 
and~\cite{PresBeamer} refers to a presentation with class \texttt{beamer} 
and to the according handout which is again an article. 
So the build process for these documents altogether 
cover the preferred usage to a wide extent. 
Thus, a bug in this \LaTeX{} builder 
is likely to be reflected in a deficiency 
in the compiled version of one of these documents. 

This shows that testing the compiled documents is a reasonable test strategy. 
It is not feasible to do this manually for whole documents, 
and it is also technically close to impossible to do it automatically. 
What can be checked automatically is coincidence with the last document. 


So the strategy is to either change the (source of the) software 
or the \emph{source of the} documents, but never both at once. 
If the software is changed, the created documents must persist. 
Changes in the manual are locally in the sources 
and result in local changes in the compiled document, 
because the software was not changed. 
Thus, the compiled document can be checked manually. 
Since the only output format being part of the preferred usage is PDF, 
only compilations into PDF must be taken into account. 

Section~\ref{sec:chkReprod} describes how this \LaTeX{} builder 
can perform an equality check on PDF documents. 
There, both Section~\ref{sec:paramRepro} on parameters for equality check is referenced 
and Table~\ref{tab:TLP} comprising build failures if the documents do not coincide. 
We highlight the parameter \texttt{chkDiff} which determines whether the check is performed 
and build failures \texttt{TLP01} thrown if the documents differ. 
In tests, \texttt{chkDiff} is set \texttt{true}, the default is \texttt{false}. 
\medskip


Although this test concept seems appealing, 
it is not always easy to realize. 

Before explaining the difficulties, let us differentiate between the two ways 
the pom of this plugin uses this plugin itself. 
The pom for performing tests 
is based on \href{\urlSite fromMain/pom4pdf.xml}{pom4pdf.xml}, 
not on the actual \texttt{pom.xml} of the project 
a version of which is 
\href{https://github.com/Reissner/latex-maven-plugin/tree/master/maven-latex-plugin}%
{on github}. 
For \texttt{pom4pdf.xml}, the current version is determined by filtering, 
which remains correct even during the release process. 
In addition, the project \texttt{pom.xml} used for development 
contains another, explicit dependency to \texttt{latex-maven-plugin}. 
This one is used for creating the documents for the site 
and also for developer tests. 
Thus, during development \texttt{pom.xml} is kept close to \texttt{pom4pdf.xml}, 
and it has a snapshot version \texttt{x.y.0-SNAPSHOT}. 

Let us first consider the case of development 
in which the version of this plugin is a SNAPSHOT version. 
Then tests refer to the (snapshot) version under consideration. 
If a change is made to the documents and all tests pass, 
the changed documents are compiled with current software, 
and go into the next snapshot deployed. 
To be precise, the documents are compiled with deployed software, 
which is equivalent with the software compiled from the current sources. 

If in contrast the software is changed, keeping the manual unchanged, 
then still compilation of the documents and also check 
is performed with the deployed version of this \texttt{latex-maven-plugin}. 
So, to decide whether the documents remain the same after the software change, 
a second build must be performed, 
because this compiles with the newly deployed snapshot of this software. 



The situation is even more complicated 
if development is finished for the current version 
and a new release must be built. 
As is state of the art, 
for this task the \texttt{maven-release-plugin} is used. 
It requires for sake of reproducibility, 
that the pom of the project, not the pom  has dependencies and plugins only in release versions, 
no snapshot versions. 

As is state of the art, 
for release the \texttt{maven-release-plugin} is used. 
Whereas it has no explicit restrictions on the pom for tests \texttt{pom4pdf.xml}, 
it requires for sake of reproducibility, 
that the pom of the project \texttt{pom.xml} has dependencies and plugins 
only in release versions, no snapshot versions. 
This applies also to this plugin. 
For development, it has version \texttt{x.y.0-SNAPSHOT} 
to deploy \texttt{x.y.0}, and this is also the version one wants to create a site with, 
but this is the one to released at present. 
A possible fallback is always to deactivate the usage of this plugin. 
As a consequence, later a version \texttt{x.y.1} shall be released, 
which uses \texttt{x.y.0} for site creation. 
Better is to use the last release version and to configure it so 
that the documents can be compiled with the old version. 
This may require a creative release planning, including features used to compile documentation 
and maybe a change in the parameters or some other change in the environment, 
which must be compensated in later releases also. 

Let us give examples of creative realizations of the described test concept 
relying on thorough release planning. 
To release \texttt{2.0.0} starting with the prior version \texttt{1.8.0}, 
almost only injections are added. 
These can be done manually using \texttt{2.0.0-SNAPSHOT}. 
Then the injected files are checked in into version control 
and then the documentation can be compiled with old version \texttt{1.8.0} 
with the same result. 
In a release \texttt{2.0.1}, version \texttt{1.8.0} can be replaced by \texttt{2.0.0}. 

For version \texttt{2.1.0} it is planned, that this plugin can use \texttt{latexmk}, 
and in the manual this is also described. 
In \texttt{2.1.0-SNAPSHOT} the manual may be compiled using \texttt{latexmk}, 
but nevertheless, in \texttt{2.1.0} the manual is still compiled without \texttt{latexmk}, 
using release \texttt{2.0.0} for creating the manual for the site. 
Only in \texttt{2.1.1}, also the manual is included in the site using \texttt{latexmk}. 

For version \texttt{2.2.0} it is planned, to support \texttt{bib2gls} directly. 
Observe that \texttt{2.1.0} supports can treat \texttt{bib2gls} via \texttt{latexmk}, 
but without all the monitoring \texttt{2.2.0} offers with direct support. 
Of course, the manual describes direct support and 
some \texttt{2.2.0-SNAPSHOT} is able to compile the manual using \texttt{bib2gls} directly. 
Nevertheless, close to release for site creation, 
\texttt{2.1.0} is used again relying on \texttt{latexmk} to invoke \texttt{bib2gls}. 
In \texttt{2.2.1} then \texttt{2.2.0} can be used for site creation invoking \texttt{bib2gls} directly. 


Note that the test concept based on preferred usage has a considerable weakness: 
It cannot test warnings, errors and exceptions 
because they are not preferred usage. 
On the other hand, it is an important design goal, 
that the result of this software is trustable if no warning, error or exception occurs. 
This requires extensive tests also on imperfect runs. 
These must be supplemented in the future. 

FIXME\@: this chapter describes the tests to be performed. 

Missing are tests on logging, tests on various input formats, 
output formats, tests including several paths 
defined by invocation of auxiliary applications for index, glossary, \dots 

