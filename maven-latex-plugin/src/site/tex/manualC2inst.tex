% !TEX root = manualLMP.tex

\chapter{Installation}\label{chap:install}

Both the ant-task\index{ant-task} and the maven-plugin just direct parameters 
from ant and from maven, respectively, 
to the programs that do the proper work. 
Thus, installation of the ant-task and of the maven-plugin 
requires that all needed programs are installed. 
These prerequisites are collected in Section~\ref{sec:prerequisites}. 

\section{Prerequisites}\label{sec:prerequisites}

The ant-task is tested with\index{ant} 
%
\begin{verbatim}
Apache Ant(TM) version 1.10.12 compiled on December 14 1969}
\end{verbatim}
%
(of course the year is not correct, but this is the version string
displayed by that release) and the maven-plugin with\index{maven} 
%
\begin{verbatim}
Apache Maven 3.9.2 
\end{verbatim}
%
Both, ant and maven are written in java and require a java installation. 
The java\index{java} version used for tests is \texttt{17.0.8}. 


So, a java installation is the base for running either the ant-task 
or the maven-plugin. 
Also, this plugin is written in java. 
To use the maven-plugin, of course maven must be installed 
and to use the ant-task, ant must be installed. 

The ant-task\index{ant-task} just passes parameters in the build file to the core 
and accordingly the maven-plugin passes parameters in the pom 
to the core of this software. 
The core just invokes various programs to do the actual work. 

Besides plain building of documentation, 
this software also supports development of documents. 
\LaTeX{} and related programs are based on text files mainly 
and so a good editor is required for development. 

The author recommends and uses VS Code, e.g.\@ 1.81.1 
in conjunction with package \LaTeX{} workshop 9.13.4 and LTeX 13.1.0. 

An alternative is good old 
%
\begin{verbatim}
  GNU Emacs 24.3.1 (x86\_64-suse-linux-gnu, GTK+ Version 3.16.7)
\end{verbatim}
%
together with several packages to support 
various file formats. 
To list the available packages type 
\texttt{M-x list-packages}. 
For comfortable development with \LaTeX, 
the \texttt{auctex} package, version \texttt{11.88} is recommended. 
The version is displayed from within Emacs 
by typing \texttt{C-h v AUCTeX-version RET}. 
For an overview on \texttt{auctex} see~\cite{AucTeX}. 


FIXME\@: gnuplot-mode expects file extension gp. 
Should be made configurable. 

To edit metapost, the mode built-in mode \texttt{Metamode} is used. 

Built-in mode \texttt{Docview} to view PDF, PS and DVI\@. 

latexmk

Builtin modes bib-mode and bibtex

Built in reftex-modes

Useful: 
ac-math, auto-complete-auctex

Depending on what kinds of graphic formats are used, 
the following programs are required: 
%
\begin{itemize}
\item
To convert the \gls{fig}-files into \gls{pdf}-files, 
by default \texttt{fig2dev}\index{fig2dev} is used. 
It makes sense to have \texttt{xfig}\index{xfig} installed 
to create and edit fig-files, but this is not mandatory. 
\item
To convert gnuplot files into PDF-files, there is no alternative, 
to have installed \texttt{gnuplot}\index{gnuplot}. 
It serves as an interpreter and also as a converter. 
Strictly speaking, only the latter functionality is required here. 
\item
To convert \gls{mp}-files into \gls{eps}-files, 
the interpreter \texttt{mpost}\index{mpost}\index{metapost} or equivalent is required. 
This comes with a standard \TeX{} installation. 
With the standard configuration, 
the resulting EPS-file can be viewed with \texttt{ghostscript} 
and for developing it is recommended to have \texttt{ghostscript} installed. 
\item
To include \gls{svg}-files into \LaTeX\index{svg}, 
\texttt{inkscape}\index{inkscape} must be installed. 
It also serves to create and to edit SVG-files. 
\end{itemize}


Currently, for including PDF-files in both cases, 
the driver \texttt{dvipdfmx} must be installed. 
Strictly speaking, this is required only for HTML-creation and related. 
Note that if no pictures created by \texttt{fig2dev}, \texttt{gnuplot}, 
\texttt{mpost} or by \texttt{inkscape} are used, of course, 
neither \texttt{fig2dev} nor \texttt{gnuplot},
\texttt{mpost}, \texttt{inkscape} 
nor \texttt{dvipdfmx} is needed. 
To include graphics, the graphics bundle described in~\cite{GraX} is required, 
except for SVG-files which requires the svg-package 
described in~\cite{SvgP}. 

As the set of required software depends on the graphic formats 
which shall be imported, 
it depends also on the set of output-formats 
to be supported: 
%
\begin{itemize}
\item
To create PDF-files from \LaTeX-files we use \lualatex\index{lualatex} 
or some other kind of \LaTeX{} creating PDF-files 
like \xelatex\index{xelatex} or \pdflatex\index{pdflatex}. 

\LaTeX{} uses several auxiliary programs. 
Above all \texttt{bibtex}\index{bibtex}, to create the bibliography 
and \texttt{makeindex}\index{makeindex} and \texttt{splitindex}\index{splitindex} for the index 
and \texttt{makeglossaries}\index{makeglossaries} for the glossary. 
The latter two 
also require the latex packages \pkg{makeidx}, optionally \pkg{showidx}, 
both described in~\cite{MkidxShIdxP}, 
the package \pkg{splitidx} documented in~\cite{SplitidxP}
and \pkg{glossaries} specified in~\cite{GloP}. 
Note that \texttt{makeglossaries} either invokes \texttt{makeindex} 
or \texttt{xindy}, depending on the parametrization of \pkg{glossaries}. 
Both, \texttt{makeglossaries} and \texttt{xindy} are written in Perl, 
which shall also be installed if a glossary is required. 

To include program code in Python, octave and other language, 
\texttt{pythontex}\index{pythontex} is needed; to eliminate that code creating an equivalent tex-file, 
one has to combine it with \texttt{depythontex}\index{depythontex}. 
Both are written in Python3 which shall be installed also as a dependency. 
To use them, one also needs to install the package \pkg{pythontex}. 

The package \pkg{rerunfilecheck} is in any standard \LaTeX-installation. 
It is almost mandatory 
because this software presupposes that package is present  
to ensure that the table of contents, list of figures, list of tables, 
the index and the glossary are up-to-date. 

It is standard to endow a PDF-file with hyperlinks. 
To support this, the package \pkg{hyperref} is required. 

****
\item
To create \gls{html}-files, 
or to be more precise any kind of \gls{sgml} and \gls{xml}, 
from \LaTeX-files, \texttt{htlatex} or alternatively \texttt{htxelatex} is used. 
Currently, the author is not aware of any alternative to the two. 
This includes also creating OpenOffice documents like ODT-files. 
Thus, OpenOffice documents are created in two steps, 
the first is to create PDF-files with the according tools, 
the second one is done by \texttt{htlatex}\index{htlatex} or that like.\index{htxelatex} 
\item
To create RTF-files, currently \texttt{latex2rtf}\index{latex2rtf} is used. 
Note that this does not require \pdflatex. 
As a drawback, not all \LaTeX-packages are supported. 
\item
MS Word documents are created from OpenOffice documents 
via the command \texttt{odt2doc}\index{odt2doc}  and thus require three steps 
and so the according tool chain. 
\item
Finally, there is a way, to create plain text files from the PDF-files 
via \texttt{pdftotext}\index{pdftotext}. 
The way from \LaTeX{} to text via PDF makes sense 
because that text is well formatted math mode symbols like $\pi$ 
and because table of contents, index, glossary and that like are included. 
So, for that task, besides \texttt{pdftotext} the whole tool chain to create
PDF-files is required. 
\item
An application which does not create a target, 
i.e.~a file in the target directory is \texttt{chktex}\index{chktex} 
which just checks the latex main files and associated files. 
\end{itemize}

So to run this software, the aforementioned programs 
or at least the subset used, must be installed.
To obtain reproducible results, the versions must fit.
This version is checked with the executables with versions given by 
an according properties file 
\href{\urlSite fromMain/version.properties}{version.properties}. 

% TBD: rework on the above list of programs
% since it has been extended.
% in particular pythontex and latexmk


There are also several \LaTeX-packages needed or at least recommended. 
The recommended ones are 
%
\begin{description}
\item[\pkg{geometry}] described in~\cite{GeomP} 
to control page layout. 
\item[\pkg{microtype}] described in~\cite{MicroTyP} improve readability 
and make the document look nicer. 
It also helps to avoid bad boxes. 
\item[\pkg{hyperref}] described in~\cite{HyperTextP} 
to insert hypertext marks, which I do not want to miss in larger documents. 
% FIXME: with latex2rtf: 
%\newif\iflatextortf% false for latex but set true for latex2rtf internally 
%\iflatextortf\else\usepackage{hyperref}\fi % includes hyperrefss if not rtf
%\iflatextortf\else\begin{appendix}\fi
%\iflatextortf\else\end{appendix}\fi
\item[\pkg{srcltx}] described in~\cite{SrcLtxP} 
which allows jumping from the DVI file to the TEX source and back.
% TBD: indicate as deprecated. 
\item[\pkg{showframe}] 
if \pkg{geometry} is not used with option \pkg{showframe}. 
There seems to be no package documentation for package \pkg{showframe}. 
\item[\pkg{booktabs}] described in~\cite{BooktP} 

\item[\pkg{anyfontsize}] described in~\cite{AnyfontsizeP} 
to allow arbitrary font sizes, eliminating certain warnings. 
An alternative may be \pkg{fix-cm} described in~\cite{FixCmP}. 
\end{description}

\noindent
Almost required are 
%
\begin{itemize}
\item
\pkg{rerunfilecheck} described in~\cite{RerunFChkP} 
which writes additional rerun warnings to the log file 
if some auxiliary files have changed. 
This software relies on these warnings 
to control rerun latex and other applications. 
\item
\pkg{iftex} described in~\cite{IfTeXP} which has two functions: 
\begin{itemize}
  \item 
  It provides the \cmd{ifpdf}-command to detect pdf-mode. 
  This is required to distinguish creation of PDF and text 
  from HTML, ODT, DOC and others, based on DVI/XDV\@. 
  \item 
  Also, it is able to detect a specific latex engine via commands 
  like \cmd{ifluatex} or \cmd{ifpdftex} but also \cmd{iftutex} 
  being true for \lualatex{} and \xelatex{}, but not for \pdflatex. 
  This is used if a document shall work for more than one engine 
  like this manual and is in particular used to create reproducible PDF files 
  which is engine specific. 
  Finally, there is a way to force an exception if the wrong engine is used, 
  e.g.~by specifying \cmd{RequireLuaTeX}. 
\end{itemize}
\item
The graphics packages described in~\cite{GraX}, 
in particular \pkg{graphicx}, \pkg{xcolor} and \pkg{transparent}, 
the latter two described in~\cite{XColorP} and in~\cite{TransP}, 
respectively. 
Sometimes also \pkg{bmpsize} described in~\cite{BmpP} 
if pixel graphics is used. 
\item
\pkg{import} described in~\cite{ImpoP} 
e.g.~to import nested graphic files from arbitrary directories. 
\item
\texttt{inputenc} described in~\cite{InputencP} 
to select an input encoding 
\texttt{fontenc} to select a font encoding. 
Font selection is described in~\cite{FontSel} in general, 
with Section 5 on font encoding and 
Section 5.1 on the \texttt{fontenc} package. 
This package is almost indispensable if you do not write English, 
e.g.~to access German umlauts. 
Note that~\cite{FontEnc} describes font encoding in more detail. 
\item 
\pkg{makeidx} and \pkg{showidx} described in~\cite{MkidxShIdxP} 
or something comparable for creating indices. 
\item 
\pkg{glossaries} described in~\cite{GloP} 
with tutorial~\cite{GloPGuide}
or something comparable for creating glossaries. 
\item 
\pkg{tocbibind} described in~\cite{TocBibIndP} 
to include bibliography and index (what about glossaries?) 
into the table of contents. 
\item 
\pkg{nag} described in~\cite{NagP} 
which performs certain checks unveiling deficiencies 
not filtered by the compiler nor by another check tool. 
\item 
\pkg{babel} described in~\cite{BabelP} for language support. 
This is not used by this manual, because it is in English. 
\end{itemize}

\noindent
Useful packages with which this software is tested: 
%
\begin{itemize}
\item
The ams-packages **** \pkg{amsmath}
\item
\pkg{longtable} described in~\cite{LongTabP} 
for long tables, i.e.~tables exceeding a page. 
\item
\pkg{listings} described in~\cite{ListingsP} for listings. 
\item
\pkg{fancyvrb} described in~\cite{FancyVerbP} 
provides useful environments to mark verbatim text. 
\end{itemize}


\section{Setting up pom.xml for the maven plugin}\label{sec:xmlPom}

If this software is used as a maven plugin,
it need not explicitly be installed, maven itself does this by need
based on the entries of the pom. 

We can distinguish between entries in the pom 
which are necessary for any kind of usage of this maven plugin 
described in Section~\ref{subsec:xmlPomBasic}, 
configuration settings to obtain behavior deviating from the default 
as sketched in Section~\ref{subsec:xmlPomSettings} 
and finally executions to integrate this plugin in the lifecycle 
as described in Section~\ref{subsec:xmlPomExecutions}. 

\subsection{Basic setup}\label{subsec:xmlPomBasic}

% TBD: add the plugin to maven central
Unfortunately, this plugin did not yet make it into maven central.
Thus, one has to add the providers' repository to the pom
as shown in Listing~\ref{lst:srcRepo} to enable maven to find the software. 

\begin{lstlisting}[language=xml, basicstyle=\footnotesize,
escapechar=|,
float, captionpos=b, label={lst:srcRepo}, 
caption={The source repository for this plugin}]
<project ...>
  ...
  <repositories>
    <repository>
      <id>publicRepoAtSimuline</id>
      <name>repo at simuline</name>
      <url>|\repo|</url>
    </repository>
  </repositories>
  ...
</project>
\end{lstlisting}

Then just adding the coordinates in the build-plugins section of the pom
as shown in Listing~\ref{lst:coords}, 
and it can be used from command line,
e.g.~to create PDF files as \texttt{mvn latex:pdf}
or for cleanup \texttt{mvn latex:clr} with default configuration. 
Alternatively, the plugin can be inserted also 
in the reporting-plugins section of the pom. 
%
%\lstset{language=xml, basicstyle=\small}
\begin{lstlisting}[language=xml, basicstyle=\footnotesize,
escapechar=|,
float, captionpos=b, label={lst:coords}, 
caption={The coordinates of this plugin}]
<project ...>
  ...
  <build>
    ...
    <plugins>
      ...
      <!-- create html and pdf and other formats from latex -->
      <plugin>
        <groupId>|\groupId|</groupId>
        <artifactId>|\artifactId|</artifactId>
        <version>|\strippedVersionID|</version>
      </plugin>
     ...
   </plugins>
    ...
  </build>
  ...
</project>
\end{lstlisting}

This plugin is indexed 
in \url{https://mvnrepository.com/artifact/eu.simuline.m2latex/latex-maven-plugin}. 

\subsection{Deviating from default settings}\label{subsec:xmlPomSettings}

This plugin is highly configurable by means of a lot of settings. 
Listing~\ref{lst:coordsConfig} shows some of them, but all are in their default settings, 
so no need to specify them explicitly: 
%
\begin{description}
\item[\texttt{targets}] defines the output formats and the checks to be run 
(\texttt{chk} signifies running \texttt{chktex}) for goal \texttt{cfg}. 
\item[\texttt{cleanUp}] whether all generated files shall be removed 
leaving the \LaTeX{} source folder untouched. 
\item[\texttt{latex2pdfCommand}] is one of the names of the tools to be invoked. 
There are more settings for treating tools. 
\end{description}

It is the experience of the author, 
that in many situations no explicit setting is necessary at all. 
The only setting needed to be configured regularly is \texttt{targets} 
which determines the output formats and whether sources are checked 
for goal \texttt{cfg}. 
It is recommended to use checking via target \texttt{chk} in any case. 
Some settings are only relevant only for document development 
as described in Section~\ref{sec:devel}, 
one of these is \texttt{cleanUp}: setting this to \texttt{false} 
keeps intermediate files, in particular log files available 
which helps to find errors and in locating the sources of warnings. 
There are further situations where the user is grateful 
of being able to configure this software, 
or even where it is not usable with default settings. 
Chapter~\ref{chap:settings} describes the complete set of settings. 
The same pieces of information 
is given in the \href{\urlSite fromMain/pom4pdf.xml}{pom.xml} 
used to test this plugin. 
Although each setting takes its default value, 
it is given explicitly and is endowed with a comment 
describing it in detail as in Chapter~\ref{chap:settings}. 
Since this pom is quite long, it is not part of this manual 
but is given by reference on the project site. 
% TBD: the explanations are in three places: 
% javadoc comments, pom and chapter {chap:settings}. 
% If it were possible to display apidocs also nested config 
% by latex:help, 
% javadoc comments alone would be sufficient. 
% This would realize single source principle 
% and would make a shorter manual which also compiles much faster. 


% TBD: analysis when configuration must be made explicit at all. 
%
% There is a general situation: output formats: affects targets 
% whether chk is desired: affects targets 
%
% one of the situations is document development. 
% This affects the following settings: 
% readTexSrcProcDirRec, targets, mainFilesIncluded, mainFilesExcluded, cleanUp, 
% debugBadBoxes, debugWarnings, maxNumRerunsLatex
%
% environment: convertersExcluded, all settings which determine names of converters. 
% Sometimes one wants to send sources and so need to adapt to foreign environments. 
% this affects mostly latex2PdfCommand and metapostCommand: mostl mpost, could also be mp. 
% also bibtexCommand with default bibtex which may be replaced by bibtex8 or by bibtexu
%
% some settings are determined by reverse engineering and may be rendered  invalid 
% just by having a new version, either of a tool or of a package 
% or just by use another package not forseen, or just a situation not taken into account 
% This affects: patternLatexMainFile, patternCreatedFromLatexMain, patternReRunLatex, 
% patternReRunMakeIndex, patternReRunMakeGlossaries, patternT4htOutputFiles
% also patterns to detect warnings or errors in log files. 
% although some like patternErrMPost seem quite stable. 
% 
% special needs: if a tool does not fit. 
% Then one could write a wrapper or its own implementation. 
% This occurs in case <pythontexCommand>pythontexW:pythontex</pythontexCommand>
% Special: pdfViaDvi if html and pdf via tex4ht
% another application: fig2dev not writing an FLS file: we could use a wrapper script 
% which does write an FLS file. 
%
% other usage: latexmk and chktex: where to read the config file. 
%
% usage of creating own converter: 
% mpost -interaction=nonstopmode -recorder -s prologues=2 -s 'outputtemplate="F3_01texUsagePlain.mps"' F3_01texUsagePlain.mp
% is what this plugin does by default. 
% so it seems sensible to define mympost just invoking mpost that way. 
% this would be a reason to 


\begin{lstlisting}[language=xml, basicstyle=\footnotesize,
  escapechar=|,
  float, captionpos=hb, label={lst:coordsConfig}, 
  caption={The coordinates of this plugin and some settings}]
        <!-- create html and pdf and other formats from latex -->
        <plugin>
          <groupId>|\groupId|</groupId>
          <artifactId>|\artifactId|</artifactId>
          <version>|\strippedVersionID|</version>
          <configuration>
            <settings>
              <!--targets>chk, pdf, html</targets-->
              <!--latex2pdfCommand>lualatex</latex2pdfCommand-->
              <!--cleanUp>true</cleanUp-->
              ...
            </settings>
          </configuration>
        </plugin>
  \end{lstlisting}

\subsection{Executions}\label{subsec:xmlPomExecutions}

% TBD: reference to section 'goals in lifecycle'. 
To make the plugin available within a build,
one has to add executions, e.g.~as shown in Listing~\ref{lst:executions}. 
For all goals specified there, a default phase is defined, 
as given as a comment but as this is hard-coded, 
one has to specify in the executions only 
when deviating from the default. 

Typically, this plugin, to be more precise its goal \texttt{cfg}, 
which allows configuring checks and the output formats in setting \texttt{targets},  
is used in the \texttt{site} lifecycle phase 
to process latex sources. 
It is perfectly ok to stick to a single format like \texttt{pdf} 
and configure \texttt{target} accordingly. 

Alternatively, one may define an execution with the required goals like \texttt{pdf}, 
but then the phase must be specified explicitly, because there is no default phase. 
Of course, then no additional check is performed. 

The goal \texttt{inj} injects files into the working directory \texttt{texSrcDirectory} 
as described in detail in Section~\ref{sec:injFiles}. 
For some files it makes sense to do this independent of the build process 
e.g.\@ by invoking \texttt{mvn latex:inj}, 
but in general it is preferable to perform the injection each build process anew 
because that way the injected file can be adapted 
to the current configuration of this plugin. 

Normally, all created files in the source directory are removed 
after the output has been copied to the target, 
but during document development, described in Section~\ref{sec:devel}, 
cleanup may be deactivated by setting \texttt{cleanUp} 
and so the source directory may be polluted. 
This may happen if other tools are used in conjunction with this plugin. 

Nevertheless, cleanup is recommended to make the individual runs of this plugin independent.
Thus, for document development, there is a dedicated goal \texttt{clr} 
to clean up the source directory in phase \texttt{clean}. 
Note that also the configuration files created by goal \texttt{inj} are cleared. 

Finally, it is recommended to add a check of the converter versions
right in the phase \texttt{validate} via goal \texttt{vrs}.
Listing~\ref{lst:executions} specifies \texttt{versionsWarnOnly=true}, 
which restricts goal \texttt{vrs} to just display a warning 
if something is wrong 
which seems appropriate in the context of validation. 

For the default \texttt{versionsWarnOnly=false}, 
the goal \texttt{vrs} 
yields a full list of registered converters, 
signifying which one may cause trouble 
because its version is out of range
as displayed in Listing~\ref{lst:vrsOut}. 
In the course of a build run, this seems too much information, 
but in fact, it is just a matter of taste. 

For details on executions of goals \texttt{inj}, \texttt{clr} and \texttt{vrs} 
see Section~\ref{sec:devel}. 
% TBD: check and make the reference more precise. 


%\lstset{language=xml, basicstyle=\small}
\begin{lstlisting}[language=XML, basicstyle=\footnotesize,
escapechar=|,
float, captionpos=hb, label={lst:executions}, 
caption={The executions of this plugin}]
<plugin>
  <groupId>|\groupId|</groupId>
  <artifactId>|\artifactId|</artifactId>
  <version>|\strippedVersionID|</version>
  <configuration>
    <settings>
      <!--targets>chk, pdf, html</targets-->
      <!--cleanUp>false</cleanUp-->
      ...
    </settings>
  </configuration>
  <executions>
    <execution>
      <id>process-latex-sources</id>
      <!-- chk, dvi, pdf, html, odt, docx, rtf, txt -->
      <goals><goal>cfg</goal></goals>
      <!-- phase>site</phase-->
    </execution>
    <execution>
      <id>clear-latex-sources</id>
      <goals><goal>clr</goal></goals>
      <!-- phase>clean</phase-->
    </execution>
    <execution>
      <?m2e execute onConfiguration?>
      <id>inject-files</id>
      <goals><goal>inj</goal></goals>
      <!-- phase>validate</phase-->
    </execution>
    <execution>
      <id>validate-converters</id>
      <goals><goal>vrs</goal></goals>
      <!-- phase>validate</phase-->
      <configuration>
        <versionsWarnOnly>true</versionsWarnOnly>
      </configuration>
    </execution>
  </executions>
</plugin>
\end{lstlisting}

The executions considered so far are appropriate 
for mavens default lifecycle. 
Typically, this maven plugin is used in the site lifecycle, 
which does not contain the phase \texttt{validate}, 
but accordingly \texttt{pre-site}. 
As a consequence, goals \texttt{inj} and \texttt{vrs} are not invoked. 
To get around, 
one could specify the phase \texttt{pre-site} in the execution explicitly. 
The author uses the \texttt{maven-jxr-plugin} 
as illustrated in Listing~\ref{lst:forkJxr}, 
which, as a side effect, forks the lifecycle 
and includes phase \texttt{validate} of the default lifecycle 
and in particular goals \texttt{inj} and \texttt{vrs}. 


\begin{lstlisting}[language=XML, basicstyle=\footnotesize,
escapechar=|,
float, captionpos=hb, label={lst:forkJxr}, 
caption={Forked execution with jxr plugin}]
<project>
  ...
  <reporting>
    <plugins>
      ...
      <!-- as a side effect, 
          triggers 'generate-sources' forked phase execution  -->
      <plugin>
        <groupId>org.apache.maven.plugins</groupId>
        <artifactId>maven-jxr-plugin</artifactId>
        <version>3.3.0</version>
      </plugin>
      ...
    </plugins>
  </reporting>
</project>
\end{lstlisting}

Note the section \texttt{configuration} in Listing~\ref{lst:executions}
which is empty and can be skipped in a default configuration
creating output in formats PDF and HTML and performing checks on the \LaTeX-sources. 
However, \href{\urlSite fromMain/pom4pdf.xml}{pom.xml} 
gives an example pom using this latex plugin 
with full configuration with default values and executions.
In addition, Chapter~\ref{chap:settings} describes each setting individually. 
\medskip



\section{Setting build.xml for the ant task}\label{sec:xmlBuild}

As you can see, the \texttt{taskdef}'s refer to java classes.
Unlike maven which loads jars with the classes inside automatically
from
% 
\begin{Verbatim}[fontsize=\small, commandchars=\\\{\}]
\repo%
\end{Verbatim}
%
the jar for the tasks, \createdJar,
must be downloaded manually from
%
\begin{Verbatim}[fontsize=\scriptsize, commandchars=\\\{\}]
\repo/eu/simuline/m2latex/\artifactId/\strippedVersionID%
\end{Verbatim}
%
Moreover, ant expects to find the jar files in an according folder.
In my installation it is \antJarDir;
as can be seen in the ant documentation,
in general it is in folder \texttt{lib} in ant's installation directory. 

However, \href{\urlSite build.xml}{build.xml} 
gives an example build file using this latex ant task\index{ant-task} 
with full configuration with default values and executions.
From that, one has to copy the following
into the \texttt{build.xml} file in the current project:
%
\begin{itemize}
\item The properties \texttt{antJarDir} and \texttt{createdJar}, 
\item The path element with the id \texttt{latex.classpath}
\item The taskdefs \texttt{latexCfg}  and \texttt{latex:Clr}
\item The  targets \texttt{latex:cfg} and \texttt{latex:clr}
\end{itemize}

As for the maven plugin, for the ant task, add configuration, 
where a deviation from the default requires to do so. 
The configuration is the same and is described in detail 
in Chapter~\ref{chap:settings}. 



\section{Installation from source}\label{sec:instSrc}

The first step to install from source, is to clone from the repository by
%
\begin{Verbatim}[commandchars=\\\{\}]
git clone \devSite{}
\end{Verbatim}
%
of course assuming that \texttt{git} has been installed.
Then change into the root repository where \texttt{pom.xml}
for maven and also \texttt{built.xml} for ant are located. 

To install the maven-plugin, ensure that maven is installed. 
One is tempted just to type 
%
\begin{Verbatim}
mvn clean install
\end{Verbatim}
%
but this does not work since the plugin needs itself to be installed
to perform even \texttt{clean}.
To solve that problem just comment out all its executions
in the local \texttt{pom.xml} by enclosing them in \texttt{<!--\ldots-->}.
In fact this is a minor bug, since, to be strict, only
the executions for verification and clearing must be deactivated.
For processing, it would be sufficient to add
\texttt{<phase>site</phase>} to execution \texttt{process-latex-sources}.
\medskip


Since the author develops with maven,
including the development of the ant task,
the maven built, creates the file \createdJar{}
defining the ant task.
To this end, also \texttt{mvn clean package} is sufficient.
After that, installation proceeds like described in Section~\ref{sec:xmlBuild}
copying that jar file ant's lib-folder where ant can find it.

With root access and after having checked the proper paths,
the build file \texttt{build.xml} can be used 
to perform copy task by \texttt{ant install},
to insert an according link by \texttt{ant link}
to remove it again with \texttt{ant uninstall}.
The build file \texttt{build.xml} works only
if \createdJar{} is placed where ant can find it
or if the parts are deactivated below the line
%
\begin{Verbatim}
  <!-- deactivate the following,
  unless the ant task is installed already -->
\end{Verbatim}

I feel building with maven and linking the jar created
is a very good way to develop the ant task,
because after changes the new ant task is available immediately.

For typical changes in the sources,
it is possible to recompile and package the ant task
by \texttt{ant jar} also cleanup is possible with \texttt{ant clean}.
Finally, the ant task can be tested with \texttt{ant latex:cfg}
and \texttt{ant latex:clr}. 

In the long run, it should be possible to build the ant task\index{ant-task} from sources
with ant alone.


