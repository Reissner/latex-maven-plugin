\RequirePackage[l2tabu, orthodox]{nag}
\documentclass[a4paper, english]{article}%12pt, 

\usepackage{listings}

%\listfiles
\synctex=1% maybe security issue: draft only? 

% for buildParams to check empty: \ifdefempty
%\usepackage{etoolbox}
% for buildParams: \verbdef 
%\usepackage{newverbs}

% provdies \ifPDFTeX, \ifXeTeX and \ifLuaTeX. 
% iftutex test is true for XeTeX and LuaTeX, 
% and an ifpdf test is provided to test the PDF or DVI output mode.
\usepackage{iftex}

% provides \newboolean, \setboolean 
% is used to integrate html production with tex4ht and pdf production
% used to define texFhtLoaded and beamerLoaded 
% maybe this is not really absolute necessary 

\usepackage{ifthen}
\newboolean{texFhtLoaded}
\setboolean{texFhtLoaded}{false}

% only with pdflatex, warnings for xelatex and for lualatex 
% ifxetex, ifluatex, ifpdf
\ifpdf%
  %\usepackage{mlmodern}
\else
  \makeatletter
  \@ifpackageloaded{tex4ht}{%
    \setboolean{texFhtLoaded}{true}
  }{%
  }% tex4ht not loaded 
  \makeatother
\fi


\newboolean{beamerLoaded}
\makeatletter
\@ifclassloaded{beamer}{%
  \setboolean{beamerLoaded}{true}
}{
  \setboolean{beamerLoaded}{false}
}
\makeatother



\iftutex%
  \usepackage{fontspec}
\else
  % this seems to work with beamer also 
  \usepackage[utf8]{inputenc}
  \usepackage[T1]{fontenc}
\fi
%\usepackage{textalpha}


% absolutely necessary. 
% for document development add certain options. 
% Then remove headline and prevent this plugin from overwriting. 

\ifthenelse{\boolean{beamerLoaded}}{
  % here nothing to do. 
  % beamer loads geometry itself. 
  % The option a4paper does not make sense; 
  % one may set aspectratio in \documentclass
}{
  \usepackage[a4paper]{geometry}% option , showframe, showcrop 
}
%\usepackage{showframe} as an alternative 
\usepackage{microtype}
%\usepackage[indent,skip=0]{parskip}% used by pandoc but not good 
% special characters
\usepackage{textcomp}
\usepackage{anyfontsize}% important e.g. for beamer class 
%\usepackage{cleveref}


% used by hyperref and also to update index and glossary 
% to avoid clash because of loading with different options: 
% declare first 
% Note that without options the check is the most strict one 
\usepackage{rerunfilecheck}

% graphics 

\ifpdf%
  % for accessability with luatex
  %\usepackage{luatex85}
  % compiles for xelatex only 
  %\usepackage[tagged, highstructure]{accessibility}
  \usepackage{xcolor}  % [pdftex]  
  \usepackage{graphicx}% [pdftex] 
  % driver [hpdftex] is autodetected 
  \usepackage[destlabel]{hyperref}
  % sometimes comes in with svg import 
  \usepackage{transparent}
  % warning transparent package: 
  % loading aborted if not pdf-mode 
  % strange: according to documentation not for xelatex; 
  % seems to work anyway 
  % can be extended using l3opacity
\else
  % No PDF, includes dvi/xdv and HTML,... via package tex4ht 
  \usepackage[dvipdfmx]{xcolor}
  \usepackage[dvipdfmx]{graphicx}
  \ifthenelse{\boolean{texFhtLoaded}}{%
    \usepackage[tex4ht, destlabel]{hyperref}
  }{%
    \ifxetex%
      \usepackage[destlabel]{hyperref}
    \else
      \usepackage[dvipdfmx, destlabel]{hyperref}%[dvipdfmx]
      % lualatex: without [dvipdfmx] option did not find 
      % converter dvi to pdf or to ps
      % pdflatex: without [dvipdfmx] option dvips still works, 
      % but no converter for pdf
    \fi
  }% tex4ht not loaded 
  %\usepackage{bmpsize}% not for xelatex 
\fi%ifpdf

\ifluatex%
  \usepackage{luamplib}
  \newcommand*\inputmpcode[1]{\begin{mplibcode}input #1\end{mplibcode}}
\else
\fi

% \@ifpackageloaded{tex4ht}{%
% \usepackage[dvipdfmx]{xcolor}
% \usepackage[dvipdfmx]{graphicx}
% \usepackage[tex4ht]{hyperref}
% \usepackage{bmpsize}
% }{%
% \usepackage{xcolor}  % [pdftex]  
% \usepackage{graphicx}% [pdftex] 
% \usepackage{hyperref}% driver [hpdftex] is autodetected 
% }


%\usepackage[clear,pdf,eps]{svg}

\usepackage{import}
\usepackage{amsmath}

% synchronization between tex and pdf 
%\usepackage[active]{srcltx}
\usepackage{longtable}
\usepackage{listings}
% this is a workaround for including listings with latexmk.. 
% This can be fixed 
% - as shown below 
% - patch in package listings 
% - patch in latexmk 
% I would prefer the latter. 
\usepackage{xpatch}
\makeatletter
\newcommand*{\NewLine}{^^J}%
\xpatchcmd{\lst@MissingFileError}
{Package Listings Error: File `#1(.#2)' not found.}
{LaTeX Error: File `#1.#2' not found.\NewLine}{%
  \typeout{File ending patch for \string\lst@MissingFileError\space done.}%
}{%
  \typeout{File ending patch for \string\lst@MissingFileError\space failed.}%
}
\makeatother

\usepackage{fancyvrb}


% index and glossary
\ifthenelse{\boolean{texFhtLoaded}}{%
  \newcommand{\pkg}[1]{\texttt{#1}}% without indexing 
}{
  \usepackage{splitidx}%[split]
%  \usepackage{makeidx}
%  \usepackage{showidx}
  \makeindex
  \usepackage[toc]{glossaries}%,automake
  % , xindy or even [xindy={language=english,codepage=utf8}]
  % mainly for index and glossaries 
  %\makeglossaries% TBD: activate later
  \newcommand{\pkg}[1]{\texttt{#1}\sindex[pkg]{#1}} % TBD: this must be extracted 
  }

% high quality tables 
\usepackage{booktabs}
\aboverulesep=0ex
\belowrulesep=0ex

\usepackage{xurl}

%\makeglossary% for rerunfilecheck 

%\usepackage{etexcmds} %still later 
\ifthenelse{\boolean{beamerLoaded}}{
  % TBD: clarify this case. 
  % maybe beamer does not support indices or glossaries. 
  % 
}{
  \usepackage[nottoc, numindex, numbib]{tocbibind}
}

%\usepackage{latex-bnf}




../../tex/headerSuppressMetaPDF.tex

\hypersetup{
  pdfinfo={
    Author      ={Ernst Reissner},
    Title       ={The file fdb\_latexmk written by latexmk},
    Subject     ={Inner workings of latexmk},
    Keywords    ={LaTeX;latexmk;fdb\_latexmk}
  }
}

\newcommand{\latexmk}{\texttt{latexmk}}

\title{The file \texttt{fdb\_latexmk} written by \latexmk}

\author{Ernst Reißner}
\date{\today}

\begin{document}

\maketitle
\tableofcontents

\section{Introduction}\label{sec:intro}

A build tool for \LaTeX{} transforms a source file \texttt{xxx.tex} 
into a file \texttt{xxx.pdf} or into other formats. 
To that end, it invokes a bunch of more basic tools, 
most notably a \LaTeX{} compiler, and possibly more than once. 
Among the most widely used build tools is \latexmk. 

To find out whether a compilation process must be repeated, 
\latexmk{} keeps track on whether a relevant file changed. 
As indicated in~\cite{LatexMk23}, Section ``Description'', 
to that end, \latexmk{} uses a file \texttt{xxx.fdb\_latexmk}: 
it writes information on dependencies 
before compilation step and records changes after. 
This is all information~\cite{LatexMk23} contains 
concerning \texttt{fdb\_latexmk} files. 

% \cite{LatexMk23}, $analyze_input_log_always, when to analyze log files 
% for contstructions <...> and (...)
% interesting: lualatex fails to add lua files to fls files. 
% \cite{LatexMk23}, $recorder [1]
% there are cases where the file is named lualatex.fls or that like. 
The manual~\cite{LatexMk23} gives only very few pieces information 
on \texttt{xxx.fdb\_latexmk}, because it is intended for internal use only; 
some results of reverse engineering on \texttt{xxx.fls} file format 
is collected in~\cite{LatexRec}. 
Also, there is some 
\href%
{https://tex.stackexchange.com/questions/705359/who-knows-about-the-format-of-fdb-latexmk}%
{statement} of the maintainer, J. Collins, 
but of course, it may be subject to change. 
It is the intention of the maintainer that if the form changes, 
so does the version number described below, so change is not without notification. 
Since \texttt{latexmk} is quite mature, 
the author of this description considers the risk as quite low. 


The understanding of the author of this document is, 
that \texttt{latexmk} uses the file \texttt{xxx.fls}, 
which is generated by \LaTeX{} compilers 
to detect dependencies, i.e.\@ files the resulting file depends on. 
The file \texttt{xxx.fdb\_latexmk} contains pieces of information on files 
to detect changes and rules on these files, 
to decide which rules must be reapplied to recreate outdated files. 

This document describes the form of these files. 
In the long run, it is desirable 
to base this documentation on an analysis of the source code of \latexmk, 
but currently, this is just collection of observations and personal communication. 


The \texttt{fdb\_latexmk} files start with a line 
%
\begin{verbatim}
  # Fdb version 4
\end{verbatim}
%
indicating the file version. 
Maybe this is the major version of \latexmk{} which produced it. 
Let us call this line the \emph{version line}. 

After that follow a sequence of productions. 
Productions are described in Section~\ref{sec:prod}. 

Note that lines starting with \texttt{\#} or \texttt{\$}, 
possibly preceded by blanks, are comment lines and as such ignored. 
So are lines consisting of blanks only, in particular empty lines. 
Nevertheless, the version is read by \texttt{latexmk} 
and if it is not as expected, the file is not read. 
This means it is treated as if not existent. 


\section{Form of a production in \texttt{fdb\_latexmk} files }\label{sec:prod}

A production describes a step of transformation of source files into target files. 

The following listing is an example of a production. 

\begin{lstlisting}[basicstyle=\scriptsize]
["bibtex manualLMP"] 1702855070 "manualLMP.aux" "manualLMP.bbl" "manualLMP" 1702855080 0
  "./lit.bib" 1702001904 30636 95f3702a4882e2bee12e1804952c0464 ""
  "/usr/share/texmf/bibtex/bst/base/alpha.bst" 1292289607 23907 a5f93555796fb564b924339521f10a7c ""
  "manualC1intro.aux" 1702855072 1784 391ed303762a58203532b3a94c1ab1ad "lualatex"
  "manualC2inst.aux" 1702855073 5712 ea0d574b315618b789351136ad49ea23 "lualatex"
  "manualC3usage.aux" 1702855074 11399 d48d9f1b29494e437814a748ac61c875 "lualatex"
  "manualC4graphics.aux" 1702855076 8936 7cd15bd8a63034e99ee74582f42ad927 "lualatex"
  "manualC5procMain.aux" 1702855077 9519 d8d947944e2816ec0f640c46e6d16bae "lualatex"
  "manualC6paramsSettings.aux" 1702855079 14104 e4dda6be3f2449b6d24bf1671c53e42c "lualatex"
  "manualC7ExcLogging.aux" 1702855079 5557 9265ce2a92076ffb1fb0c5a234bad724 "lualatex"
  "manualLMP.aux" 1702855079 4747 64f55a1e1685cf1b96b1184a428c034a "lualatex"
  (generated)
  "manualLMP.bbl"
  "manualLMP.blg"
  (rewritten before read)
\end{lstlisting}

Each production starts with a \emph{rule line} which represents a production rule. 
The rule line is the fist line of a production, and it is the sole which is not indented.  
So each production is started with a non-indented line, 
and each non-indented line except the version line described in Section~\ref{sec:intro} starts a production. 

After the rule line follow a sequence of \emph{source line}s 
which reach until the literal line ``\texttt{(generated)}''. 
This introduces the \emph{target lines} 
which reach until the literal line ``\texttt{(rewritten before read)}''. 
This follows a list of files, same form as the target files, 

So all we have to do now is to describe the rule line, 
the source lines and the target lines, all without indent. 

The rule line represents a production rule, 
each source line represents a source file, 
i.e.\@ a file needed to execute the production rule, 
and each target line represents a file created by the rule. 

The rule line starts with a rule in double quotes in brackets. 
The rest of the rule line is as follows: 
%
\begin{itemize}
  \item 
  System time at the start of the last time the rule was run.
  \item 
  Primary source file, e.g., the document's TEX file for pdflatex.
  \item 
  Primary destination file (if any), e.g., the output PDF file for pdflatex.
  \item 
  The base name for files, e.g., log files and the like.
  \item 
  The system time when the rule was last checked, 
  which is used to decide whether a rerun was needed.
  \item
  A status code for the last run: 0 for success. Currently, other possible values are: 1 if the rule failed to create the intended output file, 2 if there was some other kind of error.

\end{itemize}

The rule may be user/customer defined, but it may also be internal. 
User defined rules start with \texttt{cusdep}, 
internal rules start with the name of the command executed, 
sometimes followed by the filename used in the invocation. 

A target line is just the name of the file in double quotes. 

In contrast to this, a source line contains the following pieces of information 
separated by a single blank: 
%
\begin{itemize}
  \item the name of the file in double quotes, as for target lines, 
  \item its modification time, 
  \item its number of bytes, 
  \item its md5 signature, 
  \item the rule name by which this source file is generated in double quotes, 
  or, if it is not generated, the empty string in double quotes. 
\end{itemize}

By means of the rule name, it refers to another production section. 
The rule name is just what is given in the rule line of some (other) production. 


When non-ASCII filenames are encountered, they are coded as UTF-8.


\bibliographystyle{alpha}
\bibliography{../lit,../litTools,../litLMPpub}{}% chktex 11 

\end{document}