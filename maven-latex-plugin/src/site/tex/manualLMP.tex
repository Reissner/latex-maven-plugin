\RequirePackage[l2tabu, orthodox]{nag}
% blabla 
\PassOptionsToPackage{autostyle}{csquotes}
%\PassOptionsToPackage{dvipdfmx}{xcolor} %or dvipdfm 
%\PassOptionsToPackage{dvipdfmx}{graphicx} %or dvipdfm 
%\RequirePackage{lmodern} % does not work with htlatex
%\directlua{pdf.setminorversion(7)}
%\DocumentMetadata{pdfversion=1.7}%, lang=en-US}
\documentclass[a4paper,12pt,german,english]{book}
\newif\iflatextortf% false for latex but set true for latex2rtf internally 

%\includeonly{manualC3usage}

% for buildParams to check empty: \ifdefempty
\usepackage{etoolbox}
% for buildParams: \verbdef 
\usepackage{newverbs}

\usepackage[depythontex]{pythontex}% TBD: check whether appropriate 

% before \input{header.tex} because header.tex contains a patch for listings 
\usepackage{listings}
\usepackage{iftex}% This is a hack: is loaded later also in header.tex
% it is needed for htlatex because there is no htlualatex. 
% Even if there were, we had to change latex compiler separately for pdf and html. 
% The solution is to use make4ht and to pass the latex compiler as an argument.  
% 
\iftutex%
\usepackage{luamplib}% works with lualatex only maybe with magic comment 
\else
\fi
\input{header.tex}
\IfPackageLoadedTF{tex4ht}{
}{
  % TBD: clarify: 
  % whereas \makeindex does not cause a warning, /makeglossaries does. 
  % One has to decide whether each document must have a bibliography, an index and a glossary. 
  % I think neither. 
  % so, shall be moved out of the general header. 
  % \usepackage{splitidx}%[split]
  % %  \usepackage{makeidx}
  % %  \usepackage{showidx}
  % \makeindex
  % % TBD: eliminate; loads amsmath!!! 
  \usepackage[toc, acronym, nohypertypes={acronym},style=tree]{glossaries}%nomain, 
  % , xindy or even [xindy={language=english,codepage=utf8}]
  % mainly for index and glossaries 
  \makeglossaries%
}
\setMinorVersionPdf{7}% because of gnuplot file only 
% how to number the glossary in the toc? seemingly a gap in tocbibind 
%\usepackage[nottoc, numindex, numbib]{tocbibind}
\input{headerGrp.tex}
\newcommand{\bs}{\symbol{'134}} % backslash in typewriter T1/T1 

\newcommand{\env}[1]{\texttt{#1}}

% For tex4ht packages splitindex and glossaries/glossaries-extra are not loaded, 
% makeindex is not invoked. 
\IfPackageLoadedTF{tex4ht}{
  \newcommand{\cmd}[1]{\texttt{\bs{}#1}}
  \newcommand{\pkg}[1]{\texttt{#1}}% without indexing 
}{
  \newcommand{\cmd}[1]{\texttt{\bs{}#1}\sindex[cmd]{#1@\texttt{\bs{}#1}}}
  \newcommand{\pkg}[1]{\texttt{#1}\sindex[pkg]{#1}} % TBD: this must be extracted 
}

% currently used for latexmk only 
% but in the long shall be used for all tools 
% this is the base for creating an index of tools. 
\newcommand{\tool}[1]{\texttt{#1}}

% the following 3 shall be replaced by \tool{...}
\newcommand{\pdflatex}{\texttt{pdflatex}}
\newcommand{\lualatex}{\texttt{lualatex}}
\newcommand{\xelatex}{\texttt{xelatex}}


\newcommand{\texlive}{\TeX~Live}
\newcommand{\miktex}{MiKTeX}
\newcommand{\ltex}{L\TeX}% maybe also with \tool
\newcommand{\auctex}{AUC\TeX}

% TBD: clarify: \tool may be used in definition of acroyms, whereas \pkg may not. 
% Could be because \sindex does strange things 
\input{buildParams.tex}

% presupposes hyperref and iftex (e.g. preload header.tex)

% xelatex enters always else branch of \ifpdf 
% This is maybe because it always creates xdv 
% and the difference is only that for pdf another step is executed. 
% Whether this is considered a bug.... ? hm. 
\ifXeTeX%
  % for xelatex
  % TBD: find a way to express pdfinfoomitdate: necessary? 
  % provide valid trailerid 
  % \special{pdf:trailerid [
  %   <00112233445566778899aabbccddeeff>
  %   <00112233445566778899aabbccddeeff>
  % ]}% correct but not necessary 
\else
  \ifpdf%
    \ifLuaTeX%
      % for lualatex 
      % no adding to the trailer dictionary.
      % if invalid trailer id, e.g. empty, this is silently not set 
      % \pdfvariable trailerid{[
      %   <00112233445566778899aabbccddeeff>
      %   <00112233445566778899aabbccddeeff>
      % ]}% luatex 1.18, manual 14.1.8% chktex 1

      % omit only CreationDate and ModDate keys.
      % \pdfvariable suppressoptionalinfo 767% chktex 1
      % 767=512+0+128+64+...+1 this excludes all but 
      % We decided to omit all information 
      %\pdfvariable suppressoptionalinfo  -1%\numexpr% luatex 1.18, manual 14.1.8 % chktex 1
      %\pdfvariable suppressoptionalinfo 927%1023-32-64=1000-9-64=1000-73=927% luatex 1.18, manual 14.1.8 % chktex 1
      %\pdfvariable suppressoptionalinfo -1% luatex 1.18, manual 14.1.8 % chktex 1
      % The meaning of the bits are described in luatex reference manual 
      % \cite{LuaTexRef24}, Section 3.2.2. 
      \pdfvariable suppressoptionalinfo \numexpr % chktex 1
      0
      + 1 % PTEX.FullBanner % for sake of reproducibility and privacy 
      + 2 % PTEX.FileName   % this and the following PTEX entries refer to embedded PDF files 
      + 4 % PTEX.PageNumber
      + 8 % PTEX.InfoDict (/Producer /Creator /CreationDate /ModDate /Trapped)
      + 16 % Creator %(cannot be suppressed if hyperref is used; must be overwritten below.) 
      % + 32 % CreationDate % do not suppress: needed for reproducibility check 
      % + 64 % ModDate      % do not suppress: needed for reproducibility check 
      + 128 % Producer      % suppress for privacy/security 
      + 256 % Trapped       % xelatex no info and pdflatex does have info
      % + 512 % ID          % don't suppress trailerid: makes an individual of the PDF and needed to encrypt
      % according to \cite{Pdf20}, Section 7.5.5, the trailer ID is mandatory from PDF2.0 on. 
      % Also, consequences of absence are listed, e.g. encryption. 
      % Missing is a specification on what the trailer ID may depend. 
      % Seemingly, the trailer ID is unchanged if so is the file and the processing time. 
      % It seems not to depend on further inner states of the computer generating the PDF. 
      % This is not formulated as a requirement. 
      % If this may change, then a trailerid must be set explicitly again, but '+ 512' must remain inactive. 
      % with luatex 1.18 there are macros closer to pdftex. 
      \relax
    \else
      \ifPDFTeX%
        % the following see pdftex user manual, \cite{PdfTexUsr}, Section 7.4
        % \pdftrailerid{[
        %   <00112233445566778899aabbccddeeff>
        %   <00112233445566778899aabbccddeeff>
        % ]}%

        % omit CreationDate and ModDate keys.
        % not before pdfTeX 3.14159265-2.6-1.40.17
        % the following two see pdftex user manual, \cite{PdfTexUsr24}, Section 4.2.4
        \pdfinfoomitdate=0 % turn 0 explicitly because /CreationDate and /ModDate are needed for reproducibility check 
        % we do even more for security reasons 
        \pdfsuppressptexinfo=-1
        % eliminates PTEX info: 
        % 1 -> PTEX.Fullbanner
        % 2 -> PTEX.FileName
        % 4 -> PTEX.PageNumber
        % 8 -> PTEX.InfoDict (/Producer /Creator /CreationDate /ModDate /Trapped)

        % Replacing pdfinfoomitdate1 in conjunction with
        % \usepackage[nodocdata=true,nopdftrailerid=true]{pdfprivacy}

        % If doing without hyperref package (pdftex only): 
        % \pdfinfo{
        %   /Author       (Ernst Reissner)
        %   /Title        (The dvi-format and the program dvitype)
        %   /CreationDate (unknown)
        %   /ModDate      (unknown)
        %   /Subject      (dvi and dvitype)
        %   /Keywords     (LaTeX;dvi;dvitype)
        %}
      \else
        % Here, the tex processor is unknown. 
        \GenericError{...}{Tried compilation with unknown TeX compiler. }{}{} % chktex 11
      \fi%pdftex
    \fi%luatex
  \else%ifpdf
  \fi%ifpdf
\fi%xetex

% TBD: check of resulting pdf with exif unveils a warning duplicate author 

% TBD: there are ways to do without hypersetup. 

\hypersetup{
  % PTEX.InfoDict: (/Producer /Creator /CreationDate /ModDate /Trapped)
  % Except Trapped, this is exactly PTEX.InfoDict. 
  % TBD: deactivate this. 
  % There are settings like the banner or the trailer id, 
  % accessible only compile specific, 
  % so we cannot do without. 
  % Conversely, all accessible from hyperref and needed (consider xelatex) 
  % are accessible compiler specific. 
  % So this is not really needed. 
  pdfinfo={
    Creator     ={unknown}, 
    % 'LaTeX with hyperref'; this is uniform for hyperref 
    % and cannot be eliminated even for lualatex with suppressoptionalinfo 16. 
    % else TeX for lua/pdflatex, but  XeTeX output 2023.12.06:0106 for xelatex 
    % thus it is advisable to eliminate this 
    % maybe security issue, but it is well visible. 
    % also corrupts reproducibility. 
    % only way to do so: overwrite. 
    Producer    ={unknown}, %TBD: clarify whether this is really needed to be overwritten 
    % xdvipdfmx with version and that like for dvi or xelatex; 
    % pdfTeX-1.40.25, LuaTeX-1.17.0 for pdf
    % This shall be eliminated for sake of privacy and reproducibility. 
    %CreationDate={unknown},
    %ModDate     ={unknown}
    % these two are only overwritten to avoid wrong entry 
    % when using \texttt{SOURCE\_DATE\_EPOCH=0 FORCE\_SOURCE\_DATE=1}
    % This is recommended for tests, because else we cannot give a \date 
    % This is no longer recommended: leaving these alive is a value. 
    % In contrary, these entries are needed to reach reproducibility
    }%,
    %pdfsource   ={no latex main file}
}




\hypersetup{
  pdfinfo={
    Author      ={Ernst Reissner},
    Title       ={Maven plugin and ant-task to process latex},
    Subject     ={latex plugin for maven and latex task for ant},
    Keywords    ={svg;png;metapost;PDF;LaTeX;maven;ant}
  }
}


\IfPackageLoadedTF{tex4ht}{%
  % no glossaries and no index with tex4ht 
  % Syntax: \gls[⟨options ⟩]{⟨label ⟩}[⟨insert text ⟩] 
  % cannot be covered with this... 
  % needed e.g. \NewDocumentCommand{\gls}{O{}mO{}}{#2}
  \newcommand{\gls}[1]{#1}
  \newcommand{\Gls}[1]{#1}
  \newcommand{\GLS}[1]{#1}
  \newcommand{\glspl}[1]{#1}
  \newcommand{\Glspl}[1]{#1}
  \newcommand{\GLSpl}[1]{#1}
  % not complete, what about \glsfirst and family
  % interesting is \glsdesc
  % see glossaries manual around page 140 
  \renewcommand{\index}[1]{}
  % why no \sindex? because based on \index? 
  \newcommand{\newindex}[1][]{}
}{%
\setacronymstyle{long-short}%{footnote}%

\newglossaryentry{gls:le}{name={\LaTeX{} engine},description={%
A compiler for \LaTeX{} files 
}}

\newglossaryentry{gls:lmf}{name={\LaTeX{} main file},description={%
A \LaTeX{} file intended to be compiled by a \gls{gls:le}. 
}}


\setacronymstyle{short-long}%{footnote}%{long-short}%
% file formats 
\newacronym{pdf}{PDF}{Portable Document Format}
\newacronym{dvi}{DVI}{DeVice Independent; 
traditional output format of \LaTeX{} engines, today widely replaced by PDF}
\newacronym{xdv}{XDV}{eXtended Device Independent; 
an extension of the traditional output format \gls{dvi} of \LaTeX{} engines, 
today widely replaced by PDF}
\newacronym{eps}{EPS}{Encapsulated PostScript}
\newacronym{tex}{TEX}{\TeX{} the format, which may also be \LaTeX}
\newacronym{html}{HTML}{HyperText Markup Language}
\newacronym{xhtml}{XHTML}{eXtensible HyperText Markup Language}
\newacronym{odt}{ODT}{Open Document Text}
\newacronym{doc}{DOC}{outdated document format for MS Word }
\newacronym{docx}{DOCX}{current document format for MS Word }
\newacronym{sgml}{SGML}{Standard Generalized Markup Language }
\newacronym{xml}{XML}{eXtensible Markup Language }
\newacronym{ptx}{PTX}{pdf/postscript \TeX{} format; home-brewed }% home-brewed 

\newacronym{fig}{FIG}{native file format for xfig }
\newacronym{gp}{GP}{GnuPlot file format}
\newacronym{png}{PNG}{Portable Network Graphics}
\newacronym{jpg}{JPG}{Graphics format developed by the Joint Photographic Experts Group }
\newacronym{gif}{GIF}{Graphics Interchange Format, allows also animations }
\newacronym{bb}{BB}{Bounding Box, created by ebb }
\newacronym{xbb}{XBB}{Bounding Box, created by ebb with the option x. 
In addition to BB-files contains a high resolution bounding box }
\newacronym{svg}{SVG}{Scalable Vector Graphics}
% latex file endings 
\newacronym{toc}{TOC}{Table Of Contents: input and output format of \LaTeX{} engines}
\newacronym{lof}{LOF}{List Of Figures: input and output format of \LaTeX{} engines}
\newacronym{lot}{LOT}{List Of Tables: input and input and output format of \LaTeX{} engines}
\newacronym{lol}{LOL}{List Of Listings: input and output format of \LaTeX{} engines 
if used with package \texttt{listings}}
\newacronym{bcf}{BCF}{bibliography content file (?): generated by \LaTeX{} engines 
if used with package \texttt{biblatex}}
\newacronym{out}{OUT}{contains bookmarks: input and output format of \LaTeX{} engines 
if used with package \texttt{hyperref}, file ending seems naive}
\newacronym{ist}{IST}{(make-)Index Style File: output format of \LaTeX{} engines % chktex 36
if used with package \texttt{glossaries} configured for \tool{makeindex} }
\newacronym{xdy}{XDY}{index style file for \tool{xindy}: output format of \LaTeX{} engines 
if used with package \texttt{glossaries} configured for \tool{xindy} }


\newacronym{log}{LOG}{logging file: for \LaTeX{} engines and \tool{mpost}}
\newacronym{aux}{AUX}{auxiliary file: input and output file for \LaTeX{} engines; 
read also e.g\@. by \tool{bibtex}}
\newacronym{fls}{FLS}{FiLeS dependencies: list of files the according tex file depends on; 
output format of \LaTeX{} engines if used with option \texttt{-recorder}}
\newacronym{synctex.gz}{synctex.gz}{gnu-zipped synchronization files 
relating TEX files and their according PDF files: 
output format of \LaTeX{} engines if used with option \texttt{-synchtex=1}}

% files in conjunction with bibliographies
\newacronym{bib}{BIB}{BIBliography file: 
In particular input file for the \tool{bibtex} tool}
\newacronym{bbl}{BBL}{bibliography for a latex document in latex format: 
written by the \tool{bibtex} tool and read by \LaTeX{} processors}
\newacronym{bst}{BST}{Bibliography Style File read by the \tool{bibtex} tool}

% files in conjunction with indices 
\newacronym{idx}{IDX}{InDeX file containing unsorted and multiple index entries; 
output format of \LaTeX{} engines with package \tool{makeindex} or similar}
\newacronym{ind}{IND}{%
INDex file containing sorted, unified and formatted index entries, 
output format of \tool{makeindex} and \tool{xindy}}

% files in conjunction with glossaries 
\newacronym{glo}{GLO}{%
GLOssary file containing unsorted and multiple glossary entries; 
output format of \LaTeX{} engines with package \tool{makeglossaries}}
\newacronym{glg}{GLG}{%
\tool{makeglossaries} log file }
\newacronym{gls}{GLS}{%
glossary file containing sorted, unified and formatted glossary entries;
output format of the \tool{makeglossaries} tool read by \LaTeX{} engines}
% TBD: one could add is and xdy files. 

% files in conjunction with pythontex
\newacronym{pytxcode}{PYTXCODE}{%
Code file consisting mainly of code snippets from the TEX file; 
output format of \LaTeX{} engines with package \tool{pythontex}}
\newacronym{depytx}{DEPYTXC}{%
File containing information to replace code snippets in the TEX file 
by the result of their evaluation;  
output format of \LaTeX{} engines with package \tool{pythontex} 
if loaded with option \tool{depythontex}
}
\newacronym{plg}{PLG}{%
\tool{pythontex} log file: home-brewed since the original application does not write log files 
}
\newacronym{dplg}{DPLG}{%
\tool{depythontex} log file: home-brewed since the original application does not write log files 
}

\newacronym{mp}{MP}{MetaPost: input format for the graphic program \tool{mpost}}
\newacronym{ps}{PS}{PostScript: 
programming language for printers and printable file format; 
today mostly replaced by PDF}
\newacronym{mps}{MPS}{metapost's postscript like output including text}
\newacronym{mpx}{MPX}{metapost TEX output: texts }


\newindex[General Index]{idx}
\newindex[LaTeX Packages]{pkg}
\newindex[LaTeX Commands]{cmd}
}

% \newglossaryentry{latex}{name={latex},description={%
% First a file format based on \TeX with ending \gls{tex}, 
% and secondly a document preparation system 
% converting sources in the latex-format into a presentation format, 
% preferrably into pdf-output. 
% Historically, output was in \gls{dvi}-format 
% which is still used by htlatex as an intermediate format 
% to produce \gls{html} resp.~\gls{xhtml} and \gls{odt} output. 
% %
% Besides latex compilers, doing the proper conversion, 
% this comprises many auxiliary programs, 
% e.g. for bibliography but also to create graphics. 
% }}
% \newglossaryentry{latexCMP}{name={latex compiler},description={%
% Core part of the document preparation system \gls{latex}. 
% }}







% \newindex[General Index]{idx}
% \newindex[LaTeX Packages]{pkg}

% does not work 
%\setindexpreamble[pkg]{This index comprises all the latex packages used. }

\title{Manual for the \artifactId{} \protect\\
  and for an according ant-task \protect\\
Version \strippedVersionID}
\author{Ernst Reissner (rei3ner@arcor.de)}
\date{\today}%{\versionDate}


% seems to have a problem with pythontexW maybe clean before
%\includeonly{manualC4graphics}

\begin{document}
\maketitle

\tableofcontents
\listoffigures
\listoftables
\lstlistoflistings%


\include{manualC01intro}
\include{manualC02inst}
\include{manualC03usage}
\chapter{Graphics and Preprocessing}\label{chap:GraphConversions}

While \LaTeX{} is really strong in text processing 
and also in formula processing, 
in itself it is weak in its graphical abilities. 
Graphics in some formats can be included directly in a \LaTeX{} document, 
but all need loading of according packages. 
For an overview of the graphic formats and the packages needed for their support 
see Section~\ref{sec:graphPkg}. 
The set of available graphic formats is extended by \emph{preprocessing}, 
i.e.\@ by processing prior to the \LaTeX{} engine. 
Preprocessing mainly consists in converting 
graphic formats not supported by \LaTeX{} packages 
into graphic formats supported by some \LaTeX{} packages. 
Section~\ref{sec:preprocTarget} provide vital information 
on the target formats. 

This software uses preprocessing for graphics only. 
Note that preprocessing is a design decision on the build tool 
and e.g. \texttt{latexmk} has no preprocessing at all. 
For details see Section~\ref{subsubsec:latexmkDiff}. 


Table~\ref{tab:graphicPreprocOverview} 
gives an overview over the formats supported via preprocessing. 
The first column lists the formats, the second one at least one editor for the format, 
and the last row contains the parameter to configure the preprocessing tool 
and give the default tool as an example. 
Sections~\ref{sec:fig2dev}, \ref{sec:gnuplot2dev}, \ref{sec:metapost}, 
\ref{sec:picSvg} and \ref{sec:picAsIs} treat each format separately. 
For all but \gls{png} and \gls{jpg} considered in Section~\ref{sec:picAsIs}, 
preprocessing is just conversion of the format 
into another format directly supported as described in Section~\ref{sec:graphPkg}. 
Historically the latter two required preprocessing 
to determine the bounding box was needed. 
We still support this to support historical techniques and to be sure 
to be able to reconstruct historical documents. 
Support for further formats can be easily added. 
If there is some need, please write an email to the author. 

Of course, to support a format, the preprocessing tools must be installed. 
It is advisable to have also an editor installed. 
Sometimes the editor is used also as converter as for \tool{inkscape}. 
For human-readable formats like \texttt{fig}, it often makes sense, 
to use both the graphical editor and the textual one. 
Note that \tool{vscode} supports the given formats more properly, 
if the extensions described in Section~\ref{subsec:instExtVsCode} are installed also. 

\begin{longtable}{|l|ll|}
\toprule
Graphic format & editor & preprocessing tool\\
\midrule
\midrule
\endfirsthead%
\bottomrule
\caption{\label{tab:graphicPreprocOverview} 
Overview over the graphic formats supported via preprocessing}
\endlastfoot%
fig             & xfig, vscode     & \texttt{fig2devCommand}, e.g.~\texttt{fig2dev}    \\
gnuplot (gp)    & vscode           & \texttt{gnuplotCommand}, e.g.~\texttt{gnuplot}          \\
MetaPost (mp)   & vscode           & \texttt{metapostCommand}, e.g.~\texttt{mpost}          \\
svg             & inkscape, vscode & \texttt{svg2devCommand}, e.g.~\texttt{inkscape}  \\
jpg, png        & gimp             & \texttt{ebbCommand}, e.g.~\texttt{ebb}                   \\
\end{longtable}


\section{Graphic formats and packages supporting them}\label{sec:graphPkg}

Find below a list of packages either allowing to include directly certain graphic formats, 
or helping with graphics indirectly. 
Although strictly speaking these techniques do not need special treatment 
of a build tool, 
this software supports these techniques by providing header files 
by injection loading the needed packages. 

We also describe in which sense these packages support graphical preprocessing. 

%
\begin{itemize}
\item[\pkg{graphicx}]
is the basic graphics package which provides the command 
\cmd{includegraphics}{} 
which allows including graphics natively 
in the formats PDF, EPS, JPG and PNG at least. 
For details see~\cite{GraX}. 
Note that PDF and EPS are target formats for graphical preprocessing, 
where PDF is embedded into PDF and EPS is embedded into DVI/XDV. 
As described in Section~\ref{sec:metapost}, 
also \gls{mps}, the target format for metapost is included using \pkg{graphicx}. 
\item[\pkg{transparent}]
allows specifying transparency in graphics. 
Even if you do not use the feature, 
some source formats do (in fact only \gls{svg}) does 
and so the according converters create according information 
and so the \LaTeX{} engine must get along with it. 
Note that this applies only for output format PDF 
and in particular not for \texttt{xelatex}. 
For details see~\cite{TransP}. 
\item[\pkg{bmpsize}]
is needed for bitmap formats like \gls{jpg} and \gls{png} only. 
Used to extract resolution and bounding box. 
FIXME\@: needed more information. 
For details see~\cite{BmpP}. 
% maybe for dvi format, where jpg and png may be included also, 
% preprocessing is needed to generate the bounding boxes. 
\item[\pkg{tikz}]
The TikZ code described in~\cite{TikzPGF23} is just in \LaTeX{} format. 
Thus, it can be included directly and does not require any preprocessing. 
Still what is needed is a good graphical editor like \texttt{tikzedt} 
with online manual~\cite{TikzEdt}. 
In later versions of this software, 3.x or so, 
it is planned that TikZ is used as new target format for graphical preprocessing, 
replacing the current combination of \LaTeX{} for texts and PDF/EPS for proper graphic. 
\item[\pkg{import}]
is strictly speaking no graphics package. 
According to its documentation~\cite{ImpoP}, 
it allows an imported file to find its own inputs 
(using ``\cmd{input}'', ``\cmd{includegraphics}'' etc.) in that directory. 
This is vital for the graphic formats for which a TEX file is imported 
which itself imports a PDF/EPS file located in the same folder 
but not in the folder of the importing file. 
It is advisable to combine the \texttt{import} package with other graphic packages 
to include graphics in separate graphic files. 
\item[\pkg{xcolor}]
allows using colors in graphics. 
Even if the author does not use colors in graphics, 
several formats, like \gls{fig}, \gls{gp} and \gls{svg} 
offer it and so the according converters 
transforming them into the native formats 
create color information which can be rendered only via \pkg{xcolor}. 
In this sense its role is comparable to that of \pkg{transparent}. 
On the other hand, the use of \pkg{xcolor} is not specific to graphics. 
For details see~\cite{XColorP}. 
\item[\pkg{pythontex}]
is strictly speaking no graphics package either but more general 
a way to include and run code within a \LaTeX{} document 
as described in~\cite{PythonTexP}. 
Note that not only Python but also other languages can be used. 
Most of them offer graphic capabilities 
and so graphics can be included also via \pkg{pythontex}. 
Nevertheless, we do not treat this technique in this chapter, 
but separately in Section~\ref{sec:pythontex}. 
This is because graphics is a side aspect of \pkg{pythontex} 
and also because strictly speaking there is no preprocessing. 
First a latex processor is run, and the package extracts the code 
into a separate file which is then further processed by an external tool. 
This is more like running \cmd{bibtex} to extract a bibliography. 

If using the package \texttt{pythontex} 
a special processing interacting with the \LaTeX{} engine is required also, 
but it is not preprocessing. 
\end{itemize}

Section~\ref{sec:injFiles} is on injection of files 
and in particular header files: 
%
\begin{description}
  \item[\texttt{header.tex}] treated in Section~\ref{subsec:header}, 
  is a general header file intended to be included into all \LaTeX{} files. 
  Since the packages \pkg{import} and \pkg{xcolor} are generally useful, 
  not only in the context of graphics, they are among those loaded in \texttt{header.tex}. 
  \item[\texttt{headerGrp.tex}] described in Section~\ref{subsec:headerGrp} in contrast, 
  is a header file loading graphic specific packages related with \pkg{graphicx}, 
  loading also \pkg{transparent} and \pkg{bmpsize}. 
\end{description}

The header files adapt the loading of the packages to the context, 
in particular to the target format. 
Note that \texttt{headerGrp.tex} must follow \texttt{header.tex}. 

The package \pkg{tikz}, although a pure graphic package 
is very specific and not related to \pkg{graphicx}. 
Thus, it must be loaded separately. 
The same holds for \pkg{pythontex}. 









Besides the converter external to \LaTeX, 
also several \LaTeX-packages are required 
to use graphics. 


This section describes the conversions of 
graphical source files into target files 
in detail. 

But PDF also occurs as an intermediate format for pictures. 
For historical reasons, still \gls{eps} is used. 
%Section~\ref{sec:figpdf} shows that it suffices to stick to pictures 
%in PDF format. 
% FIXME\@: REALLY? 
%FIXME\@: missing pdf als basic input format. What about (e)ps? 
Section~\ref{sec:fig2dev} shows how \texttt{fig2dev} converts fig-files 
into \LaTeX-files containing text and including graphics in as PDF files. 
Likewise, Section~\ref{sec:gnuplot2dev} describes 
how gnuplot converts gnuplot-files into PDF files. 
An interesting alternative to gnuplot for computing pictures 
is MetaPost described in Section~\ref{sec:metapost}. 
A more elaborate alternative to fig-pictures are SVG pictures 
described in Section~\ref{sec:picSvg}
Also several formats collected in Section~\ref{sec:picAsIs} 
may be included as is. 



\section{Target formats for preprocessing}\label{sec:preprocTarget}

At a first sight, PDF seems the ideal target format for any kind of preprocessing: 
It is really mighty enough to display pictures in any source format 
without loss of information and even without change in appearance, 
and for modern \LaTeX{} implementations directly creating PDF files, 
the \LaTeX-package \pkg{graphicx} allows including graphics as PDF files in \LaTeX-files. 

At a second sight, the source formats under consideration 
offer pictures mixing vector graphics and texts and in particular formulae 
set in \LaTeX{} style. 
Preprocessing is based on on-the-shelf converters 
and if targeting PDF, the texts originally in \LaTeX{} style change their appearance. 
To keep up \LaTeX{} style, they provide mixed export 
consisting of a PDF file containing proper graphics 
without texts and a TEX file containing the texts in proper location 
and an \cmd{includegraphics} command including the created PDF file. 
This mixed conversion is used for all kind of preprocessing. 

Note that we could have used the ending TEX for the texts, 
but we opted for a specific ending \gls{ptx} signifying that the file is created 
and thus does not slow down search of \LaTeX{} main files. 

But still there is another problem with PDF as target format: 
Traditionally \LaTeX{} produced output in the \gls{dvi}/\gls{xdv}-format 
which is still used to create \gls{html}-output. 
For \LaTeX{} engines \pdflatex{} and \lualatex, 
DVI output is specified with option \texttt{--output-format=dvi}. 
It turns out, that with this setting, 
PDF files cannot be incorporated with \cmd{includegraphics} command. 
Instead, one must use \gls{eps} files. 
Fortunately, the graphic converters used also support combined TEX/\gls{ptx} and EPS formats. 
We ensured that \cmd{includegraphics} in the \gls{ptx} file 
specifies the file without ending so that the \gls{ptx} file is the same, 
whether it encloses a PDF file or an EPS file,
and we provide both, a PDF file and an EPS file\footnote%
{Of course, here a more sophisticated technique is conceivable, 
recognizing the required format and generating the specific one if missing. }. 
That way, both, \pdflatex{} and \lualatex{} choose the EPS file or the PDF file 
depending on whether the output format is \texttt{--output-format=dvi} 
or \texttt{--output-format=pdf} which is the default. 
Note that \xelatex, which always creates an intermediate XDV file 
(which is a special kind of DVI file), 
acts differently: If present, it prefers including the PDF file, 
if absent, but there is an EPS file instead, it uses this without making any difference. 

Although this is beyond necessity, 
let me state that \pdflatex{} and \lualatex, 
while not accepting inclusion of PDF files in DVI mode, 
EPS files are accepted in PDF mode for more modern versions of the \LaTeX{} engines, 
but this leads to creation of intermediate files \texttt{xxx-eps-converted-to.pdf}, 
which are not cleaned up in target \texttt{clr}. 

Whereas PDF and EPS files both are offered, 
only one of them is included for a specific configuration. 
This is in contrast to other formats described in Section~\ref{sec:picAsIs}. 

Although \gls{ptx} is just a TEX format, 
it is special in that it presupposes that some packages are loaded before being included. 
The packages which are not specific for graphics like \pkg{xcolor} are loaded in \texttt{header.tex} 
described in Section~\ref{subsec:header}, 
whereas the ones specific for graphics, above all \pkg{graphicx}, 
are loaded in \texttt{headerGrp.tex} 
as described in Section~\ref{subsec:headerGrp}. 
The packages actually to be loaded and their respective options 
depend on the configuration. 

Note that PDF and EPS file may be created by preprocessing 
but also as proper sources not created at all, 
even in a single document. 
Goal \texttt{clr} deletes the according files \texttt{xxx.pdf} or \texttt{xxx.eps} only, 
if an according source like \texttt{xxx.fig} exists. 
Else it is treated as proper source and is not deleted. 


In the future, the combination of PDF/EPS and PTX files may be replaced, 
at least partially, or supplemented by TikZ files. 
It turned out, that the converters under consideration 
support more and more conversion into the TikZ format 
which can represent both, proper vector graphic and also \LaTeX{} texts like formulae. 
Using TikZ as intermediate format has the advantage, 
that the working space is polluted less with generated files, 
preprocessing is speeded up because fewer files are created and in some cases, 
less processing steps are needed. 
Another advantage is, that the internal dependency recording of \LaTeX{} engines 
made available through the \gls{fls} file is accessible. 
% TBD: not so easy: The created 
As in the current technique using PTX files instead of TEX files, 
we could put the TikZ into TEX files, but we opt against it for the same reasons. 

Note that PS is not supported because it misses the bounding box. 
% TBD: clarify whether we could add a bounding box by preprocessing 
If adding it, one arrives at the EPS format. 



\section{Conversion of fig-files}\label{sec:fig2dev}



A simple but still useful tool to draw figures is \texttt{xfig}\index{xfig}\index{fig2dev} 
which stores graphics in a native format 
described in~\cite{XFigF} with file extension \texttt{.fig}. 
The file extension \texttt{.fig} is also used by MATLAB to store plots, 
but this is something different. 
Graphics in xfig format cannot be directly included in latex files 
but must be exported into a \LaTeX-readable format. 

To export a file \texttt{xxx.fig} residing in directory \texttt{yyy} 
into several external formats, 
\texttt{xfig} uses \texttt{fig2dev}. 
A look in~\cite{XFigF}, Section~3.4 shows that texts with set ``special''-flag 
are interpreted as latex-code. 
For these texts the appropriate export language would be \texttt{latex}. 
On the other hand, \texttt{latex} is weak in graphics 
and \texttt{pdf} would be the ideal export format for all kinds of objects, 
except for texts with set ``special''-flag. 
In \texttt{pdf} format, texts are interpreted literally, 
independent of the ``special''-flag. 
Thus, \texttt{fig2dev} offers a mixed solution: 
export \texttt{xxx.fig} in format \texttt{pdftex} which yields a pdf-file 
\texttt{xxx.pdf} containing all but text with set ``special''-flag 
and complementary \texttt{pdftex\_t} which yields a tex-file \texttt{xxx.ptx} 
including the pdf-file and the texts with set ``special''-flag\index{special-flag}. %maybe glossary 
The exported files are in the same directory \texttt{yyy} 
as the original file \texttt{xxx.fig}. 

For example, 
the fig-file \texttt{F4\_01fig2dev.fig} defining Figure~\ref{fig:fig2dev}, 
is transformed into a file \texttt{F4\_01fig2dev.ptx} 
in format \texttt{pdftex\_t} which starts as given by Listing~\ref{lst:ptx}. 

%\lstset{language=tex, breaklines, basicstyle=\footnotesize}
% FIXME\@: nag complains for htlatex 
\lstinputlisting[language=TeX, basicstyle=\tiny,
breaklines, lastline=25,
float, captionpos=b, label={lst:ptx}, 
caption={The ptx-file for a fig-file}]{F4_01fig2dev.ptx}

The file \texttt{xxx.ptx} is ``imported'' into the tex-file of this manual 
by the command 
%
\begin{lstlisting}[language=TeX]
\import{yyy}{xxx.ptx}
\end{lstlisting}
%texttt
and includes \texttt{xxx.pdf} automatically the file \texttt{xxx.pdf} 
via \cmd{includegraphics\{xxx\}}{} (line 2). 
Note the following remarkable details: 
%
\begin{itemize}
\item
Observe that we can drop the suffix of the included file \texttt{xxx.pdf} 
which is expressed as ``\texttt{xxx}'' 
because \LaTeX{} chooses the right suffix: 
If instead of \texttt{xxx.pdf} there is a file \texttt{xxx.eps}, 
the latter is chosen if no suffix is specified. 
As we will see below, 
omitting the suffix is crucial to make \texttt{xxx.ptx} work 
for both \LaTeX-output formats: 
the pdf-format can include pdf-files, 
whereas the dvi-format which is required to create html- and
odt-files can include eps-files. 
\item
If \texttt{xxx.pdf} is included in \texttt{xxx.ptx} 
with the full path name, 
we may use \cmd{input\{xxx.ptx\}}{} instead of \cmd{import\{yyy\}\{xxx.ptx\}}. 

If in contrast, \texttt{xxx.pdf} is included in \texttt{xxx.ptx} 
with the short name only, 
\texttt{xxx.pdf} is assumed to be in the same directory 
as the file inputting \texttt{xxx.ptx}. 
So in general, i.e.~if this is not \texttt{yyy}, we need import 
\cmd{import\{yyy\}\{xxx.ptx\}}. 
If the directories coincide, 
in the import the string \texttt{yyy} may be empty. 
If the string \texttt{yyy} is not empty, it must end with the path delimiter, 
i.e.~\texttt/ for Unix like systems and 
\texttt{\textbackslash} for win-like systems. 
\end{itemize}

As indicated in Section~\ref{sec:graphPkg}, 
the commands in \texttt{xxx.ptx} 
require the packages \pkg{graphicx} and \pkg{xcolor}. 
Also, the \cmd{import}{} command 
requires the \pkg{import} package. 

To export \texttt{xxx.fig} into \texttt{xxx.ptx} and \texttt{xxx.pdf} 
this software invokes two commands: 
%
\begin{Verbatim}[fontsize=\scriptsize]
fig2dev -L pdftex   <fig2devGenOptions> <fig2devPdfEpsOptions>        xxx.fig xxx.pdf   
fig2dev -L pdftex_t <fig2devGenOptions> <fig2devPtxOptions>    -p xxx xxx.fig xxx.ptx
\end{Verbatim}
%
Both commands specify the input file \texttt{xxx.fig}, 
both use the options given by the parameter \texttt{fig2devGenOptions} 
while each invocation allows to specify also specific options, 
\texttt{fig2devPdfEpsOptions} and \texttt{fig2devPtxOptions}, respectively, 
and both use the option \texttt{-L} 
to specify the output format (``language''). 

%FIXME\@: ptx-->tex and then: convention: intuitive suffixes 

The parameters specific for \texttt{pdftex} 
are called \texttt{fig2devPdfEpsOptions} 
because the options available are the same 
as for output format \texttt{pstex} creating eps-files. 
An example for a common option would be \texttt{-b width} 
which shall specify the same boundary for both formats; 
otherwise they do not fit. 

For the output format \texttt{pdftex\_t}, 
the option \texttt{-p xxx} says, 
that the string \texttt{xxx} must be included in \texttt{xxx.ptx} 
as \cmd{includegraphics\{xxx\}}. 
Note that the option \texttt{-p} shall not be specified 
in \texttt{fig2devPtxOptions}, because it is automatically added. 

Equivalent to mixed export with formats \texttt{pdftex} and \texttt{pdftex\_t} 
which is appropriate for \LaTeX-output format pdf, 
is the mixed export with the according formats 
\texttt{pstex} and \texttt{pstex\_t} appropriate for \LaTeX-output format dvi. 
The difference is that \texttt{pstex} creates an eps-file instead of a pdf-file 
with the same content 
and \texttt{pstex\_t} creates a tex-file which looks like that 
created by \texttt{pdftex\_t} except including the eps-file 
instead of the pdf-file. 
If the suffix is not given, 
\texttt{pstex\_t} and \texttt{pdftex\_t} create identical files. 
Thus exporting \texttt{xxx.fig} via 
%
\begin{Verbatim}[fontsize=\scriptsize]
fig2dev -L pstex    <fig2devGenOptions> <fig2devPdfEpsOptions>        xxx.fig xxx.eps   
fig2dev -L pdftex   <fig2devGenOptions> <fig2devPdfEpsOptions>        xxx.fig xxx.pdf   
fig2dev -L pdftex_t <fig2devGenOptions> <fig2devPtxOptions>    -p xxx xxx.fig xxx.ptx
\end{Verbatim}
%
and ``inputting'' \texttt{xxx.ptx} works for both \LaTeX{} output formats. 

Table~\ref{tab:xfigSuffixes} relates the language specified with the \texttt{-L} option 
with the suffix of the output file chosen canonically, the suffix we choose 
and the actual file format. 
In contrast to \texttt{fig2dev}, 
we choose the actual file format, except if this is TEX\@. 
We opted for the quite unusual suffix \texttt{.ptx} 
instead of \texttt{.tex} 
to avoid that TEX-files may be both, 
source files and created files, 
but this is not compulsory, 
since the same holds and is accepted for pdf-files. 

\begin{longtable}{|l|lll|}
\toprule
Output format (language) & xfig suffix & our suffix & format \\
\midrule
\midrule
\endfirsthead%
\bottomrule%
\caption{\label{tab:xfigSuffixes} Language, suffixes and file format }
\endlastfoot%
pstex                    & pstex       & eps        & eps \\
pstex\_t                 & pstex\_t    & ptx        & tex \\
pdftex                   & pdf         & pdf        & pdf \\
pdftex\_t                & pdf\_t      & pdf        & pdf \\
\end{longtable}


Maybe xfig is intended to export from within the export dialog 
and not directly via a script like \texttt{fig2dev}. 
This may be the reason 
why the magnification must be set in the export dialog, 
but it is stored in the fig-file nevertheless. 

Figure~\ref{fig:fig2dev} shows the transformation 
of figures with \texttt{fig2dev} 
and the inclusion of the eps-file and of the pdf-file in the ptx-file. 
Note that the \texttt{fig2dev}-command is configurable 
via the parameter \texttt{fig2devCommand}, 
but there will be hardly any command with the same command line interface 
performing exactly the transformations given in Figure~\ref{fig:fig2dev}, 
except \texttt{fig2dev} itself. 

At the same time, Figure~\ref{fig:fig2dev} is an example 
for a \LaTeX-file \texttt{xxx.ptx} created from a fig-file 
and embedded in this \LaTeX-file 
with the \cmd{input}-command. 
More than that, 
Figure~\ref{fig:fig2dev} describes the way it has been created. 
Note that all text labels are specified with set ``special''-flag, 
and are thus included as \LaTeX-text, 
except the text \textbf{\tiny postscript} 
which is typeset with a postscript font to make the difference visible. 


\begin{figure}[htb]
\centering
\IfPackageLoadedTF{tex4ht}{%
should be a picture 
}{
\import{}{F4_01fig2dev.ptx}
}
\caption{\label{fig:fig2dev}Conversion of a fig-file 
into pdf-, eps- and ptx-files with inclusions}
\end{figure}


\section{Conversion of gnuplot-files}\label{sec:gnuplot2dev}

The term ``gnuplot'' refers to a file format
and to a program \texttt{gnuplot}
which can read this format, both described in~\cite{GnuPlot6_0}. 

Note that there seems no official file extension 
to identify gnuplot files. 
From the most common extensions \texttt{.plt}, \texttt{.gpi} and \texttt{.gp} 
we have chosen the one with the least collision 
and supported by Emacs, vscode and by my file browser: \texttt{.gp}. 
% FIXME\@: should be configurable.
% FIXME\@: look at https://www.file-extensions.org/
% FIXME\@: have a look in general on extensions for vscode 
% and include them into this manual 

The gnuplot format is a textual command language you can even program with 
and may thus be created with any editor but
for sake of reproducibility it is recommended to use only files
created by \texttt{gnuplot}.
To ensure that a handwritten gnuplot file \texttt{xxx.gp},
e.g.\@ with a single line like
%
\begin{verbatim}
plot [-10:10] sin(x), atan(x), cos(atan(x))
\end{verbatim}
%
really works
with the current \texttt{gnuplot} and to see how it is interpreted,
it is recommended to convert it via
%
\begin{lstlisting}
gnuplot -persist -e "load 'xxx.gp'; save 'xxx.gp'"
\end{lstlisting}
%
If you have a look inside the resulting file 
\href{\urlSite fromTex/F4_03someGnuplot.gp}{\texttt{F4\_03someGnuplot.gp}}, 
you can see, that in a comment line
the current version of \texttt{gnuplot} is documented
and also all the settings implicitly used.
The original line is the last but one. 
Pasting the into VS Code, one can see the highlighting, 
of course provided the extensions 
described in Section~\ref{subsec:instExtVsCode} are installed. 

Also, if a gnuplot file is created with an old version of \texttt{gnuplot},
it is recommended to update version with the same command.
Note that \texttt{gnuplot} does not offer full backward compatibility. 


This software supports including 
figures stored in \texttt{.gp}-files created by \texttt{gnuplot}.
To export a file \texttt{xxx.gp} into several external formats, 
it uses \texttt{gnuplot} itself. 
According to the manual~\cite{GnuPlot6_0}, Part IV, 
\texttt{gnuplot} supports output formats through so-called \emph{terminals}. 
Among those are several ones intended for inclusion into \LaTeX-files, 
like \texttt{Cairolatex}, \texttt{Epscairo}, \texttt{Epslatex}, 
\texttt{Latex}, \texttt{Lua (tikz)}, \texttt{Postscript}, \texttt{Ps(la)tex}, % chktex 36
\texttt{Pstricks}, \texttt{Texdraw} and \texttt{Tikz} 
which is in fact equivalent with \texttt{Lua (tikz)}. 
Comparison with the manual~\cite{GnuPlot5_0} for older versions of \texttt{gnuplot} 
shows that support of \texttt{Eepic}, \texttt{Mp} and \texttt{Tpic} ended. 
Note that also export into the fig-format via the terminal \texttt{Fig} 
is supported which in turn may be included in latex 
as described in Section~\ref{sec:fig2dev}. 
Also, \texttt{gnuplot} pictures may be exported in MetaPost format 
which in turn may be included in latex 
as described in Section~\ref{sec:metapost}. 
% TBD: clarify maybe tikz is superior. see backlog. 

This software supports the export of a file \texttt{xxx.gp} 
only via the terminal \texttt{Cairolatex} 
which offers export to mixed PDF and \LaTeX\@: 
graphics in PDF and text in \LaTeX{}
which yields the fonts typical for \LaTeX. 
This is as described for fig-files in Section~\ref{sec:fig2dev}, 
except that text is generally converted in \LaTeX{}-format, 
and not selectively those text marked with special flag. 

Accordingly, the export yields two files \texttt{xxx.ptx} and
\texttt{xxx.pdf}, both in the directory \texttt{yyy} 
in which \texttt{xxx.gp} resides. 
The file \texttt{xxx.ptx} must be imported via 
%
\begin{lstlisting}[language=TeX]
\import{yyy}{xxx.ptx}
\end{lstlisting}
%
It contains the texts and includes \texttt{xxx.pdf} 
via \cmd{includegraphics\{xxx\}}{} without specifying a suffix. 

Unlike for fig-files, 
\texttt{xxx.ptx} and \texttt{xxx.pdf} are created with a single command: 
%
\begin{verbatim}
gnuplot -e "set terminal cairolatex pdf <gnuplotOptions>;
            set output 'xxx.ptx';
            load 'xxx.gp'"
\end{verbatim}

Accordingly, 
\texttt{xxx.ptx} and \texttt{xxx.eps} are created with a single command: 
%
\begin{verbatim}
gnuplot -e "set terminal cairolatex eps <gnuplotOptions>;
            set output 'xxx.ptx';
            load 'xxx.gp'"
\end{verbatim}
%
Note that this writes another but identical file \texttt{xxx.ptx} 
as no file endings are written 
and so \texttt{xxx.ptx} can include both, \texttt{pdf} and \texttt{eps}. 
When creating both performance is not optimal, 
but \texttt{gnuplot} offers no way to avoid this. 
If being strict, 
\texttt{xxx.ptx} is perfectly correct only for output \texttt{eps},
% TBC: in which respect? 
if comments and error messages are taken into account 
but as long as no error occurs, 
the result is perfectly ok also for \texttt{pdf}. 

As for inclusion of fig-files, 
packages \pkg{graphicx} and \texttt{color} are needed. 


%FIXME\@: 
%Here further options are missing: 
%   // set terminal pdf {monochrome|color|colour}
%    //                      {{no}enhanced}
%    //                      {fname "<font>"} {fsize <fontsize>}
%    //                      {font "<fontname>{,<fontsize>}"}
%    //                      {linewidth <lw>} {rounded|butt}
%    //                      {solid|dashed} {dl <dashlength>}}
%    //                      {size <XX>{unit},<YY>{unit}}

Figure~\ref{fig:gp2pdf} shows the transformation of the plots 
and the inclusion of graphic files. 
In addition, Figure~\ref{fig:gnuplot} shows an example of a \LaTeX-file 
created from a gnuplot file 
and embedded in this \LaTeX-file. 
%Note that the \texttt{gnuplot}-command is configurable 
%via the parameter \texttt{gnuplotCommand}, 
%but there will be hardly any command with the same command line interface 
%performing exactly the transformations given in Figure~\ref{fig:gp2pdf}, 
%except \texttt{gnuplot} itself. 

\begin{figure}[htb]
\centering
\IfPackageLoadedTF{tex4ht}{%
should be a picture 
}{
\import{}{F4_02gp2pdf.ptx}
}
\caption{\label{fig:gp2pdf}Conversion of a gnuplot-file 
into pdf-, eps- and ptx-files with inclusions}
\end{figure}

\begin{figure}[htb]
\centering
\IfPackageLoadedTF{tex4ht}{%
should be a picture 
}{
\import{}{F4_03someGnuplot.ptx}
}
\caption{\label{fig:gnuplot}
Converted sample gnuplot-file into ptx and pdf files }
\end{figure}


\section{Inclusion of MetaPost files}\label{sec:metapost}

A vector graphic format, very native to TeX is \texttt{MetaPost}, 
a derivative of \texttt{Metafont} originally used to describe shape of fonts. 
Although seemingly supported by \TeX{} only, 
\texttt{MetaPost} is interesting in its own right, 
as it is a graphical programming language, 
Turing complete, much like postscript, and allows also declarative programming. 
The manual describing the language is~\cite{MPost24}, 
seemingly complete, but it is not. 
Thus, one can be thankful for~\cite{MPostGuid} 
which offers some introduction and for the really helpful tutorial~\cite{MPostTut}. 

Files containing \texttt{MetaPost} have the ending \texttt{.mp}. 
Note that there are other graphic formats 
like monochrome pictures in TIFF-format 
which are identified with the same extension 
but the MetaPost format has nothing to do with this. 

Since \texttt{MetaPost} is a programming language, 
MetaPost files are created with an editor. 
Since \texttt{MetaPost} is very versatile, 
it is impossible to give an impression by a single example. 
We decided to choose an example 
using a MetaPost library, \texttt{MetaUML}, 
described in~\cite{MetaUml} for some reasons apparent later. 
The example file is given in Listing~\ref{lst:metapost} 
and also on the web as 
\href{\urlSite fromTex/F4_05someMetapost.mp}{\texttt{F4\_05someMetapost.mp}}. 
It is the source file of Figure~\ref{fig:metapost}. 
Pasting the into VS Code, one can see the highlighting, 
of course provided the extensions 
described in Section~\ref{subsec:instExtVsCode} are installed. 


\lstinputlisting[language=MetaPost,
basicstyle=\footnotesize,
float,
showstringspaces={false},
numbers={left}, stepnumber={5}, 
captionpos=b, label={lst:metapost},
caption={An example file in MetaPost}]
{F4_05someMetapost.mp}

Listing~\ref{lst:metapost} illustrates some structure of MetaPost. 
As in \TeX, comments start with \texttt{\%} and end with the line or with the file. 
The proper figures are enclosed between \texttt{beginfig(n)} and \texttt{endfig}, %chktex 36
where $n$ is the number of the figure, the so called \texttt{charcode}\footnote%
{This is a relict from Metafont, where each figure showed a character}, 
and the file ends with \texttt{end}. 
This software relies on specifying a single figure per file; the \texttt{charcode} is irrelevant. 

Code outside figures is possible, but does not belong to a figure 
and is thus not displayed. 
In our example, besides \texttt{end} commands outside the figure 
are just \texttt{input xxx}, where \texttt{xxx} names a so-called library 
defined by the file \texttt{xxx.mp} 
and a sequence of settings of internal variables 
of the MetaPost compiler 
controlling how the following figure is compiled. 
Most of them even in comments. 

The compiler for \texttt{MetaPost} 
is given by the parameter \texttt{metapostCommand} 
which defaults to \texttt{mpost}, occasionally just \texttt{mp}. 

Each internal variable which can be set in the MP file 
can also be set when invoking \texttt{mpost} 
using the option \texttt{-s \textlangle{}variable\textrangle=\textlangle{}value\textrangle} 
as described in~\cite{MPost24},~Section~B.2.1. 
There it is stated that the option is read just before the file is read, 
which implies that the setting in the file overrides the command line setting. 
Caution: in the manual, the variable is referred to as ``key''. 

The most basic setting is \texttt{outputformat:="eps"} %chktex 18
which is the only setting appropriate for latex. 
So don't change\footnote%
{Note that \texttt{metapostCommand} may also besides \gls{eps} 
output \gls{svg} and \gls{png}, 
just by setting \texttt{outputformat:="svg"} or that like. %chktex 18
Caution: case-sensitive, assuming silently \texttt{eps} if the format is not recognized. 
Whereas \gls{svg} is a vector format as MetaPost ifself, \gls{png} is a raster format}. 
Note the strange default setting for the names of the output files, \texttt{outputtemplate}, 
which reflects the \texttt{charcode} of the individual figures as file ending. 
For inclusion in latex, the file ending \texttt{mps} is required 
and so frequently \texttt{outputtemplate} is set to reflect the ending. 
It seems more appropriate to make the setting in the command line 
which yields the following invocation
%
\begin{verbatim}
  mpost -s 'outputtemplate="%{jobname}%{charcode}.mps"' xxx.mp
\end{verbatim}

As we agreed that a MetaPost file shall contain a single figure only, 
we also ignore the \texttt{charcode} which unifies MetaPost with other formats supported. 
This yields 
%
\begin{verbatim}
  mpost -s 'outputtemplate="%{jobname}.mps"' xxx.mp
\end{verbatim}
%
The MetaPost file shall not overwrite the command line settings. 

The setting of \texttt{prologues} controls where fonts come from and 
becomes relevant when using \TeX{} for typesetting. 
Listing~\ref{lst:metapost}, line 21 includes a label via a note implicitly, 
and for the material between \texttt{btex} and \texttt{etex} uses \TeX{}. 
The manual~\cite{MPost24}, Section 8.1 is on typesetting labels 
and specifies the meaning of \texttt{prologues}. 
If we stick to including in \LaTeX{} and creating PDF out of that only, 
the default setting $0$ is appropriate always 
but since this software uses DVI as intermediate format, e.g.\@ to create HTML, 
or because for debugging one wants to view the MPS files standalone in a viewer 
things are not so easy. 
For details see~\cite{MPost24}, Section 14.2. 
Setting \texttt{prologues:=1} is deprecated. 
The only save way to get the correct display is to include fonts in the MPS file, 
setting \texttt{prologues:=3}, but this makes the MPS file quite big. 
So a good compromise is to set \texttt{prologues:=2} as a command line option 
resulting in 
%
\begin{verbatim}
  mpost -s prologues=2 -s 'outputtemplate="%{jobname}.mps"' xxx.mp
\end{verbatim}
%
and overwriting by need as in Listing~\ref{lst:metapost}, line 2. 

As mentioned above, \texttt{input xxx} includes a library 
making the program dependent on a file \texttt{xxx.mp}. 
As for latex processors, also \texttt{mpost} records dependencies recursively 
in an FLS file if invoked with option \texttt{-recorder}. 
Also like latex processors, an error shall not cause break or interaction 
so adding the option \texttt{-interaction=nonstopmode}. 
Thus, we arrive finally at the default invocation 
%
\begin{verbatim}
  mpost -interaction=nonstopmode -recorder \
    -s prologues=2 -s 'outputtemplate="%{jobname}.mps"' xxx.mp
\end{verbatim}






% TBD: treat mpxerr.tex, mpxerr.log
% this must be done both in software and in this manual 





Figure~\ref{fig:mp2mps} illustrates 
how \texttt{mpost} converts an \gls{mp}-file \texttt{xxx.mp} 
with the given settings into various result files: 
%
\begin{itemize}
\item
an \gls{mps}-file or with setting 
\begin{verbatim}outputtemplate="\%{jobname}\%{charcode}.mps"\end{verbatim}
more \gls{mps}-files \texttt{xxx1.mps}\dots \texttt{xxxn.mps}, 
\item
a log-file \texttt{xxx.log} and a fls-file \texttt{xxx.fls} much like \LaTeX{} does 
\item
and an \gls{mpx}-file \texttt{xxx.mpx} containing the \LaTeX{} text of the figure; 
this is not created if there is no such text. 
\end{itemize}



\begin{figure}[htb]
\centering
\IfPackageLoadedTF{tex4ht}{%
should be a picture 
}{
\import{}{F4_04mp2mps.ptx}
}
\caption{\label{fig:mp2mps}Conversion of a MetaPost-file into an mps-file}
\end{figure}


Figure~\ref{fig:metapost} gives an example of a MetaPost file 
included in this \LaTeX-file as ab mps-file 
created from the MetaPost file 
and embedded in this \LaTeX-file 
with the \cmd{includegraphics}-command. 
Normally, \cmd{includegraphics} is invoked with the filename without extension, 
but for mps-files, the extension is needed. 
As for inclusion of fig-files, the package \pkg{graphicx} is needed. 

\begin{figure}[htb]
\centering
\IfPackageLoadedTF{tex4ht}{%
should be a picture 
}{ 
%\includegraphics{F4_05someMetapost1.mps}
\includegraphics{F4_05someMetapost.mps}
}
\caption{\label{fig:metapost}
Converted sample MetaPost-file included as mps-file  }
\end{figure}

One of the descendants of MetaPost is TikZ (see introductory text~\cite{TikzIntro})
and one of the deficiencies resolved is that it allows passing information 
from the main document to the proper figure. 

With \lualatex{} this can be reached for MetaPost also using package \pkg{luamplib}. 
The package itself provides an environment \texttt{mplibcode}. 
Essentially, \lualatex{} interprets all code 
enclosed in the \texttt{mplibcode} environment as MetaPost. 
As described in Section~\ref{subsec:header}, 
this software can inject a header which loads the header 
and enhances it providing the additional command \cmd{inputmpcode} 
which allows also load MetaPost from a file. 
The latter is preferred to direct inclusion with the \texttt{mplibcode} environment, 
e.g.\@ for sake of proper code highlighting. 
Note that the package declaration is enclosed in an if construct, 
ensuring that the package is loaded only if lualatex or that like is run. 

That this allows better integration within the enclosing latex document 
is illustrated by redefining the letter $\aleph$ as $\alpha$ which is really related. 
%
\begin{lstlisting}[language=TeX]
  {% make redefine local 
  \renewcommand{\aleph}{\alpha}
  \inputmpcode{F4_05someMetapost}
  }% to recover from redefine {manualC4graphics.tex}
\end{lstlisting}

Figure~\ref{fig:metapostLua} Documents, 
that the redefinition really influences rendering in the MetaPost file. 

\begin{figure}[htb]
  \centering
  \IfPackageLoadedTF{tex4ht}{%
  should be a picture 
  }{
  \ifluatex%
  {% make redefine local 
  \renewcommand{\aleph}{\alpha}
  \inputmpcode{F4_05someMetapost}
  }% to recover from redefine 
  \else
  \includegraphics{F4_05someMetapost.mps}
  \fi
  }%texFhtLoaded
  \caption{\label{fig:metapostLua}
  Sample MetaPost-file included via \texttt{luamplib} for lua(hb)tex } % chktex 36
\end{figure}

  
  %Question: how to center, how to scale? 
  %Interesting: interaction between MetaPost code and enclosing tex: 
  
  



% I cannot detect an mps file. 
  
% This example is simple but not very illustrative. 
% In the long run we shall use a state diagram based on~\cite{MetaUml}. 

% There must be also a section on installation of packages: 
% in suse linux it is ready to go, but in general, it is not. 
% kpsewhich metauml.mp
% /usr/share/texmf/metapost/metauml/metauml.mp
% shows the location. 
% Put https://github.com/ogheorghies/MetaUML/ there 
% and mktexlsr and then check again with kpsewhich. 
% docs is in 
% /usr/share/texmf/doc/metapost/metauml/README

% What is interesting here also is, that mp can be included directly in lualatex. 
% There shall be some illustration for this. 

% Similarly one can install latex packages. 
% Example: tikz-uml: 
% kpsewhich tikz-uml.sty shows 
% /usr/share/texmf/tex/latex/tikz-uml/tikz-uml.sty

% ok download, move accordingly and also move the docs 
% /usr/share/texmf/doc/amstex/base/README
% /usr/share/texmf/doc/latex/geometry/README.md



\section{Inclusion of SVG-files}\label{sec:picSvg}

Comparable with the xfig-format described in Section~\ref{sec:fig2dev} 
but much more elaborate and widely used is the \gls{svg}-format. 
There is a huge up-to-date official SVG 1.1 specification,~\cite{Svg11} 
and a specification~\cite{Svg12Tiny} for SVG Tiny 1.2, 
which is itself quite short and more readable 
and gives also a good overview on ``SVG Big''. 
For a tutorial, see~\cite{SvgTut}. 
As stated in~\cite{Svg12Tiny}, Section~1.1, 
SVG-files may contain vector graphics, raster images and text. 
It may also contain video and audio elements 
and may be interactive and dynamic, 
which goes beyond what can be included in \LaTeX-files. 

Figure~\ref{fig:svgWithText} shows a picture in \gls{svg}-format. 
As PDF-files are included directly 
via the \cmd{includegraphics}-command, 
using the \LaTeX-packages \pkg{xcolor} and \pkg{graphicx}, 
virtually, 
\texttt{xxx.svg} can be included directly via 
% FIXME\@: done differently 
\begin{lstlisting}[language=TeX]
%\includesvg[width=0.5\textwidth]{xxx}%
\end{lstlisting}
%
using the \LaTeX-packages \pkg{svg} described in~\cite{SvgP}. 
Note that the suffix of the file name shall be omitted. 

A closer look shows, that graphic preprocessing is done behind the scenes 
in the course of a \LaTeX-run 
creating files \texttt{xxx.pdf} and \texttt{xxx.pdf\_tex}. 
As described for fig-files in Section~\ref{sec:fig2dev} 
and for gnuplot-files in Section~\ref{sec:gnuplot2dev}: 
The latter is a \LaTeX-file containing text 
and including the former. 
To include \texttt{xxx.pdf} 
of course the \LaTeX-packages \pkg{xcolor} and \pkg{graphicx} 
are required. 
Moreover, it may happen that the \LaTeX-package \pkg{transparent} 
is required also, depending on the features used in \texttt{xxx.svg}. 

As indicated in~\cite{SvgP}, Section~1, 
the \pkg{svg}-package delegates the transformation 
of \texttt{xxx.svg} \texttt{xxx.pdf} and \texttt{xxx.pdf\_tex} 
to \texttt{inkscape}. 
This is a graphical editor with export functions 
which can be invoked in batch-mode also. 
Of course using the \pkg{svg}-package has the advantage 
that no explicit preprocessing is required, 
the created files updated by need. 
It is worth thinking about whether it is worthwhile 
writing according packages \texttt{fig} and \texttt{gnuplot}. 

On the other hand, 
this breaks the workflow this software normally applies to graphic files. 
In particular, the package creates \LaTeX{} main files 
which are not removed after the latex run 
if parametrized accordingly or if something goes wrong. 
Also, the \pkg{svg}-package does not provide the full flexibility 
of a standard solution. 
Since this software is still under construction 
and more than that, is in an experimental phase, 
we provide explicit preprocessing of SVG-files using \texttt{inkscape}. 
Another problem with the \pkg{svg}-package is, 
that according to~\cite{SvgP}, Section~1, 
it does not work on Windows platforms. 


Some research shows,
that \texttt{inkscape} in the version current at time of this writing
exports mixed PDF and latex: If invoked as 
%
\begin{Verbatim}[fontsize=\footnotesize]
inkscape --export-filename=xxx.pdf --export-area-drawing --export-latex xxx.svg 
\end{Verbatim}
%
\texttt{inkscape} creates a file \texttt{xxx.pdf}
containing all graphics but text and another file \texttt{xxx.pdf\_tex}
containing text and including \texttt{xxx.pdf}.
The file \texttt{xxx.pdf\_tex}
can be integrated into the latex document as
%
\begin{lstlisting}[language=TeX]
\def\svgwidth{0.5\textwidth}
\import{yyy}{xxx.pdf\_tex}%
\end{lstlisting}
%
Unlike \texttt{fig2dev} and \texttt{gnuplot}, 
specifying the files with their full path, 
has no effect, i.e.~inclusion uses the file name only. 
Thus, \cmd{import}{} cannot be replaced by \cmd{input}{} 
and so the \LaTeX-package \pkg{import} is required. 
 

This is essentially the same technique as applied for fig-files 
and for gnuplot-files as described 
in Sections~\ref{sec:fig2dev}~and~\ref{sec:gnuplot2dev}. 

Analogously,
%
\begin{Verbatim}[fontsize=\footnotesize]
inkscape --export-filename=xxx.eps --export-area-drawing --export-latex xxx.svg 
\end{Verbatim}
%
exports files \texttt{xxx.eps\_tex} and \texttt{xxx.eps}.

In older versions of \texttt{inkscape},
there was a configuration allowing \texttt{xxx.eps\_tex}
to include uniformly both \texttt{xxx.pdf} and \texttt{xxx.eps}.
Thus, \texttt{xxx.pdf\_tex} could be deleted
and \texttt{xxx.eps\_tex} moved to \texttt{xxx.ptx}
which in turn could be included into the main document.

As shown in Figure~\ref{fig:svg2pdf},
for the current version of \texttt{inkscape},
this software filters \texttt{xxx.eps\_tex} into \texttt{xxx.ptx}
``manually'' so that both \texttt{xxx.pdf} and \texttt{xxx.eps}
are included in \texttt{xxx.ptx}.
Then it deletes the original files
\texttt{xxx.pdf\_tex} and \texttt{xxx.eps\_tex}.

The author has filed a bug report to the inkscape team,
to avoid this workaround in the future. 

\begin{figure}[htb]
\centering
\IfPackageLoadedTF{tex4ht}{%
should be a picture 
}{
\import{}{F4_06svg2pdf.ptx}%
}
\caption{\label{fig:svg2pdf}Conversion of an SVG-file 
into pdf-, eps- and ptx-files with inclusions}
\end{figure}


\begin{figure}[htb]
\centering
%\def\svgwidth{0.5\textwidth}
\IfPackageLoadedTF{tex4ht}{%
should be a picture 
}{
  \import{}{F4_07someSvg.ptx}% this should be the standard
}
\caption{\label{fig:svgWithText}Some svg-picture with text FIXME\@: uniformity  }
\end{figure}

In contrast to the FIG format, 
SVG pictures can be created by several programs. 
Among those, is also \texttt{inkscape} 
which can be used like \texttt{xfig}  
as a graphical editor with export functionality. 
In contrast to FIG format, 
SVG is essentially human-readable, 
in fact an XML derivative. 
The author calls it ``essentially'', 
referring to the fact, that the format is quite wordy 
as is illustrated by the source code 
\href{\urlSite fromTex/F4_07someSvg.svg}{\texttt{F4\_07someSvg.svg}} 
for the above picture. 
Nevertheless, it can be an advantage 
to go into internals and manipulate with a text editor. 
Pasting the into VS Code, one can see the highlighting and a preview, 
of course provided the extensions 
described in Section~\ref{subsec:instExtVsCode} are installed. 


\section{Pictures which are not transformed}\label{sec:picAsIs}

Figure~\ref{fig:asIsJpg} shows some picture included as JPG\@. 
This is done as usual with the command \cmd{includegraphics}{}
provided by the package \pkg{graphicx}. 
According to the documentation~\cite{GraX}, page 13, 
the bounding box must be provided somehow. 

This may be done via the package \pkg{bmpsize} 
but alternatively also using the command \texttt{ebb}. 
There is some hint, that \pkg{bmpsize} does not work with \texttt{xelatex}. 
So maybe \texttt{ebb} is the better alternative. 
Note that both techniques are available in distribution \texlive, 
but not in \miktex. 

Research shows, that inclusion is seamlessly if PDF files are created. 
So the problem addressed is specific for creating DVI files. 
Also, at time of this writing, it seems that also in DVI mode, no problems occur. 
Nevertheless, the author experienced errors on missing bounding box 
and to be safe, provides a way to invoke \texttt{ebb} on the file \texttt{xxx.jpg}. 

With parameter \texttt{-m}, this creates a file \texttt{xxx.bb} 
containing the bounding box for \texttt{dvipdfm}, 
and with parameter \texttt{-x} a file \texttt{xxx.xbb} 
containing an extended bounding box for \texttt{dvipdfmx}. 
The current implementation seems not to make any difference, 
whether the bounding boxes are created or not. 

Sizes seem to differ in DVI/XDV output after conversion to PDF, 
depending on whether \texttt{dvipdfm} or \texttt{dvipdfmx} is used. 
Only the latter yields the same size as direct conversion to PDF creates. 

Since bounding boxes seem superfluous, 
we control their creation with a parameter \texttt{createBoundingBoxes} 
whether to invoke \texttt{ebb}, which is false by default. 
Nevertheless, if we invoke, then we do twice, 
creating bounded boxes and extended bounding boxes. 


FIXME\@: further research and further documentation is required. 

Note that both for \pdflatex{} and siblings creating PDF-output 
and for \texttt{htlatex} in conjunction with \texttt{dvipdfmx} 
files in the format \gls{pdf}, \gls{png}, \gls{jpg} are supported. 
% **** what about \gls{gif}? 
This list may be incomplete. 

\begin{figure}[htb]
\centering
\IfPackageLoadedTF{tex4ht}{%
should be a picture 
}{
\setkeys{Gin}{width=0.9\textwidth}
\includegraphics%[width=0.9\textwidth]
{F4_08someJpgOboeBaroqueDennerMIR370}%.jpg
}
\caption{\label{fig:asIsJpg}Some JPG-picture, directly included }
\end{figure}

As an example, Figure~\ref{fig:asIsPng} shows the same picture 
as PNG-file. 

FIXME\@: At the moment, \texttt{htlatex} does not work with pictures at all. 

\begin{figure}[htb]
\centering
\IfPackageLoadedTF{tex4ht}{%
should be a picture 
}{
\includegraphics[width=0.9\textwidth]
{F4_09somePngOboeBaroqueDennerMIR370}%.png
}
\caption{\label{fig:asIsPng}Some PNG-picture, directly included }
\end{figure}

Note that in DVI/XDV mode all usual \LaTeX{} engines 
can include BMP-pictures, whereas in PDF mode only \texttt{xelatex} 
can do that, maybe because it creates XDV internally in any case. 
In contrast, \texttt{lualatex} and \texttt{pdflatex} can not. 
% The following works for lualatex but not for pdflatex 
% And still the same picture, Figure~\ref{fig:asIsPng} shows the oboe 
% as bmp-file. 

% \begin{figure}[htb]
%   \centering
%   \IfPackageLoadedTF{tex4ht}{%
%   should be a picture 
%   }{
%   \includegraphics[width=0.9\textwidth]
%   {F4_10someBmpOboeBaroqueDennerMIR370.bmp}%.bmp
%   }
%   \caption{\label{fig:asIsBmp}Some bmp-picture, directly included }
%   \end{figure}
  


\chapter{Processing of \LaTeX{} Main Files}\label{chap:latexMainConversions}

Given graphics in formats includable in TEX files, 
which may require preprocessing described in 
Chapter~\ref{chap:GraphConversions}, 
this section describes the conversions of \LaTeX{} main files 
into target files in detail. 
The most important target file format is \gls{pdf}. 
Conversion into this format is described in Section~\ref{sec:tex2pdf}. 
Note that \gls{pdf} also occurs as source format 
for included pictures and as intermediate files. 
Specific for \LaTeX{} is the \gls{dvi} format, 
which is supported mainly for historical reasons. 
% FIXME\@: nowhere described. 

Almost independent of the format created, 
inclusion of bibliographies, indices and glossaries 
requires additional conversions 
done by several auxiliary programs. 
Bibliographies are described in Section~\ref{sec:bibtex}, 
indices in Section~\ref{sec:indices} 
and glossaries in Section~\ref{sec:glossaries}. 
Only at the first sight different 
but behind the scenes quite analogous 
is inclusion of results of code evaluations, 
code in python and other languages described in Section~\ref{sec:pythontex}. 
Here, an auxiliary program essentially invokes the language interpreter. 

Sections~\ref{sec:runRerunAux} and~\ref{sec:rerunLatex} 
describe running and rerunning auxiliary programs 
like \texttt{makeindex} and the \LaTeX{} engine, respectively. 
The latter may be necessary if certain lists are present 
like table of contents list of figures or list of tables. 
Section~\ref{sec:runRerunAux} clarifies the exchange of information  
between the \LaTeX{} engines and auxiliary programs, 
whereas Section~\ref{sec:rerunLatex} 
essentially describes the exchange of information 
between individual runs of the \LaTeX{} engine. 

Section~\ref{sec:chkReprod} is special in that it is not related with conversion 
but with checking reproducibility. 
This \LaTeX{} builder has some built-in build algorithm, 
but one can also use \tool{latexmk} as a build tool 
in a way that invokes all tools with parameters given by the configuration. 
Note that \tool{latexmk} has a different build algorithm, 
but the results should be the same. 
This is mainly to integrate document development more seamlessly. 
For details on motivation and implementation see Section~\ref{sec:latexmk}. 

Besides the output formats traditional for \LaTeX, 
\gls{pdf} and \gls{dvi} describing e.g.\@ books, 
Section~\ref{sec:tex2html} describes creation of 
\gls{html}, Section~\ref{sec:tex2odt} the creation of \gls{odt} and 
Section~\ref{sec:tex2doc} creation of MS Word formats like \gls{docx}. 
Finally, also pure text can be generated 
as described in Section~\ref{sec:tex2txt}. 

\newpage


\section{Transforming \LaTeX{} files into PDF files}\label{sec:tex2pdf}

The next step is to create a PDF file from the TEX files. 
\LaTeX{} distinguishes master TEX files from TEX files intended to be inputted
from elsewhere. 
Not taking comments and that like into account, 
master TEX files roughly have the form 
%
%\lstset{language=tex, basicstyle=\small}
\begin{lstlisting}[language=tex, basicstyle=\small]
\RequirePackage[l2tabu, orthodox]{nag} % optional 
\documentclass{...}

\begin{document}
...
\end{document}
\end{lstlisting}

The core of conversion of a TEX file into a PDF file 
is running a \LaTeX{} engine \texttt{latex2pdf} 
to a master TEX file \texttt{xxx.tex}.
The \LaTeX{} engine \texttt{latex2pdf} 
is configurable via the parameter \texttt{latex2pdfCommand}. 
Possible values are \lualatex{}, \xelatex{} and \pdflatex, 
where the first is the default for which this software is also tested. 
It is also possible to pass parameters to the \LaTeX{} engine. 
Besides conversion into \gls{pdf} format, 
all engines offer conversion to the older \gls{dvi} format 
via option \texttt{-{}-output-format} as \lualatex{} and \pdflatex, 
or the alternative \gls{xdv} generalizing \gls{dvi} 
as \xelatex{} does with the option \texttt{-{}-no-pdf}. 

In fact, the engine \texttt{latex2pdf} 
does much more than converting TEX files to PDF files. 
Figure~\ref{fig:tex2pdf} shows for \texttt{latex2pdf} set e.g.~to \lualatex{}, 
that besides the PDF file also a LOG file and an AUX file is created. 
The LOG file contains logging information on the run of the conversion 
and the AUX file transports information from one run to the next, 
writing in one run and reading in the next run. 
Thus, conversion goes without it, but it is read if present. 
This is why it is depicted at input side in dashed lines. 

Optionally, an FLS file is created containing paths to the files 
the converted \LaTeX{} file depends on 
and a file with ending \texttt{synctex.gz} 
with information for mapping locations at the created PDF file 
to the according input files. 
This is to support backward search, meaning click on a place in the PDF viewer 
opens an editor in the source file. 

What is in fact in the AUX file depends on the package. 
Among other information, 
also citations and the location of the bibliography file with ending bib 
are present. 
This cannot be used directly in the next \texttt{latex2pdf} run 
to create the bibliography, 
because the entries referenced in the document must be extracted from the BIB file 
and sorted. 
This is done by invoking \texttt{bibtex} between two \texttt{latex2pdf} runs. 
Based on the AUX file, \texttt{bibtex} creates a BBL file 
containing the bibliography, which is read in the next \texttt{latex2pdf} run. 
For details see Section~\ref{sec:bibtex}. 

Alternatively to \texttt{bibtex} a bibliography can be created 
with the package \pkg{biblatex} in conjunction with the auxiliary program \texttt{biber}. 
Running a \LaTeX{} engine with package \pkg{biblatex} loaded 
creates a \gls{bcf} file read by \texttt{biber}. 
At time of this writing, this software does not support that option. 
Nevertheless, for sake of completeness we added this data path to Figure~\ref{fig:tex2pdf}. 

If an index is demanded, 
in addition \texttt{latex2pdf} creates a \gls{idx} file. 
As the citations, it cannot be used directly to create an index in
the next \texttt{latex2pdf} run, 
because the index entries must be collected and sorted before. 
This is done by invoking \texttt{makeindex} 
between the two \texttt{latex2pdf} runs. 
Based on the \gls{idx} file, \texttt{makeindex} creates a \gls{ind} file 
containing the index, which is read in the next \texttt{latex2pdf} run. 
For details see Section~\ref{sec:indices}. 

If more than one index is demanded, 
we suggest using \texttt{splitindex} instead of \texttt{makeindex} 
which creates one \gls{ind} file per index. 

A more modern technique to create an index is via \texttt{xindy}, 
but at time of this writing, this software does not support \texttt{xindy} yet. 

If a glossary is demanded, this can be read off the \gls{aux} file 
and a \gls{glo} file containing the index entries 
is created and a file with style information. 
Depending on the configuration, 
this may be a \gls{ist} file or a \gls{xdy} file. 
As for the index the \gls{idx} file, 
the \gls{glo} file cannot be used directly to create a glossary in
the next \texttt{latex2pdf} run, 
because the glossary entries must be collected and sorted before. 
This is done by invoking \texttt{makeglossaries} 
between the two \texttt{latex2pdf} runs. 
Based on the \gls{glo} file, \texttt{makeglossaries} creates a \gls{gls} file 
containing the glossary, which is read in the next \texttt{latex2pdf} run. 
For details see Section~\ref{sec:glossaries}. 

Besides \texttt{makeglossaries}, there is a more modern tool, 
\texttt{bib2gls}, which is not yet supported by this software 
at time of this writing. 

The package \pkg{pythontex} allows including python code or related 
in the \gls{tex} file and to evaluate it. 
The first \texttt{latex2pdf} run creates a \gls{pytxcode} file 
which contains essentially the code parts of the \LaTeX{} file. 
Invoking \texttt{pythontex} creates by default 
a folder \texttt{pythontex-files-xxx} 
with material where code is already evaluated. 
In the next \texttt{latex2pdf} run, this material is included in the document. 
The \pkg{pythontex} comes with a second command line utility, 
\texttt{depythontex}, eliminating all python code from the original TEX file. 
Optionally, \texttt{latex2pdf} also creates a \gls{depytx} file 
with all information to replace python code in the original TEX file 
with evaluated material from \texttt{pythontex-files-xxx}. 
Replacement is done by \texttt{depythontex} 
which by default, sends the result to stdout, 
but there is an option to write into another \LaTeX{} file. 
Converting this new \LaTeX{} file 
yields the same result as converting the original one. 
Depythonization is a feature needed e.g.~for papers 
when the publisher does not accept included code. 
For details see Section~\ref{sec:pythontex}. 

In addition, if
a table of contents, a list of figures, a list of tables 
or a list of listings is required, 
also a TOC file, a LOF file, a LOT file and a LOL file is created,
respectively, 
collecting the according information. 
Also, if hyper-references are built, an \gls{out} file 
containing bookmarks is created. 
If such a file is present, it is read in and is used
to create a table of contents, a list of figures, of tables and of listings 
or bookmarks in the second run of \texttt{latex2pdf}. 

To summarize, 
if a table of contents, a list of figures, a list of tables, a list of listings or 
a bibliography, an index or a glossary is present, 
or if code must be replaced by their evaluation, 
a second \LaTeX{} run is required to make that material appear in the PDF output. 

If a table of contents and at the same time 
a bibliography, an index or a glossary is present, 
even two further \LaTeX{} runs are required: 
After the first one, the bibliography, the index or the glossary 
occurs in the PDF file but not yet in the table of contents. 
This happens after the second additional \LaTeX{} run. 
As described in Sections~\ref{sec:runRerunAux} and~\ref{sec:rerunLatex}, 
further runs of auxiliary programs mainly to create index or glossaries, 
but also under certain circumstances bibliographies and inserting invoked code, 
followed by invocation of the \LaTeX{} engine \texttt{latex2pdf} may be necessary. 

\begin{figure}[htb]
\centering
\IfPackageLoadedTF{tex4ht}{%
should be a picture 
}{
\import{}{F5_01tex2pdf.ptx}
}
\caption{\label{fig:tex2pdf}Conversion of a TEX file into a PDF, DVI, XDV file }
\end{figure}

\section{Bibliographies}\label{sec:bibtex}

For each occurrence 
of a command \cmd{cite} in the TEX file, 
referring to a document with given key,
\texttt{latex2pdf} writes an according entry \cmd{citation} with that key 
into an AUX file. 
Note that, 
if the \LaTeX{} main file includes other TEX files with \cmd{include}, 
and the \cmd{cite}-command is invoked in the included TEX file, 
the \cmd{citation} commands go into the AUX file of that TEX file. 
Moreover, a \cmd{bibliography}-command in the TEX file 
writes a link to the BIB files containing the bibliography data 
into the (top level) AUX file as \cmd{bibdata}. 
Note that \cmd{bibliography} accepts a list of BIB files, not only a single one, 
as maybe suggested by the singular name. 
The key given by \cmd{cite} commands must refer to exactly one key in the BIB files. 
Last not least, a \cmd{bibliographystyle}-command in the TEX file 
writes a link to the bibliography style file 
which determines the appearance of the bibliography 
and also the labels and the ordering 
into the AUX file as \cmd{bibstyle}. 
Typically, the style file comes from the \TeX{} distribution rather than the user. 
Its ending is \gls{bst}. 

To create a bibliography, 
a \texttt{bibtexCommand} must be run after the \LaTeX{} run. 
The default command is the traditional \tool{bibtex}, 
but there are more modern alternatives also supported
like \tool{bibtexu} and \tool{bibtex8} supporting utf8 encoding 
and others. 
Among the tools which are not supported are \tool{biber} and \tool{mlbibtex}. 

We run \tool{bibtex} if either command \cmd{bibliography} or \cmd{bibliographystyle} 
is in the top level AUX file. 
If there is no \cmd{cite}-command, \tool{bibtex} yields an error. 
If neither \cmd{bibliography}-command nor \cmd{bibliographystyle}-command 
are present, then presence of \cmd{cite} yields an error when running the \LaTeX{} engine. 
So, there is an error if not either all three ingredients are present or neither. 

Essentially, \tool{bibtex} extracts the citations in the AUX files, 
unifies them, i.e.\@ a citation is listed once even if it is used more than once, 
retrieves the according entries from the BIB files specified, 
sorts and formats these entries 
according to the \gls{bst} file and writes all into a \gls{bbl} file 
which can be included in the next run of \texttt{latex2pdf}. 
Formatting includes associating a label with each key 
and sorting is based typically on the label. 
The BBL file consists essentially in a \env{thebibliography} environment 
listing the \cmd{bibitem}s. 
These relate the key and the label given by the BST file 
and show the text of the bibliography entry. 

Note that after a \tool{bibtex}-run, 
two \LaTeX{} runs are required: 
The first one just puts the bibliography found in the BBL file 
into the PDF file at place of \cmd{bibliography} 
(which shows why it is singular, although a list of BIB files may serve as source)
and the labels of the citations into the AUX file 
as \cmd{bibcite}-commands. 
The second run places the labels of the citations found in the AUX file 
at the citations given by \cmd{cite}. 
The package \pkg{tocbibind} described in~\cite{TocBibIndP}, 
then writes the headline of the bibliography 
into the table of contents.\index{table of contents}
% The package \pkg{rerunfilecheck} described in~\cite{RerunFChkP110}, 
% ensures that \texttt{latex2pdf} is rerun if needed, 
% provided loaded with option \texttt{aux}. 

% TBD: One could do some postprocessing to bibtex 
% writing 


This software presupposes, that \tool{bibtex} reads the AUX file 
and creates a BBL file and also a BLG file with logging output 
as illustrated by Figure~\ref{fig:aux2bbl}. 
From the BLG file this software may determine 
whether \tool{bibtex} emitted an error or warnings. 


\begin{figure}[htb]
\centering
\IfPackageLoadedTF{tex4ht}{%
should be a picture 
}{
\import{}{F5_02aux2bbl.ptx}
}
\caption{\label{fig:aux2bbl}
Conversion of an AUX file into a BBL file using bibliographies}
\end{figure}

Vital information on \tool{bibtex} can be found in~\cite{BibPat} 
and in~\cite{BibMar}. 
Also,~\cite{Gra}, Chapter 10 is worth reading in this context. 

Note that in the master AUX file one can find also entries \cmd{bibcite} 
relating the labels for bibliography entries to the representations 
to be inserted for the \cmd{cite} commands, 
but it is the \LaTeX{} engine which extracts these mappings 
from the \cmd{bibitem} entries in the BBL file written by \tool{bibtex}. 

The package \pkg{tocbibind} described in~\cite{TocBibIndP}, 
then writes the headline of the index into the table of contents, 
if the option \texttt{numibib} is given.\index{table of contents}


\section{Indices}\label{sec:indices}

Let us first assume that only a single index is wanted. 
For each occurrence 
of a command \cmd{index} or similar (details see below) in the TEX file, 
referring to an entry of the index,
\texttt{latex2pdf} writes an according entry \cmd{indexentry} 
into the \gls{idx} file, provided before the command \cmd{makeindex} was issued. 

In case the \LaTeX{} engine writes index information, 
into its \gls{idx} file, at least one index must be generated. 
Since the \gls{idx} file contains nothing but index information, 
an index is created if and only if the \gls{idx} file is created. 
% Well, this is not really the truth: it is needed only if ...
Essentially, 

the command \cmd{makeindex} tells \texttt{latex2pdf} 
to open the \gls{idx} file for writing. 
Then for each occurrence of the \cmd{index}-command in the TEX file 
specifying an index entry, 
an \cmd{indexentry} command is written into the \gls{idx} file 
comprising the keyword to be written into the index given by the \cmd{index}-command 
and the page number where the \cmd{index}-command occurred. 

 is written to the \gls{idx} file as 
\cmd{indexentry} 
For example \cmd{index\{ant-task\}} in occurring on page 3 creates an entry 
%
\begin{lstlisting}[language=TeX]
\indexentry{ant-task}{3}
\end{lstlisting}
%
in the IDX file. 

Then the \texttt{makeindex}-command is applied to the \gls{idx} file 
which sorts keywords and for each keyword collects the according page numbers, 
sorts it and and writes the result into a \gls{ind} file. 
In the next run of \texttt{latex2pdf}, 
the \cmd{prindindex}-command includes the index as a separate section; 
typically at the end of the PDF file. 
The most basic package to provide this command 
is \pkg{makeidx} described in~\cite{MkidxShIdxP}. 
In addition, \pkg{makeidx} provides the command \cmd{see}{} 
which is for cross-reference within an index. 
The package \pkg{tocbibind} described in~\cite{TocBibIndP}, 
then writes the headline of the index into the table of contents, 
if the option \texttt{numindex} is given.\index{table of contents}
% The package \pkg{rerunfilecheck} described in~\cite{110}, 
% ensures that \texttt{latex2pdf} is rerun if needed, 
% provided loaded with option \texttt{aux}. 

The same document,~\cite{MkidxShIdxP} 
also describes the package \pkg{showidx} 
which prints index entries at the margin of the document. 
This is for debugging only. 
\medskip


The main restriction of the package \pkg{makeidx} is, 
that only a single index can be created. 
The reason is that, \texttt{latex2pdf} creates a single \gls{idx} file 
and, as illustrated in Figure~\ref{fig:idx2ind}, 
\texttt{makeindex} creates a single ind file from that, 
representing a single index. 

To overcome this restriction, 
replace package \pkg{makeidx} and \texttt{makeindex} 
with package \pkg{splitidx} and \texttt{splitindex} 
both described in~\cite{SplitidxP}. 

The package \pkg{splitidx} is used 
in conjunction with the program \texttt{splitindex}. 
It must be possible to create a single index 
without using \pkg{splitidx} and \texttt{splitindex}. **** 

% TBD: clarify whether this option is for splitidx or for makeidx 
Package option \texttt{split} makes \texttt{latex2pdf} 
creating \gls{idx} files \texttt{xxx-y.idx} directly. 
Here \texttt{y} represents the identifier of an individual index. 
These \gls{idx} files can be transformed individually with \texttt{makeindex} 
into ind files as illustrated in Figure~\ref{fig:idx2indMult}. 
Since \texttt{latex2pdf} can keep open only up to 16 output streams, 
not all of which can be used to create a file \texttt{xxx-y.idx}, 
this approach allows a limited number of indices 
and is thus not recommended and not supported. 
% Another reason is, that this approach undermines 
% the package \pkg{rerunfilecheck} described in~\cite{RerunFChkP110}, 
% and so it is not guaranteed that \texttt{latex2pdf} is rerun if needed. 
% This explains why option \texttt{split} is not allowed. 
% **** check? 

Instead, without option \texttt{split}, 
\texttt{latex2pdf} creates a single \gls{idx} file. 
The program \texttt{splitindex} splits it up into several \gls{idx} files 
and applies \texttt{makeindex} to each of them separately 
as illustrated in Figure~\ref{fig:idx2indSplit}. 

For usage of further packages supporting multiple indices 
which are not intended to be used with this software, 
see Chapter~\ref{chap:gaps}. 

This software presupposes, that \texttt{makeindex} converts the \gls{idx} file 
into an ind file containing the index 
and creating also an ilg file with logging output 
as shown in Figure~\ref{fig:idx2ind}. 
From the ilg file this software may determine 
whether \texttt{makeindex} emitted an error or warnings. 

\begin{figure}[htb]
\centering
\IfPackageLoadedTF{tex4ht}{%
should be a picture 
}{
\import{}{F5_03idx2ind.ptx}
}
\caption{\label{fig:idx2ind}Conversion of an \gls{idx} file into an ind file}
\end{figure}

\begin{figure}[htb]
\centering
\IfPackageLoadedTF{tex4ht}{%
should be a picture 
}{
\import{}{F5_04idx2indMult.ptx}
}
\caption{\label{fig:idx2indMult}
Not supported: Conversion of \gls{idx} files into ind files}
\end{figure}

\begin{figure}[htb]
\centering
\IfPackageLoadedTF{tex4ht}{%
should be a picture 
}{
\import{}{F5_05idx2indSplit.ptx}
}
\caption{\label{fig:idx2indSplit}Conversion of an \gls{idx} file into ind files}
\end{figure}

It is possible to configure the makeindex-command 
and to pass arbitrary options. 
CAUTION\@: For the usual \texttt{makeindex}-command, 
the options \texttt{-o} specifying an output file 
and \texttt{-t} (transcript) specifying the logging file are not allowed, 
because this breaks the expectation to find the sorted index 
in file \texttt{xxx.ind} 
and bypasses the detection of errors and warnings of this software, 
respectively. 
Also specifying a style file via option \texttt{-s} 
is not recommended because this is used to create a glossary 
and so breaks glossary creation 
as described in Section~\ref{sec:glossaries}. 

Information on the makeindex program can be found in~\cite{MkIdxMoe} 
and in~\cite{MkIdxLam}. 
Also, there is a site~\cite{MakeIdxOpts} 
describing all available options for \texttt{makeindex}. 

As indicated above, the program \texttt{splitindex} 
invokes \texttt{makeindex}. 
Its options are described in~\cite{SplitidxP}, Section~3.10. 
Since the long option names are not understood in all environments, 
only the short options are recommended. 

Since \texttt{splitindex} must satisfy the interface 
given by Figure~\ref{fig:idx2indSplit}, 
the option \texttt{--help} and its shortcut \texttt{-h} are not allowed. 
Likewise for option \texttt{--version} and its shortcut \texttt{-V}. 
The option \texttt{--makeindex <makeindex>}, resp.~\texttt{-m <makeindex>}, 
is used with the \texttt{makeindex} command used for single indices. 
Thus, this may not be given explicitly but is specified implicitly. 
Also, the option \texttt{--identify <regex>}, resp.~\texttt{-i <regex>} 
must be set implicitly because it must be the same expression 
as used to ***** 
Then splitindex.tlu is not allowed, 
because this has another expression. 

Only allowable seems \texttt{-V}, the shortcut for \texttt{--verbose}. 

Then comes the name of the index file to be processed 
without suffix. 

The program \texttt{splitindex} invokes \texttt{makeindex}. 
The option \texttt{--} coming after the filename, 
indicates that all following options are passed to \texttt{makeindex} 



\section{Glossaries}\label{sec:glossaries}

CAUTION\@: The method described here, 
has at least two severe bugs: 
The number of reruns of the \LaTeX{} engine and also of \texttt{makeglossaries} 
is not guaranteed as a consequence of a bug in \pkg{rerunfilecheck} 
and the fact, that it does not fit current versions of \pkg{makeglossaries}. 
In addition, entries of the glossaries not mentioned directly in the document 
but must be included because they are used in the explanation of entries to be included 
are not treated properly. 

As a consequence, this document, or to be more precise its glossary, 
could not always be reproduced and so the author excluded the glossary until the problem is fixed. 

In addition, it is a conceptual weakness that a glossary data base 
shall be centralized and shall thus not be included in a \LaTeX{} document 
and not even be written in \LaTeX. 
All weaknesses, bugs and conceptual shortcomings are overcome 
by the package \pkg{glossaries-extra} in conjunction with the auxiliary program \texttt{bib2gls} 
which will replace \pkg{glossaries} and \texttt{makeglossaries}. 
For the time being, use glossaries with caution. 
\medskip


Creating glossaries 
requires the package \pkg{glossaries} described in~\cite{GloP4_54}. 
By default, package \pkg{glossaries} creates a single ``main glossary'', 
which can be switched off specifying the option \texttt{nomain} 
described in Section~2.6. 
In this case at least, more specific glossary types with according headline must be specified. 
As specified in~\cite{GloP4_54}, Section~2.6, 
\pkg{glossaries} offers \texttt{acronyms}, \texttt{symbols},
\texttt{numbers} and \texttt{index}. 
To avoid collision with indexing as described in Section~\ref{sec:indices}, 
this software does not allow the latter. 
Moreover, the package \pkg{glossaries} even supports user-defined glossary types, 
but this software does not, 
mainly to keep the internal build in line with build using \tool{latexmk}. 
For details see Section~\ref{sec:gapGlossaries}. 

Also, the package \pkg{glossaries} offers sorting and unifying 
either via \texttt{makeindex} as for indices or via \texttt{xindy}, 
and it offers also to do without external programs. 
In contrast, this software supports only the variant using \texttt{makeindex}. 





As for creating indices there is a \LaTeX-command \cmd{makeindex}, 
to create a glossary there is a \LaTeX-command \cmd{makeglossaries}, 
but the latter is not built-in as \cmd{makeindex} 
but provided by the package \pkg{glossaries}. 
If \texttt{xxx.tex} is the \LaTeX{} main file, 
\cmd{makeglossaries} opens the glo file \texttt{xxx.glo} 
containing glossary entries for writing. 
As the built-in command \cmd{index} 
writes entries into the \gls{idx} file defining the index, 
the command \cmd{gls} defined by the package \pkg{glossaries} 
writes an entry into the glo file. 
Note that \texttt{xxx.glo} typically contains entries more than once 
and that the entries are not sorted. 

To perform sorting, formatting and typically also unification, 
the package \pkg{glossaries} allows three mechanisms. 
This software supports two of them: 
via the shell command \texttt{makeindex}, which is also used for indices, 
and via the shell command \texttt{xindy}. 
Using \texttt{makeindex} is the default but can also be activated through 
\cmd{usepackage[makeindex]\{glossaries\}}. 
Using \texttt{xindy} instead of \texttt{makeindex} is triggered through 
\cmd{usepackage[xindy]\{glossaries\}}. 
Accordingly, for option \texttt{makeindex} the AUX file receives lines 
%
\begin{lstlisting}[language=TeX]
\providecommand\@istfilename[1]{}
\@istfilename{manualLMP.ist}
\end{lstlisting}
%
whereas for option \texttt{xindy}, there are lines 
%
\begin{lstlisting}[language=TeX]
\providecommand\@istfilename[1]{}
\@istfilename{manualLMP.xdy}
...
\providecommand\@xdylanguage[2]{}
\@xdylanguage{main}{english}
\providecommand\@gls@codepage[2]{}
\@gls@codepage{main}{}
\end{lstlisting}



This software neither invokes \texttt{makeindex} nor \texttt{xindy} directly. 
Instead, it invokes the shell command \texttt{makeglossaries}
invoked without file ending  
which determines from the AUX file 
whether to invoke \texttt{makeindex} nor \texttt{xindy}. 
Accordingly, it writes the style definition 
by creating an ist file \texttt{xxx.ist} or an xdy file \texttt{xxx.xdy} 
if \texttt{makeindex} or \texttt{xindy} is specified as package option, 
respectively. 

Seemingly, \texttt{makeglossaries} relies on the AUX file 
to determine whether to invoke \texttt{makeindex} or \texttt{xindy} 
for sorting and unification. 
Then it invokes the according command and writes a LOG file 
with ending \texttt{glg}, 
redirecting the logging output of \texttt{makeindex} or \texttt{xindy} 
adding own output so that a glg file may be written, 
even if e.g.~\texttt{makeindex} is invoked and does not. 
In any case, if the glg file is written, 
\texttt{makeglossaries} writes text matching 
%
\begin{verbatim}
(^\*\*\* unable to execute: )
\end{verbatim}
%
in the glg file if an error occurs, 
no matter whether \texttt{makeindex} or \texttt{xindy} is invoked. 
Possibly, there are cases where an error causes no glg file to be written. 
If no error occurs, a glg file is written 
and if warnings are emitted, 
they either come from \texttt{makeindex} or from \texttt{xindy}. 
Thus warnings may be detected with the patterns 
defined by \texttt{makeindex} and by \texttt{xindy}. 

The style \texttt{list} (which is the default) is set in the form 
%
\begin{lstlisting}[language=TeX]
\usepackage[style=list]{glossaries}
\end{lstlisting}
%
where~\cite{GloP4_54}, Section~13 lists predefined styles. 
So, the style determines the content of the style definition, 
whereas the options \texttt{makeindex} and \texttt{xindy} 
specify the form in which the style is encoded 
and thus the ending of the style file, 
which is either \texttt{ist} or \texttt{xdy}. 

Sorting the glo file, as said above, 
currently is only supported using the command \texttt{makeglossaries}. 
The allowed options are essentially those 
making sense for \texttt{makeindex} and those making sense for \texttt{xindy}. 
If the shell command \texttt{makeglossaries} 
invokes \texttt{makeindex} of course only the according options 
are passed supplemented by additional options 
\texttt{-s}, \texttt{-t}, \texttt{-o}, to specify the
ist file, the glg file (the transcript file) and the gls file,
respectively, 
which is the result of sorting, the output file, 
and contains the entries of the glo file 
just sorted, formatted and unified.
So for a tex main file \texttt{xxx.tex} the program 
\texttt{makeglossaries} invokes
%
\begin{verbatim}
makeindex  -s "xxx.ist" -t "xxx.glg" -o "xxx.gls" "xxx.glo"
\end{verbatim}
%
Accordingly, if the shell command \texttt{makeglossaries} 
invokes \texttt{xindy} of course only the according options 
are passed supplemented by additional options 
\texttt{-M}, \texttt{-t}, \texttt{-o}. 
This is illustrated in Figure~\ref{fig:glo2gls}. 


\begin{figure}[htb]
\centering
\IfPackageLoadedTF{tex4ht}{%
should be a picture 
}{
\import{}{F5_06glo2gls.ptx}
}
\caption{\label{fig:glo2gls}Conversion of a glo file into a gls file 
using \texttt{makeglossaries}}
\end{figure}


\section{Including code via \texttt{pythontex}}\label{sec:pythontex}

The package \pkg{pythontex}, described in~\cite{PythonTexP} 
originally allowed including Python code into a latex document. 
Later on, further languages were added, most notably octave or Matlab, 
and the user can easily extend it to further languages 
as sketched in~\cite{PythonTexP}. Section 7. 
Of course, to that end, the interpreter for the desired language must be installed.\index{pythontex}
The meaning of the term ``including'' used above 
ranges from mere listing to pure execution and comprises also inserting results of execution. 
A field of application is also creating figures. 

Note that like the package \pkg{splitindex}, also \pkg{pythonindex} 
comes with an according auxiliary program, 
in this case, besides \texttt{pythontex} also \texttt{depythontex}. 
Consequently,~\cite{PythonTexP} is not only on the package 
but also on the corresponding command line tools. 
Since~\cite{PythonTexP} is quite detailed, 
there is an introduction~\cite{PythonTexQ} and a gallery~\cite{PythonTexG}. 
For background on the intentions of package \pkg{pythontex}, consult~\cite{PythonTexRepr}. 
Information required to integrate pythontex into this software 
partially goes much beyond the official documentation and is collected in~\cite{PyTexInOut}. 
It could also be interesting for the user for debugging. 

Running the \LaTeX{} engine on a file \texttt{xxx.tex} 
with package \pkg{pythontex} loaded 
yields a file \texttt{xxx.pytxcode} 
and if the package is loaded with option \texttt{depythontex} 
also a file \texttt{xxx.depytx}.
If the file \texttt{xxx.pytxcode} is present, 
this software invokes the command line tool \texttt{pythontex} 
(same name as the according package) 
to \texttt{xxx.pytxcode} (without ending) 
which converts this into a variety of output files, 
which are, without further configuration, 
all in the folder \texttt{pythontex-files-xxx}
as shown in Figure~\ref{fig:py2dir}, 
which is described in more detail in~\cite{PyTexInOut}, Section~3. 
Note that this software uses the wrapper \texttt{pythontexW}\index{pythontexW} 
of \texttt{pythontex} described in Section~\ref{subsec:pythontexW}, 
instead of \texttt{pythontex} itself. 
The figure reflects this. 

% TBD: later on, if \texttt{xxx.depytx} after \texttt{pythontex} also \texttt{pythontex} is invoked, 
% creating a TEX file without python code. 
% TBD: clarify what if input files or include files are present. 

Running the \LaTeX{} engine again, 
includes all the output files \texttt{*.stdout} 
in the PDF file or whatever output file created. 

% Note that if depythontex is invoked, it is immaterial whether the subsequent run of LaTeX-to-pdf converter 
% is on the original TEX file or on the TEX file created by depythontex. 

An important remark is that \lualatex{} is the preferred engine, 
because files \texttt{*.stdout} can impose heavy memory usage 
and currently \lualatex{} is the only engine allocating memory dynamically. 

As one can see, \texttt{pythontex} cooperates with \lualatex{} in a way 
also \texttt{bibtex} or the other auxiliary programs do. 
Although \texttt{pythontex}, at time of this writing in version 0.18, 
is quite mature, it refrains from writing a log file and indicates errors and warnings 
just on standard output or error output. 
This is unlike all the other auxiliary programs in a line with \texttt{pythontex}. 
As a consequence, in particular warnings are difficult to detect 
and cannot be detected in a uniform way. 
Thus, the author wrote a little wrapper, called \texttt{pythontexW} 
and place it where it can be found, e.g.~in the folder of \texttt{pythontex}. 

Accordingly, \texttt{depythontex} behaves in a non-standard way: 
Firstly, by default, it does not output a result file but outputs on standard output. 
This can be changed using the option \texttt{--output} or \texttt{-o} for short. 
Also, \texttt{depythontex} changes into interactive mode 
if the output file is already present. 
To avoid this, the option \texttt{--overwrite} is required. 
Overwriting without asking is the standard behavior of all other auxiliary programs. 
As \texttt{pythontex} also \texttt{depythontex} does not write a log file 
but just prints its errors and warnings. 
Thus, the author wrote a little wrapper, 
called \texttt{depythontexW} and described in Section~\ref{subsec:pythontexW}, 
and place it where it can be found, e.g.~in the folder of \texttt{depythontex}\index{depythontexW}. 
% TBD: evaluate: maybe it is a better solution to enable this software itself to write the output to a log file. 
\medskip


The package \pkg{pythontex} and the according auxiliary programs are highly configurable, 
more than this software allows. 

In particular, in the \LaTeX{} document, 
the commands \cmd{setpythontexoutputdir} setting the output directory 
and \cmd{setpythontexworkingdir} setting the working directory shall not be used, 
because this software assumes the default, that the working directory is the directory 
containing the \LaTeX{} main file \texttt{xxx.tex}
and the output directory is in the working directory 
and its name is \texttt{pythontex-files-xxx}. 

Further, the package \pkg{pythontex} can be configured with package options when loading the package. 
Since this software is designed for reproducibility, 
most appropriate would be to specify \texttt{runall=true} meaning that even if no python code is modified 
the auxiliary program \texttt{pythontex} executes the python code in the document. 
Also, it is appropriate to specify \texttt{rerun=always}. 
Note that the defaults are \texttt{runall=false} and \texttt{rerun=errors}. 
This behavior makes sense to speed up creation of the document, 
but it differs from the behavior of all other auxiliary programs 
and causes the check for update of output files to fail. 
Moreover, reproducibility is not as easily shown. 

The package documentation~\cite{PythonTexP} suggests, 
that this makes a difference between \texttt{runall=true/false} 
and \texttt{rerun=always/errors} if external sources are modified, 
but as is proved in~\cite{PyTexInOut}, Section 2.1, 
the package translates package option \texttt{runall=true/false} into key value pair \texttt{rerun=always/errors} 
and this is the only information \texttt{pythontex} obtains from the package, 
so there is no difference. 

Also, the auxiliary program \texttt{pythontex} itself can be configured via command line arguments. 
For the package options \texttt{runall} and \texttt{rerun}, 
there are according command line options \texttt{-{}-runall} and \texttt{-{}-rerun} with the same scope. 
Whereas the package merges options \texttt{runall} and \texttt{rerun} silently, 
the auxiliary program \texttt{pythontex} emits an error, if both are combined. 
Essentially one can forget about \texttt{runall} and stick to \texttt{rerun}. 

Strange enough, according to~\cite{PythonTexP}, Section 4.1, package options overwrite command line options. 
This software shall invoke \texttt{pythontex} 
with the option \texttt{-{}-rerun=always} which is thus specified as the default. 
To force unconditional update, this is not sufficient. 
Instead, this software relies on an undocumented feature of auxiliary program \texttt{pythontex} 
which is likely not to change: 
If one of the expected output files is missing, it recreates all output files, independent of command line options and package options. 
Thus, this software deletes one output file if present, before executing \texttt{pythontex}. 

When this software invokes \texttt{pythontex} 
the exit codes may not be changed via \texttt{-{}-error-exit-code}, 
i.e.~if specified then with value \texttt{true}. 
Neither the options \texttt{-{}-interactive}, \texttt{-h}, \texttt{-{}-help} or \texttt{-{}-version} are allowed. 
Currently, this software does not check for options which are not allowed. 
Fortunately, the latter two command line options have no counterpart in the package configuration. 




If we place some code, e.g.~python code as inline code using \cmd{pyc}
%
\begin{lstlisting}[basicstyle=\footnotesize]
  \usepackage[depythontex]{pythontex}
  ...
  \pyc|print(rf'Python inside latex says: "Hello World; 1+1={1+1}"')|
\end{lstlisting}
%
the code is really evaluated, and the string result is included at proper place 
as illustrated by the following text which is created by python: 
%
\begin{quote}
  \texttt{\pyc|print(rf'Python inside latex says: "Hello World; 1+1={1+1}"')|}. % chktex 26  chktex 18 chktex 36
\end{quote}
%
Note that the typewriter font is not created by python, 
it is explicitly set to highlight the string created by python, 
but it is python which evaluates the little computation 
and which prints the string. 

Since \texttt{pythontex} is written in python, 
including python code in the \LaTeX{} document 
uses the python interpreter already installed, as a prerequisite of \texttt{pythontex}. 
To use another language, the according interpreter must be installed in addition to python. 



% TBD: create a figure 


% TBD: depythontex


\begin{figure}[!htb]
  \centering
  \IfPackageLoadedTF{tex4ht}{%
  should be a picture 
  }{
  \import{}{F5_07py2dir.ptx}
  }
  \caption{\label{fig:py2dir}Conversion of a \texttt{pytxcode} file using \texttt{pythontex}}
  \end{figure}


  Figure~\ref{fig:depy2out} shows the files converted by \texttt{depythontex}. 
  As for \texttt{depythontex}, this software uses the wrapper \texttt{depythontexW} 
  of \texttt{depythontex} instead of \texttt{depythontex} itself. 
  This is reflected in the figure. 


  \begin{figure}[!htb]
    \centering
    \IfPackageLoadedTF{tex4ht}{%
    should be a picture 
    }{
    \import{}{F5_08depy2out.ptx}
    }
    \caption{\label{fig:depy2out}Conversion of a \texttt{depytx} file using \texttt{depythontex}}
    \end{figure}
  
\section{Running and rerunning auxiliary programs}%
\label{sec:runRerunAux}

After describing the interface 
between the \LaTeX{} engine and the auxiliary programs 
in Section~\ref{subsec:latexAux}, 
Section~\ref{subsec:noRerunfilecheck} explains 
why we don't use the package \pkg{rerunfilecheck} 
to determine when to (re-) run auxiliary programs. 


\subsection{The interface between \LaTeX{} and auxiliary programs}%
\label{subsec:latexAux}

Auxiliary programs perform tasks which \LaTeX{} cannot carry out at all 
or only with bad performance, 
for example adding bibliographies which comprises sorting 
or executing program code. 

The interface between the \LaTeX{} engine and an auxiliary program 
is always implemented via files: 
In the first run, the \LaTeX{} engine writes a file or files 
specific for the auxiliary program 
or at least writes entries specific for the auxiliary program 
in a standard file or even both. 
Then the auxiliary program is run which creates other files 
which in turn must be read back, in a second run of the \LaTeX{} engine. 
So the run of an auxiliary program 
is always enclosed between two runs of a \LaTeX{} engine. 

Typically, the \LaTeX{} run needs a \LaTeX{} package 
associated with the auxiliary tool 
which performs reading and writing. 
An exception is \tool{bibtex} and friends 
for which \LaTeX{} engines support communication out of the box. 
An example with more complicated communication 
is \tool{makeglossaries} with associated package \pkg{makeglossaries} 
which writes lines into the AUX file 
and which typically writes the main glossary into a GLO file. 
The tool \tool{makeglossaries} which is invoked without ending, 
reads the AUX file, determines which other files to read, 
typically the GLO file also 
and writes the result into the GLS file. 
This is read back by the package \pkg{makeglossaries} 
in the next run of the \LaTeX{} engine. 


\subsection{When running an auxiliary program}\label{subsec:firstRunAux}

After the first run of the \LaTeX{} engine, 
one must decide which auxiliary programs to run. 
For each auxiliary program, there is a specific file it reads 
or at least specific entries in a general file, typically the AUX file. 
If this file or these entries exist, the auxiliary program must be run 
and after the \LaTeX{} engine must be rerun 
to read in the data created by the auxiliary program. 
As is discussed for each auxiliary program separately 
in Section~\ref{subsec:noRerunfilecheck}, 
this file or these entries may change after each run of the \LaTeX{} engine 
and as a result, the auxiliary program must be rerun as well. 
So, \LaTeX{} engine and auxiliary program maybe must be run alternately. 

Instead of checking whether the relevant data really changes, 
only the number of relevant lines and a hash is taken into account. 
% TBD: clarify: Does it really have advantages to take the number of lines into account? 
% latexmk uses the length instead. 
% is it advantageous to add length as additional identifier 
% or instead number of lines? 
% Maybe some discussion is in place 
This bears a minimal risk of not rerunning the auxiliary program although needed. 
Note that also package \pkg{rerunfilecheck} is based on hashes and bears the same risk. 
% TBD: could improve just compressing the entries losslessly. 
% Essentially, there is a whole spectrum between hashes with high overlap 
% and hashes with no overlap, i.e.\@ lossless. 
% Maybe even Hamming distance plays a role. 

It is an interesting detail, 
that deciding whether an auxiliary program must be run at all, 
i.e.\@ for the first time, 
is just based on the existence of a specific file 
or of a specific line in a file, 
not comprising all pieces of information read by the auxiliary program. 
Nevertheless, if it is decided that the auxiliary program must be run, 
it is clear that the \LaTeX{} engine must be run after also 
and so the information may change. 
So one must be prepared for a rerun check. 
For this, all the information in the file(s) %chktex 36
relevant for the auxiliary program must be hashed. 

From the second run of the \LaTeX{} engine on, 
only those auxiliary programs must be checked for rerun condition, 
for which a hash is present. 
\medskip

After these quite abstract considerations, 
let us apply these to the concrete auxiliary programs supported. 





% TBD: \pkg in headlines 
\subsection{Why \texttt{rerunfilecheck} is not used for auxiliary programs}%
\label{subsec:noRerunfilecheck}

As described in Section~\ref{sec:prerequisites}, 
package \pkg{rerunfilecheck} is used to check 
whether the \LaTeX{} engine must be rerun, 
and its authors also intended it to check 
for need of rerun of auxiliary programs. 
While this works satisfactory for a single index, 
it fails for multiple indices. 
Likewise, support for glossaries is buggy and works only in case of a single glossary, 
which in addition must be the main glossary. 
In contrast, the package \pkg{glossaries} supports multiple glossaries, 
with and without main glossary 
and even allows user-defined glossaries. 
It is awkward to implement rerun check 
for all this functionality with \pkg{rerunfilecheck}. 
% TBD: link to discussion for rerun check on bibliography and others.. 

It may be surprising, that there are situations 
where even bibliography processors need to be rerun, 
among these backlinks, and citations in headlines and glossaries. 
Package \pkg{rerunfilecheck} does not take this into account. 
Accordingly, even \texttt{pythontex} may need a rerun, 
e.g.\@ if code is executed in headlines or in captions of floating objects, 
because this may insert additional invocations and may change invocation order 
which may lead to different results. 

While many auxiliary programs depend only on a subset of entries 
in their source file, 
\pkg{rerunfilecheck} can take files into account only as a whole. 
As a consequence, even if no rerun is required 
because the relevant entries did not change, 
\pkg{rerunfilecheck} could trigger useless rerun, 
because irrelevant entries in the relevant file changed. 

Tanking all these aspects into account, 
we decided to provide an internal algorithm for rerun check of auxiliary programs, 
which is based on the ideas of \pkg{rerunfilecheck} 
but avoiding all its shortcomings. 

Note also, that besides whether to rerun an auxiliary program, 
there is also the question in which case to run it at all, i.e.\@ for a first time. 
Since package \pkg{rerunfilecheck} interprets a newly occurring file 
as a changed file, this case is addressed implicitly. 

Unfortunately, not all packages associated with auxiliary tools 
give a hint if the auxiliary program must be run. 
% TBD: talk with the maintainers 
\medskip


As described in Section~\ref{sec:tex2pdf}, 
running a \LaTeX{} engine as \texttt{latex2pdf} 
may detect the presence of a bibliography, an index and/or of a glossary 
and writes raw files to describe them. 
After that, an intermediate step is required, 
sorting, unifying and formatting the entries. 
This is always done by an external program, we call an auxiliary program. 
Similarly, the presence of code to be interpreted 
may be detected which is also written in a separate file 
and an external program, 
\tool{pythontex} must be run to run the code in sequence 
and in many cases to determine the result of invocation. 

In the next step, 
the \LaTeX{} processor must read in the results of the auxiliary programs again 
to write bibliography, indices and glossaries 
and to insert the results of code invocations. 
Also, except the code invocations, 
all other pieces of information typically go into the table of contents. 
If code is invoked in a headline or in a caption, 
the result of the code invocation goes into the TOC and in the list of captions, 
e.g.\@ the list of figures LOF also. 
So in any case, after an auxiliary program the \LaTeX{} processor must be rerun. 

Obviously, the run of a \LaTeX{} processor may change page numbers 
and thus invalidate the index or the glossary. 
So the auxiliary program to create the index or the glossary must be rerun 
if the \LaTeX{} processor changes the input file for the auxiliary program 
creating index or glossary and after that, 
the \LaTeX{} processor must be run again. 

What is less obvious is, that bibliographies may be invalidated also, 
e.g.\@ because of a backlink 
or because a bibliographic reference occurs in a glossary. 
Even code may be invalidated by a run of the \LaTeX{} processor 
if some code occurs in a floating object, e.g.\@ in the caption or in a glossary. 
So code invocations may change order 
and also there may be additional code 
occurring not before later runs of the \LaTeX{} processor. 
So also in this case, the according auxiliary program, \tool{pythontex} 
must be rerun after the run of the \LaTeX{} processor. 

Summarizing, a run of the \LaTeX{} processor 
may trigger invocation of each auxiliary program. 
This must be done if the according raw file changes. 
Note that various auxiliary programs share the AUX file to get information. 
So only the aspects relevant for the specific auxiliary program 
shall be taken into account. 
What makes things a bit more complicated is, 
that including TEX files yields included AUX files 
which must be taken into account also. 

To implement rerun check completely reliable, 
huge parts of text files, a lot of information must be stored. 
Thus, we go a way like package \pkg{rerunfilecheck}, 
detecting only the change of number of relevant lines and the according hash. 
In extremely rare cases, this software may fail to rerun a program although needed, 
because number of relevant lines or its hash don't change 
although contents change. 

Note that we only use the concept of \pkg{rerunfilecheck} 
to detect running and rerunning auxiliary programs, 
but we do not use the package \pkg{rerunfilecheck} itself for this task. 
This is because supporting all relevant auxiliary programs and also 
included AUX files would require considerable extensions on \pkg{rerunfilecheck} 
and would impact considerable dependencies. 
So, as described in Section~\ref{sec:rerunLatex}, 
\pkg{rerunfilecheck} is used to control rerunning the \LaTeX{} processor 
as far as auxiliary programs are not involved, 
whereas detecting auxiliary programs to be rerun is done internally 
while the algorithm is inspired by the package \pkg{rerunfilecheck}. 


\section{Rerunning the \LaTeX{} processor}\label{sec:rerunLatex}

CAUTION\: rework needed 

FIXME\@: a word on change in toc, lof, lot and lol. 

As indicated in the previous sections, 
\texttt{latex2pdf} must be rerun, 
if an auxiliary program like \texttt{bibtex}, \texttt{makeindex} 
or \texttt{makeglossaries} 
had been run. 

Likewise, if a toc file, a lof file, a lot file or a lol file
had been created in the first \texttt{latex2pdf} run, 
another run is needed to read in these files 
to create a table of contents, a list of figures or a list of tables, 
respectively. 
Note that for all these cases, 
the LOG file does not allow to detect that \texttt{latex2pdf} has to be rerun, 
by matching a fixed pattern. 

After the second run of \texttt{latex2pdf}, 
the table of contents,
the list of figures, the list of tables and the list of listings 
are included and a section with the bibliography, 
the index and the glossary are inserted. 
It takes a third run of \texttt{latex2pdf} 
to include the bibliography the index and the glossary 
into the table of contents. 
Also, it takes that third run to replace the citations 
with the proper labels given in the bibliography. 

Inserting the table of contents,
the list of figures, the list of tables and the list of listings 
may shift the subsequent text 
which may require another run of \texttt{latex2pdf} 
to get the page numbers right. 
As described in Section~\ref{sec:runRerunAux} 
intermediate runs of auxiliary programs like \texttt{makeindex} 
may be required 
and these also require another run of \texttt{latex2pdf} 
also to get the page numbers right. 

The package \pkg{rerunfilecheck} allows detecting file changes via a hash 
almost for sure, and writes an according message into the LOG file. 
This is offered for pure rerun control of \texttt{latex2pdf} 
based on TOC, LOL, LOF and LOT, but also on the OUT file written by package \pkg{hyperref}. 
Partially, it supports also the need to rerun auxiliary programs, 
but for sake of uniformity, we refrain from using this, 
and rely on in internal algorithm also based on hashes. 

Only for rerunning \texttt{latex2pdf} alone, we rely on package \pkg{rerunfilecheck}. 
This software just reruns texttt{latex2pdf} 
if it detects the pattern of warning written by \pkg{rerunfilecheck} into the LOG file. 



Note that there are several packages which require additional runs, 
such as the package \pkg{longtable}, 
which may vary dimensions of tables. 
This software presupposes, that all these reruns 
may be detected by matching a fixed pattern in the LOG file. 
Since packages are frequently changed and new packages are written, 
also the pattern cannot be fixed. 
Thus, it is configurable. 
 
Note that, if a package requires running other programs 
between two runs of \texttt{latex2pdf}, 
this may require a change in this software. 

\section{Checking reproducibility}\label{sec:chkReprod}

There are use cases, where it is extremely important 
that the according artifacts are really reproducible. 
One is when we have to deliver the sources 
and the receiver has to reconstruct the artifacts. 
Another obvious use case is integration test for this software 
by ensuring that each artifact created 
is equivalent with a confirmed version, 
although this software changed. 
Details are given in Section~\ref{chap:tests}. 


Currently, reproducibility checks are supported for PDF files only. 
% TBD: change that 
The problem with PDF files is, that besides visible contents 
they contain also metadata (see~\cite{Pdf17} or~\cite{Pdf20}, each Section 14.3), 
which depends on the run of the conversion. 
For example the timestamp and the timezone of conversion goes into 
and derived from these other values. 

There are two strategies to deal with the problem: 
%
\begin{itemize}
  \item 
  Make the build process reproducible. 
  The advantage of this approach is that diffing is quite simple, 
  fast and reproducible: it is byte by byte. 
  This is easily done with a fixed installation 
  but tends to break with update of tools. 
  % TBD: allow own version check. 
  % Also, at time of this writing, 
  % the different latex engines cannot be treated uniformly. 
  % TBD: a feature request to hyperref is already posted. 
  % let us keep an eye on that. 
  \item 
  Use diff tools implementing a weaker notion of equivalence, 
  in a sense visibility equivalence of some degree. 
  One approach is the script \texttt{vmdiff} 
  described in Section~\ref{subsec:pythontexW} 
  which combines visibility equivalence 
  with equivalence of part of metadata. 
\end{itemize}%

Since the first one works very well, it is the one we describe here, 
but it is always possible to configure a diff tool with a weaker equivalence check. 

The first question is, whether reproducibility is requested. 
It is, if there is according magic comment in the \LaTeX{} main file requires this 
as described in Section~\ref{subsubsec:openingMagComm}. 
If there is no such magic comment is present, if the setting \texttt{chkDiff} specifies so. 
If in this section settings are given without explicit reference, 
they are described in Table~\ref{tab:paramDiffPdf} on page~\pageref{tab:paramDiffPdf} 
in Section~\ref{sec:paramRepro}. 

Since date and time both visible and in the metadata of a PDF document 
is given relative to a timezone, 
for reproducible builds compilers must run with a fixed timezone 
and, as reproducibility shall not break if changing a timezone 
or if the country running the build changes between daylight saving time and standard time, 
we chose a uniform timezone namely UTC\@. 

If a \LaTeX{} main file is already under reproducibility control, 
then there is an according original PDF file in \texttt{diffDirectory} or in a subfolder 
to be compared with a newly created PDF file 
which occurs in a subfolder of the TEX source directory \texttt{texSrcDirectory} 
described in Table~\ref{tab:paramGen} on page~\pageref{tab:paramGen}. 
The PDF file for comparison has the same path relative to \texttt{diffDirectory} 
as the created PDF file relative to \texttt{texSrcDirectory}. 

First \texttt{pdfMetainfoCommand} is used 
to extract metadata \texttt{CreationTime} from the original PDF file. 
This comprises time and timezone which is UTC\@. 

The compilation to create the new PDF file is run in an environment 
with that timezone and with that creation time. 
In addition, there is an environment variable forcing 
that the timestamp does not only affect metadata but also visual data of the PDF file 
to be created, 
as e.g.\@ typically the date at the front page. 
Note that if the PDF file is created from TEX files via DVI/XDV files, 
both engines need the appropriate environment. 

After creating the new PDF file with this environment, 
coincidence with the original PDF file is checked 
using the tool given by setting \texttt{diffPdfCommand} described in Table~\ref{tab:paramDiffPdf}. 
If the actual artifact does not coincide with predefined one 
according to the chosen diff tool, 
a build exception is thrown as specified in Table~\ref{tab:TLP}. 
\medskip


If a \LaTeX{} main file is not already under reproducibility control, 
then no original PDF file exists. 
In this case, the environment for compilation only ensures the timezone UTC\@. 
Then the created PDF file is copies at proper place into \texttt{diffDirectory} 
-- that's all for setting a document under reproducibility control. 

Finally, if a \LaTeX{} main f8ile file is under reproducibility control 
but is to be changed in a way that also the according PDF file is affected, 
then before compilation just the original PDF file is deleted, 
and the workflow is as setting under reproducibility control. 
\medskip


Reproducibility is affected or even supported by various injections 
as defined in Section~\ref{sec:injFiles}. 
First, the generic header described in Section~\ref{subsec:header} 
affects metadata, above all because it loads the package \pkg{hyperref}. 
Part of this metadata is overwritten by another header 
described in Section~\ref{subsec:headerSuppressMetaPDF}, 
to improve security and privacy, 
but enough metadata remains to keep up reproducibility. 
Reproducibility is guaranteed with the full set of metadata 
or with somehow reduced metadata. 
The only piece of information needed for reproducibility is \texttt{CreationDate} 
and this is preserved by the headers. 
Removing this also has severe consequences 
so that we can assume it is preserved. 
On the other hand, removing metadata may stabilize reproducibility 
as this is true for the banner which identifies the latex compiler and its version 
and consequently breaks reproducibility in any version change. 
Details to reproducibility with a focus on metadata are given in~\cite{LatexGen}, Section 4. 

Obviously, reproducibility checks cause work 
when putting a document under check, 
i.e.\@ in the end phase of document development 
as defined in Section~\ref{sec:devel}
or if the source document changes, i.e.\@ if document development is entered again, 
or if the output PDF changes unintended 
normally, although the sources did not change in an obvious way, 
which triggers again document development searching the cause of the change in the sources. 

This \LaTeX{} builder is not the tool for document development. 
Instead, Section~\ref{subsec:develLatexmk} suggests to use \tool{latexmk} for, 
and describes how \tool{latexmk} is integrated in this \LaTeX{} builder: 
This builder writes a config file \texttt{.latexmkrc} 
reflecting the settings of this software, at least to some extent. 
The config file \texttt{.latexmkrc} is again written as an injection 
and is described in Section~\ref{subsec:latChkRc}. 
It supports reproducibility checks even reading magic comments, 
checking existence of original PDF file 
and reading its timestamp if the PDF file is present. 
Creation of the new PDF file takes timestamp and timezone into account. 

Two further injections may be helpful in the context of reproducibility checks, 
both described in Section~\ref{subsec:ntlatexVmdiff}: 
\tool{ntlatex} to create a PDF file and \tool{vmdiff} 
realizing a weaker variant of diffing tool as described above: 
It checks for visual equality and equality of metadata. 
% TBD: maybe not true: only in trailer directory of the PDF. 
\medskip


For updating metadata only, we suggest the following technique: 
Keep the original PDF file in \texttt{diffDirectory} 
and check with \tool{vmdiff} that visually, the PDf file remains the same 
and that the correct metadata is updated. 
Of course, a new timestamp is wanted. 
So in a second step, the original PDF file is deleted, 
compilation is repeated, e.g.\@ by \tool{ntlatex} and copied into \texttt{diffDirectory}. 
\medskip


There are rare occasions where the timestamp shall be set explicitly. 
This is not possible directly as it is read off from the original PDF file. 
We suggest to use \texttt{exiftool} to modify the \texttt{CreationDate} 
of the original PDF file in \texttt{diffDirectory} before compilation. 
This is done by something like
%
\begin{verbatim}
  exiftool -PDF:CreateDate=2020-01-01T00:01:02Z xxx.pdf 
\end{verbatim}
%
Here, the option \texttt{PDF:CreateDate} is in fact the name of the tag to be written. 
Note that the timezone must be UTC represented by the \texttt Z 
signifying zero time offset compared to UTC\@. 
The attentive reader may wonder why the option is \texttt{PDF:CreateDate} instead of \texttt{CreationDate}. 
One may check with \texttt{pdfinfo}, that really \texttt{CreationDate} is modified. 
Note that \tool{exiftool} writes the original PDF file into \texttt{xxx.pdf\_original}

Two important details are not so obvious: 
%
\begin{itemize}
\item
Not only the given metadata is changed but also all metadata depending on it, 
in this case the trailer ID\@. 
This is to keep the PDF file consistent. 
\item 
The metadata is not really overwritten, but it is hidden by new metadata. 
In fact, \texttt{exiftool} uses incremental update specified for the PDF format, 
adding a layer describing the modification. 
All modifications done can also be undone by 
%
\begin{verbatim}
  exiftool -PDF-update:All= xxx.pdf
\end{verbatim}
%
unless the PDF file has been linearized. 
\LaTeX{} to PDF compilers always create linearized PDF files and never update incrementally. 
\end{itemize}


To know that changing metadata is done by incremental update is important, 
insofar as a PDF file with modified timestamp and timezone  
differs from a PDF file compiled directly with the given timestamp and timezone; 
it is shorter. 
So, updating the timestamp of the PDF file in \texttt{diffDirectory} 
does not yield a PDF file which is reproduced. 
Compilation leads to another PDF file and only the updated timestamp is reproduced. 
This compiled PDF file is reproduced, so 
copying it the into \texttt{diffDirectory} solves the problem: 
Next compilation yields a PDF file with the correct timestamp and timezone, 
and it coincides with the PDF file in \texttt{diffDirectory}. 

When subjecting a document under reproduction control with a predefined timestamp, 
then initially there is no original PDF file. 
One could place any PDF file in \texttt{diffDirectory}, 
overwrite the timestamp and timezone by \tool{exiftool}. 
Is content is immaterial. 




\section{Alternative build process with \protect\tool{latexmk}}\label{sec:latexmk}%\protect\tool{latexmk}
\index{latexmk}

This section is on running the build process of \LaTeX{} main files 
with \tool{latexmk} or equivalent. 
Currently, that way only PDF files can be created. 
% TBD: at least DVI/XDV is possible as well 
% and also chk are possible. 
% what if magic comment specifies another target but a build using latexmk? 
% This shall be a warning. 
% currently, just setting latexmk is ignored. 
Although the functionality is readily explained, 
the intention is not so obvious: 
In Section~\ref{subsec:develLatexmk} 
describes the role of \tool{latexmk} as a build tool 
in the course of document development, 
whereas this \LaTeX{} builder is for final, quality checked build. 
So the two tools seem to be complementary. 
Section~\ref{subsec:latChkRc} describes that this \LaTeX{} builder 
can write its own configuration as 
a config file \texttt{.latexmkrc} for \tool{latexmk} 
so that builds with \tool{latexmk} are in line 
with final builds by this \LaTeX{} builder itself internally. 

So running \tool{latexmk} from within this \LaTeX{} builder 
seems superfluous at first sight. 
A closer look onto \texttt{.latexmkrc} unveils that this is just a Perl script 
which is very flexible realizing new or special functionality, 
whereas this \LaTeX{} builder is tied to a quite rigid configuration in the pom. 
So, for example if for building a document tools are needed 
which are not supported by this \LaTeX{} builder, 
their invocation can be implemented directly in \texttt{.latexmkrc}. 
Since this \LaTeX{} builder writes a single \texttt{.latexmkrc} 
in the root directory \texttt{texSrcDirectory}, 
which must be made available in each subfolder by adding a link, 
the config \texttt{.latexmkrc} by this \LaTeX{} builder 
may be replaced by a hand-crafted config file for each folder separately. 

Another advantage being able to run \tool{latexmk} from within this builder: 
It is conceivable, that the artifacts created in the course of document development 
using \tool{latexmk} cannot be reproduced by this builder. 
Most likely because \texttt{.latexmkrc} does not reimplement the internal functionality properly. 
Invoking \tool{latexmk} in a final build reduces this risk to a minimum. 

Further motivations for integrating latexmk in this builder, 
in particular for individual files: 
there are cases where the build process of latexmk works, 
but not the internal build process of this builder. 
Integrating latexmk offers the strengths of latexmk. 
Note that there are also cases 
where the built-in build process of this builder 
is mightier than that of latexmk. 
Another reason for integrating latexmk here, 
is the use case of source distribution: 
The document(s) may be passed to someone as the source, % chktex 36
not as a target, like PDF\@. 
It is not clear that the ``customer'' uses this latex builder, 
but maybe (s)he uses latexmk. % chktex 36
In this case it makes sense to check, 
whether the document can be built with latexmk alone. 

Having explained this, the question arises 
why this \LaTeX{} builder does not in general rely on \tool{latexmk} 
and invokes \LaTeX{} engines and other converters directly. 
One reason is that \LaTeX{} builder does not only invoke converters, 
it also checks return values and, depending on the converter, 
log files emitting errors and warnings if appropriate. 
So, delegating to \tool{latexmk} 
the user can no longer check that the build process passed without warning or error. 
A second aspect is, that the build algorithms differ: 
\tool{latexmk} runs the \LaTeX{} main file then detecting which files are missing 
and then tries to build these based on rules. 
The basic idea behind is ``backward discovery'' of dependencies, 
whereas this \LaTeX{} builder first builds the graphic files globally 
(\tool{latexmk} detects last) 
before for each \LaTeX{} main file is compiled. 
So this \LaTeX{} builder combines ``forward discovery'' and backwards discovery. 
Pure backward discovery is more elegant 
but as the \LaTeX{} compiler stops at each graphic file not present 
before creating it and rerunning compilation of the \LaTeX{} main file, 
it may result in excessive reruns of the \LaTeX{} engine 
if there are many created graphics in the document. 

So there are strong reasons to avoid \tool{latexmk}, 
but there are also reasons to allow in special cases. 
The parameter \texttt{\$latexmkUsage} described in Table~\ref{tab:paramGen} 
on page~\pageref{tab:paramGen} allows gradually use of \tool{latexmk}, 
not at all, fully or as backend where \tool{latexmk} is invoked 
after graphic files have been created with an internal process. 
As a rule, \tool{latexmk} shall be used as much as required and as little as possible. 

This shows also, that it is a good thing 
to be able to activate \tool{latexmk} in individual \LaTeX{} main files 
which is realized with the magic comment \texttt{latexmk}. 
It can take the form \texttt{latexmk=false}, \texttt{latexmk=true} or just \texttt{latexmk} 
which is the short form of the latter. 
Magic comments are described in Section~\ref{subsubsec:openingMagComm}. 
In general, they overwrite settings. 
Here, the situation is a bit more complicated. 
Whereas \texttt{\$latexmkUsage} allows three levels of usage, 
the magic comment can choose to use \tool{latexmk} or not. 
If \tool{latexmk} shall be used due to the magic comment, 
then it is used to compile the TEX file in any case, 
but it compiles graphic files only, if \texttt{\$latexmkUsage} takes the value \texttt{NotAtAll}. 
If \tool{latexmk} shall not be used due to the magic comment, 
then it will never compile the TEX file itself, 
and if \texttt{\$latexmkUsage} takes the value \texttt{Fully}, 
all required graphic files must be compiled for some reason, 
e.g.\@ there is none to be compiled. 
\medskip




% special: latexmk does not update pdf regularly provoking EEX03 false positive 
% We must find a way to prevent this. 
% Also: this is another disadvantage of build with latexmk: 
% It can be checked for update of files only if they were not present before. 
% One could remove pdf before running, but that way an advantage gets lost: 
% that latexmk does not compile anything if all is up to date. %


By the way, invoking \tool{latexmk} from within this software is the same as invoking manually. 
Both are based on \texttt{.latexmkrc}. 
The features supported are described in Section!\ref{subsec:latChkRc}. 
Among those are the supported targets, 
reading magic comments independently from internal implementations 
and support for reproducibility checks. 


% this is only for goal pdf 
% and maybe even only if pdfViaDvi is not set: LatexProcessor.processLatex2pdf
% changes for latexmk must be done in LatexProcessor.processLatex2dev
% bypassing processLatex2devCore... not clear whether also bypassing warnings. 
% maybe better not doing so: at least part of warnings. 
% on the other hand: misleading. 
% maybe warning that warnings are bypassed. 
% This could be done globally if \texttt{\$latexmkUsage} is not NotAtAll. 
% Also specifically if triggered by a magic comment. 

% make a note that still the return codes of latexmk are recorded resulting in EEX01 
% and also the presence of a pdf is monitored triggering EEC03 if not updated. 
% also the manual shows how to convert warnings into errors. 



% latexmk is never used for cleanup with latexmk -c or latexmk -C 

% on magic comments: 
% - target: currently, latexmk is applied only to pdf creation. 
%           the target is determined on a higher level, so it is taken into account 
% - program: taken into account. 




% preferred usage: 
% latexmkUsage=NotAtAll, but activation by magic comment shall be preferred usage. 
% this brings problems finding out whether really latexmk has been used. 
% an idea would be to include the fdb_latexmk file as this is specific for latexmk 
% But this has to be done in a way, that compilation still works if not present. 
% Only the diff to the original fails then. 
% In fact, even two \LaTeX{} main files are required: 
% - one for which latexmk is activated although latexmkUsage is set to the default NotAtAll
% - another one for which latexmk is not requested by magic comment. 
% This can be any document which is already present. 

% Maybe check that latexmk respects also the compiler given by magic comment. 
% Decision: no: is checked for one document once and this is suficient although later it is compiled generically. 
% This is decided because wrong compiler is detected quickly. 

% latexmk can be invoked as 
% SOURCE_DATE_EPOCH=0 FORCE_SOURCE_DATE=1 latexmk xxx.tex 
% as well. 
% That way all converters are invoked with the same environment variables. 
% The deceicive ones are xxxlatex and dvipdf 




% baglock: chkDiff as a magic comment. 
% this may be better in many cases than a global setting chkDiff (although this has applications also)
% extended patternLatexMainFile
% TBD: adapt manual. 
% This is part of preferred usage: both chkDiff setting in pom 
% and according magic comment. 
% unlike for latexmk, it is not possible to check both preferred usages: 
% for this project the setting chkDiff=true holds 
% so the magic comment must be checked outside the project. 





\section{Creating hypertext}\label{sec:tex2html}

To create HTML and XHTML from TEX files (more precise from \LaTeX{} files), 
a \texttt{tex4htCommand}-command is used 
Together with its parameters, 
it is described in Section!\ref{sec:settingsLatex2Html}. 
This may be \texttt{htlatex}, the default based on \texttt{latex} 
and \texttt{htxelatex} based on \xelatex. 

Figure~\ref{fig:tex2xml} shows the steps \texttt{htlatex} performs: 
From the input \LaTeX{} file \texttt{xxx.tex} 
another \LaTeX{} file \texttt{yyy.tex} is created 
which arises from \texttt{xxx.tex} by adding 
%FIXME\@: maybe instead: \RequirePackage which may be placed before documentclass
\begin{lstlisting}[language=TeX]
\usepackage[...]{tex4ht}. 
\end{lstlisting}
%
Then \texttt{htlatex} runs \texttt{latex} on \texttt{yyy.tex} 
which results in \texttt{yyy.dvi}. 
Note that this is in contrast to \lualatex{} 
which would create some \texttt{yyy.pdf} unless otherwise specified. 

Then comes the converter \pkg{tex4ht} into the game 
which creates several html files among those also \texttt{xxx.html}. 
The other files, \texttt{yyy.idv} and \texttt{yyy.lg}, 
are further processed by \texttt{t4ht} 
creating the stylesheet \texttt{xxx.css} and graphic files. 
\medskip


Let us make this more precise. 
The output of latex is a standard \gls{dvi} file 
interleaved with special instructions 
for the post-processor \pkg{tex4ht} to use. 
Note that \pkg{tex4ht} is the name both of the post-processor 
and of the \LaTeX-package. 
The special instructions come from implicit and explicit requests 
made in the source file through commands for TeX4ht. 

The utility \pkg{tex4ht} translates the dvi-code into standard text, 
while obeying the requests it gets from the special instructions. 
The special instructions may request the creation of files, 
insertion of html code, filtering of pictures, and so forth. 
In the extreme case that the source code contains no commands of TeX4ht, 
\pkg{tex4ht} gets pure dvi-code and it outputs (almost) plain text 
with no hypertext elements in it.

The special (\cmd{special}) 
instructions seeded in the dvi-code 
are not understood by dvi processors other than those of TeX4ht.

\texttt{t4ht}
This is an interpreter 
for executing the requests made in the \texttt{xxx.lg} script.

\texttt{xxx.idv}
This is a dvi file extracted from \texttt{xxx.dvi}, 
and it contains the pictures needed in the html files.

\texttt{xxx.lg}
This is a log file listing the pictures of \texttt{xxx.idv}, 
the \gls{png} files that should be created, CSS information, 
and user directives introduced 
through the ``\cmd{Needs\{\ldots\}}'' command.

\raggedbottom{}


\begin{figure}[!htb]
\centering
\IfPackageLoadedTF{tex4ht}{%
should be a picture 
}{
\import{}{F5_09tex2xml.ptx}
}
\caption{\label{fig:tex2xml}Conversion of a TEX file into an xml file}
\end{figure}

% LTeX: enabled=false
\begin{Verbatim}[fontsize=\tiny]
(/usr/local/texlive/2014/texmf-dist/tex/generic/tex4ht/tex4ht.4ht
version 2009-01-07-07:11
--------------------------------------
Note --- for additional information, use the command line option `info'
--------------------------------------

(/usr/local/texlive/2014/texmf-dist/tex/generic/tex4ht/html4.4ht

Note: to remove the <?xml version=...?> processing instruction 
use the command line option `no-VERSION'

Note: to remove the DOCTYPE declaration 
use the command line option `no-DOCTYPE'
)

--------------------------------------
Note: for marking of the base font, use the command line option `fonts+'
Note: for non active _, use the command line option `no_'
Note: for _ of catcode 13, use the command line option `_13'
Note: for non active ^, use the command line option `no^'
Note: for ^ of catcode 13, use the command line option `^13'
--------------------------------------

(/usr/local/texlive/2014/texmf-dist/tex/generic/tex4ht/html4.4ht
--------------------------------------
Note: For section filenames that reflect on their titles 
use the command line option `sec filename'

Note: for alternative charset, use the command line option `charset=...'

Note: to ignore CSS font decoration, use the `NoFonts' command line option

Note: for jpg bitmaps of pictures, 
use the `jpg' command line option. 
(Character bitmaps are controled only by `g' 
records of tex4ht.env and `-g' switches of tex4ht.c) 

Note: for gif bitmaps of pictures, use the `gif' command line option. 
(Character bitmaps are controled only by `g' 
records of tex4ht.env and `-g' switches of tex4ht.c) 

Note: for content and toc in 2 frames, 
use the command line option `frames'

Note: for content, toc, and footnotes in 3 frames, 
use the command line option `frames-fn'

Note --- for file extension name xht, use the command line option `xht'
--------------------------------------
TeX4ht package options: xhtml,uni-html4,2,pic-tabular,html
--------------------------------------
Note: to ignore CSS code, use the command line option `-css

Note: for inline CSS code, use the command line option `css-in'

Note: for pop ups on mouse over, use the command line option `mouseover'

Note: for addressing images in a subdirectory, 
use the command line option `imgdir:.../'
)

Note --- for back links to toc, use the command line option `sections+'

Note --- for linear crosslinks of pages, use the command line option `next'

(/usr/local/texlive/2014/texmf-dist/tex/generic/tex4ht/latex.4ht
version 2009-05-21-09:32
--------------------------------------
Note --- for links into captions, instead of float heads, use the command l
ine option `refcaption'
--------------------------------------

(/usr/local/texlive/2014/texmf-dist/tex/generic/tex4ht/html4.4ht
--------------------------------------
Note --- For mini tocs immediately aftter the header 
use the command line option `minitoc<'

Note --- for enumerated list elements with valued data, 
use the command line option `enumerate+'

Note --- for enumerated list elements li's with value attributes, use the c
ommand line option `enumerate-'

Note --- for CSS2 code, use the command line option `css2'

Note --- for bitmap fbox'es, use the command line option `pic-fbox'

Note --- for bitmap framebox'es, use the command line option `pic-framebox'

Note --- for inline footnotes use command line option `fn-in'

Note --- for tracing of latex font commands, 
use the command line option `fonts'
--------------------------------------
--------------------------------------
Note --- for width specifications of tabular p entries, 
use the `p-width' command line option 
or a configuration similar to 
\Configure{HColWidth}{\HCode{style="width:\HColWidth"}}
--------------------------------------
)
(/usr/local/texlive/2014/texmf-dist/tex/generic/tex4ht/html4-math.4ht
version 2009-05-18-23:01
--------------------------------------
Note --- for pictorial eqnarray, use the command line option `pic-eqnarray'

Note --- for pictorial array, use the command line option `pic-array'

Note --- for pictorial $...$ environments, 
use the command line option `pic-m' (not recommended!!)

Note --- for pictorial $...$ and $$...$$ environments with latex alt, 
use the command line option `pic-m+' (not safe!!)

Note --- for pictorial array, use the command line option `pic-array'
)
(/usr/local/texlive/2014/texmf-dist/tex/generic/tex4ht/unicode.4ht
version 2010-12-18-17:40
)
(/usr/local/texlive/2014/texmf-dist/tex/generic/tex4ht/html4-uni.4ht))


(/usr/local/texlive/2014/texmf-dist/tex/generic/tex4ht/html4.4ht
--------------------------------------
Note --- for tocs without * entries, use command line option `notoc*'

Note --- for tocs without * entries, use command line option `notoc*'

Note --- to eliminate mini tables of contents, 
use the command line option `nominitoc'

Note --- for frames-like object-based table of contents, 
use the command line option `obj-toc'

Note --- for files named derived from section titles, 
use the command line option `sec filename'

Note --- for i-columns index, 
use the command line option `index=i' (e.g., index=2)
--------------------------------------
)

(/usr/local/texlive/2014/texmf-dist/tex/generic/tex4ht/html4.4ht

Note --- if included graphics are of degraded quality, 
try the command line options `graphics-num' or `graphics-'. 
The `num' should provide the density of pixels in the bitmaps (e.g., 110). 

Note --- for key dimensions try the option `Gin-dim'; 
for key dimensions when bounding box is unavailable 
try `Gin-dim+'; neither is recommended
)

(/usr/local/texlive/2014/texmf-dist/tex/generic/tex4ht/html4.4ht
Note --- for URL encoding within href use the command line option `url-enc'
)

(/usr/local/texlive/2014/texmf-dist/tex/generic/tex4ht/html4.4ht

Note --- for pictorial longtable, 
use the command line option `pic-longtable'
)

(/usr/local/texlive/2014/texmf-dist/tex/generic/tex4ht/html4.4ht

Note --- to ensure proper alignments use fixed size fonts (see listings.dtx
)
)
\end{Verbatim}
% LTeX: enabled=true

\pkg{tex4ht} yields 

\begin{Verbatim}[fontsize=\scriptsize]
----------------------------
tex4ht.c (2012-07-25-19:36 kpathsea)
tex4ht 
--- error --- improper command line
tex4ht [-f<path-separator-ch>]in file[.dvi]
   [-.<ext>]            replacement to default file extension name .dvi
   [-c<tag name>]       choose named segment in env file
   [-e<env file>]
   [-f<path-separator-ch>]        remove path from the file name
   [-F<ch-code>]        replacement for missing font characters; 0--255; default 0
   [-g<bitmap file-ext>]
   [-h(e|f|F|g|s|v|V)]  trace: e-errors/warnings, f-htf, F-htf search
                            g-groups, s-specials, v-env, V-env search
   [-i<htf-font-dir>]
   [-l<bookkeeping file>]
   [-P(*|<filter>)]     permission for system calls: *-always, filter
   [-S<image-script>]
   [-s<css file-ext>]   default: -s4cs; multiple entries allowed
   [-t<tfm-font-dir>]
   [-u10]               base 10 for unicode characters
   [-utf8]              utf-8 encoding for unicode characters
   [-v<idv version>]    replacement for the given dvi version
   [-xs]           ms-dos file names for automatically generated gifs
\end{Verbatim}


\texttt{t4ht} yields 

\begin{Verbatim}[fontsize=\footnotesize]
--------------------------------------------------------------------
t4ht [-f<dir char>]filename ...
  -b     ignore -d -m -M for bitmaps
  -c...  choose named segment in env file
  -d...  directory for output files       (default:  current)
  -e...  location of tex4ht.env
  -i     debugging info
  -g     ignore errors in system calls
  -m...  chmod ... of new output files (reused bitmaps excluded)
  -p     don't convert pictures           (default:  convert)
  -r     replace bitmaps of all glyphs    (default:  reuse old ones)
  -M...  chmod ... of all output files
  -Q     quit, if tex4ht.c had problems
  -S...  permission for system calls: *-always, filter
  -X...  content for field %%3 in X scripts
  -....  content for field %%2 in . scripts

Example: 
   t4ht name -d/WWW/temp/ -etex4ht-32.env -m644
--------------------------------------------------------------------
\end{Verbatim}

\flushbottom

\section{Creating odt files}\label{sec:tex2odt}

\section{Creating MS word files}\label{sec:tex2doc}

The best way to convert \LaTeX{} files into MS word files is via ODT files. 
Conversion from \LaTeX{} to odt 
is already described in Section~\ref{sec:tex2odt}. 
The last step can be done by \texttt{odt2doc} which can create both 
doc-format and docx-format and many others 
which is illustrated in Figure~\ref{fig:tex2doc}. 


\begin{figure}[htb]
\centering
\IfPackageLoadedTF{tex4ht}{%
should be a picture 
}{
\import{}{F5_10tex2doc.ptx}
}
\caption{\label{fig:tex2doc}Conversion of a TEX file into a docx file}
\end{figure}



\section{Creating plain text files}\label{sec:tex2txt}

Why should one create plain text from \LaTeX{} files? 
Maybe this is the minimal format the receiver can work with. 
Another common application is word-count, 
in particular if writing a paper for a journal. 

Plain text files can be created from \LaTeX{} files 
just by stripping off the tex-commands. 
The disadvantage is, 
that references, bibliography, index, glossary, 
table of contents, list of figures, list of tables, \dots 
and symbols get lost. 
Thus, the first step we take is complete creation of a PDF file 
except display of warnings like bad boxes 
as described in Section~\ref{sec:tex2pdf}. 
This creates an appropriate pdf file, 
with correct numbering and links, 
possibly with overfull boxes and that like. 
As a final step, we convert the pdf file into a text file 
using, as a default \texttt{pdftotext} with ending \texttt{txt}. 
Figure~\ref{fig:tex2txt} illustrates the translation process. 

\begin{figure}[htb]
\centering
\IfPackageLoadedTF{tex4ht}{%
should be a picture 
}{
\import{}{F5_11tex2txt.ptx}
}
\caption{\label{fig:tex2txt}Conversion of a TEX file into a txt file}
\end{figure}

Note that \texttt{pdftotext} produces a text file with page numbers 
and signifies the end of a page 
(to see how, just have a look at the end of the file), 
so that one can identify page numbers as such. 
Thus references, index, glossary, table of contents and that like 
referring to page numbers carry valuable information. 
Also symbols available in utf8 encoding are preserved. 
In contrast, heavily stacked formulae become unreadable, 
because \texttt{pdftotext} displays them line by line 
and drops fraction bars completely. 
Also, formulae with complex subformulae in a root operator  
become unreadable because the root operator becomes just a root symbol. 
Likewise for integrals and that like. 

Aspects of figures kept are the captions of course but also the \LaTeX-texts. 
This is displayed line-wise. 
What gets lost is the postscript/pdf-parts, i.e.~the plain graphics. 

\raggedbottom{}


\include{manualC06paramsSettings}
\include{manualC07ExcLogging}
\chapter{Gaps}\label{chap:gaps}

Only figures created with xfig and stored as files pdf and ptx 
may be integrated into a \LaTeX{} document. 
This could be extended to a broader variety of export file formats. 
The problem is, that fig-files to not contain information on the export
format. 
This has to be either given elsewhere in a config file 
or determined by pre-parsing the tex-files. 

There is no support for pictures in \gls{gif}-format 
but maybe a converter to \gls{png} is all needed. 

There is no proper make-mechanism taking dependencies into account. 
Thus all documents in all formats specified are remade, 
whether they changed or not. 

Also, if more than one target is created from one \LaTeX{} source, 
common steps are redone for each target. 
E.g.~if pdf and html are created, 
pdf creation is done twice and if pdf, html, odt and docx are created, 
odt is done twice (once for odt second for docx) 
and pdf is done even trice: 
once for pdf itself, once for odt and once for docx. 

% This has to be removed. 
Creating more than one index is supported only 
via package \pkg{splitidx} in conjunction with \texttt{SplitIndex}. 
There are the following packages also supporting multiple indices 
but not supported officially: 
\pkg{index} described in~\cite{IndexP}, 
\pkg{amsmidx} described in~\cite{AmsmidxP} and 
\pkg{imakeidx} described in~\cite{ImakeidxP}.
Note that the package \pkg{multind} is obsolete. 

According to~\cite{GloP4_54}, Section~1.3, 
there are various options to create a glossary, 
whereas this software supports option two only described in Section~1.3.2, 
which uses \texttt{makeindex} for indexing. 
Currently, indexing with \texttt{xindy} is not supported. 
The last two options are available only with package \pkg{glossaries-extra} 
which this software will support in later versions. 
% planned in 2.2. 
% Then this technique is devoted a separate section and this becomes the preferred usage. 

By default, package \pkg{glossaries} creates a single ``main glossary'', 
which can be switched off specifying the option \texttt{nomain} 
described in Section~2.6. 
In this case at least, more specific glossary types must be specified. 
This can be done by options like \texttt{acronyms} described in Section~2.7 
or the \texttt{symbols},
\texttt{numbers} or \texttt{index} options described in Section~2.9. 
As the \texttt{index} option collides with indexing as performed by this software, 
the option \texttt{index} is not allowed. 

The package \pkg{glossaries} itself 
supports new glossary types via the command
%
\begin{lstlisting}[language=TeX, basicstyle=\small]
\newglossary[log-ext]{name}{in-ext}{out-ext}{title}[counter]
\end{lstlisting}
%
described in~\cite{GloP4_54}, Section~9. 
In fact, the glossary types accessible via options and even the main glossary 
are defined internally that way. 
Although the glossary algorithm of this software, 
in particular rerun management as described in Section~\ref{sec:rerunAux}
can create any kind of glossaries, 
this software does not support defining new glossary types 
because collision, e.g. with indexing cannot be excluded 
and also because each glossary type creates auxiliary files 
with user-defined extensions currently neither this software 
nor \texttt{latexmk} can cleanup. 

Reading~\cite{GloP4_54}, Section~13.1, the glossary style \texttt{index} 
seems to allow creating indices through the \pkg{glossaries} package 
making any index-package obsolete. 
This software does not support that option provided by the package \pkg{glossaries}. 

For development given the \LaTeX{} main file \texttt{xxx.tex}, 
the files \texttt{xxx.pdf}, \texttt{xxx.pdf}, \texttt{xxx.synctex.gz} 
and \texttt{xxx.log} are vital. 
Thus, it would be fine to have a goal which touches these files 
or to have a parameter to touch these prior to creation 
to avoid that these are cleaned up after the run. 
This is an alternative to setting parameter \texttt{cleanup} to \texttt{false}. 
On the other hand, goal \texttt{grp} creating graphics 
in conjunction with a development tool like \auctex{} for \texttt{emacs}, 
allows to compile a latex main file in that tool 
and thus to access \texttt{xxx.log} and \texttt{xxx.pdf}. 

There are lots of possible improvements to be done on the goal \emph{check}. 


The ant-task does not allow creating single formats, e.g.~pdf selectively. 

The ant-build is not completed: tests are not run and 
test runs are no prerequisite for installation. 

This manual is not finished. 
To test the overall functionality of the maven-plugin and of the ant-task 
described here, this manual is created through plugin and task. 

Support for djvu via pdf2djvu: 
pdf2djvu -o output\_file input\_file

pdf2dsc
(ps with document structuring convention) 

pdf2svg is not so useful. 

pdftohtml -c is also not bad, 

consider also pdftocairo for creation of tiff and ps and many others. 

\include{manualC09Bugs}
\include{manualC10prefUsage}


\bibliographystyle{alpha}
\bibliography{lit,litLMPpub,litTools}

\IfPackageLoadedTF{tex4ht}{%
  % no index and no glossaries with tex4ht 
}{
%\chapter{Indices}
  \printindex[idx]
  \printindex[pkg]
  \printindex[cmd]
  \printglossaries%
}


%  "-src-specials -shell-escape -recorder" 
\end{document}

