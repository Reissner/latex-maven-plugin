\chapter{Exceptions and Logging}\label{chap:exceptionLogging}

If during execution of this software something goes wrong, 
and it is possible to detect that, the user shall be notified. 

Maven foresees a mechanism to abort the whole build, i.e.~lifecycle phase 
or a single goal and accordingly ant allows to abort a task. 
In both cases, abortion is implemented by throwing an exception. 


A maven plugin aborts a goal throwing a 
%
\begin{verbatim}
org.apache.maven.plugin.MojoFailureException
\end{verbatim}
%
and a 
%
\begin{verbatim}
org.apache.maven.plugin.MojoExecutionException 
\end{verbatim}
%
to abort the life-cycle phase. 
Since this plugin is just for documentation, 
there is no need to abort site creation altogether, 
so only the former exception occurs. 

An ant-task aborts an ant-build throwing a
%
\begin{verbatim}
org.apache.tools.ant.BuildException
\end{verbatim}
%
without further distinction. 

This software provides both a maven plugin and an ant task 
built on the same code base. 
Thus, the maven plugin throws a \texttt{MojoFailureException} 
if and only if the according ant-task throws an \texttt{BuildException} 
in the same situation. 

Section~\ref{sec:exception} describes 
the philosophy of throwing an exception and 
defines in detail under what circumstances which exception is thrown. 

Roughly speaking, an exception is thrown only if something is really wrong, 
e.g.~a non-recoverable error or an indication 
that the build system is out of control or if this plugin/task 
is likely to destroy the work of another plugin/task. 

If something went wrong, but no exception is thrown, 
the user must be notified by logging 
and the build process to go on, 
skipping a section of a task as small as possible. 
Both, maven and ant provide a logging mechanism 
with the levels error, warning, info and debug. 
Section~\ref{sec:logWarnErr} describes the errors and warnings; 
the lot of infos and debugging output are not described here. 

Verbosity is chosen by the following command line options: 
%
\begin{itemize}
\item[\texttt{-e}] shows error messages, 
\item[\texttt{-X}] shows debug-messages, 
\item[\texttt{-q}] quiet hides the info-level and shows \emph{only} errors. 
\end{itemize}
%
There seems no way to get warnings only. 

Each exception offers a message and also each warning has a warning message. 
The messages are endowed with a unique identifier of the form 
\texttt{KCCDD}, where \texttt{K} is the kind which is one of 
%
\begin{itemize}
\item[T] Throwable, 
which results in a \texttt{MojoFailureException} for the maven-plugin 
and \texttt{BuildException} for the ant-task. 
This is described in detail in Section~\ref{sec:exception} 
\item[E] logging as ERROR, 
\item[W] logging as WARNING 
\item[I] logging as INFO which occurs frequently
\item[D] logging as DEBUGGING output, which is lengthy 
\end{itemize}

The shortcut \texttt{CC} describes the class where the exception is thrown 
or the warning is logged: 
%
\begin{itemize}
\item[EX] CommandExectutorImpl: 
a class executing applications on a command line. 
\item[PP] LatexPreProcessor: preprocessing of \LaTeX-files: 
Processing of graphic files and detection of latex main files. 
\item[LP] LatexProcessor: processing of \LaTeX-main files: 
conversion into various output formats. 
\item[SS] Settings: A container holding the values of all parameters. 
These are either default or read from the configuration 
in the pom for the maven plugin and in the build file for the ant task. 
\item[MI] MetaInfo: offering meta information
  as expected and actual versions of converters. 
\item[FU] TexFileUtilsImpl: a class providing access to files. 
\end{itemize}
%
Finally, \texttt{DD} is a two digit number enumerating the messages.


\begin{longtable}{|ll|}
\toprule
Identifier        & Message  \\
\multicolumn2{|l|}{Explanation }  \\
\midrule
\midrule
\endfirsthead%
\bottomrule
\caption{\label{tab:LogMI} The logging for MetaInfo   }
\endlastfoot%
\texttt{\footnotesize WMI01} 
& \texttt{\footnotesize Version string from converter \$conv did not match expected form:      } \\
& \texttt{\footnotesize \$conv: 'version'not?in\$interv } \\% chktex 32
\multicolumn2{|l|}{
\begin{minipage}{0.95\linewidth}
  Indicates that the version string coming from the converter \$conv
  is not as expected.
  Programming error excluded, this means that the version does not fit,
  i.e.~is not in \texttt{\$interv}.
\end{minipage}
} \\
\texttt{\footnotesize WMI02} 
& \texttt{\footnotesize \$conv: '\$actVersion'not in\$interv} \\
\multicolumn2{|l|}{
\begin{minipage}{0.95\linewidth}
  Indicates that the version of converter \$conv can be detected
  and is \$actVersion but does not fit the expectation which is \$expVersion. 
\end{minipage}
}
\end{longtable}




\begin{longtable}{|ll|}
\toprule
Identifier        & Message  \\
\multicolumn2{|l|}{Explanation }  \\
\midrule
\midrule
\endfirsthead%
\bottomrule
\caption{\label{tab:LogFU} The logging for TexFileUtils   }
\endlastfoot%
\texttt{\footnotesize WFU01} 
& \texttt{\footnotesize Cannot read directory '\$dir';  } \\
& \texttt{\footnotesize build may be incomplete. } \\
\multicolumn2{|l|}{
\begin{minipage}{0.95\linewidth}
TBD 
\end{minipage}
} \\
\texttt{\footnotesize XFU02} 
& \texttt{\footnotesize  TBD} \\
\multicolumn2{|l|}{
\begin{minipage}{0.95\linewidth}
TBD
\end{minipage}
} \\
\texttt{\footnotesize WFU03}  & \texttt{\footnotesize Cannot close '\$file'.} \\
\multicolumn2{|l|}{
\begin{minipage}{0.95\linewidth}
TBD
\end{minipage}
} \\
\texttt{\footnotesize EFU05}  & \texttt{\footnotesize Cannot delete file '\$file'.} \\
\multicolumn2{|l|}{
\begin{minipage}{0.95\linewidth}
TBD
\end{minipage}
} \\
\texttt{\footnotesize EFU06}  & \texttt{\footnotesize Cannot move file '\$src'
                                to '\$dest'.} \\
\multicolumn2{|l|}{
\begin{minipage}{0.95\linewidth}
TBD
\end{minipage}
} \\
\texttt{\footnotesize EFU07}  & \texttt{\footnotesize File '\$srcFile' to be filtered cannot be read.} \\
\multicolumn2{|l|}{
\begin{minipage}{0.95\linewidth}
  WORKAROUND for \texttt{inkscape}
  filtering \texttt{eps\_tex}-file into \texttt{ptx} file:
  The former is not a readable regular file. 
\end{minipage}
} \\
\texttt{\footnotesize EFU08}  & \texttt{\footnotesize Destination file '\$destFile' for filtering cannot be written.} \\
\multicolumn2{|l|}{
\begin{minipage}{0.95\linewidth}
 WORKAROUND for \texttt{inkscape}
  filtering \texttt{eps\_tex}-file into \texttt{ptx} file:
  The latter is not a writable regular file. 
\end{minipage}
} \\
\texttt{\footnotesize EFU09}  & \texttt{\footnotesize Cannot filter file '\$srcFile' into '\$destFile'. } \\
\multicolumn2{|l|}{
\begin{minipage}{0.95\linewidth}
 WORKAROUND for \texttt{inkscape}
  filtering \texttt{eps\_tex}-file into \texttt{ptx} file:
  Either reading a line or writing a line failed. 
\end{minipage}
} \\
\end{longtable}
TBD\@: check whether workaround still necessary. 
TBD\@: complete list
TBD\@: add missing lists 


\section{Exceptions}\label{sec:exception}

Exceptions are thrown only if no substantial part of 
this maven-goal or this ant-task may be completed 
as if the tex source directory does not exist or is no directory 
or if a failure occurs which indicates 
that the underlying system does not work properly, 
as if the tex source directory or a subdirectory is not readable 
or if execution of an external program fails. 
The latter does not mean that the program returns with an error code, 
but it means that execution from within java fails.

\begin{longtable}{|ll|}
\toprule
Identifier        & Message  \\
\multicolumn2{|l|}{Explanation }  \\
\midrule
\midrule
\endfirsthead%
\bottomrule
\caption{\label{tab:TEX} The \texttt{BuildFailureException}s of the class
\texttt{CommandExecutorImpl} }
  % TBD: find out why the above texttt does not show up in the text. 
\endlastfoot%
\texttt{\footnotesize TEX01} & \texttt{\footnotesize Error running \$command. }  \\
\multicolumn2{|l|}{
\begin{minipage}{0.95\linewidth}
Compare with EEX01 in Table~\ref{tab:WarnCEI}: 
Error execution means 
%
\begin{itemize}
\item
the file expected to be the working directory 
does not exist or is not a directory. 
\item
method \texttt{Runtime.exec(String, String[], File)} fails % chktex 36
throwing an \texttt{IOException}. 
\item
an error inside systemOut parser occurs 
\item
an error inside systemErr parser occurs 
\item
Wrapping an \texttt{InterruptedException} 
on the process to be executed thrown by \texttt{Proces.waitFor()}. % chktex 36
\end{itemize}
%
whereas for EEX01 just a failure code is returned. 
\end{minipage}
} \\
\end{longtable}



\begin{longtable}{|ll|}
\toprule
Identifier        & Message  \\
\multicolumn2{|l|}{Explanation }  \\
\midrule
\midrule
\endfirsthead%
\bottomrule
\caption{\label{tab:TSS} The \texttt{BuildFailureException}s of the class \texttt{Settings} }
\endlastfoot%
\texttt{\footnotesize TSS01} 
& \texttt{\footnotesize The tex source directory '\$texSrcDirectoryFile'} \\
& \texttt{\footnotesize should be an existing directory, but is not. } \\
\multicolumn2{|l|}{
\begin{minipage}{0.95\linewidth}
The tex source directory is given in the pom/build-file 
with default value \texttt{./src/site/tex}. 
It contains or is \texttt{\$texSrcProcDirectoryFile}. 
Thus is must be a directory. 
\end{minipage}
} \\
\texttt{\footnotesize TSS02} 
& \texttt{\footnotesize The tex source processing directory '\$texSrcProcDirectoryFile'} \\
& \texttt{\footnotesize should be an existing directory, but is not. } \\
\multicolumn2{|l|}{
\begin{minipage}{0.95\linewidth}
The tex source processing directory is given in the pom/build-file 
relative to \texttt{\$texSrcDirectoryFile}
with default value \texttt{.}. 
It contains all files to be processed. 
Thus is must be a directory. 
\end{minipage}
} \\
\texttt{\footnotesize TSS03}  & \texttt{\footnotesize The output directory '\$outputDirectory'} \\
& \texttt{\footnotesize should be a directory if it exists, but is not. } \\
\multicolumn2{|l|}{
\begin{minipage}{0.95\linewidth}
The output directory is given in the pom/build-file 
with default value \texttt{./target/site/.}. 
The output directory is where the result of the goal/task 
are copied to. 
If it does not yet exist, it is created 
but if it exists and is a regular file, it cannot be created any more. 
\end{minipage}
} \\
\texttt{\footnotesize TSS04}  & \texttt{\footnotesize The target set '\$targets' } \\
& \texttt{\footnotesize should be a subset of the registered targets '\ldots', but is not. } \\
\multicolumn2{|l|}{
\begin{minipage}{0.95\linewidth}
The target set is given in the pom/build-file 
with default value \texttt{pdf, html}. 
A single target can be given on the command line
as e.g.~via \texttt{mvn latex:pdf} and also in this case,
the validity of the target is checked, so that e.g.~\texttt{mvn latex:invalid}
throws an exception, but the mechanism relies directly on mavens ability
to check the targets of this plugin. 
\end{minipage}
} \\
\texttt{\footnotesize TSS05}  & \texttt{\footnotesize The excluded converters '\$convertersExcluded' } \\
& \texttt{\footnotesize should form a subset of the registered converters '\ldots'. } \\
\multicolumn2{|l|}{
\begin{minipage}{0.95\linewidth}
From the possible ``registered'' converters the ones not used may be excluded
to avoid that they cause errors when trying to check correctness of version in
target \texttt{vrs} accessed via \texttt{mvn latex:vrs}.
These converters may not even be installed. 
\end{minipage}
} \\
\texttt{\footnotesize TSS06}  & \texttt{\footnotesize Tried to use converter '\$convStr' } \\
& \texttt{\footnotesize although not among the registered converters '\ldots' as expected. } \\
\multicolumn2{|l|}{
\begin{minipage}{0.95\linewidth}
Only registered converters may be used. 
\end{minipage}
} \\
\texttt{\footnotesize TSS07}  & \texttt{\footnotesize Tried to use converter '\$convStr' } \\
& \texttt{\footnotesize although among the excluded converters '\ldots'. } \\
\multicolumn2{|l|}{
\begin{minipage}{0.95\linewidth}
Among the registered converters only those may be used,
which are not excluded, i.e.~listed in configuration in section \texttt{convertersExcluded}.
\end{minipage}
} \\
\texttt{\footnotesize TSS08}  & \texttt{\footnotesize Tried to use converter '\$convStr' } \\
 & \texttt{\footnotesize in configuration '\ldots' instead of configuration '\ldots'. } \\
 \multicolumn2{|l|}{
 \begin{minipage}{0.95\linewidth}
   Each converter may occur in a specified configuration only.
   So e.g.\@ \lualatex{} is only allowed in configuration
   '\texttt{latex2pdfCommand}'.
   If used in configuration '\texttt{makeIndexCommand}' this causes this exception,
   because in that configuration, e.g \texttt{makeindex} is allowed. 
 \end{minipage}
 } \\
 \texttt{\footnotesize TSS09}  & \texttt{\footnotesize The diff directory '\$diffDirectoryFile' } \\
 & \texttt{\footnotesize should be a directory if it exists, but is not.  } \\
 \multicolumn2{|l|}{
 \begin{minipage}{0.95\linewidth}
   The \texttt{\$diffDirectoryFile} shall exist and be a directory. 
   In it shall be stored the artifacts the actually created shall be compared with 
   if \texttt{chkDiff} is set using the command \texttt{diffPdfCommand}. 
   As the name suggests, currently only pdf-files are compared. 
 \end{minipage}
 } \\
 \texttt{\footnotesize TSS10}  & \texttt{\footnotesize Specified unregistered converter '\$convStrProper'} \\
 & \texttt{\footnotesize with invalid category '\$catStr'; should be '\dots'.} \\
 \multicolumn2{|l|}{
 \begin{minipage}{0.95\linewidth}
  The converter \texttt{convName} 
  is specified in the setting \texttt{<catCommand>} 
  in the form \texttt{convName:notCat} 
  with category \texttt{notCat} not coinciding with \texttt{cat} as required. 
 \end{minipage}
 } \\
\end{longtable}


\begin{longtable}{|ll|}
\toprule
Id.        & Message  \\
\multicolumn2{|l|}{Explanation }  \\
\midrule
\midrule
\endfirsthead%
\bottomrule
\caption{\label{tab:TMI} The \texttt{BuildFailureException}s of the class \texttt{MetaInfo} }
\endlastfoot%
\texttt{\footnotesize TMI01} 
& \texttt{\footnotesize Cannot get stream to file '\$fileName'. }  \\
\multicolumn2{|l|}{
\begin{minipage}{0.95\linewidth}
  Stream to file within jar.
  This may be the manifest file, pom.properties or git.properties. 
\end{minipage}
} \\
\texttt{\footnotesize TMI02} 
& \texttt{\footnotesize Cannot load properties from file '\$fileName'.  }  \\
\multicolumn2{|l|}{
\begin{minipage}{0.95\linewidth}
  Provided the stream to the file is ok,
  could not load property.
  This may occur for pom.properties or git.properties. 
\end{minipage}
} \\
\texttt{\footnotesize TMI03} 
& \texttt
{\footnotesize IOException reading manifest. } \\
\multicolumn2{|l|}{
\begin{minipage}{0.95\linewidth}
Provided the stream to the manifest file is ok, could not read completely.
\end{minipage}
} \\
\end{longtable}

\begin{longtable}{|ll|}
\toprule
Id.        & Message  \\
\multicolumn2{|l|}{Explanation }  \\
\midrule
\midrule
\endfirsthead%
\bottomrule
\caption{\label{tab:TFU} The \texttt{BuildFailureException}s of the class \texttt{TexFileUtilsImpl}  }
\endlastfoot%
\texttt{\footnotesize TFU01} 
& \texttt{\footnotesize Cannot create destination directory '\$targetDir'. }  \\
\multicolumn2{|l|}{
\begin{minipage}{0.95\linewidth}
This is mainly because of writing permissions. 
\end{minipage}
} \\
\texttt{\footnotesize TFU04} 
& \texttt{\footnotesize Cannot overwrite directory '\$destFile'. }  \\
\multicolumn2{|l|}{
\begin{minipage}{0.95\linewidth}
Because this plugin shall not turn directories into regular files 
and vice versa. 
This failure indicates that another plugin/task disturbs this one. 
\end{minipage}
} \\
\texttt{\footnotesize TFU06} 
& \texttt
{\footnotesize Cannot copy file '\$srcFileName' to directory '\$targetDir'. } \\
\multicolumn2{|l|}{
\begin{minipage}{0.95\linewidth}
This is mainly because of writing permissions. 
\end{minipage}
} \\
\end{longtable}

\begin{longtable}{|ll|}
  \toprule
  Id.        & Message  \\
  \multicolumn2{|l|}{Explanation }  \\
  \midrule
  \midrule
  \endfirsthead%
  \bottomrule
  \caption{\label{tab:TLP} The \texttt{BuildFailureException}s of the class \texttt{LatexProcessor} }
  \endlastfoot%
  \texttt{\footnotesize TLP01} 
  & \texttt{\footnotesize Artifact '\$pdfFileAct' from '\$texFile' could not be reproduced. }  \\
  \multicolumn2{|l|}{
  \begin{minipage}{0.95\linewidth}
  Processing \$texFile' yields \$pdfFileAct which is not ``alike'' the stored version. 
  Currently, that kind of check can be performed for \texttt{pdf} files only. 
  Also the diff check is executed only if parameter \$chkDiff described in Section~\ref{sec:paramRepro} is set. 
  Then the diff command \$diffPdfCommand is performed to determine whether the artifacts are equivalent
  in the sense given by the diff command. 
  The concrete meaning of that equivalence 
  may range from strict equivalence to some kind of visual equivalence. 
  \end{minipage}
  } \\
  \texttt{\footnotesize TLP02} 
  & \texttt{\footnotesize  No file '\$pdfFileCmp' to compare with artifact from '\$texFile'. }  \\
  \multicolumn2{|l|}{
  \begin{minipage}{0.95\linewidth}
  The PDF file '\$pdfFileCmp' expected for comparison with the PDF file created from '\$texFile' does not exist. 
  It is expected only if a diff check is configured according to \$chkDiff described in Section~\ref{sec:paramRepro}. 
  Currently, that kind of check can be performed for \texttt{pdf} files only. 
  \end{minipage}
  }
\end{longtable}
  
  



FIXME\@: to be added. 


\section{Logging of Warnings and Errors}\label{sec:logWarnErr}

The rules for logging warnings and errors is, 
that the user must be notified, 
if something went wrong, but the run is not aborted, 
by a warning or an error. 
It is not required that for each detail going wrong, 
there is a separate notification, 
but the user must be sure, that all is ok, 
if no warning and no error occurs. 

To decide whether it is an error or a warning to be logged, 
one has to distinguish, 
whether the problem occurs when running an external application 
or within internal code. 
In the first case, the decision whether it is an error or a warning 
is left to that application: 
%
\begin{itemize}
\item 
If the application returns an error code other than 0, 
it is an error. 
\item
If the application is expected to write a log file, 
but none is found, it is an error. 
The applications used here, 
return a nontrivial error code if no log file is written. 
\item
The applications used here, writing a log file 
distinguish between error and warning. 
If a log file is written both are logged in the log file 
and can be distinguished by the form of the entry via pattern matching. 
If no error occurs, the return code is 0, even if warnings occur. 
\item
If an application writes at least one error into the log file, 
this software logs an error. 
\item
If an application writes no error into the log file 
but at least one warning, principally this software logs a warning. 
There may be parameters to switch off warnings partially 
or all of them, 
but there must be also a configuration of parameter values 
that allow logging all warnings. 
\end{itemize}

If an application does not create the expected output file, 
this software logs an error. 
This may be because of an internal error as described above, 
but also because of wrong parameters. 
So, 
% LTeX: enabled=false
e.g.~\texttt{lualatex -v xxx.tex} 
% LTeX: enabled=true
does not create a pdf-file as expected. 


\begin{longtable}{|ll|}
\toprule
\small Id.        & \small Message  \\
\multicolumn2{|l|}{\small Explanation }  \\
\midrule
\midrule
\endfirsthead%
\bottomrule
\caption{\label{tab:WarnALP} The errors and warnings on running a command  }
\endlastfoot%
\texttt{\footnotesize EEX01} & \texttt{\footnotesize Running \$command 
			    failed with return code \$returnCode. } \\
\multicolumn2{|l|}{
\begin{minipage}{0.95\linewidth}
Compare with TEX01 in Table~\ref{tab:TEX}: 
Error execution means that there is even no valid return code. 
% Variables: 
% %
% \begin{itemize}
% \item \texttt{\$command}: The command executed. 
% \item \texttt{\$returnCode}: The failure code which is not $0$ in this case. 
% \end{itemize}
\end{minipage}
} \\
\texttt{\footnotesize EEX02} & \texttt{\footnotesize Running \$command failed: 
			  No target file '\$fileName' written. } \\
\multicolumn2{|l|}{
\begin{minipage}{0.95\linewidth}
FIXME
\end{minipage}
} \\
\texttt{\footnotesize EEX03} 
& \texttt{\footnotesize Running \$command failed: 
Target file '\$fileName' is not updated. } \\
\multicolumn2{|l|}{
\begin{minipage}{0.95\linewidth}
The command \texttt{\$command} is expected 
to write to the file '\texttt{\$fileName}' but this file is not updated. 
This indicates an error executing \texttt{\$command}. 
\end{minipage}
} \\
\texttt{\footnotesize WEX04} 
& \texttt
{\footnotesize Cannot read target file '\$fileName'; may be outdated. } \\
\multicolumn2{|l|}{
\begin{minipage}{0.95\linewidth}
FIXME
\end{minipage}
} \\
\texttt{\footnotesize WEX05} 
& \texttt{\footnotesize Update control may emit false warnings. } \\
\multicolumn2{|l|}{
\begin{minipage}{0.95\linewidth}
FIXME
\end{minipage}
} \\
\texttt{\footnotesize EAP02} & \texttt{\footnotesize Running \$command failed: 
No log file '\$logFileName' written. }  \\
\multicolumn2{|l|}{
\begin{minipage}{0.95\linewidth}
The command \texttt{\$command} is expected 
to write a log file '\texttt{\$logFileName}' but no such file exists. 
This indicates an error executing \texttt{\$command}. 
\end{minipage}
} \\
\texttt{\footnotesize EAP01}  & \texttt{\footnotesize Running \$command failed. 
Errors logged in '\$logFileName'. } \\
\multicolumn2{|l|}{
\begin{minipage}{0.95\linewidth}
The command \texttt{\$command} logged at least one error 
in the file '\texttt{\$logFileName}', 
where more details can be found. 
\end{minipage}
} \\
\texttt{\footnotesize WAP03} 
& \texttt{\footnotesize Running \$command emitted warnings 
logged in '\$logFileName'. }  \\
\multicolumn2{|l|}{
\begin{minipage}{0.95\linewidth}
The command \texttt{\$command} logged at least one warning 
in the file '\texttt{\$logFileName}', 
where more details can be found. 
Note that if \texttt{\$command} is a latex processor, 
this warning comes only iff the parameter \texttt{\$debugWarnings} is set. 
Note also that notifications on bad boxes are not counted as warnings here. 
\end{minipage}
} \\
\texttt{\footnotesize WLP03} 
& \texttt{\footnotesize Running \$command created bad boxes 
logged in '\$logFileName'. } \\
\multicolumn2{|l|}{
\begin{minipage}{0.95\linewidth}
Here, \texttt{\$command} is a latex processor. 
It logged at least one bad box, 
overfull or underfull, horizontal or vertical 
in \texttt{\$logFileName} where more details can be found. 
Note that 
this warning comes only iff the parameter \texttt{\$debugBadBoxes} is set. 
\end{minipage}
} \\
\texttt{\footnotesize WLP06} 
& \texttt{\footnotesize Running \$command 
found issues logged in '\$logFileName'. } \\
\multicolumn2{|l|}{
\begin{minipage}{0.95\linewidth}
\textbf{This warning does no longer occur. The following is the original explanation: }
Here, \texttt{\$command} is a checker tool. 
Strictly speaking, unlike the other warnings here, 
this does not signify that running \texttt{\$command} went wrong 
but uncovered an issue (warning/error/message) logged in a file. 
\end{minipage}
} \\
\texttt{\footnotesize WLP05} 
& \texttt{\footnotesize Use package \texttt{splitidx} 
without option \texttt{split} in \texttt{\$texFileName}. } \\
\multicolumn2{|l|}{
\begin{minipage}{0.95\linewidth}
This indicates that an extended idx-file ``\texttt{xxx-yy.idx}'' 
has been found without \texttt{xxx.idx} or without according entry 
\texttt{\textbackslash indexentry[yy]\{\dots\}\{\dots\}} 
in \texttt{xxx.idx}. 
\end{minipage}
} \\
\texttt{\footnotesize WLP07} 
& \texttt{\footnotesize Found both '\texttt{\$dviFile}' and '\texttt{\$xdvFile}'; 
convert the latter. } \\
\multicolumn2{|l|}{
\begin{minipage}{0.95\linewidth}
This indicates that for conversion to PDF there are a DVI-file and a XDV-file 
which may come from mixed application of \texttt{xelatex} and another converter. 
In this case, the \texttt{\$xdvFile} is converted. 
\end{minipage}
} \\
\end{longtable}


\begin{longtable}{|ll|}
\toprule
Id.        & Message  \\
\multicolumn2{|l|}{Explanation }  \\
\midrule
\midrule
\endfirsthead%
\bottomrule
\caption{\label{tab:WarnCEI} The errors and warnings on files/streams  }
\endlastfoot%
\texttt{\footnotesize WFU01} 
& \texttt{\footnotesize Cannot read directory '\$dir'; 
build may be incomplete. } \\
\multicolumn2{|l|}{
\begin{minipage}{0.95\linewidth}
FIXME
\end{minipage}
} \\
\texttt{\footnotesize WPP02} 
& \texttt{\footnotesize Cannot read tex file '\$texFile'; 
may bear latex main file. } \\
\multicolumn2{|l|}{
\begin{minipage}{0.95\linewidth}
FIXME
\end{minipage}
} \\
\texttt{\footnotesize WAP04} 
& \texttt{\footnotesize Cannot read log file '\$logFileName'; 
  may hide warnings/errors. } \\
\multicolumn2{|l|}{
\begin{minipage}{0.95\linewidth}
FIXME
\end{minipage}
} \\
\texttt{\footnotesize WLP02} 
& \texttt{\footnotesize Cannot read log file '\$logFileName'; 
  \$kind may require rerun. } \\
\multicolumn2{|l|}{
\begin{minipage}{0.95\linewidth}
FIXME
\end{minipage}
} \\
\texttt{\footnotesize WLP04} 
& \texttt{\footnotesize Cannot read idx file '\$idxFileName'; 
skip creation of index. } \\
\multicolumn2{|l|}{
\begin{minipage}{0.95\linewidth}
FIXME
\end{minipage}
} \\
\texttt{\footnotesize WFU03} 
& \texttt{\footnotesize Cannot close '\$closeable'. } \\
\multicolumn2{|l|}{
\begin{minipage}{0.95\linewidth}
FIXME
\end{minipage}
} \\
\texttt{\footnotesize EFU05} 
& \texttt{\footnotesize Cannot delete file '\$delFile'. } \\
\multicolumn2{|l|}{
\begin{minipage}{0.95\linewidth}

\end{minipage}
} \\
\texttt{\footnotesize EFU06} 
& \texttt{\footnotesize Cannot move file '\$fromFile' to '\$toFile'. } \\
\multicolumn2{|l|}{
\begin{minipage}{0.95\linewidth}
FIXME
\end{minipage}
} \\
\end{longtable}


\begin{longtable}{|ll|}
\toprule
Id.        & Message  \\
\multicolumn2{|l|}{Explanation }  \\
\midrule
\midrule
\endfirsthead%
\bottomrule
\caption{\label{tab:WarnLPP} Miscellaneous errors and warnings  }
\endlastfoot%
\texttt{\footnotesize WPP03} 
& \texttt{\footnotesize Skipped processing of files with suffixes \$skipped. }\\
\multicolumn2{|l|}{
\begin{minipage}{0.95\linewidth}
FIXME
\end{minipage}
} \\
\texttt{\footnotesize WPP04} 
& \texttt{\footnotesize Skip processing \texttt{\$srcFile}: 
  interpreted as target of \texttt{\$lmFile}. } \\
\multicolumn2{|l|}{
\begin{minipage}{0.95\linewidth}
FIXME
\end{minipage}
} \\
\texttt{\footnotesize WPP05} 
& \texttt{\footnotesize Included latex files which are not latex main files: } \\
& \texttt{\footnotesize \texttt{\$includedNotMainFiles}. } \\
\multicolumn2{|l|}{
\begin{minipage}{0.95\linewidth}
In parameter \texttt{mainFilesIncluded} 
only latex main files shall be mentioned. 
The above message shows files specified 
which are not recognized as latex main files. 
This is also affected by parameter \texttt{patternLatexMainFile}. 
\end{minipage}
} \\
\texttt{\footnotesize WPP06} 
& \texttt{\footnotesize Excluded latex files which are not latex main files: } \\
& \texttt{\footnotesize \texttt{\$excludedNotMainFiles}. } \\
\multicolumn2{|l|}{
\begin{minipage}{0.95\linewidth}
In parameter \texttt{mainFilesExcluded} 
only latex main files shall be mentioned. 
The above message shows files specified 
which are not recognized as latex main files. 
This is also affected by parameter \texttt{patternLatexMainFile}. 
\end{minipage}
} \\
\texttt{\footnotesize WPP07} 
& \texttt{\footnotesize Included/Excluded latex main files not identified by their name: } \\
& \texttt{\footnotesize \texttt{\$inclExcl}. } \\
\multicolumn2{|l|}{
\begin{minipage}{0.95\linewidth}
This indicates that there are different latex main files 
with the same name (of course in different directories) 
and that \texttt{\$inclExcl} are those given in parameter 
\texttt{mainFilesIncluded} or \texttt{mainFilesExcluded}. 
\end{minipage}
} \\



\texttt{\footnotesize WLP01} 
& \texttt{\footnotesize LaTeX requires rerun 
but maximum number \$maxNumRerunsLatex reached. } \\
\multicolumn2{|l|}{
\begin{minipage}{0.95\linewidth}
FIXME
\end{minipage}
} \\
\texttt{\footnotesize ELP01} 
& \texttt{\footnotesize For command '\$command' 
found unexpected return code \$returnCode. } \\
\multicolumn2{|l|}{
\begin{minipage}{0.95\linewidth}
Here, \texttt{\$command} is a checker tool. 
The return codes are determined by reverse engineering. 
So possibly \texttt{\$returnCode} cannot be interpreted. 
\end{minipage}
} \\
\texttt{\footnotesize ELP02} 
& \texttt{\footnotesize Checker '\$command' logged an error in \$clgFile. } \\
\multicolumn2{|l|}{
\begin{minipage}{0.95\linewidth}
Indicates that the checker found an error. 
Note that errors are warnings declared explicitly as errors. 
There is also the case that warnings are declared as simple messages 
and thus causes neither a warning nor an error. 
\end{minipage}
} \\
\texttt{\footnotesize WLP08} 
& \texttt{\footnotesize Checker '\$command' logged a warning in \$clgFile. } \\
\multicolumn2{|l|}{
\begin{minipage}{0.95\linewidth}
Indicates that the checker found a warning. 
Implicitly it means that no error was found 
since this would cause EPL02. 
Note that warnings can be declared as simple messages 
and thus cause neither a warning nor an error. 
\end{minipage}
} \\

\end{longtable}








%\begin{longtable}
  
%\end{longtable}

FIXME\@: to be added. 
