\chapter{Gaps}\label{chap:gaps}

Only figures created with xfig and stored as files pdf and ptx 
may be integrated into a \LaTeX{} document. 
This could be extended to a broader variety of export file formats. 
The problem is, that fig-files to not contain information on the export
format. 
This has to be either given elsewhere in a config file 
or determined by pre-parsing the tex-files. 

There is no support for pictures in \gls{gif}-format 
but maybe a converter to \gls{png} is all needed. 

There is no proper make-mechanism taking dependencies into account. 
Thus all documents in all formats specified are remade, 
whether they changed or not. 

Also, if more than one target is created from one \LaTeX{} source, 
common steps are redone for each target. 
E.g.~if pdf and html are created, 
pdf creation is done twice and if pdf, html, odt and docx are created, 
odt is done twice (once for odt second for docx) 
and pdf is done even trice: 
once for pdf itself, once for odt and once for docx. 

% This has to be removed. 
Creating more than one index is supported only 
via package \pkg{splitidx} in conjunction with \texttt{SplitIndex}. 
There are the following packages also supporting multiple indices 
but not supported officially: 
\pkg{index} described in~\cite{IndexP}, 
\pkg{amsmidx} described in~\cite{AmsmidxP} and 
\pkg{imakeidx} described in~\cite{ImakeidxP}.
Note that the package \pkg{multind} is obsolete. 

According to~\cite{GloP}, Section~1, 
there are three options to create a glossary, 
whereas this software supports option two only, 
which uses \texttt{makeindex}. 
Also, although the package \pkg{glossaries} itself 
supports multiple glossaries via the command
%
\begin{lstlisting}[language=TeX, basicstyle=\small]
\newglossary[log-ext]{name}{in-ext}{out-ext}{title}[counter]
\end{lstlisting}
%
described in~\cite{GloP}, Section~12, 
this software only supports creating a single glossary. 

Reading~\cite{GloP}, Section~15.1, the glossarystyle \texttt{index} 
seems to allow creating indices through the \pkg{glossaries} package 
making any index-package obsolete. 

For development given the \LaTeX{} main file \texttt{xxx.tex}, 
the files \texttt{xxx.pdf}, \texttt{xxx.pdf}, \texttt{xxx.synctex.gz} 
and \texttt{xxx.log} are vital. 
Thus it would be fine to have a goal which touches these files 
or to have a parameter to touch these prior to creation 
to avoid that these are cleaned up after the run. 
This is an alternative to setting parameter \texttt{cleanup} to \texttt{false}. 
On the other hand, goal \texttt{grp} creating graphics 
in conjunction with a development tool like \auctex{} for \texttt{emacs}, 
allows to compile a latex main file in that tool 
and thus to access \texttt{xxx.log} and \texttt{xxx.pdf}. 

There are lots of possible improvements to be done on the goal \emph{check}. 


The ant-task does not allow creating single formats, e.g.~pdf selectively. 

The ant-build is not completed: tests are not run and 
test runs are no prerequisite for installation. 

This manual is not finished. 
To test the overall functionality of the maven-plugin and of the ant-task 
described here, this manual is created through plugin and task. 

Support for djvu via pdf2djvu: 
pdf2djvu -o output\_file input\_file

pdf2dsc
(ps with document structuring convention) 

pdf2svg is not so useful. 

pdftohtml -c is also not bad, 

consider also pdftocairo for creation of tiff and ps and many others. 
