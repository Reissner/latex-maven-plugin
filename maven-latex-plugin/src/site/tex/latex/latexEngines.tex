\RequirePackage[l2tabu, orthodox]{nag}
\documentclass{article}
\synctex=1
\usepackage{iftex}

\iftutex%
\usepackage{fontspec}
\else
\usepackage[utf8]{inputenc}
\usepackage[T1]{fontenc}
\fi
\usepackage{fancyvrb}


%%\pdfinfoomitdate1
%\pdfsuppressptexinfo-1
%\pdftrailerid{}


\usepackage{ifthen}
\newboolean{texFhtLoaded}
\setboolean{texFhtLoaded}{false}

\usepackage{booktabs}
\aboverulesep=0ex
\belowrulesep=0ex

\usepackage[a4paper]{geometry}
\usepackage{listings}
\usepackage{longtable}
% used by hyperref and also to update index and glossary 
% to avoid clash because of loading with different options: 
% declare first 
\usepackage[aux]{rerunfilecheck}

% graphics 
\makeatletter

\ifpdf%
\usepackage{xcolor}  % [pdftex]
\usepackage{graphicx}% [pdftex] 
\usepackage{hyperref}% driver [hpdftex] is autodetected 

\ifLuaTeX%
% for lualatex
\pdfvariable minorversion=7% chktex 1
% omit CreationDate and ModDate keys.
\pdfvariable suppressoptionalinfo 767% chktex 1
% no adding to the trailer dictionary.
\pdfvariable trailerid{}% chktex 1
\pdfvariable suppressoptionalinfo -1% chktex 1
\else
\ifXeTeX%
% for xelatex
\special{pdf:minorversion 7}
% TBD: find a way to express pdfinfoomitdate: necessary? 
\special{pdf:trailerid []}
\else
\ifPDFTeX%
\pdfminorversion=7         % for pdflatex
% omit CreationDate and ModDate keys.
% not before pdfTeX 3.14159265-2.6-1.40.17
\pdfinfoomitdate=1                   % for pdflatex
% no adding to the trailer dictionary.
%\pdftrailer=0                        % for pdflatex
\pdftrailerid{}                       % for pdflatex
\pdfsuppressptexinfo=-1               % for pdflatex
\else
% Here, the tex processor is unknown. 
\fi%pdftex
\fi%xetex
\fi%luatex



\hypersetup{
  pdfinfo={
    Author      ={Ernst Reissner},
    Title       ={Special and common aspects of pdf generators },
    CreationDate={unknown},
    ModDate     ={unknown},
    Producer    ={unknown},
    Subject     ={pdf generators},
    Keywords    ={LaTeX;pdflatex; xelatex; lualatex}
  }
}


\else
\usepackage[dvipdfmx]{xcolor}
\usepackage[dvipdfmx]{graphicx}
\@ifpackageloaded{tex4ht}{%
  % \usepackage[tex4ht]{hyperref}
  \setboolean{texFhtLoaded}{true}%
}{%
%\usepackage[dvipdfmx]{hyperref}
}
\usepackage{bmpsize}% not for xelatex; lualatex and newer pdflatex
\fi

\usepackage{import}
%\usepackage[keeptemps, depythontex, runall=false, fvextfile=55]{pythontex}%rerun=always, 


\usepackage[nottoc, numindex, numbib]{tocbibind}

%\usepackage{latex-bnf}
\title{Special and common aspects of pdf generators }
\author{Ernst Reissner (rei3ner@arcor.de)}
\date{}

\newcommand{\pdflatex}{\texttt{pdflatex}}
\newcommand{\lualatex}{\texttt{lualatex}}
\newcommand{\xelatex}{\texttt{xelatex}}


\begin{document}

\maketitle
\tableofcontents
%\listoffigures
\listoftables
%\lstlistoflistings%

\section{Introduction}

This document is created with \lualatex{} or that like 
with output format 
\ifpdf%
PDF%
\else
DVI% TBD: take XDV into account also 
\fi.
The package \texttt{tex4ht} 
is \ifthenelse{\boolean{texFhtLoaded}}{}{not} loaded. 

This document is about features the three generators on latex, 
\pdflatex, \lualatex{} and \xelatex{} have in common 
and discusses also aspects under which they are specific. 
These programs are just underlying tex engines preloading the the \LaTeX{} format. 
The names of the underlying engines just drop the inner sylable "`la"' 
For PdfTeX there is a user manual~\cite{PdfTexUsr}, 
for LuaTeX there is a reference manual~\cite{LuaTexRef} and 
for XeTeX a reference guide~\cite{XeTexRef}. 
Note that in fact we use a variant of luatex, called LuaHBTeX. 

This document is valid for versions of the underlying tex engine 
as given in Table~ref{tab:versions}. 


\begin{longtable}{|ll|}
  \toprule
  converter & version \\
  \midrule
  \midrule
  \endfirsthead%
  \bottomrule
  \caption{\label{tab:versions} Converters and the version this document refers to }
  \endlastfoot%
  \pdflatex{}  & pdfTeX 3.141592653-2.6-1.40.24 \\
  \xelatex{}   & XeTeX 3.141592653-2.6-0.999994 \\
  \lualatex{}  & LuaHBTeX, Version 1.15.0 \\
\end{longtable}

The first aspect we cover are the options (among those to display the version). 

\section{The options}\label{sec:options}

We start with a synopsis of the options. 
Table~\ref{tab:latexOptionsCommon} shows options common to all, 
or all but one of the converters under consideration. 
Note that in contrast to the other converters, 
\lualatex{} defines options starting with \texttt{-{}-} 
but it can also process the options if given with a single dash also. 
This is to allow giving options valid for all given converters. 


{\footnotesize
\begin{longtable}{|lll|}
\toprule
option & excluded & explanation \\
\midrule
\midrule
\endfirsthead%
\bottomrule
\caption{\label{tab:latexOptionsCommon} Common options, with up to one exception }
\endlastfoot%
\texttt{(-)-cnf-line=STRING}       & ---         & parse STRING as a configuration file line \\
\texttt{(-)-draftmode}             & \xelatex{}  & switch on draft mode (generates no output PDF) \\
\texttt{(-)-[no-]file-line-error}  & ---         & disable/enable file:line:error style messages \\
\texttt{-fmt=FMTNAME}              & ---         & use FMTNAME instead of program name or a \%\& line\footnote%
{in fact for \lualatex{} the explanation deviates a bit: \texttt%
{-{}-fmt=FORMAT\@: load the format file FORMAT}} \\
\texttt{(-)-ini}                   & ---         & for dumping formats \\
\texttt{(-)-halt-on-error}         & ---         & stop processing at the first error \\
\texttt{(-)-help}                  & ---         & display this help and exit \\
\texttt{(-)-version}               & ---         & output version information and exit \\
\texttt{-8bit}                     & \lualatex{} & make all characters printable by default\footnote%
{for \xelatex: don't use \^{}\^{}X sequences} \\
\texttt{(-)-interaction=STRING}    & ---         & set interaction mode \\
&& (\texttt{STRING=batchmode/nonstopmode/scrollmode/errorstopmode}) \\
\texttt{(-)-jobname=STRING}        & ---         & set the job name to STRING \\
\texttt{(-)-kpathsea-debug=NUMBER} & ---         & set path searching debugging flags \\
                                   &             & according to the bits of NUMBER \\
\texttt{(-)-[no-]mktex=FMT}        & ---         & disable/enable mktexFMT generation\footnote%
{\texttt{(FMT=tex/tfm/pk)} for \pdflatex; else \texttt{(FMT=tex/tfm)}} \\
\texttt{-mltex}                    & \lualatex{} & enable MLTeX extensions such as \texttt{\textbackslash{}charsubdef} \\
\texttt{(-)-output-comment=STRING} & ---         & use STRING for DVI file comment instead of date \\
                                   &             & (no effect for PDF)\footnote%
{For \xelatex{} it is XDV instead of DVI and the remark \texttt{(no effect for PDF)} is missing} \\
\texttt{(-)-output-directory=DIR}  & ---         & use existing DIR as the directory to write files in \\
\texttt{(-)-output-format=FORMAT}     & \xelatex{}  & use FORMAT for job output; FORMAT is `dvi' or `pdf'\footnote%
{\xelatex{} offers option \texttt{-no-pdf} instead. } \\
\texttt{[-no]-parse-first-line}    & \lualatex{} & disable/enable parsing of first line of input file \\
\texttt{(-)-progname=STRING}       & ---         & set program (and fmt) name to STRING\footnote%
{\lualatex{} does not mention \texttt{(and fmt)}} \\
\texttt{(-)-recorder}              & ---         & enable filename recorder \\
\texttt{(-)-[no-]shell-escape}     & ---         & disable/enable \texttt{\textbackslash{}write18{SHELL COMMAND}}\footnote%
{For \lualatex{} the explanation is \texttt{disable/enable system commands}} \\
\texttt{(-)-shell-restricted}      & ---         & enable restricted \texttt{\textbackslash{}write18}\footnote%
{For \lualatex{} the explanation is \texttt{restrict system commands to a list of commands given in texmf.cnf}} \\
\texttt{-src-specials}             & \lualatex{} & insert source specials into the DVI file \\
\texttt{-src-specials=WHERE}       & \lualatex{} & insert source specials in certain places of
                          the DVI/XDV\footnote{DVI for \pdflatex; XDV for \xelatex} file. \\
\texttt{(-)-synctex=NUMBER}        & ---         & generate SyncTeX data for previewers\footnote%
{Explanation differs for \lualatex} \\ 
                        

%-translate-file=TCXNAME (ignored)
           
\end{longtable}
} %\small


To ensure that a configuration works for all of these converters, 
the options must be restricted to those listed in Table~~\ref{tab:latexOptionsCommon} 
and not assigned a restriction. 

If one really needs functionality special for a certain converter, 
one has further options at disposal. 
Table~\ref{tab:latexOptionsPdflatex} presents options specific to \pdflatex{}, 
Table~\ref{tab:latexOptionsLualatex} Options specific to \lualatex{} and 
Table~\ref{tab:latexOptionsXelatex} Options specific to \xelatex. 





pdflatex: 

\begin{verbatim}
Usage: pdftex [OPTION]... [TEXNAME[.tex]] [COMMANDS]
   or: pdftex [OPTION]... \FIRST-LINE
   or: pdftex [OPTION]... &FMT ARGS
  Run pdfTeX on TEXNAME, usually creating TEXNAME.pdf.
  Any remaining COMMANDS are processed as pdfTeX input, after TEXNAME is read.
  If the first line of TEXNAME is %&FMT, and FMT is an existing .fmt file,
  use it.  Else use `NAME.fmt', where NAME is the program invocation name,
  most commonly `pdftex'.

  Alternatively, if the first non-option argument begins with a backslash,
  interpret all non-option arguments as a line of pdfTeX input.

  Alternatively, if the first non-option argument begins with a &, the
  next word is taken as the FMT to read, overriding all else.  Any
  remaining arguments are processed as above.

  If no arguments or options are specified, prompt for input.
\end{verbatim}

  {\small
  \begin{longtable}{|ll|}
  \toprule
  option & explanation \\
  \midrule
  \midrule
  \endfirsthead%
  \bottomrule
  \caption{\label{tab:latexOptionsPdflatex} Options specific to \pdflatex{} }
  \endlastfoot%
  \texttt{-enc}                    & Enable encTeX extensions such as \texttt{\textbackslash{}mubyte} \\
  \texttt{-etex}                   & enable e-TeX extensions \\
  \texttt{-ipc}                    & send DVI output to a socket as well as the usual output file \\
  \texttt{-ipc-start}              & as \texttt{-ipc}, and also start the server at the other end \\
  \texttt{-translate-file=TCXNAME} & use the TCX file \texttt{TCXNAME} \\
  \end{longtable}
  }



{\small
\begin{longtable}{|ll|}
\toprule
option & explanation \\
\midrule
\midrule
\endfirsthead%
\bottomrule
\caption{\label{tab:latexOptionsLualatex} Options specific to \lualatex{} }
\endlastfoot%
\texttt{-{}-credits}                    & Display credits and exit. \\
\texttt{-{}-debug-format}               & enable format debugging \\
\texttt{-{}-[no-]file-line-error-style} & aliases of \texttt{-{}-[no-]file-line-error} \\
\texttt{-{}-lua=FILE}                   & Load and execute a lua initialization script. \\
\texttt{-{}-nosocket}                   & Disable the Lua socket library. \\
\texttt{-{}-safer}                      & Disable easily exploitable Lua commands. \\
\texttt{--utc}                          & Init time to UTC \\
\midrule
\texttt{--luaonly}                      & run a lua file, then exit \\
\texttt{--luaconly}                     & byte-compile a lua file, then exit \\
\texttt{--luahashchars}                 & the bits used by current Lua interpreter for strings hashing \\
\end{longtable}
}


{\small
\begin{longtable}{|ll|}
\toprule
option & explanation \\
\midrule
\midrule
\endfirsthead%
\bottomrule
\caption{\label{tab:latexOptionsXelatex} Options specific to \xelatex{} }
\endlastfoot%
\texttt{-etex}                   & enable e-TeX extensions \\
\texttt{-output-driver=CMD}      & use CMD as the XDV-to-PDF driver instead of xdvipdfmx \\
\texttt{-no-pdf}                 & generate XDV (extended DVI) output rather than PDF \\
\texttt{-papersize=STRING}       & set PDF media size to STRING \\
\end{longtable}
}

\section{The options for miktex}

Well this section is preliminary only. 
It turned out that Section{ref:options} is valid only for distribution texlive. 
So in this section we venture to find out the options for the other big distribution, 
miktex. 
We shall also investigate whether there are further distributions. 

Whereas the description~\cite{Texlive} seems not to mention the options explicitly, 
the MiKTeX manual~\cite{MiKTeX} describes each program in section II, 6, 
in particular also the tex converters. 
This is the source of the following tables. 


The first thing observed is that, for miktex all options start with two dashes, 
whereas for texlive this is the case only for luatex. 
One has to clarify, whether the maven latex plugin under consideration really works for miktex. 

\begin{longtable}{|lll|}
  \toprule
  option & excluded & explanation \\
  \midrule
  \midrule
  \endfirsthead%
  \bottomrule
  \caption{\label{tab:latexOptionsCommonMiktex} Common options, with up to one exception }
  \endlastfoot%
  \texttt{-{}-alias=name}            & ---         & Pretend to be program name, \dots\footnote%
{Using this option is equivalent to copying the program file to name and invoking name.} \\
\texttt{-{}-aux-directory=dir}       & ---         & 
Set \texttt{dir} as the directory to write auxiliary files to. \\
\texttt{-{}-buf-size=n}              & \lualatex{} & Set the the maximum number of characters \dots \\
\texttt{-{}-c-style-errors}          & ---         & Change the way, error messages are printed. \\
\texttt{-{}-disable-8bit-chars}      & \lualatex{} & Make only 7-bit characters printable. \\
\texttt{-{}-disable/enable-installer} & ---         & Disable/Enable automatic installation of packages. \\
\texttt{-{}-disable-write18}         & ---         & Disable the \texttt{\textbackslash{}write18\{command\}} construct.\\
\texttt{-{}-enable-write18}          & ---         & Fully enable the \texttt{\textbackslash{}write18\{command\}} construct. \footnote%
{Corresponds roughly to \texttt{(-)-shell-escape} in texlive. } \\
\texttt{-{}-restrict-write18}        & ---         & Partially enable the \texttt{textbackslash{}write18{command}} construct. \\
%\texttt{-{}-parse-first-line}   & \lualatex{} & Disable checking \dots the first line \dots \\
\texttt{-{}-[dont-]parse-first-line}        & \lualatex{} & [Dont p|P]arse first line of input file under definite conditions\footnote%
{Similar for texlive. Note that there is also a converse option. } \\
\texttt{-{}-draftmode}               & \xelatex{}  & switch on draft mode (generates no output PDF)\footnote%
{Some differences in formulation between the converter and also between distributions} \\
\texttt{-{}-enable-8bit-chars}       & \lualatex{} & Make all characters printable. \\
\texttt{-{}-enable-etex}             & \lualatex{} & Enable eTeX extensions. \\
%\texttt{-{}-enable-installer}        & ---         & Enable automatic installation of packages. \\
\texttt{-{}-enable-mltex}            & \lualatex{} & Enable MLTeX extensions such as \texttt{\textbackslash{}charsubdef}. \\
\texttt{-{}-error-line=n}            & \lualatex{} & Set the width of context lines on \dots error messages. \\
\texttt{-{}-extra-mem-bot=n}         & \lualatex{} & Set the extra size \dots{} for large data structures \dots \\
\texttt{-{}-extra-mem-top=n}         & \lualatex{} & Set the extra size (in memory words) for chars, tokens, \dots. \\
\texttt{-{}-font-max=n}              & \lualatex{} & Set the maximum internal font number. \\
\texttt{-{}-font-mem-size=n}         & \lualatex{} & Set the size, in TeX memory words, of the font memory. \\
\texttt{-{}-half-error-line=n}       & \lualatex{} & Set the width of first lines of contexts in terminal error messages. \\
\texttt{-{}-halt-on-error}           & ---         & Quit after the first error. \\
\texttt{-{}-hash-extra=n}            & \lualatex{} & Set the extra space for the hash table of control sequences \dots \\
\texttt{-{}-help}                    & ---         & Give help and exitt\footnote%
{The same as for texlive. }.. \\
\texttt{-{}-hhelp}                   & \lualatex{} & manual page in an HTML Help window\footnote%
{This option is only available on Windows systems.} \\
\texttt{-{}-include-directory=dir}   & ---         & Add the directory \texttt{dir} to [those] to be searched for input files. \\
\texttt{-{}-initialize}              & ---         & Become the INI variant of the program. \\
\texttt{-{}-interaction=mode}        & ---         & Set the interaction mode. \\
&& (\texttt{STRING=batchmode/nonstopmode/scrollmode/errorstopmode}) \\
\texttt{-{}-job-name=name}           & ---         & Set the name of the job (\texttt{\textbackslash{}jobname}). \\
\texttt{-{}-job-time=file}           & \lualatex{} & Set the time-stamp of all output files equal to file's time-stamp. \\
\texttt{-{}-main-memory=n}           & \lualatex{} & Change the total size \dots{} of the main memory array. \\
\texttt{-{}-max-in-open=n}           & \lualatex{} & Set the maximum number of input files \dots \\
\texttt{-{}-max-print-line=n}        & \lualatex{} & Set the width of longest text lines output. \\
\texttt{-{}-max-strings=n}           & \lualatex{} & Set the maximum number of strings. \\
\texttt{-{}-nest-size=n}             & \lualatex{} & Set the maximum number of semantic levels simultaneously active. \\
\texttt{-{}-no-c-style-errors}       & ---         & Don't change the way, error messages are printed. \\
\texttt{-{}-output-directory=dir}    & ---         & Write output files in \texttt{dir}\footnote%
{Similar as for texlive. }. \\
\texttt{-{}-output-format=format}    & \xelatex{}  & Use format for job output (one of: dvi, pdf)\footnote%
{\pdflatex{} and \lualatex{} differ a bit in text. Seems similar to texlive. }. \\
\texttt{-{}-param-size=n}            & \lualatex{} & Set the the maximum number of simultaneous macro parameters. \\
\texttt{-{}-pool-free=n}             & \lualatex{} & Set the minimum pool space left after loading the format. \\
\texttt{-{}-pool-size=n}             & \lualatex{} & Set the maximum number of characters in strings, \dots \\
\texttt{-{}-quiet}                   & \lualatex{} & Suppress all output, except errors. \\
\texttt{-{}-record-package-usages=file} & \lualatex{} & Record all package usages and write them into \texttt{file}. \\
\texttt{-{}-recorder}                & ---         & Enable the file name recorder\footnote%
{The same as in texlive. }. \\
\texttt{-{}-save-size=n}             & \lualatex{} & Set the the amount of space for saving values outside of current group. \\
\texttt{-{}-src-specials}            & \lualatex{} & Embed source file information (source specials) in the DVI file\footnote%
{Similar as in texlive. }. \\
\texttt{-{}-stack-size=n}            & \lualatex{} & Set the maximum number of simultaneous input sources. \\
\texttt{-{}-string-vacancies=n}      & \lualatex{} & Set the minimum number of characters \dots \\
\texttt{-{}-synctex=n}               & ---         & Generate SyncTeX data for previewers\footnote%
{Explanation with more detail than for texlive. } \\
\texttt{-{}-time-statistics}         & \lualatex{} & Show processing time statistics. \\
\texttt{-{}-trace[=tracestreams]}    & \lualatex{} & Enable trace messages. \\
\texttt{-{}-trie-size=n}             & \lualatex{} & Set the amount of space for hyphenation patterns. \\
\texttt{-{}-undump=name}             & ---         & Use name as the name of the format to be used, \dots \\
\texttt{-{}-version}                 & ---         & Show version information and exit\footnote%
{The same as for texlive. }. \\
\end{longtable}

Strange, there are \texttt{-{}-enable-etex} and \texttt{-{}-enable-mltex} 
but no way to disable. 
Maybe disable is the default. 

\begin{verbatim}
  miktex-pdftex [option...] [[file] | [\command...]]
\end{verbatim}

\begin{longtable}{|ll|}
  \toprule
  option & explanation \\
  \midrule
  \midrule
  \endfirsthead%
  \bottomrule
  \caption{\label{tab:latexOptionsPdflatexMiktex} Options specific to \pdflatex{} }
  \endlastfoot%
  \texttt{--enable-enctex} & Enable encTeX extensions such as \texttt{\textbackslash{}mubyte}\footnote%
  {Corresponds with \texttt{-enc} in texlive. }. \\
  texttt{--tcx=tcxname}    & Use the tcxname translation table \dots \\
  \end{longtable}




\begin{verbatim}
  miktex-luatex [option...] [[command...] | [file]]

The following options are ignored:

--8bit, --etex, --parse-first-line, --no-parse-first-line
These are always on.

--default-translate-file=tcxname, --translate-file=tcxname
These are always off.
\end{verbatim}

\begin{longtable}{|ll|}
  \toprule
  option & explanation \\
  \midrule
  \midrule
  \endfirsthead%
  \bottomrule
  \caption{\label{tab:latexOptionsLualatexMiktex} Options specific to \lualatex{} }
  \endlastfoot%
  \texttt{-{}-credits}                    & Display credits and exit. \footnote%
  {The same as for texlive. }. \\
  \texttt{-{}-debug-format}               & Enable format debugging\footnote%
  {The same as for texlive. }. \\
  % \texttt{-{}-[no-]file-line-error-style} & aliases of --[no-]file-line-error \\
  \texttt{-{}-lua=FILE}                   & load and execute a lua initialization script\footnote%
  {The same as for texlive. }.   \\
  \texttt{-{}-nosocket}                   & Disable the Lua socket library. \\
  \texttt{-{}-safer}                      & Disable easily exploitable Lua commands\footnote%
  {The same as for texlive. }.   \\
  \texttt{--utc}                          & Init time to UTC\footnote%
  {The same as for texlive. }.  \\
  \midrule
  \texttt{-{}-[no-]mktex=fmt} & Enable/Disable \texttt{fmt} generation, where \texttt{fmt} must be either \texttt{tex} or \texttt{tfm}. \\
  \texttt{-{}-output-comment=string} & Use \texttt{string} for DVI file comment instead of date. \\
  \midrule
  \texttt{--luaonly}                      & Start LuaTeX as a Lua interpreter\footnote%
  {Could be the same as for texlive. }.  \\
  \texttt{--luaconly}                     & byte-compile a lua file, then exit\footnote%
  {Could be the same as for texlive. }. \\
  % \texttt{--luahashchars}                 & the bits used by current Lua interpreter for strings hashing \\
  \end{longtable}


  
  
  

\begin{verbatim}
  miktex-xetex [option...] [[file] | [\command...]]
\end{verbatim}



\begin{longtable}{|ll|}
  \toprule
  option & explanation \\
  \midrule
  \midrule
  \endfirsthead%
  \bottomrule
  \caption{\label{tab:latexOptionsXelatexMiktex} Options specific to \xelatex{} }
  \endlastfoot%
  % \texttt{-etex}                   & enable e-TeX extensions \\
  \texttt{-output-driver=CMD}      & use CMD as the XDV-to-PDF driver instead of xdvipdfmx \\
  \texttt{-no-pdf}                 & generate XDV (extended DVI) output rather than PDF \\
  \texttt{-papersize=STRING}       & set PDF media size to STRING \\
  \end{longtable}

  Same options as for texlive, except \texttt{-etex} which is specific to texlive. 


\section{To be clarified}\label{sec:TBC}

Note that \xelatex{} works as the other engines for PDF, 
whereas it has XDV instead of DVI as alternative format. 
Thus the format DVI is ignored. 
See the manual. 





\bibliographystyle{alpha}
\bibliography{../lit}{}% chktex 11 


\end{document}


