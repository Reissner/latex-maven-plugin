\RequirePackage[l2tabu, orthodox]{nag}
\documentclass[a4paper]{article}
\synctex=1
\usepackage{iftex}

\iftutex%
\usepackage{fontspec}
\else
\usepackage[utf8]{inputenc}
\usepackage[T1]{fontenc}
\fi

%\pdfinfoomitdate1
%\pdfsuppressptexinfo-1
%\pdftrailerid{}
\ifpdf%
\usepackage{hyperref}
\ifLuaTeX%
% for lualatex
\pdfvariable minorversion=7% chktex 1
% omit CreationDate and ModDate keys.
\pdfvariable suppressoptionalinfo 767% chktex 1
% no adding to the trailer dictionary.
\pdfvariable trailerid{}% chktex 1
\pdfvariable suppressoptionalinfo -1% chktex 1
\else
\ifXeTeX%
% for xelatex
\special{pdf:minorversion 7}
% TBD: find a way to express pdfinfoomitdate: necessary? 
\special{pdf:trailerid []}
\else
\ifPDFTeX%
\pdfminorversion=7         % for pdflatex
% omit CreationDate and ModDate keys.
% not before pdfTeX 3.14159265-2.6-1.40.17
\pdfinfoomitdate=1                   % for pdflatex
% no adding to the trailer dictionary.
%\pdftrailer=0                        % for pdflatex
\pdftrailerid{}                       % for pdflatex
\pdfsuppressptexinfo=-1               % for pdflatex
\else
% Here, the tex processor is unknown. 
\fi%pdftex
\fi%xetex
\fi%luatex

\hypersetup{
  pdfinfo={
    Author      ={Ernst Reissner},
    Title       ={The recorder option for latex processors},
    CreationDate={unknown},
    ModDate     ={unknown},
    Producer    ={unknown},
    Subject     ={dependency management in latex documents},
    Keywords    ={LaTeX;latexmk}
  }
}
\else
\ifXeTeX%
\usepackage{hyperref}
\else
\usepackage[dvipdfmx]{hyperref}
% lualatex: without [dvipdfmx] option did not find converter dvi to pdf or to ps
% pdflatex: without [dvipdfmx] option dvips still works, but no converter for pdf
% with [dvipdfmx] all works fine 
\fi%
\fi%ifpdf

\title{The recorder option for latex processors}

\author{Ernst Reißner}
\date{}

\begin{document}

\maketitle
\tableofcontents

\section{Introduction}
This document is about the \texttt{recorder} option available for all modern \LaTeX{} processors, 
among those also \texttt{lualatex}. 
If the \LaTeX{} processor is applied to a file \texttt{xxx.tex} with this option set, 
it writes a file \texttt{xxx.fls} in the FLS format which is described here. 
Essentially, it contains dependency information for that file 
which can be used to decide on rebuilding the document. 
This is mostly to spare processing time 
while still ensuring that the created documents are up-to-date. 
Among others, \texttt{latexmk} uses this information. 
The problem here is, 
that no official information on the FLS format seems available 
and so there is no way but reverse engineering. 

The risk to have an incomplete specification is always present, 
also that the format is updated and the reverse engineered specification is rendered invalid. 
Thus, software based on this specification must be failure tolerant. 
The goal that the result is perfect if no warning is displayed 
forces 


\section{The FLS format}

The FLS format seems to be line based and so is the description. 
The format is described in terms of the possible form of lines. 

In addition, the form of a line is quite restricted. 
It is of the form \texttt{type file}, consisting of line type and a file path. 
The available \texttt{type}s are 
%
\begin{description}
  \item[PWD] The working folder. 
  Seemingly this is the folder containing the latex file \texttt{xxx.tex} to be processed. 
  This type of line occurs once and is the first line. 
  We refer to it as the \emph{current directory}. 
  \item[INPUT] All input files. Among those \texttt{xxx.tex}, \texttt{xxx.aux}, 
  Possibly graphic files and files related with fonts. 
  \item[OUTPUT] Almost all output files. 
  Among those \texttt{xxx.log}, \texttt{xxx.tex}, \texttt{xxx.pdf}
  but also very strange cache files. 
  Note that \texttt{xxx.fls} itself and also the file \texttt{xxx.synctex.gz} are not mentioned. 
  Maybe there are more files not mentioned. 
\end{description}

The line \texttt{PWD} is first and there is a single one specifying the working folder. 
Paths are not restricted to the local folder but may point also into the latex installation. 
Paths may be absolute and relative. 
Relative paths seem to be relative to the working folder. 
TBD\@: do research on paths with blanks. 

Note that a file may occur as input and output like \texttt{xxx.aux}. 

The relative paths either start with \texttt{./} or with \texttt{../}. 
In the first case, it seems that these paths refer to files in a subdirectory 
of the current directory, not in the second. 

The absolute paths either refer to the installation or to a cache. 
In my linux box, the cache is at \texttt{/home/ernst/.cache/texmf/}, so it is in my home directory. 

The installation directory of a \TeX{} Live installation, 
is given by \texttt{kpsewhich latex.ltx}, 
just eliminating three containment levels from the path. 

The rest of the input lines of the fls files are the following on my linux box: 
%
\begin{verbatim}
INPUT /var/lib/texmf/web2c/luahbtex/lualatex.fmt
INPUT /var/lib/texmf/fonts/map/pdftex/updmap/pdftex.map
INPUT /etc/texmf/web2c/mktex.cnf
\end{verbatim}

\end{document}