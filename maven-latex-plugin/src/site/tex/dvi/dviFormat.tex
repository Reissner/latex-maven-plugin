\RequirePackage[l2tabu, orthodox]{nag}
\documentclass[10pt, a4paper]{article}
\synctex=1

\usepackage{iftex}

\iftutex%
  \usepackage{fontspec}
\else
  \usepackage[utf8]{inputenc}
  \usepackage[T1]{fontenc}
\fi

\usepackage{fancyvrb}

\usepackage[nottoc,numbib]{tocbibind}

\ifpdf%
  \usepackage{hyperref}


  \ifLuaTeX%
    % for lualatex
    \pdfvariable minorversion=7
    % omit CreationDate and ModDate keys.
    \pdfvariable suppressoptionalinfo 767% chktex 1
    % no adding to the trailer dictionary.
    \pdfvariable trailerid{}% chktex 1
    \pdfvariable suppressoptionalinfo -1% chktex 1
  \else
    \ifXeTeX%
      % for xelatex
      \special{pdf:minorversion 7}
      % TBD: find a way to express pdfinfoomitdate: necessary? 
      \special{pdf:trailerid []}
    \else
      \ifPDFTeX%
        \pdfminorversion=7         % for pdflatex
        % omit CreationDate and ModDate keys.
        % not before pdfTeX 3.14159265-2.6-1.40.17
        %\pdfinfoomitdate=1                   % for pdflatex
        % no adding to the trailer dictionary.
        %\pdftrailer=0                        % for pdflatex
        \pdftrailerid{}                       % for pdflatex
        \pdfsuppressptexinfo=-1               % for pdflatex
      \else
        % Here, the tex processor is unknown. 
      \fi%pdftex
    \fi%xetex
  \fi%luatex

  % removes the /CreationDate and /ModDate 
  % superfluous as we override anyway
  %\pdfinfoomitdate1 
  % pdfsuppressptexinfo-1 is the same as pdfsuppressptexinfo15
  % 1 -> PTEX.Fullbanner
  % 2 -> PTEX.FileName
  % 4 -> PTEX.PageNumber
  % 8 -> PTEX.InfoDict (/Producer /Creator /CreationDate /ModDate /Trapped)
  % The following is to control metadata: 
  % - erase docdata and set the ones needed thereafter
  % - erase trailer to make repreoducible 
  %\pdfsuppressptexinfo-1
  % CAUTION: This is not recommended as the docs are no longer individual. 
  %\pdftrailerid{}% removed by nopdftrailerid

  \hypersetup{
    pdfinfo={
      Author      ={Ernst Reissner},
      Title       ={The DVI-format and the program DVItype},
      CreationDate={unknown},
      ModDate     ={unknown},
      Producer    ={unknown},
      Subject     ={DVI and DVItype},
      Keywords    ={LaTeX;dvi;DVItype}
    }
  }
\else
  \ifXeTeX%
    \usepackage{hyperref}
  \else
    \usepackage[dvipdfmx]{hyperref}
    % lualatex: without [dvipdfmx] option did not find converter dvi to pdf or to ps
    % pdflatex: without [dvipdfmx] option dvips still works, but no converter for pdf
    % with [dvipdfmx] all works fine 
  \fi%
\fi%ifpdf

\title{The DVI-format and the program \texttt{DVItype}}
\author{Ernst Reissner (rei3ner@arcor.de)}
\date{}

\begin{document}

\maketitle
\tableofcontents

\section{Introduction}
This document describes the DVI file format 
traditionally used by \LaTeX{} 
and still in use with \texttt{htlatex} and that like. 

This self written document is strictly speaking obsolete: 
The ultimate reference is~\cite{DviType}. 
Nevertheless, this document may serve as a description of a very simple subset of the language. 

A \texttt{DVI} file is a stream of 8-bit bytes, which may be regarded as a
series of commands in a machine-like language. The first byte of each command
is the operation code, and this code is followed by zero or more bytes
that provide parameters to the command. The parameters themselves may consist
of several consecutive bytes; 
for example, the `\texttt{set\_rule}' command has two
parameters, each of which is four bytes long. Parameters are usually
regarded as non-negative integers; but four-byte-long parameters,
and shorter parameters that denote distances, can be
either positive or negative. Such parameters are given in two's complement
notation. For example, a two-byte-long distance parameter has a value between
$-2^{15}$ and $2^{15}-1$.


A \texttt{DVI} file consists of a ``preamble,'' 
followed by a sequence of one or more ``pages,'' 
followed by a ``postamble''. 
The preamble is simply a \texttt{pre} command, 
with its parameters that define the dimensions used in the file; 
this must come first.  
Each ``page'' consists of a \texttt{bop} command,
followed by any number of other commands that tell where characters are to
be placed on a physical page, followed by an \texttt{eop} command. 
The pages appear in the order that they were generated, 
not in any particular numerical order. 
If we ignore \texttt{nop} commands and \texttt{fnt\_def} commands
(which are allowed between any two commands in the file), 
each \texttt{eop} command is immediately followed by a \texttt{bop} command, 
or by a \texttt{post} command; 
in the latter case, there are no more pages in the file, 
and the remaining bytes form the postamble.  
Further details about the postamble will be explained later.

Some parameters in \texttt{DVI} commands are ``pointers''. 
These are four-byte quantities that give the location number 
of some other byte in the file; 
the first byte is number~0, then comes number~1, and so on. 
For example, one of the parameters of a \texttt{bop} command 
points to the previous \texttt{bop};
this makes it feasible to read the pages in backwards order, 
in case the results are being directed to a device 
that stacks its output face up. 
Suppose the preamble of a \texttt{DVI} file occupies bytes 0 to 99. 
Now if the first page occupies bytes 100 to 999, say, 
and if the second page occupies bytes 1000 to 1999, 
then the \texttt{bop} that starts in byte 1000
points to 100 and the \texttt{bop} that starts in byte 2000 points to 1000. 
(The very first \texttt{bop}, 
i.e., the one that starts in byte 100, has a pointer of $-1$.)


The  \texttt{DVI} format is intended to be both 
compact and easily interpreted by a machine. 
Compactness is achieved by making most of the information implicit 
instead of explicit. 
When a \texttt{DVI}-reading program reads the commands for a page, 
it keeps track of several quantities: 
%
\begin{itemize}
\item
The current font $f$ is an integer; 
this value is changed only by \texttt{fnt} and \texttt{fnt\_num} commands. 
\item
The current position on the page 
is given by two numbers called the horizontal and vertical coordinates,
$h$ and $v$. 
Both coordinates are zero in the upper left corner of the page; 
moving to the right corresponds to increasing the horizontal coordinate, and
moving down corresponds to increasing the vertical coordinate. 
Thus, the coordinates are essentially Cartesian, 
except that vertical directions are flipped; 
the Cartesian version of $(h,v)$ would be $(h,-v)$.  
\item
The current spacing amounts 
are given by four numbers $w$, $x$, $y$, and $z$,
where $w$ and~$x$ are used for horizontal spacing and where $y$ and~$z$
are used for vertical spacing. 
\item
There is a stack containing $(h,v,w,x,y,z)$ values; 
the \texttt{DVI} commands \texttt{push} and \texttt{pop} are used 
to change the current level of operation. 
Note that the current font~$f$ is not pushed and popped; 
the stack contains only information about positioning.
\end{itemize}

The values of $h$, $v$, $w$, $x$, $y$, and $z$ are signed integers having up
to 32 bits, including the sign. Since they represent physical distances,
there is a small unit of measurement such that increasing $h$ by~1 means
moving a certain tiny distance to the right. The actual unit of
measurement is variable, as explained below. 

Here is a list of all the commands that may appear in a \texttt{DVI} file. 
Each command is specified by its symbolic name (e.g., \texttt{bop}), 
its opcode byte (e.g., 139), and its parameters (if any). 
The parameters are followed by a bracketed number 
telling how many bytes they occupy; for example,
`$p[4]$' means that parameter $p$ is four bytes long. 

\section{The Op-Codes}\label{sec:opCodes}

\begin{itemize}

\item[0--127] \texttt{set\_char\_$<d>$ $d$} where $d\in\{0,\dots,127\}$. 

Typeset character number~$d$ from font~$f$ 
such that the reference point of the character is at $(h,v)$. 
Then increase $h$ by the width of that character. 
Note that a character may have zero or negative width, 
so one cannot be sure that $h$ will advance after this command; 
but $h$ usually does increase.

\item[128--131] \texttt{set\_$<d>$ $128+d$ $c[d]$} where $d\in\{1,\dots,4\}$. 
Same as \texttt{set\_char\_0}, except that character number~$c$ is typeset. 
Note that $d$ determines the number of characters. 
\texttt{set\_$2$} is intended for oriental languages 
with a script with a lot of characters. 


\item[132] \texttt{setrule $132$ $a[4]$ $b[4]$}. 
Typeset a solid black rectangle of height $a$ and width $b$, 
with its bottom left corner at $(h,v)$. 
Then set $h:=h+b$. If either $a<=0$ or $b<=0$, nothing should be typeset. 
Note that if $b<0$, the value of $h$ will decrease 
even though nothing else happens. 
Programs that typeset from \texttt{DVI} files 
should be careful to make the rules line up carefully 
with digitized characters, 
as explained in connection with the \texttt{rule\_pixels} subroutine below.

\item[137] \texttt{put$<d>$ $132+d$ $c[d]$} where $d\in\{1,\dots,4\}$. 
Typeset character number~$c$ from font~$f$
such that the reference point of the character is at $(h,v)$. 
(The `put' commands are exactly like the `set' commands, 
except that they simply put out a character or a rule 
without moving the reference point afterwards.)


\item[137] \texttt{put\_rule $137$ $a[4]$ $b[4]$}. 
Same as \texttt{set\_rule}, except that $h$ is not changed. 

\item[138] \texttt{nop}. 
No operation, do nothing. Any number of \texttt{nop}'s
may occur between \texttt{DVI} commands, 
but a \texttt{nop} cannot be inserted 
between a command and its parameters or between two parameters. 

\item[139] \texttt{bop $139$ $c_0[4]$ $c_1[4]$ $\ldots$ $c_9[4]$ $p[4]$}. 
Indicates the beginning of a page: 
Set $(h,v,w,x,y,z):=(0,0,0,0,0,0)$ and set the stack empty. 
Set the current font $f$ to an undefined value.  
The ten $c_i$ parameters can be used to identify pages, 
if a user wants to print only part of a \texttt{DVI} file; 
\TeX82 gives them the values of \texttt{count0} $\ldots$ \texttt{count9} 
at the time \texttt{shipout} was invoked for this page.  
The parameter $p$ points to the previous \texttt{bop} command in the file, 
where the first \texttt{bop} has $p=-1$.

\item[140] \texttt{eop $140$} 
End of page: 
Print what you have read since the previous \texttt{bop}. 
At this point the stack should be empty. 
(The \texttt{DVI}-reading programs that drive most output devices 
will have kept a buffer of the material 
that appears on the page that has just ended. 
This material is largely, but not entirely, 
in order by $v$ coordinate and (for fixed $v$) by $h$~coordinate; 
so it usually needs to be sorted into some order that is
appropriate for the device in question. 
\texttt{DVItype} does not do such sorting.)


\item[141] \texttt{push $141$}. 
Push the current values of $(h,v,w,x,y,z)$ onto the top of the stack; 
do not change any of these values. Note that $f$ is not pushed. 


\item[142] \texttt{pop $142$}. 
Pop the top six values off of the stack and assign them to $(h,v,w,x,y,z)$. 
The number of pops should never exceed the number of pushes, 
since it would be highly embarrassing if the stack were empty 
at the time of a \texttt{pop} command.

\item[143--146] \texttt{right$<d>$ $142+d$ $b[d]$} where $d\in\{1,\dots,4\}$. 
Set $h:=h+b$, i.e., move right $b$ units.
The parameter is a signed number in two's complement notation; 
if $b<0$, the reference point actually moves left.

\item[147] \texttt{w0 $147$}. 
Set $h:=h+w$; i.e., move right $w$ units. With luck,
this parameter-less command will usually suffice, because the same kind of motion
will occur several times in succession; the following commands explain how
$w$ gets particular values. 

\item[148--151] \texttt{w$<d>$ $147+d$ $b[d]$} where $d\in\{1,\dots,4\}$. 
Set $w:=b$ and $h:=h+b$. 
The value of $b$ is a signed quantity in two's complement notation. 
This command changes the current $w$~spacing and moves right by $b$.


\item[152] \texttt{x0 $152$}. 
Set $h:=h+x$; i.e., move right $x$ units. The `$x$'
commands are like the `$w$' commands except that they involve $x$ instead
of $w$.

\item[153--156] \texttt{x$<d>$ $152+d$ $b[d]$} where $d\in\{1,\dots,4\}$. 
Set $x:=b$ and $h:=h+b$. 
The value of $b$ is a signed quantity in two's complement notation. 
This command changes the current $x$~spacing and moves right by $b$.


\item[159] \texttt{down$<d>$ $156+d$ $b[d]$} where $d\in\{1,\dots,4\}$. 
Set $v:=v+a$, i.e., move down $a$ units. 
The parameter is a signed number in two's complement notation; 
if $a<0$, the reference point actually moves up.


\item[161] \texttt{y0 $161$}. 
Set $v:=v+y$; i.e., move down $y$ units. 
With luck, this parameter-less command will usually suffice, 
because the same kind of motion will occur several times in succession.

\item[162--165] \texttt{y$<d>$ $161+d$ $a[d]$} where $d\in\{1,\dots,4\}$. 
Set $y:=a$ and $v:=v+a$. 
The value of $a$ is a signed quantity in two's complement notation. 
This command changes the current $y$~spacing and moves down by $a$. 

\item[166] \texttt{z0 $166$}. 
Set $v:=v+z$; i.e., move down $z$ units. 
With luck, this parameter-less command will usually suffice, 
because the same kind of motion will occur several times in succession.
The `\texttt{z}' commands are like the `\texttt{y}' commands 
except that they involve $z$ instead of $y$. 

\item[167--170] \texttt{z$<d>$ $166+d$ $a[d]$} where $d\in\{1,\dots,4\}$. 
Set $y:=a$ and $v:=v+a$. 
The value of $a$ is a signed quantity in two's complement notation. 
This command changes the current $z$~spacing and moves down by $a$. 


\item[171] \texttt{fntnum0 $171$}. 
Set $f:=0$. Font 0 must previously have been
defined by a \texttt{fnt\_def} instruction, as explained below.

\item[172--234] \texttt{fntnum$<d>$ $171+d$ } where $d\in\{1,\dots,63\}$. 
Set $f:=d$. 


\item[235--238] \texttt{fnt$<d>$ $235+d$ $k[d]$} where $d\in\{1,\dots,4\}$. 
Set $f:=k$. 
An application of \texttt{fnt$2$} or higher 
maybe used for fonts with colors and texture information. 

\item[239--242] \texttt{xxx$<d>$ $238+d$ $k[2]$ $x[k]$} 
where $d\in\{1,\dots,4\}$. 
This command is undefined in general; 
it functions as a $(k+2)$-byte \texttt{nop} 
unless special \texttt{DVI}-reading programs are being used. 
\TeX82 generates \texttt{xxx1} when a short enough \texttt{special} appears, 
setting $k$ to the number of bytes being sent. 
It is recommended that $x$ be a string having the form of a keyword 
followed by possible parameters relevant to that keyword.

\item[243--246] \texttt{fntdef$<d>$ $242+d$ 
$k[d]$ $c[4]$ $s[4]$ $d[4]$ $a[1]$ $l[1]$ $n[a+l]$} 
where $d\in\{1,\dots,4\}$. 
Define font $k$; font definitions will be explained shortly.

\item[247] \texttt{pre $247$ $i[1]$ $num[4]$ $den[4]$ $mag[4]$ $k[1]$ $x[k]$}.
Beginning of the preamble; 
this must come at the very beginning of the file. 
Parameters $i$, $num$, $den$, $mag$, $k$, and $x$ are explained below.

\item[248] \texttt{post $248$}. 
Beginning of the postamble, see below. 

\item[249] \texttt{post\_post $249$}. 
Ending of the postamble, see below. 

\item[250--255] undefined currently. 
\end{itemize}

\section{The Preamble}\label{sec:preamble}

The preamble contains basic information about the file as a whole. 
As stated above, there are six parameters:
%
\[
\hbox{i[1]| num[4], den[4], mag[4], k[1], x[k]}
\]
%
The $i$ byte identifies \texttt{DVI} format; 
currently this byte is always set to~2. 
(The value $i=3$ is currently used for an extended format that
allows a mixture of right-to-left and left-to-right typesetting.
Some day we will set $i=4$, when \texttt{DVI} format makes another
incompatible change---perhaps in the year 2048.)

The next two parameters, $num$ and $den$, are positive integers that define
the units of measurement; they are the numerator and denominator of a
fraction by which all dimensions in the \texttt{DVI} file could be multiplied
in order to get lengths in units of $10^{-7}$ meters. (For example, there are
exactly 7227 \TeX\ points in 254 centimeters, and \TeX82 works with scaled
points where there are $2^{16}$ sp in a point, so \TeX82 sets $num=25400000$
and $den=7227\cdot2^{16}=473628672$.)


The $mag$ parameter is what \TeX82 calls \texttt{mag}, i.e., 1000 times the
desired magnification. The actual fraction by which dimensions are
multiplied is therefore $mn/1000d$. Note that if a \TeX\ source document
does not call for any `{true}' dimensions, and if you change it only by
specifying a different \texttt{mag} setting, the \texttt{DVI} file that \TeX\
creates will be completely unchanged except for the value of $mag$ in the
preamble and postamble. (Fancy \texttt{DVI}-reading programs allow users to
override the $mag$~setting when a \texttt{DVI} file is being printed.)

Finally, $k$ and $x$ allow the \.{DVI} writer to include a comment, which is not
interpreted further. The length of comment $x$ is $k$, where $0<=k<256$.


\section{Font definitions}\label{sec:fontDef}

Font definitions for a given font number $k$ contain further parameters
\[
\hbox{$c[4]$ $s[4]$ $d[4]$ $a[1]$ $l[1]$ $n[a+l]$.}
\]
The four-byte value $c$ is the check sum that \TeX\ (or whatever program
generated the \texttt{DVI} file) found in the \texttt{TFM} file for this font;
$c$ should match the check sum of the font found by programs that read
this \texttt{DVI} file.


Parameter $s$ contains a fixed-point scale factor that is applied to the
character widths in font $k$; font dimensions in \texttt{TFM} files and other
font files are relative to this quantity, which is always positive and
less than $2^{27}$. 
It is given in the same units as the other dimensions of the \texttt{DVI} file. 
Parameter $d$ is similar to $s$; it is the ``design size'' 
and (like~$s$) it is given in \texttt{DVI} units. 
Thus, font $k$ is to be used at $mag\cdot s/1000d$ times its normal size.

The remaining part of a font definition gives the external name of the font,
which is an ASCII string of length $a+l$. The number $a$ is the length
of the ``area'' or directory, and $l$ is the length of the font name itself;
the standard local system font area is supposed to be used when $a=0$.
The $n$ field contains the area in its first $a$ bytes.

Font definitions must appear before the first use of a particular font number.
Once font $k$ is defined, it must not be defined again; 
however, we shall see below 
that font definitions appear in the postamble as well as in the pages, 
so in this sense each font number is defined exactly twice, if at all. 
Like \texttt{nop} commands, 
font definitions can appear before the first \texttt{bop}, 
or between an \texttt{eop} and a \texttt{bop}. 

\section{The Postamble}\label{sec:postamble}

The last page in a \texttt{DVI} file is followed by `\texttt{post}'; 
this command introduces the postamble, 
which summarizes important facts that \TeX\ has accumulated about the file, 
making it possible to print subsets of the data with reasonable efficiency. 
The postamble has the form
%$$\vbox{\halign{\hbox{#\hfil}\cr
%  \texttt{post} |p[4]| |num[4]| |den[4]| |mag[4]| |l[4]| |u[4]| |s[2]| |t[2]|\cr
%  $\langle\,$font definitions$\,\rangle$\cr
%  \texttt{post_post} |q[4]| |i[1]| 223's$[{\G}4]$\cr}}$$
Here $p$ is a pointer to the final \texttt{bop} in the file. The next three
parameters, $num$, $den$, and $mag$, are duplicates of the quantities that
appeared in the preamble.

Parameters $l$ and $u$ give respectively the height-plus-depth of the tallest
page and the width of the widest page, in the same units as other dimensions
of the file. These numbers might be used by a \texttt{DVI}-reading program to
position individual ``pages'' on large sheets of film or paper; however,
the standard convention for output on normal size paper is to position each
page so that the upper left-hand corner is exactly one inch from the left
and the top. 
Experience has shown that it is unwise to design \texttt{DVI}-to-printer
software that attempts cleverly to center the output; a fixed position of
the upper left corner is easiest for users to understand and to work with.
Therefore, $l$ and~$u$ are often ignored.

Parameter $s$ is the maximum stack depth 
(i.e., 
the largest excess of \texttt{push} commands over \texttt{pop} commands)
 needed to process this file. 
Then comes $t$, the total number of pages (\texttt{bop} commands) present.

The postamble continues with font definitions, 
which are any number of \texttt{fnt\_def} commands as described above, 
possibly interspersed with \texttt{nop} commands. 
Each font number that is used in the \texttt{DVI} file must be defined
exactly twice: Once before it is first selected by a \texttt{fnt} command, 
and once in the postamble.

@ The last part of the postamble, following the \texttt{post\_post} byte 
that signifies the end of the font definitions, 
contains $q$, a pointer to the \texttt{post} command 
that started the postamble.  
An identification byte, $i$, comes next; 
this currently equals~2, as in the preamble.

The $i$ byte is followed by four or more bytes that are all equal to
the decimal number 223 (i.e., @'337 in octal). \TeX\ puts out four to seven of
these trailing bytes, until the total length of the file is a multiple of
four bytes, since this works out best on machines that pack four bytes per
word; but any number of 223's is allowed, as long as there are at least four
of them. In effect, 223 is a sort of signature that is added at the very end.


This curious way to finish off a \texttt{DVI} file makes it feasible for
\texttt{DVI}-reading programs to find the postamble first, on most computers,
even though \TeX\ wants to write the postamble last. Most operating
systems permit random access to individual words or bytes of a file, so
the \texttt{DVI} reader can start at the end and skip backwards over the 223's
until finding the identification byte. Then it can back up four bytes, read
$q$, and move to byte $q$ of the file. This byte should, of course,
contain the value 248 (\texttt{post}); now the postamble can be read, so the
\texttt{DVI} reader discovers all the information needed for typesetting the
pages. Note that it is also possible to skip through the \texttt{DVI} file at
reasonably high speed to locate a particular page, if that proves
desirable. 
This saves a lot of time, 
since \texttt{DVI} files used in production jobs tend to be large.

\begin{Verbatim}[fontsize=\scriptsize]
ftp://ftp.rrzn.uni-hannover.de/pub/mirror/tex-archive/systems/knuth/dist/texware/dvitype.web
\end{Verbatim}

\bibliographystyle{alpha}
\bibliography{../lit}{}% chktex 11 

\end{document}

%%% Local Variables:
%%% mode: latex
%%% TeX-master: t
%%% End:
