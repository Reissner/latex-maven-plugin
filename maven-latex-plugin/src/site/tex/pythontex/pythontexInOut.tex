\RequirePackage[l2tabu, orthodox]{nag}
%\RequirePackage{pdfmanagement-testphase}
% \DocumentMetadata{%
% pdfversion=1.7,backend=,pdfstandard=,debug={options},
% lang=,uncompress,testphase=,,colorprofiles={options},xmp
% }
\documentclass{article}

\synctex=1
\usepackage{iftex}

\iftutex%
  \usepackage{fontspec}
\else
  \usepackage[utf8]{inputenc}
  \usepackage[T1]{fontenc}
\fi


\usepackage{ifthen}
\newboolean{texFhtLoaded}
\setboolean{texFhtLoaded}{false}


% absolutely necessary. 
% for document development add certain options. 
% Then remove headline and prevent this plugin from overwriting. 
\usepackage[a4paper]{geometry}% option , showframe, showcrop 
%\usepackage{showframe} as an alternative 
\usepackage{microtype}
% special characters
\usepackage{textcomp}
%\usepackage{anyfontsize}
%\usepackage{cleveref}

% used by hyperref and also to update index and glossary 
% to avoid clash because of loading with different options: 
% declare first 
\usepackage[aux]{rerunfilecheck}




\ifpdf%
\usepackage{xcolor}  % [pdftex]
\usepackage{graphicx}% [pdftex] 
\usepackage{hyperref}% driver [hpdftex] is autodetected 

\ifLuaTeX%
% for lualatex
\pdfvariable minorversion=7% chktex 1
% omit CreationDate and ModDate keys.
\pdfvariable suppressoptionalinfo 767% chktex 1
% no adding to the trailer dictionary.
\pdfvariable trailerid{}% chktex 1
\pdfvariable suppressoptionalinfo -1% chktex 1
\else
\ifXeTeX%
% for xelatex
\special{pdf:minorversion 7}
% TBD: find a way to express pdfinfoomitdate: necessary? 
\special{pdf:trailerid []}
\else
\ifPDFTeX%
\pdfminorversion=7         % for pdflatex
% omit CreationDate and ModDate keys.
% not before pdfTeX 3.14159265-2.6-1.40.17
\pdfinfoomitdate=1                   % for pdflatex
% no adding to the trailer dictionary.
%\pdftrailer=0                        % for pdflatex
\pdftrailerid{}                       % for pdflatex
\pdfsuppressptexinfo=-1               % for pdflatex
\else
% Here, the tex processor is unknown. 
\fi%pdftex
\fi%xetex
\fi%luatex



\hypersetup{
  pdfinfo={
    Author      ={Ernst Reissner},
    Title       ={Files, errors and warnings of pythontex },
    CreationDate={unknown},
    ModDate     ={unknown},
    Producer    ={unknown},
    Subject     ={pythontex},
    Keywords    ={LaTeX;pythontex}
  }
}


\else
\usepackage[dvipdfmx]{xcolor}
\usepackage[dvipdfmx]{graphicx}
\@ifpackageloaded{tex4ht}{%
  \usepackage[tex4ht]{hyperref}
  \renewcommand{\url}[1]{\texttt{#1}}% not active, replacement for hyperref
  \setboolean{texFhtLoaded}{true}%
}{%
\ifXeTeX%
\usepackage{hyperref}
\else
\usepackage[dvipdfmx]{hyperref}
% lualatex: without [dvipdfmx] option did not find converter dvi to pdf or to ps
% pdflatex: without [dvipdfmx] option dvips still works, but no converter for pdf
% with [dvipdfmx] all works fine 
\fi%
}
%\usepackage{bmpsize}% not for xelatex; lualatex and newer pdflatex
\fi

\usepackage{listings}
\usepackage{longtable}

\usepackage{booktabs}
\aboverulesep=0ex
\belowrulesep=0ex


\usepackage{import}
\usepackage[keeptemps, depythontex, runall=false, fvextfile=55]{pythontex}%rerun=always, 

\renewcommand{\lstlistoflistings}{\begingroup
\tocfile{\lstlistingname}{lol}
\endgroup}
%and to number the Listof heading do:
\renewcommand{\lstlistoflistings}{\begingroup
%\tocsection
%\tocchapter
\tocfile{List of \lstlistingname{}s}{lol}
\endgroup}
\usepackage[nottoc, numindex, numbib]{tocbibind}

%\usepackage{latex-bnf}
\title{Files, Errors and Warnings of \texttt{pythontex} 0.18 }
\author{Ernst Reissner (rei3ner@arcor.de)}
\date{}

\newcommand{\lualatex}{\texttt{lualatex}}
\newcommand{\makeIndex}{\texttt{makeindex}}
\newcommand{\pytex}{\texttt{pythontex}}
\newcommand{\Ppytex}{package \texttt{pythontex}}
\newcommand{\pytexP}{\texttt{pythontex} package}
\newcommand{\dpytex}{\texttt{depythontex}}
\newcommand{\cmd}[1]{\texttt{\textbackslash#1}}

\begin{document}

\maketitle
\tableofcontents
\listoffigures
\listoftables
\lstlistoflistings%

\section{Introduction}

This document is created with \lualatex{} or that like 
with output format 
\ifpdf%
PDF%
\else
DVI% TBD: take XDV into account also 
\fi.
The package \texttt{tex4ht} 
is \ifthenelse{\boolean{texFhtLoaded}}{}{not} loaded. 

The \pytexP{} together with the auxiliary program with the same name \pytex{}, 
allows including code, e.g.~in Python into a \LaTeX{} document. 
This document describes the input/output behavior of the auxiliary program \pytex{}, version 0.18
which includes all files read and written and uses pythontex. 
For example, \texttt{\pys[sname]{1+1=!{1+1}}} has been computed by python. 
% \begin{pycode}
% import sys
% sys.stdout.write("bla") 
% \end{pycode}

Interaction of \pytex{} with a \LaTeX-to-pdf converter like \lualatex{} 
is comparable to that of other auxiliary programs like \makeIndex: 
A latex package makes the \LaTeX-to-pdf converter extract information for the auxiliary program 
into a separate file or more. 
Then the auxiliary program is run which creates further files 
which the \LaTeX-to-pdf converter reads in a second run. 

Both, the package \texttt{pythontex} and the auxiliary programs \pytex{} and \dpytex{}, 
are described in~\cite{PythonTexP}. 
Moreover, there is an introduction~\cite{PythonTexQ} and a gallery~\cite{PythonTexG}. 
For background on the intentions of package \texttt{pythontex}, consult~\cite{PythonTexRepr}. 

The integration of \texttt{pythontex} into the latex maven plugin in this project 
is given in~\cite{LatexPlugin}, Section 5.5. 

Another source of knowledge on pythontex is the source code hosted at 
\url{https://github.com/gpoore/pythontex}. 
Note that \texttt{pythontex} is written in python and we only take into account the code for python3. 

At least the following properties are special to \Ppytex: 
%
\begin{itemize}
  \item The number of files \pytex{} may create is variable 
  and so by default they are put into a subfolder. 
  \item The output files generated are highly configurable. 
  \item There is more than one auxiliary program tied to the package, besides \pytex{} also \dpytex{}. 
  \item The errors and warnings of a \pytex{} run and of a \dpytex{} run are not written into a log file. 
\end{itemize}

In~\cite{LatexPlugin}, Section 5.5 a wrapper for \pytex{} is suggested 
writing the errors and warnings normally coming at standard output or error output into a log file \texttt{xxx.plg}. 
Nevertheless, currently no log file is written. 

The \Ppytex{} is highly configurable, 
more than this software allows. 
Thus, also in this document 
we assume that neither \cmd{setpythontexoutputdir} setting the output directory 
nor \cmd{setpythontexworkingdir} setting the working directory are used, 
because this software assumes the default that the working directory is the directory 
containing the \LaTeX{} main file \texttt{xxx.tex}
and the output directory is in the working directory 
and its name is \texttt{pythontex-files-xxx}. 

Note that we assume python 3.x is installed only. 


\section{The converter \pytex}\label{sec:pytex}

As already pointed out in the introduction, 
we restrict ourselves to the default case 
in which \pytex{} writes output files only in folder \texttt{pythontex-files-xxx}. 

Under these assumptions, Figure~\ref{fig:py2dir} shows the input and output files of \pytex{}. 

\begin{figure}[htb]
  \centering
  \ifthenelse{\boolean{texFhtLoaded}}{
  should be a picture
  }{
  \import{}{F5_07py2dir.ptx}
  }
  \caption{\label{fig:py2dir}Conversion of a \texttt{pytxcode}-file using \texttt{pythontex}}
  \end{figure}

  The input file is described in Section~\ref{subsec:pytexInFile} in full detail. 
  Section~\ref{subsec:pytexOutFile} is devoted to the output files of \pytex{}. 
  Note that unlike the wrapper \texttt{pythontexW}, the original \pytex{} 
  just prints errors and warnings. 
  These are all collected in Section~\ref{subsec:pytexErrWarn}. 
  Finally, Section~\ref{subsec:pytexFCodes} is on the failure codes. 

\subsection{The Input File \texttt{xxx.pytxcode}}\label{subsec:pytexInFile}


If a file \texttt{xxx.tex} loading package \texttt{pythontex} is processed, 
as is the case for this document, 
a file \texttt{xxx.pytxcode} is created, 
whether there is python code within \texttt{xxx.tex} or not. 
This file contains a line 
%
\begin{verbatim}
  =>PYTHONTEX:SETTINGS#
\end{verbatim}
%
and below that are specified the package options in the form given by Listing~\ref{lst:pytxcodeSettings}. 



\lstinputlisting%
[basicstyle=\scriptsize,
firstline=3, print=true,
breaklines, showstringspaces=false,
float,
captionpos=b, label={lst:pytxcodeSettings},
caption={The settings section of \texttt{pythontexInOut.pytxcode}}]
{./pythontexInOut.pytxcode.cpy}


% question: how is debug encoded? 

There is one key which does not refer to a package option: it is \texttt{version} 
which refers to the version of the \texttt{pythontex} package 
which is also the expected version of \pytex. 
If the versions deviate, running \pytex{} emits the fatal error with line number 491 in Table~\ref{tab:pytexFErr}. 

Interesting: \texttt{runall} is a package option, but it is not a valid key in \texttt{xxx.pytxcode}: 
instead, \texttt{runall=true/false} is converted into \texttt{rerun=always/default}. 
Note that \texttt{pythontex} is not able to process the key \texttt{runall} 
but emits a warning with line number 484 given in Table~\ref{tab:pytexWarn}. 
This document is compiled with option \texttt{runall=false}. 

For each python code in \texttt{xxx.tex}, there is a separate code section in \texttt{xxx.pytxcode}. 
The code sections come in proper order and precede the settings section. 
This document has a single section with python code, right at the beginning of the introduction. 
The code is 
%
\begin{verbatim}
  \pys[sname]{1+1=!{1+1}}
\end{verbatim}

Listing~\ref{lst:pytxcodeCode1} shows the according section in \texttt{xxx.pytxcode}. 
As always there is a headline starting with \texttt{=>PYTHONTEX} 
then follow, separated by \texttt\# symbols 
%
\begin{itemize}
  \item the family, i.e.~the interpreter, here \texttt{py} representing python, 
  coming from the command \texttt{\textbackslash pys}; 
  accordingly for environments, 
  \item the session, here \texttt{sname}, 
  which is the optional parameter of the command, 
  \item next suspected the restart identifier, seemingly always \texttt{default}
  \item the command, here \texttt{s}, 
  also determined by the command \texttt{\textbackslash pys}, 
  \item the context which is empty, % TBD: clarify 
  \item arguments which are empty here, 
  \item the number of the instance, which runs from 0 to the number of commands minus one 
  \item and the line number which is the line in the \LaTeX{} file, 
  where the command or the according environment starts. 
\end{itemize}
% the following is in the generated file xxx.py: 
% pytex.id = 'py_sname_default'
% pytex.family = 'py'
% pytex.session = 'sname'
% pytex.restart = 'default'

% pytex.command = 's'
% pytex.set_context('')
% pytex.args = ''
% pytex.instance = '0'
% pytex.line = '152'

% LTeX: enabled=false
\lstinputlisting%
[basicstyle=\scriptsize,
firstline=1, lastline=2, print=true,
breaklines, showstringspaces=false,
float,
captionpos=b, label={lst:pytxcodeCode1},
caption={The sole code section of \texttt{pythontexInOut.pytxcode}}]
{./pythontexInOut.pytxcode.cpy}
% LTeX: enabled=true


If running \texttt{pythontex} on the job \texttt{xxx}, 
we obtain for this manual with a trailing empty line. 
%
\begin{verbatim}
  This is PythonTeX 0.18

  --------------------------------------------------
  PythonTeX:  manualLMP - 0 error(s), 0 warning(s)  
\end{verbatim}

The folder \texttt{pythontex-files-manualLMP} is created 
but may be empty because there is no code. %TBD: clarify whether this is true. 

% TBD: analyze whether also other sections may occur: 
% =>PYTHONTEX:STDERR#
% =>PYTHONTEX:STDOUT#
% =>PYTHONTEX:DEPENDENCIES
% =>PYTHONTEX:CREATED
% =>PYTHONTEX:FIELD_DELIM


\subsection{The Output Files}\label{subsec:pytexOutFile}

  Figure~\ref{fig:py2dir} shows that the output files of \pytex{} 
  are all in folder \texttt{pythontex-files-xxx}. 
  Temporary files in dotted boxes, 
  so these can be seen only if the \pytex{} run is interrupted, e.g.~by failure. 
  The other files are called \emph{final}. 
  The Figure also indicates, that the wrapper \texttt{pythontexW} writes a log file in addition. 
  
  Among the final files, there is \texttt{xxx.pytxmcr} 
  which starts something like 
%
\begin{lstlisting}
  %Last time of file creation:  1656851667.5282867
\end{lstlisting}
%
and contains processed \texttt{pygments} code according to~\cite{PythonTexP}, page~107. 

  Although indicates the time of the last \pytex{} wrote the file, 
  seemingly, \pytex{} does not update if it is unchanged. 
  So it does not indicate the last run. %run and is updated if it has been run. 


\subsection{Errors and Warnings at standard/error output}\label{subsec:pytexErrWarn}

{\small
\begin{longtable}{|rlll|}
\toprule
Line No       & * & Message & RC \\
\midrule
\midrule
\endfirsthead%
\bottomrule
\caption{\label{tab:pytexFErr} Fatal errors with number code of \texttt{pythontex}}
\endlastfoot%
 219 & ---   & \texttt{Invalid --interpreter argument} & 2 \\
 246 & error & \texttt{You have launched PythonTeX using pythontex2/3.py directly.}   & 2 \\
 271 & error & \texttt{You have launched PythonTeX using pythontex2/3.py directly.}   & 2 \\
 292 & error & \texttt{Code file xxx.pytxcode does not exist. Run LaTeX \dots }       & 1 \\
 327 & error & \texttt{Directory naming collision between the following files:\dots}  & 1 \\
 362 & error & \texttt{Code file xxx.pytxcode does not exist. Run \LaTeX{} \dots }    & 1 \\
 370 & ---   & \texttt{The .pytxcode file appears to have an outdated format \dots}   & 1 \\% chktex 26
     &       & \texttt{Run LaTeX to make sure the file is current }                   &   \\
 406 & error & \texttt{Unable to parse package option fvextfile.  }                   & 1 \\
 491 & error & \texttt{The version of the PythonTeX scripts does not match }          & 1 \\
     &       & \texttt{ the last code saved by the document \dots}                    &   \\
2864 & ---   & ---                                                                    & 1 \\
\end{longtable}
} %\small


Fatal errors are understood to be those which exit via \texttt{sys.exit} 
\emph{immediately} with an error code other than 0 or with a message. 
Thus, at most one fatal error can occur. 
This document doesn't treat fatal errors with error message. 
% TBD: fatal errors with message but no error code. 
Table~\ref{tab:pytexFErr} gives an overview of fatal errors with number code. 
It gives the line number of the according \texttt{sys.exit} command, 
whether the message is preceded by a line \texttt{* PythonTeX error}, 
followed by an abbreviation of the proper message and finally the error code. 
Of course, just the error code indicates that it is an error, 
even if the message does not start with \texttt{* PythonTeX error}, 
and for the cases with lines 219 and 370, 
the missing indication in the message seems to be just a bug or a weakness. 
% bug report already filed. 

A special case is in line 2864: 
\texttt{pythontex} has a counter for non-fatal errors and another counter for warnings. 
Error code 1 is returned also, if at least one non-fatal error was counted 
and if command line option \texttt{-{}-error-exit-code} is set to \texttt{true}, 
which is the default according to~\cite{PythonTexP}, Section 3.2. 
Fatal errors cannot be suppressed via that command line option. 
Strictly speaking, the case of line 2864 is a fatal error itself, 
but it shall not be treated as such, and it is appropriate that it has no message, 
just an error code if configured so. 
% TBD: the package documentation does not distinguish between fatal errors and non-fatal ones. 
% As a consequence, the above option is not described properly. 
% An according bug report is filed already. 

In contrast, non-fatal errors, 
are errors which do not immediately cause \texttt{pythontex} to exit. 
Have a look at the messages collected in Table~\ref{tab:pytexNFErr}: 
It is structured similar to Table~\ref{tab:pytexFErr} 
except that the exit code (which non-fatal errors don't have) 
is replaced by the line number of the increment of the error counter, 
except in one case, where there is no increment at all. 
In this case, the line number, which is 2422, refers to the error message. 
The author considers this a bug in \texttt{pythontex}. 
% bug report filed already. 
% TBD: Check on reaction. 
Observe that all these errors have messages starting with \texttt{* PythonTeX error}. 
All entries in Table~\ref{tab:pytexNFErr} refer to non-fatal errors: 
They are errors because of either the message or the increment of the error count. 
Also, they are non-fatal because they don't lead to an immediate exit with failure code. 




{\small
\begin{longtable}{|rlll|}
  \toprule
  Line No       & * & Message & inc err/warn  \\
  \midrule
  \midrule
  \endfirsthead%
  \bottomrule
  \caption{\label{tab:pytexNFErr} Non-fatal errors of \texttt{pythontex}  }
  \endlastfoot%
   655 & error  & Cannot find dependency \dots                                   & e   \\%temp_data['errors'] += 1
  1359 & error  & Currently, non-Python consoles are not supported               & e   \\%errors += 1
  1605 & error  & Missing output file for \dots                                  & e   \\%errors += 1
  1611 & error  & Running code for Julia console failed                          & e   \\%errors += 1
  1696 & error  & Cannot find dependency. It belongs to \dots                    & e   \\%errors += 1
  1765 & error  & Missing stderr file for \dots                                  & e   \\%errors += 1
  1960 & error  & Line number xxx could not be synced with the document \dots    & e   \\%errors += 1
  2343 & error  & An error occurred but no error messages were identified. \dots & e   \\
  2422 & error  & Could not find external file xxx The file was not pygmentized  & --- \\ % TBD: ask whether correct. 
\end{longtable}
} % \small 

Table~\ref{tab:pytexIOErr} collects messages starting with \texttt{* PythonTeX stderr} 
indicating that they are handed over from included code.  
They are treated either as errors or warnings, 
increasing exactly one of the according counters, 
which is indicated by the last column of the table. 
Note that the message gives no indication on whether it is counted as an error or as a warning: 
One and the same message form can be both. 
Distinction is just by the counter incremented. 
Since there are at least two lines of code where the increments are performed, 
at least one for an error and one for a warning, 
the line number given in the table refers to the code where the message 
\texttt{* PythonTeX stderr} is printed. 

Without the irregularity given in Table~\ref{tab:pytexNFErr} line with number 2422, 
a non-fatal error is just tied with messages for which the error count is incremented. 
The irregularity can be included by specifying 
that a non-fatal error is if the error counter is incremented 
or the message starts with \texttt{* PythonTeX error}. 

{\small
\begin{longtable}{|rlll|}
  \toprule
  Line No       & * & Message & inc err/warn  \\
  \midrule
  \midrule
  \endfirsthead%
  \bottomrule
  \caption{\label{tab:pytexIOErr} StdErr (non-fatal) errors of \texttt{pythontex}  }
  \endlastfoot%
  1899 & stderr & \dots on line \dots in ``\dots''                               & e/w \\
  1899 & stderr & \dots on line \dots                                            & e/w \\
  2061 & stderr & \dots near line \dots in ``\dots''                             & e/w \\
  2063 & stderr & \dots near line \dots                                          & e/w \\
  2164 & stderr & \dots near line \dots in ``\dots''                             & e/w \\
  2166 & stderr & \dots near line \dots                                          & e/w \\
  2654 & stderr & \dots in console startup code                                  & e/w \\
  2677 & stderr & \dots near line \dots in custom code for console               & e/w \\
  2679 & stderr & \dots near line \dots in console code                          & e/w \\
\end{longtable}
} % \small 

Table~\ref{tab:pytexWarn} contains proper warnings 
always incrementing the warnings counter. 
So definition of warnings is simple: A warning is what increases the warning counter. 

\begin{longtable}{|rlll|}
  \toprule
  Line No       & * & Message & inc warn \\
  \endfirsthead%
  \bottomrule
  \caption{\label{tab:pytexWarn} Warnings of \texttt{pythontex}  }
  \endlastfoot%
   340 & warning & Potential directory naming collision \dots  & yes \\%temp_data['warnings'] += 1
   413 & warning & Invalid value for package option fvextfile  & yes \\%temp_data['warnings'] += 1
   484 & warning & Unknown option \dots                        & yes \\%temp_data['warnings'] += 1
   685 & warning & Session xxx has rerun=never                 & yes \\%temp_data['warnings'] += 1
       &   & But its code or dependencies have been modified & \\
  1446 & warning & The following have dependencies that have been modified & yes \\%warnings += 1
  1737 & warning & Custom code for xxx attempted to print or write to stdout & yes \\%ewarnings += 1
   \end{longtable}


Finally, Table~\ref{tab:pytexNotice} states notices, 
seemingly mere info not directly tied to an error or a warning. 
The according message is identified by its setart \texttt{* PythonTeX stderr}. 



   \begin{longtable}{|rll|}
    \toprule
    Line Number       & * & Message  \\
    \endfirsthead%
    \bottomrule
    \caption{\label{tab:pytexNotice} Notices of \texttt{pythontex}  }
    \endlastfoot%
  2276 & notice & Line number \dots could not be synced with the document \\
  2336 & notice & x message(s) could not be classified          \\% chktex 36
       &        & Interpreted as y, based on the return code(s) \\% chktex 36
\end{longtable}

It is difficult to analyze the code around line 2276, 
but it seems as if synchronization of line numbers 
occurs only in conjunction with non-fatal errors 
and warnings, because synchronization is needed only to locate those in code text. 

Analyzing the code preceding line 2336 shows that the according notice 
comes up only if a stderr message could not be identified 
as a non-fatal error or a warning, so both counters are increased 
by the number of events which could not be classified. 
So again, an error or a warning is indicated by a nonzero counter, 
but one of the counters may be too high. 

So also notices are recognized via error count and warning count. 

At the end of the log, which is currently written to stdio, 
\texttt{pythontex} summarizes the (non-fatal) errors and warnings which occurred. 
So besides the proper messages, there is a summary. 
It takes one of the following forms: 
Either 
%
\begin{Verbatim}[showspaces=true]
--------------------------------------------------
PythonTeX:  pythontexInOut 
    - Old:      0 error(s), 0 warning(s)
    - Current:  0 error(s), 0 warning(s)
\end{Verbatim}
%
or 
%
\begin{Verbatim}[showspaces=true]
--------------------------------------------------
PythonTeX:  pythontexInOut - 0 error(s), 0 warning(s)
\end{Verbatim}
%
where of course \texttt{pythontexInOut} is to be replaced by the jobname 
and the number of errors and warnings may be different from 0. 

The pattern to match at least one non-fatal error in java style is 
%
\begin{Verbatim}[showspaces=true, fontsize=\small]
(PythonTeX:  .+ -|    - Current: ) [1-9][0-9]* error\\(s\\), [0-9]+ warning\\(s\\)
\end{Verbatim}
%
The last thing to do is to take into account the irregularity 
of the non-fatal error in Table~\ref{tab:pytexNFErr}, line number 2422, 
by adding the alternative \texttt{* PythonTeX error} to the regular expression, 
which detects nothing additional but this case. 
So we arrive at the following pattern, where the dots must be replaced by the above pattern 
%
\begin{Verbatim}[showspaces=true, fontsize=\small]
\\* PythonTeX error|...
\end{Verbatim}

For warnings we can use the same but do not need the bugfix. 
The result is 
%
\begin{Verbatim}[showspaces=true, fontsize=\small]
(PythonTeX:  .+ -|    - Current: ) [0-9]+ error\\(s\\), [1-9][0-9]* warning\\(s\\)
\end{Verbatim}
  


\subsection{Failure codes}\label{subsec:pytexFCodes}

\section{The converter \dpytex}\label{sec:dpytex}

\begin{figure}[htb]
  \centering
  \ifthenelse{\boolean{texFhtLoaded}}{
  should be a picture 
  }{
  \import{}{F5_08depy2out.ptx}
  }
  \caption{\label{fig:depy2out}Conversion of a \texttt{depytx}-file using \texttt{depythontex}}
  \end{figure}


\subsection{The Input File \texttt{xxx.dpytx}}\label{subsec:dpytexInFile}

\subsection{The Output Files}\label{subsec:dpytexOutFile}

\subsection{Errors and Warnings at standard/error output}\label{subsec:dpytexErrWarn}

\subsection{Failure codes}\label{subsec:dpytexFCodes}


\bibliographystyle{alpha}
\bibliography{../lit}{}% chktex 11 


\end{document}


