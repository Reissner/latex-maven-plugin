\RequirePackage[l2tabu, orthodox]{nag}
\documentclass[10pt]{article}
\synctex=1
\usepackage{iftex}

\iftutex%
\usepackage{fontspec}
\else
\usepackage[utf8]{inputenc}
\usepackage[T1]{fontenc}
\fi
\usepackage{fancyvrb}


%%\pdfinfoomitdate1
%\pdfsuppressptexinfo-1
%\pdftrailerid{}


\usepackage{ifthen}
\newboolean{texFhtLoaded}
\setboolean{texFhtLoaded}{false}

\usepackage{booktabs}
\aboverulesep=0ex
\belowrulesep=0ex

\usepackage[a4paper]{geometry}
\usepackage{listings}
\usepackage[aux]{rerunfilecheck}

% graphics 
\makeatletter

\ifpdf%
\usepackage{xcolor}  % [pdftex]
\usepackage{graphicx}% [pdftex] 
\usepackage{hyperref}% driver [hpdftex] is autodetected 

\ifLuaTeX
% for lualatex
\pdfvariable minorversion=7
% omit CreationDate and ModDate keys.
\pdfvariable suppressoptionalinfo 767
% no adding to the trailer dictionary.
\pdfvariable trailerid{}
\pdfvariable suppressoptionalinfo -1
\else
\ifXeTeX
% for xelatex
\special{pdf:minorversion 7}
% TBD: find a way to express pdfinfoomitdate: necessary? 
\special{pdf:trailerid []}
\else
\ifPDFTeX
\pdfminorversion=7         % for pdflatex
% omit CreationDate and ModDate keys.
% not before pdfTeX 3.14159265-2.6-1.40.17
\pdfinfoomitdate=1                   % for pdflatex
% no adding to the trailer dictionary.
%\pdftrailer=0                        % for pdflatex
\pdftrailerid{}                       % for pdflatex
\pdfsuppressptexinfo=-1               % for pdflatex
\else
% Here, the tex processor is unknown. 
\fi%pdftex
\fi%xetex
\fi%luatex



\hypersetup{
  pdfinfo={
    Author      ={Ernst Reissner},
    Title       ={Files, errors and warnings of pythontex },
    CreationDate={unknown},
    ModDate     ={unknown},
    Producer    ={unknown},
    Subject     ={pythontex},
    Keywords    ={LaTeX;pythontex}
  }
}


\else
\usepackage[dvipdfmx]{xcolor}
\usepackage[dvipdfmx]{graphicx}
\@ifpackageloaded{tex4ht}{%
  % \usepackage[tex4ht]{hyperref}
  \setboolean{texFhtLoaded}{true}
}{%
%\usepackage[dvipdfmx]{hyperref}
}
\usepackage{bmpsize}% not for xelatex; lualatex and newer pdflatex
\fi

\usepackage{import}
\usepackage[keeptemps, depythontex, rerun=always, fvextfile=55]{pythontex}%
\usepackage[nottoc, numindex, numbib]{tocbibind}

%\usepackage{latex-bnf}
\title{Files, Errors and Warnings of \texttt{pythontex} 0.18 }
\author{Ernst Reissner (rei3ner@arcor.de)}
\date{}

\newcommand{\lualatex}{\texttt{lualatex}}
\newcommand{\makeIndex}{\texttt{makeindex}}
\newcommand{\pytex}{\texttt{pythontex}}
\newcommand{\Ppytex}{package \texttt{pythontex}}
\newcommand{\pytexP}{\texttt{pythontex} package}
\newcommand{\dpytex}{\texttt{depythontex}}
\newcommand{\cmd}[1]{\texttt{\textbackslash#1}}

\begin{document}

\maketitle
\tableofcontents
\listoffigures
%\listoftables

\section{Introduction}

This document is created with \lualatex{} or that like 
with output format 
\ifpdf%
pdf%
\else
dvi%
\fi.
The package \texttt{tex4ht} 
is \ifthenelse{\boolean{texFhtLoaded}}{}{not} loaded. 

The \pytexP{} together with the auxiliary program with the same name \pytex, 
allows including code, e.g.~in Python into a \LaTeX{} document. 
This document describes the input/output behavior of the auxiliary program \pytex, version 0.18
which includes all files read and written and uses pythontex. 
For example, \texttt{\pys[sname]{1+1=!{1+1}}} has been computed by python. 
% \begin{pycode}
% import sys
% sys.stdout.write("bla") 
% \end{pycode}

Interaction of \pytex{} with a \LaTeX-to-pdf converter like \lualatex{} 
is comparable to that of other auxiliary programs like \makeIndex: 
A latex package makes the \LaTeX-to-pdf converter extract information for the auxiliary program 
into a separate file or more. 
Then the auxiliary program is run which creates further files 
which the \LaTeX-to-pdf converter reads in a second run. 

Both, the package \texttt{pythontex} and the auxiliary programs \pytex{} and \dpytex, 
are described in~\cite{PythonTexP}. 
Moreover, there is an introduction~\cite{PythonTexQ} and a gallery~\cite{PythonTexG}. 
For background on the intentions of package \texttt{pythontex}, consult~\cite{PythonTexRepr}. 

The integration of \texttt{pythontex} into the latex maven plugin in this project 
is given in~\cite{LatexPlugin}, Section 5.5. 

Another source of knowledge on pythontex is the source code hosted at 
\url{https://github.com/gpoore/pythontex}. 

At least the following properties are special to \Ppytex: 
%
\begin{itemize}
  \item The number of files \pytex{} may create is variable 
  and so by default they are put into a subfolder. 
  \item The output files generated are highly configurable. 
  \item There is more than one auxiliary program tied to the package, besides \pytex{} also \dpytex. 
  \item The errors and warnings of a \pytex{} run and of a \dpytex{} run are not written into a log file. 
\end{itemize}

In~\cite{LatexPlugin}, Section 5.5 a wrapper for \pytex{} is suggested 
writing the errors and warnings normally coming at standard output or error output into a log file \texttt{xxx.plg}. 
Nevertheless, currently no log file is written. 

The \Ppytex{} is highly configurable, 
more than this software allows. 
Thus, also in this document 
we assume that neither \cmd{setpythontexoutputdir} setting the output directory 
nor \cmd{setpythontexworkingdir} setting the working directory are used, 
because this software assumes the default that the working directory is the directory 
containing the \LaTeX{} main file \texttt{xxx.tex}
and the output directory is in the working directory 
and its name is \texttt{pythontex-files-xxx}. 


\section{The converter \pytex}\label{sec:pytex}

As already pointed out in the introduction, 
we restrict ourselves to the default case 
in which \pytex{} writes output files only in folder \texttt{pythontex-files-xxx}. 

Under these assumptions, Figure~\ref{fig:py2dir} shows the input and output files of \pytex{}. 

\begin{figure}[htb]
  \centering
  \ifthenelse{\boolean{texFhtLoaded}}{
  should be a picture 
  }{
  \import{}{F5_07py2dir.ptx}
  }
  \caption{\label{fig:py2dir}Conversion of a \texttt{pytxcode}-file using \texttt{pythontex}}
  \end{figure}

  The input file is described in Section~\ref{subsec:pytexInFile} in full detail. 
  Section~\ref{subsec:pytexOutFile} is devoted to the output files of \pytex{}. 
  Note that unlike the wrapper \texttt{pythontexW}, the original \pytex{} 
  just prints errors and warnings. 
  These are all collected in Section~\ref{subsec:pytexErrWarn}. 
  Finally, Section~\ref{subsec:pytexFCodes} is on the failure codes. 

\subsection{The Input File \texttt{xxx.pytxcode}}\label{subsec:pytexInFile}

\subsection{The Output Files}\label{subsec:pytexOutFile}

  Figure~\ref{fig:py2dir} shows that the output files of \pytex{} 
  are all in folder \texttt{pythontex-files-xxx}. 
  Temporary files in dotted boxes, 
  so these can be seen only if the \pytex{} run is interrupted, e.g. by failure. 
  The other files are called {\em final}. 
  The Figure also indicates, that the wrapper \texttt{pythontexW} writes a log file in addition. 
  
  Among the final files, there is \texttt{xxx.pytxmcr} 
  which starts something like 

\begin{lstlisting}
  %Last time of file creation:  1656851667.5282867
\end{lstlisting}

and contains processed pygments code according to~\cite{PythonTexP}, page~107. 

  Although indicates the time of the last \pytex{} wrote the file, 
  seemingly, \pytex{} does not update if it is unchanged. 
  So it does not indicate the last run. %run and is updated if it has been run. 


\subsection{Errors and Warnings at standard/error output}\label{subsec:pytexErrWarn}

\subsection{Failure codes}\label{subsec:pytexFCodes}

\section{The converter \dpytex}\label{sec:dpytex}

\begin{figure}[htb]
  \centering
  \ifthenelse{\boolean{texFhtLoaded}}{
  should be a picture 
  }{
  \import{}{F5_08depy2out.ptx}
  }
  \caption{\label{fig:depy2out}Conversion of a \texttt{depytx}-file using \texttt{depythontex}}
  \end{figure}


\subsection{The Input File \texttt{xxx.dpytx}}\label{subsec:dpytexInFile}

\subsection{The Output Files}\label{subsec:dpytexOutFile}

\subsection{Errors and Warnings at standard/error output}\label{subsec:dpytexErrWarn}

\subsection{Failure codes}\label{subsec:dpytexFCodes}


\bibliographystyle{alpha}
\bibliography{../lit}{}% chktex 11 


\end{document}


