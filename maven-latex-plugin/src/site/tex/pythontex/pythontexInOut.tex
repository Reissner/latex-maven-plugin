\RequirePackage[l2tabu, orthodox]{nag}
\documentclass{article}
\synctex=1
\usepackage{iftex}

\iftutex%
\usepackage{fontspec}
\else
\usepackage[utf8]{inputenc}
\usepackage[T1]{fontenc}
\fi
%\usepackage{fancyvrb}


%%\pdfinfoomitdate1
%\pdfsuppressptexinfo-1
%\pdftrailerid{}


\usepackage{ifthen}
\newboolean{texFhtLoaded}
\setboolean{texFhtLoaded}{false}

\usepackage{booktabs}
\aboverulesep=0ex
\belowrulesep=0ex

\usepackage[a4paper]{geometry}
\usepackage{listings}
\usepackage{longtable}
% used by hyperref and also to update index and glossary 
% to avoid clash because of loading with different options: 
% declare first 
\usepackage[aux]{rerunfilecheck}

% graphics 
\makeatletter

\ifpdf%
\usepackage{xcolor}  % [pdftex]
\usepackage{graphicx}% [pdftex] 
\usepackage{hyperref}% driver [hpdftex] is autodetected 

\ifLuaTeX
% for lualatex
\pdfvariable minorversion=7
% omit CreationDate and ModDate keys.
\pdfvariable suppressoptionalinfo 767
% no adding to the trailer dictionary.
\pdfvariable trailerid{}
\pdfvariable suppressoptionalinfo -1
\else
\ifXeTeX
% for xelatex
\special{pdf:minorversion 7}
% TBD: find a way to express pdfinfoomitdate: necessary? 
\special{pdf:trailerid []}
\else
\ifPDFTeX
\pdfminorversion=7         % for pdflatex
% omit CreationDate and ModDate keys.
% not before pdfTeX 3.14159265-2.6-1.40.17
\pdfinfoomitdate=1                   % for pdflatex
% no adding to the trailer dictionary.
%\pdftrailer=0                        % for pdflatex
\pdftrailerid{}                       % for pdflatex
\pdfsuppressptexinfo=-1               % for pdflatex
\else
% Here, the tex processor is unknown. 
\fi%pdftex
\fi%xetex
\fi%luatex



\hypersetup{
  pdfinfo={
    Author      ={Ernst Reissner},
    Title       ={Files, errors and warnings of pythontex },
    CreationDate={unknown},
    ModDate     ={unknown},
    Producer    ={unknown},
    Subject     ={pythontex},
    Keywords    ={LaTeX;pythontex}
  }
}


\else
\usepackage[dvipdfmx]{xcolor}
\usepackage[dvipdfmx]{graphicx}
\@ifpackageloaded{tex4ht}{%
  % \usepackage[tex4ht]{hyperref}
  \setboolean{texFhtLoaded}{true}
}{%
%\usepackage[dvipdfmx]{hyperref}
}
\usepackage{bmpsize}% not for xelatex; lualatex and newer pdflatex
\fi

\usepackage{import}
\usepackage[keeptemps, depythontex, runall=false, fvextfile=55]{pythontex}%rerun=always, 

\renewcommand{\lstlistoflistings}{\begingroup
\tocfile{\lstlistingname}{lol}
\endgroup}
%and to number the Listof heading do:
\renewcommand{\lstlistoflistings}{\begingroup
%\tocsection
%\tocchapter
\tocfile{List of \lstlistingname{}s}{lol}
\endgroup}
\usepackage[nottoc, numindex, numbib]{tocbibind}

%\usepackage{latex-bnf}
\title{Files, Errors and Warnings of \texttt{pythontex} 0.18 }
\author{Ernst Reissner (rei3ner@arcor.de)}
\date{}

\newcommand{\lualatex}{\texttt{lualatex}}
\newcommand{\makeIndex}{\texttt{makeindex}}
\newcommand{\pytex}{\texttt{pythontex}}
\newcommand{\Ppytex}{package \texttt{pythontex}}
\newcommand{\pytexP}{\texttt{pythontex} package}
\newcommand{\dpytex}{\texttt{depythontex}}
\newcommand{\cmd}[1]{\texttt{\textbackslash#1}}

\begin{document}

\maketitle
\tableofcontents
\listoffigures
\listoftables
\lstlistoflistings

\section{Introduction}

This document is created with \lualatex{} or that like 
with output format 
\ifpdf%
pdf%
\else
dvi%
\fi.
The package \texttt{tex4ht} 
is \ifthenelse{\boolean{texFhtLoaded}}{}{not} loaded. 

The \pytexP{} together with the auxiliary program with the same name \pytex, 
allows including code, e.g.~in Python into a \LaTeX{} document. 
This document describes the input/output behavior of the auxiliary program \pytex, version 0.18
which includes all files read and written and uses pythontex. 
For example, \texttt{\pys[sname]{1+1=!{1+1}}} has been computed by python. 
% \begin{pycode}
% import sys
% sys.stdout.write("bla") 
% \end{pycode}

Interaction of \pytex{} with a \LaTeX-to-pdf converter like \lualatex{} 
is comparable to that of other auxiliary programs like \makeIndex: 
A latex package makes the \LaTeX-to-pdf converter extract information for the auxiliary program 
into a separate file or more. 
Then the auxiliary program is run which creates further files 
which the \LaTeX-to-pdf converter reads in a second run. 

Both, the package \texttt{pythontex} and the auxiliary programs \pytex{} and \dpytex, 
are described in~\cite{PythonTexP}. 
Moreover, there is an introduction~\cite{PythonTexQ} and a gallery~\cite{PythonTexG}. 
For background on the intentions of package \texttt{pythontex}, consult~\cite{PythonTexRepr}. 

The integration of \texttt{pythontex} into the latex maven plugin in this project 
is given in~\cite{LatexPlugin}, Section 5.5. 

Another source of knowledge on pythontex is the source code hosted at 
\url{https://github.com/gpoore/pythontex}. 

At least the following properties are special to \Ppytex: 
%
\begin{itemize}
  \item The number of files \pytex{} may create is variable 
  and so by default they are put into a subfolder. 
  \item The output files generated are highly configurable. 
  \item There is more than one auxiliary program tied to the package, besides \pytex{} also \dpytex. 
  \item The errors and warnings of a \pytex{} run and of a \dpytex{} run are not written into a log file. 
\end{itemize}

In~\cite{LatexPlugin}, Section 5.5 a wrapper for \pytex{} is suggested 
writing the errors and warnings normally coming at standard output or error output into a log file \texttt{xxx.plg}. 
Nevertheless, currently no log file is written. 

The \Ppytex{} is highly configurable, 
more than this software allows. 
Thus, also in this document 
we assume that neither \cmd{setpythontexoutputdir} setting the output directory 
nor \cmd{setpythontexworkingdir} setting the working directory are used, 
because this software assumes the default that the working directory is the directory 
containing the \LaTeX{} main file \texttt{xxx.tex}
and the output directory is in the working directory 
and its name is \texttt{pythontex-files-xxx}. 

Note that we assume python 3.x is installed only. 


\section{The converter \pytex}\label{sec:pytex}

As already pointed out in the introduction, 
we restrict ourselves to the default case 
in which \pytex{} writes output files only in folder \texttt{pythontex-files-xxx}. 

Under these assumptions, Figure~\ref{fig:py2dir} shows the input and output files of \pytex{}. 

\begin{figure}[htb]
  \centering
  \ifthenelse{\boolean{texFhtLoaded}}{
  should be a picture 
  }{
  \import{}{F5_07py2dir.ptx}
  }
  \caption{\label{fig:py2dir}Conversion of a \texttt{pytxcode}-file using \texttt{pythontex}}
  \end{figure}

  The input file is described in Section~\ref{subsec:pytexInFile} in full detail. 
  Section~\ref{subsec:pytexOutFile} is devoted to the output files of \pytex{}. 
  Note that unlike the wrapper \texttt{pythontexW}, the original \pytex{} 
  just prints errors and warnings. 
  These are all collected in Section~\ref{subsec:pytexErrWarn}. 
  Finally, Section~\ref{subsec:pytexFCodes} is on the failure codes. 

\subsection{The Input File \texttt{xxx.pytxcode}}\label{subsec:pytexInFile}


If a file \texttt{xxx.tex} loading package \texttt{pythontex} is processed, 
as is the case for this document, 
a file \texttt{xxx.pytxcode} is created, 
whether there is python code within \texttt{xxx.tex} or not. 
This file contains a line 
%
\begin{verbatim}
  =>PYTHONTEX:SETTINGS#
\end{verbatim}
%
and below that are specified the package options in the form given by Listing~\ref{lst:pytxcodeSettings}. 



\lstinputlisting%
[basicstyle=\scriptsize,
firstline=3, print=true,
breaklines, showstringspaces=false,
float,
captionpos=b, label={lst:pytxcodeSettings},
caption={The settings section of \texttt{pythontexInOut.pytxcode}}]
{./pythontexInOut.pytxcode.cpy}


% question: how is debug encoded? 

There is one key which does not refer to a package option: it is \texttt{version} 
which refers to the version of the \texttt{pythontex} package 
which is also the expected version of \pytex. 
If the versions deviate, running \pytex{} emits the fatal error with line number 491 in Table~\ref{tab:pytexFErr}. 

Interesting: \texttt{runall} is a package option, but it is not a valid key in \texttt{xxx.pytxcode}: 
instead, \texttt{runall=true/false} is converted into \texttt{rerun=always/default}. 
Note that \texttt{pythontex} is not able to process the key \texttt{runall} 
but emits a warning with line number 484 given in Table~\ref{tab:pytexWarn}. 
This document is compiled with option \texttt{runall=false}. 

For each python code in \texttt{xxx.tex}, there is a separate code section in \texttt{xxx.pytxcode}. 
The code sections come in proper order and precede the settings section. 
This document has a single section with pythoncode, right at the beginning of the introduction. 
The code is 
%
\begin{verbatim}
  \pys[sname]{1+1=!{1+1}}
\end{verbatim}

Listing~\ref{lst:pytxcodeCode1} shows the according section in \texttt{xxx.pytxcode}. 
As always there is a headline starting with \texttt{=>PYTHONTEX} 
then follow, separated by \texttt\# symbols 
%
\begin{itemize}
  \item the family, i.e.~the interpreter, here \texttt{py} representing python, 
  coming from the command \texttt{\textbackslash pys}; 
  accordingly for environments, 
  \item the session, here \texttt{sname}, 
  which is the optional parameter of the command, 
  \item next suspected the restart identifier, seemingly always \texttt{default}
  \item the command, here \texttt{s}, 
  also determined by the command \texttt{\textbackslash pys}, 
  \item the context which is empty, % TBD: clarify 
  \item arguments which are empty here, 
  \item the number of the instance, which runs from 0 to the number of commands minus one 
  \item and the line number which is the line in the \LaTeX{} file, 
  where the command or the according environment starts. 
\end{itemize}
% the following is in the generated file xxx.py: 
% pytex.id = 'py_sname_default'
% pytex.family = 'py'
% pytex.session = 'sname'
% pytex.restart = 'default'

% pytex.command = 's'
% pytex.set_context('')
% pytex.args = ''
% pytex.instance = '0'
% pytex.line = '152'


\lstinputlisting%
[basicstyle=\scriptsize,
firstline=1, lastline=2, print=true,
breaklines, showstringspaces=false,
float,
captionpos=b, label={lst:pytxcodeCode1},
caption={The sole code section of \texttt{pythontexInOut.pytxcode}}]
{./pythontexInOut.pytxcode.cpy}



If running \texttt{pythontex} on the job \texttt{xxx}, 
we obtain for this manual with a trailing empty line. 
%
\begin{verbatim}
  This is PythonTeX 0.18

  --------------------------------------------------
  PythonTeX:  manualLatexMavenPlugin - 0 error(s), 0 warning(s)  
\end{verbatim}

The folder \texttt{pythontex-files-manualLatexMavenPlugin} is created 
but may be empty because there is no code. %TBD: clarify whether this is true. 



\subsection{The Output Files}\label{subsec:pytexOutFile}

  Figure~\ref{fig:py2dir} shows that the output files of \pytex{} 
  are all in folder \texttt{pythontex-files-xxx}. 
  Temporary files in dotted boxes, 
  so these can be seen only if the \pytex{} run is interrupted, e.g. by failure. 
  The other files are called {\em final}. 
  The Figure also indicates, that the wrapper \texttt{pythontexW} writes a log file in addition. 
  
  Among the final files, there is \texttt{xxx.pytxmcr} 
  which starts something like 

\begin{lstlisting}
  %Last time of file creation:  1656851667.5282867
\end{lstlisting}

and contains processed pygments code according to~\cite{PythonTexP}, page~107. 

  Although indicates the time of the last \pytex{} wrote the file, 
  seemingly, \pytex{} does not update if it is unchanged. 
  So it does not indicate the last run. %run and is updated if it has been run. 


\subsection{Errors and Warnings at standard/error output}\label{subsec:pytexErrWarn}

\begin{longtable}{|rlll|}
\toprule
Line No       & * & Message & RC \\
\endfirsthead
\bottomrule
\caption{\label{tab:pytexFErr} Fatal errors of \texttt{pythontex}  }
\endlastfoot
219 & - & \texttt{Invalid --interpreter argument} & 2 \\
246 & y & \texttt{You have launched PythonTeX using pythontex2/3.py directly.}  & 2 \\
271 & y & \texttt{You have launched PythonTeX using pythontex2/3.py directly.}  & 2 \\
292 & y & \texttt{Code file xxx.pytxcode does not exist. Run LaTeX \dots }      & 1 \\
327 & y & \texttt{Directory naming collision between the following files:\dots} & 1 \\
362 & y & \texttt{Code file xxx.pytxcode does not exist.Run \LaTeX{} \dots }    & 1 \\
370 & - & \texttt{The .pytxcode file appears to have an outdated format or be invalid  } & 1 \\
    &   & \texttt{Run LaTeX to make sure the file is current }                  & \\
406 & y & \texttt{Unable to parse package option fvextfile.  }                  & 1\\
491 & y & \texttt{The version of the PythonTeX scripts does not match the last code }& 1 \\
    &   & \texttt{saved by the document-{}-run LaTeX to create an updated version.} & \\
2864 &  &                                                                       & 1 \\
\end{longtable}

% TBD: fatal errors with message but no error code. 
Fatal errors are those which exit directly with an error code other than 0. 
They cannot be suppressed via command line option \texttt{-{}-error-exit-code}; 
this refers to the non-fatal errors only which cause error code 1 if the option is set to \texttt{true}. 
If such a non-fatal error occurs, and 
The proper error message always starts with \texttt{* PythonTeX error}. 

\begin{longtable}{|rlll|}
  \toprule
  Line No       & * & Message & inc err  \\
  \endfirsthead
  \bottomrule
  \caption{\label{tab:pytexNFErr} Non-fatal errors of \texttt{pythontex}  }
  \endlastfoot
   655 & y & Cannot find dependency \dots                                   & yes \\
  1359 & y & Currently, non-Python consoles are not supported               & yes \\
  1605 & y & Missing output file for \dots                                  & yes \\
  1611 & y & Running code for Julia console failed                          & yes \\
  1696 & y & Cannot find dependency. It belongs to \dots                    & yes \\
  1765 & y & Missing stderr file for \dots                                  & yes \\
  1960 & y & Line number xxx could not be synced with the document \dots    & yes \\
  2343 & y & An error occurred but no error messages were identified. \dots & yes \\
  2422 & y & Could not find external file xxx The file was not pygmentized  & no \\ % TBD: ask whether correct. 
\end{longtable}


The non fatal errors go into the error count displayed at the end of the log. 
To this number also errors from the python run are added. 


\begin{longtable}{|rlll|}
  \toprule
  Line No       & * & Message & inc warn \\
  \endfirsthead
  \bottomrule
  \caption{\label{tab:pytexWarn} Warnings of \texttt{pythontex}  }
  \endlastfoot
   340 & y & Potential directory naming collision \dots  & yes \\
   413 & y & Invalid value for package option fvextfile  & yes \\
   484 & y & Unknown option \dots                        & yes \\
   685 & y & Session xxx has rerun=never                 & yes \\
       &   & But its code or dependencies have been modified & \\
  1446 & y & The following have dependencies that have been modified & yes \\
  1737 & y & Custom code for xxx attempted to print or write to stdout & yes \\
   \end{longtable}


   \begin{longtable}{|rll|}
    \toprule
    Line Number       & * & Message  \\
    \endfirsthead
    \bottomrule
    \caption{\label{tab:pytexNotice} Notices of \texttt{pythontex}  }
    \endlastfoot
  2276 & y & Line number xxx could not be synced with the document \\
  2336 & y & x message(s) could not be classified \\
       &   & Interpreted as y, based on the return code(s) \\
\end{longtable}

  

\subsection{Failure codes}\label{subsec:pytexFCodes}

\section{The converter \dpytex}\label{sec:dpytex}

\begin{figure}[htb]
  \centering
  \ifthenelse{\boolean{texFhtLoaded}}{
  should be a picture 
  }{
  \import{}{F5_08depy2out.ptx}
  }
  \caption{\label{fig:depy2out}Conversion of a \texttt{depytx}-file using \texttt{depythontex}}
  \end{figure}


\subsection{The Input File \texttt{xxx.dpytx}}\label{subsec:dpytexInFile}

\subsection{The Output Files}\label{subsec:dpytexOutFile}

\subsection{Errors and Warnings at standard/error output}\label{subsec:dpytexErrWarn}

\subsection{Failure codes}\label{subsec:dpytexFCodes}


\bibliographystyle{alpha}
\bibliography{../lit}{}% chktex 11 


\end{document}


