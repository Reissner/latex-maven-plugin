

% ACHTUNG: BRITISCHES ENGLISCH

% Die beiden Anreden im Briefkopf und im Text müssen unterschieden werden. 
% VORSICHT: Ms. Mr. usw in Amerikanischem Englisch. 
\newboolean{istMaennlich}
\newcommand*{\AdressatTitel}{}
\newcommand*{\AnredeKopf}%
{\ifthenelse{\boolean{istMaennlich}}%
{\iflanguage{german}{Herrn}{Mr}}{\iflanguage{german}{Frau}{Ms}}}
\newcommand*{\AnredeText}%
{\ifthenelse{\boolean{istMaennlich}}%
{\iflanguage{german}{r Herr}{ Mr}}{\iflanguage{german}{ Frau}{ Ms}}}
% wenn unbekannt: 
% Dear Sir or Madam	eher brit. Englisch
% Ladies and Gentlemen eher amerik. Englisch
% To whom it may concern	eher amerik. Englisch

% Adressat
\newcommand{\Anschrift}{%\vspace{3.5cm}%
    \Firma{} \\
    \AnredeKopf{} 
    \AdressatTitel{}\AdressatVorname{} \AdressatNachname\\
    \AdressatAnschriftStrasse\\
    \AdressatAnschriftPLZOrt\\
}

% Etwa Abstand vor dem Ort, um die Seite besser aufteilen zu können.
%\newlength{\AbstandVorOrt}
%\setlength{\AbstandVorOrt}{-1em}

% Hier kommt "Ort, den Datum" 

% Betreff
%\newcommand*{\Betreff}{Bewerbung als ****}%{\textbf{Bewerbung als \Bewerberstelle}}

% Anrede im Brief
\newcommand*{\Anrede}%
{%
\iflanguage{german}{Sehr geehrte}{Dear}%
\AnredeText{} \AdressatTitel\AdressatNachname%
\iflanguage{german}{,}{}%
}
% Deutsch: ',' 
% britisch: '' (nix)
% amerikanisch: ':' 


\newcommand*{\AnredeAnonym}%
{%
\iflanguage{german}{Sehr geehrte Damen und Herren,}{Dear Sir or Madam}%
}
% Deutsch: ',' 
% britisch: '' (nix)
% amerikanisch: ':' 



% Hier kommt der Text 

\newcommand{\Grussworte}{%
%Über die Einladung zu einem persönlichen Gespräch freue ich mich und verbleibe
%
\iflanguage{german}{Mit freundlichen Grüßen,}{Yours sincerely}%
% Deutsch: mit Komma, 
% britisch: ohne Komma, 
% amerikanisch: mit Komma 
%
%\Unterschrift\\[1em] 
}

%\renewcommand{\enclname}{Anlagen}

% Hinweis auf Anlagen
% Wenn nicht benötigt, dann einfach auskommentieren.
\newcommand{\AnschreibenAnlage}
{%
%   Anlagen
}


% \iflanguage{german}
% {\LoadLetterOption{../absenderDE}}
% {\LoadLetterOption{../absenderUK}}


\iflanguage{german}
{
    % Erklaerungen zu den Optionen sind in
% http://www.tex.ac.uk/tex-archive/macros/latex/contrib/koma-script/scrguide.pdf
% zu finden. Kommentiere aus / deaktiviere, was Du nicht brauchst
% (z.B. `fromphone`).
\KOMAoptions{
    %addrfield=false,
    backaddress=false,
    foldmarks=false,
    fromalign=right, % german 
    %fromrule=aftername,
    footsepline=off,
    fromphone=false,
    frommobilephone=true,
    fromemail=true,
    fromurl=false,
    fromfax=false, 
%refline=nodate
}
}{
    % Erklaerungen zu den Optionen sind in
% http://www.tex.ac.uk/tex-archive/macros/latex/contrib/koma-script/scrguide.pdf
% zu finden. Kommentiere aus / deaktiviere, was Du nicht brauchst
% (z.B. `fromphone`).
\KOMAoptions{
    %addrfield=false,
    backaddress=false,
    foldmarks=false,
    fromalign=left, % english
    %fromrule=aftername,
    footsepline=off,
    fromphone=false,
    frommobilephone=true,
    fromemail=true,
    fromurl=false,
    fromfax=false, 
    refline=dateright% english
}
}

\iflanguage{german}
{
    % Absenderdaten:
\setkomavar{fromname}{\VollerName}
\setkomavar{frommobilephone}[]{\Mobil}
%\setkomavar{fromphone}[]{\Telefon}
\setkomavar{fromemail}[]{\EMail}
%\setkomavar{fromurl}[]{\vspace*{0.5\baselineskip}\today}
\setkomavar{fromfax}[]{\ }
\setkomavar{fromaddress}{%ohne name
\AbsenderStrasse\\
\AbsenderDPLZOrt
}

\setkomavar{subject}[]{Bewerbung als \Bewerberstelle}



% signatur linksbündig mit ordentlichem Abstand vorher für pic. 
\setkomavar{signature}{\VollerName}
\renewcommand*{\raggedsignature}{\raggedright}
\setplength{sigbeforevskip}{0.0cm}
}{
    % Absenderdaten:
\setkomavar{fromname}{\VollerName}
\setkomavar{frommobilephone}[]{\mobile}
%\setkomavar{fromphone}[]{\myPhone}
\setkomavar{fromemail}[]{\EMail}
%\setkomavar{fromurl}[]{\vspace*{0.5\baselineskip}\today}% hack: date 
\setkomavar{fromfax}[]{\ }
\setkomavar{fromaddress}{%ohne name
\AbsenderStrasse\\
\AbsenderDPLZOrt
}
%\setkomavar{refline}[]{dateright}
% This must be put in the 1st section of the letter 
\setkomavar{subject}[afteropening]{Application as \Bewerberstelle}


% signatur linksbündig mit ordentlichem Abstand vorher für pic. 
\setkomavar{signature}{\VollerName}
\renewcommand*{\raggedsignature}{\raggedright}
\setplength{sigbeforevskip}{0.0cm}

}