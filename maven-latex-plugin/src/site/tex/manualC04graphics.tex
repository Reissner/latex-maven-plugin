\chapter{Graphics and Preprocessing}\label{chap:GraphConversions}

While \LaTeX{} is really strong in text processing 
and also in formula processing, 
in itself it is weak in its graphical abilities. 
Graphics in some formats can be included directly in a \LaTeX{} document, 
but all need loading of according packages. 
For an overview of the graphic formats and the packages needed for their support 
see Section~\ref{sec:graphPkg}. 
The set of available graphic formats is extended by \emph{preprocessing}, 
i.e.\@ by processing prior to the \LaTeX{} engine. 
Preprocessing mainly consists in converting 
graphic formats not supported by \LaTeX{} packages 
into graphic formats supported by some \LaTeX{} packages. 
Section~\ref{sec:preprocTarget} provide vital information 
on the target formats. 

This software uses preprocessing for graphics only. 
Note that preprocessing is a design decision on the build tool 
and e.g. \texttt{latexmk} has no preprocessing at all. 
For details see Section~\ref{subsubsec:latexmkDiff}. 


Table~\ref{tab:graphicPreprocOverview} 
gives an overview over the formats supported via preprocessing. 
The first column lists the formats, the second one at least one editor for the format, 
and the last row contains the parameter to configure the preprocessing tool 
and give the default tool as an example. 
Sections~\ref{sec:fig2dev}, \ref{sec:gnuplot2dev}, \ref{sec:metapost}, 
\ref{sec:picSvg} and \ref{sec:picAsIs} treat each format separately. 
For all but \gls{png} and \gls{jpg} considered in Section~\ref{sec:picAsIs}, 
preprocessing is just conversion of the format 
into another format directly supported as described in Section~\ref{sec:graphPkg}. 
Historically the latter two required preprocessing 
to determine the bounding box was needed. 
We still support this to support historical techniques and to be sure 
to be able to reconstruct historical documents. 
Support for further formats can be easily added. 
If there is some need, please write an email to the author. 

Of course, to support a format, the preprocessing tools must be installed. 
It is advisable to have also an editor installed. 
Sometimes the editor is used also as converter as for \tool{inkscape}. 
For human-readable formats like \texttt{fig}, it often makes sense, 
to use both the graphical editor and the textual one. 
Note that \tool{vscode} supports the given formats more properly, 
if the extensions described in Section~\ref{subsec:instExtVsCode} are installed also. 

\begin{longtable}{|l|ll|}
\toprule
Graphic format & editor & preprocessing tool\\
\midrule
\midrule
\endfirsthead%
\bottomrule
\caption{\label{tab:graphicPreprocOverview} 
Overview over the graphic formats supported via preprocessing}
\endlastfoot%
fig             & xfig, vscode     & \texttt{fig2devCommand}, e.g.~\texttt{fig2dev}    \\
gnuplot (gp)    & vscode           & \texttt{gnuplotCommand}, e.g.~\texttt{gnuplot}          \\
MetaPost (mp)   & vscode           & \texttt{metapostCommand}, e.g.~\texttt{mpost}          \\
svg             & inkscape, vscode & \texttt{svg2devCommand}, e.g.~\texttt{inkscape}  \\
jpg, png        & gimp             & \texttt{ebbCommand}, e.g.~\texttt{ebb}                   \\
\end{longtable}


\section{Graphic formats and packages supporting them}\label{sec:graphPkg}

Find below a list of packages either allowing to include directly certain graphic formats, 
or helping with graphics indirectly. 
Although strictly speaking these techniques do not need special treatment 
of a build tool, 
this software supports these techniques by providing header files 
by injection loading the needed packages. 

We also describe in which sense these packages support graphical preprocessing. 

%
\begin{itemize}
\item[\pkg{graphicx}]
is the basic graphics package which provides the command 
\cmd{includegraphics}{} 
which allows including graphics natively 
in the formats PDF, EPS, JPG and PNG at least. 
For details see~\cite{GraX}. 
Note that PDF and EPS are target formats for graphical preprocessing, 
where PDF is embedded into PDF and EPS is embedded into DVI/XDV. 
As described in Section~\ref{sec:metapost}, 
also \gls{mps}, the target format for metapost is included using \pkg{graphicx}. 
\item[\pkg{transparent}]
allows specifying transparency in graphics. 
Even if you do not use the feature, 
some source formats do (in fact only \gls{svg}) does 
and so the according converters create according information 
and so the \LaTeX{} engine must get along with it. 
Note that this applies only for output format PDF 
and in particular not for \texttt{xelatex}. 
For details see~\cite{TransP}. 
\item[\pkg{bmpsize}]
is needed for bitmap formats like \gls{jpg} and \gls{png} only. 
Used to extract resolution and bounding box. 
FIXME\@: needed more information. 
For details see~\cite{BmpP}. 
% maybe for dvi format, where jpg and png may be included also, 
% preprocessing is needed to generate the bounding boxes. 
\item[\pkg{tikz}]
The TikZ code described in~\cite{TikzPGF23} is just in \LaTeX{} format. 
Thus, it can be included directly and does not require any preprocessing. 
Still what is needed is a good graphical editor like \texttt{tikzedt} 
with online manual~\cite{TikzEdt}. 
In later versions of this software, 3.x or so, 
it is planned that TikZ is used as new target format for graphical preprocessing, 
replacing the current combination of \LaTeX{} for texts and PDF/EPS for proper graphic. 
\item[\pkg{import}]
is strictly speaking no graphics package. 
According to its documentation~\cite{ImpoP}, 
it allows an imported file to find its own inputs 
(using ``\cmd{input}'', ``\cmd{includegraphics}'' etc.) in that directory. 
This is vital for the graphic formats for which a TEX file is imported 
which itself imports a PDF/EPS file located in the same folder 
but not in the folder of the importing file. 
It is advisable to combine the \texttt{import} package with other graphic packages 
to include graphics in separate graphic files. 
\item[\pkg{xcolor}]
allows using colors in graphics. 
Even if the author does not use colors in graphics, 
several formats, like \gls{fig}, \gls{gp} and \gls{svg} 
offer it and so the according converters 
transforming them into the native formats 
create color information which can be rendered only via \pkg{xcolor}. 
In this sense its role is comparable to that of \pkg{transparent}. 
On the other hand, the use of \pkg{xcolor} is not specific to graphics. 
For details see~\cite{XColorP}. 
\item[\pkg{pythontex}]
is strictly speaking no graphics package either but more general 
a way to include and run code within a \LaTeX{} document 
as described in~\cite{PythonTexP}. 
Note that not only Python but also other languages can be used. 
Most of them offer graphic capabilities 
and so graphics can be included also via \pkg{pythontex}. 
Nevertheless, we do not treat this technique in this chapter, 
but separately in Section~\ref{sec:pythontex}. 
This is because graphics is a side aspect of \pkg{pythontex} 
and also because strictly speaking there is no preprocessing. 
First a latex processor is run, and the package extracts the code 
into a separate file which is then further processed by an external tool. 
This is more like running \cmd{bibtex} to extract a bibliography. 

If using the package \texttt{pythontex} 
a special processing interacting with the \LaTeX{} engine is required also, 
but it is not preprocessing. 
\end{itemize}

Section~\ref{sec:injFiles} is on injection of files 
and in particular header files: 
%
\begin{description}
  \item[\texttt{header.tex}] treated in Section~\ref{subsec:header}, 
  is a general header file intended to be included into all \LaTeX{} files. 
  Since the packages \pkg{import} and \pkg{xcolor} are generally useful, 
  not only in the context of graphics, they are among those loaded in \texttt{header.tex}. 
  \item[\texttt{headerGrp.tex}] described in Section~\ref{subsec:headerGrp} in contrast, 
  is a header file loading graphic specific packages related with \pkg{graphicx}, 
  loading also \pkg{transparent} and \pkg{bmpsize}. 
\end{description}

The header files adapt the loading of the packages to the context, 
in particular to the target format. 
Note that \texttt{headerGrp.tex} must follow \texttt{header.tex}. 

The package \pkg{tikz}, although a pure graphic package 
is very specific and not related to \pkg{graphicx}. 
Thus, it must be loaded separately. 
The same holds for \pkg{pythontex}. 









Besides the converter external to \LaTeX, 
also several \LaTeX-packages are required 
to use graphics. 


This section describes the conversions of 
graphical source files into target files 
in detail. 

But PDF also occurs as an intermediate format for pictures. 
For historical reasons, still \gls{eps} is used. 
%Section~\ref{sec:figpdf} shows that it suffices to stick to pictures 
%in PDF format. 
% FIXME\@: REALLY? 
%FIXME\@: missing pdf als basic input format. What about (e)ps? 
Section~\ref{sec:fig2dev} shows how \texttt{fig2dev} converts fig-files 
into \LaTeX-files containing text and including graphics in as PDF files. 
Likewise, Section~\ref{sec:gnuplot2dev} describes 
how gnuplot converts gnuplot-files into PDF files. 
An interesting alternative to gnuplot for computing pictures 
is MetaPost described in Section~\ref{sec:metapost}. 
A more elaborate alternative to fig-pictures are SVG pictures 
described in Section~\ref{sec:picSvg}
Also several formats collected in Section~\ref{sec:picAsIs} 
may be included as is. 



\section{Target formats for preprocessing}\label{sec:preprocTarget}

At a first sight, PDF seems the ideal target format for any kind of preprocessing: 
It is really mighty enough to display pictures in any source format 
without loss of information and even without change in appearance, 
and for modern \LaTeX{} implementations directly creating PDF files, 
the \LaTeX-package \pkg{graphicx} allows including graphics as PDF files in \LaTeX-files. 

At a second sight, the source formats under consideration 
offer pictures mixing vector graphics and texts and in particular formulae 
set in \LaTeX{} style. 
Preprocessing is based on on-the-shelf converters 
and if targeting PDF, the texts originally in \LaTeX{} style change their appearance. 
To keep up \LaTeX{} style, they provide mixed export 
consisting of a PDF file containing proper graphics 
without texts and a TEX file containing the texts in proper location 
and an \cmd{includegraphics} command including the created PDF file. 
This mixed conversion is used for all kind of preprocessing. 

Note that we could have used the ending TEX for the texts, 
but we opted for a specific ending \gls{ptx} signifying that the file is created 
and thus does not slow down search of \LaTeX{} main files. 

But still there is another problem with PDF as target format: 
Traditionally \LaTeX{} produced output in the \gls{dvi}/\gls{xdv}-format 
which is still used to create \gls{html}-output. 
For \LaTeX{} engines \pdflatex{} and \lualatex, 
DVI output is specified with option \texttt{--output-format=dvi}. 
It turns out, that with this setting, 
PDF files cannot be incorporated with \cmd{includegraphics} command. 
Instead, one must use \gls{eps} files. 
Fortunately, the graphic converters used also support combined TEX/\gls{ptx} and EPS formats. 
We ensured that \cmd{includegraphics} in the \gls{ptx} file 
specifies the file without ending so that the \gls{ptx} file is the same, 
whether it encloses a PDF file or an EPS file,
and we provide both, a PDF file and an EPS file\footnote%
{Of course, here a more sophisticated technique is conceivable, 
recognizing the required format and generating the specific one if missing. }. 
That way, both, \pdflatex{} and \lualatex{} choose the EPS file or the PDF file 
depending on whether the output format is \texttt{--output-format=dvi} 
or \texttt{--output-format=pdf} which is the default. 
Note that \xelatex, which always creates an intermediate XDV file 
(which is a special kind of DVI file), 
acts differently: If present, it prefers including the PDF file, 
if absent, but there is an EPS file instead, it uses this without making any difference. 

Although this is beyond necessity, 
let me state that \pdflatex{} and \lualatex, 
while not accepting inclusion of PDF files in DVI mode, 
EPS files are accepted in PDF mode for more modern versions of the \LaTeX{} engines, 
but this leads to creation of intermediate files \texttt{xxx-eps-converted-to.pdf}, 
which are not cleaned up in target \texttt{clr}. 

Whereas PDF and EPS files both are offered, 
only one of them is included for a specific configuration. 
This is in contrast to other formats described in Section~\ref{sec:picAsIs}. 

Although \gls{ptx} is just a TEX format, 
it is special in that it presupposes that some packages are loaded before being included. 
The packages which are not specific for graphics like \pkg{xcolor} are loaded in \texttt{header.tex} 
described in Section~\ref{subsec:header}, 
whereas the ones specific for graphics, above all \pkg{graphicx}, 
are loaded in \texttt{headerGrp.tex} 
as described in Section~\ref{subsec:headerGrp}. 
The packages actually to be loaded and their respective options 
depend on the configuration. 

Note that PDF and EPS file may be created by preprocessing 
but also as proper sources not created at all, 
even in a single document. 
Goal \texttt{clr} deletes the according files \texttt{xxx.pdf} or \texttt{xxx.eps} only, 
if an according source like \texttt{xxx.fig} exists. 
Else it is treated as proper source and is not deleted. 


In the future, the combination of PDF/EPS and PTX files may be replaced, 
at least partially, or supplemented by TikZ files. 
It turned out, that the converters under consideration 
support more and more conversion into the TikZ format 
which can represent both, proper vector graphic and also \LaTeX{} texts like formulae. 
Using TikZ as intermediate format has the advantage, 
that the working space is polluted less with generated files, 
preprocessing is speeded up because fewer files are created and in some cases, 
less processing steps are needed. 
Another advantage is, that the internal dependency recording of \LaTeX{} engines 
made available through the \gls{fls} file is accessible. 
% TBD: not so easy: The created 
As in the current technique using PTX files instead of TEX files, 
we could put the TikZ into TEX files, but we opt against it for the same reasons. 

Note that PS is not supported because it misses the bounding box. 
% TBD: clarify whether we could add a bounding box by preprocessing 
If adding it, one arrives at the EPS format. 



\section{Conversion of fig-files}\label{sec:fig2dev}



A simple but still useful tool to draw figures is \texttt{xfig}\index{xfig}\index{fig2dev} 
which stores graphics in a native format 
described in~\cite{XFigF} with file extension \texttt{.fig}. 
The file extension \texttt{.fig} is also used by MATLAB to store plots, 
but this is something different. 
Graphics in xfig format cannot be directly included in latex files 
but must be exported into a \LaTeX-readable format. 

To export a file \texttt{xxx.fig} residing in directory \texttt{yyy} 
into several external formats, 
\texttt{xfig} uses \texttt{fig2dev}. 
A look in~\cite{XFigF}, Section~3.4 shows that texts with set ``special''-flag 
are interpreted as latex-code. 
For these texts the appropriate export language would be \texttt{latex}. 
On the other hand, \texttt{latex} is weak in graphics 
and \texttt{pdf} would be the ideal export format for all kinds of objects, 
except for texts with set ``special''-flag. 
In \texttt{pdf} format, texts are interpreted literally, 
independent of the ``special''-flag. 
Thus, \texttt{fig2dev} offers a mixed solution: 
export \texttt{xxx.fig} in format \texttt{pdftex} which yields a pdf-file 
\texttt{xxx.pdf} containing all but text with set ``special''-flag 
and complementary \texttt{pdftex\_t} which yields a tex-file \texttt{xxx.ptx} 
including the pdf-file and the texts with set ``special''-flag\index{special-flag}. %maybe glossary 
The exported files are in the same directory \texttt{yyy} 
as the original file \texttt{xxx.fig}. 

For example, 
the fig-file \texttt{F4\_01fig2dev.fig} defining Figure~\ref{fig:fig2dev}, 
is transformed into a file \texttt{F4\_01fig2dev.ptx} 
in format \texttt{pdftex\_t} which starts as given by Listing~\ref{lst:ptx}. 

%\lstset{language=tex, breaklines, basicstyle=\footnotesize}
% FIXME\@: nag complains for htlatex 
\lstinputlisting[language=TeX, basicstyle=\tiny,
breaklines, lastline=25,
float, captionpos=b, label={lst:ptx}, 
caption={The ptx-file for a fig-file}]{F4_01fig2dev.ptx}

The file \texttt{xxx.ptx} is ``imported'' into the tex-file of this manual 
by the command 
%
\begin{lstlisting}[language=TeX]
\import{yyy}{xxx.ptx}
\end{lstlisting}
%texttt
and includes \texttt{xxx.pdf} automatically the file \texttt{xxx.pdf} 
via \cmd{includegraphics\{xxx\}}{} (line 2). 
Note the following remarkable details: 
%
\begin{itemize}
\item
Observe that we can drop the suffix of the included file \texttt{xxx.pdf} 
which is expressed as ``\texttt{xxx}'' 
because \LaTeX{} chooses the right suffix: 
If instead of \texttt{xxx.pdf} there is a file \texttt{xxx.eps}, 
the latter is chosen if no suffix is specified. 
As we will see below, 
omitting the suffix is crucial to make \texttt{xxx.ptx} work 
for both \LaTeX-output formats: 
the pdf-format can include pdf-files, 
whereas the dvi-format which is required to create html- and
odt-files can include eps-files. 
\item
If \texttt{xxx.pdf} is included in \texttt{xxx.ptx} 
with the full path name, 
we may use \cmd{input\{xxx.ptx\}}{} instead of \cmd{import\{yyy\}\{xxx.ptx\}}. 

If in contrast, \texttt{xxx.pdf} is included in \texttt{xxx.ptx} 
with the short name only, 
\texttt{xxx.pdf} is assumed to be in the same directory 
as the file inputting \texttt{xxx.ptx}. 
So in general, i.e.~if this is not \texttt{yyy}, we need import 
\cmd{import\{yyy\}\{xxx.ptx\}}. 
If the directories coincide, 
in the import the string \texttt{yyy} may be empty. 
If the string \texttt{yyy} is not empty, it must end with the path delimiter, 
i.e.~\texttt/ for Unix like systems and 
\texttt{\textbackslash} for win-like systems. 
\end{itemize}

As indicated in Section~\ref{sec:graphPkg}, 
the commands in \texttt{xxx.ptx} 
require the packages \pkg{graphicx} and \pkg{xcolor}. 
Also, the \cmd{import}{} command 
requires the \pkg{import} package. 

To export \texttt{xxx.fig} into \texttt{xxx.ptx} and \texttt{xxx.pdf} 
this software invokes two commands: 
%
\begin{Verbatim}[fontsize=\scriptsize]
fig2dev -L pdftex   <fig2devGenOptions> <fig2devPdfEpsOptions>        xxx.fig xxx.pdf   
fig2dev -L pdftex_t <fig2devGenOptions> <fig2devPtxOptions>    -p xxx xxx.fig xxx.ptx
\end{Verbatim}
%
Both commands specify the input file \texttt{xxx.fig}, 
both use the options given by the parameter \texttt{fig2devGenOptions} 
while each invocation allows to specify also specific options, 
\texttt{fig2devPdfEpsOptions} and \texttt{fig2devPtxOptions}, respectively, 
and both use the option \texttt{-L} 
to specify the output format (``language''). 

%FIXME\@: ptx-->tex and then: convention: intuitive suffixes 

The parameters specific for \texttt{pdftex} 
are called \texttt{fig2devPdfEpsOptions} 
because the options available are the same 
as for output format \texttt{pstex} creating eps-files. 
An example for a common option would be \texttt{-b width} 
which shall specify the same boundary for both formats; 
otherwise they do not fit. 

For the output format \texttt{pdftex\_t}, 
the option \texttt{-p xxx} says, 
that the string \texttt{xxx} must be included in \texttt{xxx.ptx} 
as \cmd{includegraphics\{xxx\}}. 
Note that the option \texttt{-p} shall not be specified 
in \texttt{fig2devPtxOptions}, because it is automatically added. 

Equivalent to mixed export with formats \texttt{pdftex} and \texttt{pdftex\_t} 
which is appropriate for \LaTeX-output format pdf, 
is the mixed export with the according formats 
\texttt{pstex} and \texttt{pstex\_t} appropriate for \LaTeX-output format dvi. 
The difference is that \texttt{pstex} creates an eps-file instead of a pdf-file 
with the same content 
and \texttt{pstex\_t} creates a tex-file which looks like that 
created by \texttt{pdftex\_t} except including the eps-file 
instead of the pdf-file. 
If the suffix is not given, 
\texttt{pstex\_t} and \texttt{pdftex\_t} create identical files. 
Thus exporting \texttt{xxx.fig} via 
%
\begin{Verbatim}[fontsize=\scriptsize]
fig2dev -L pstex    <fig2devGenOptions> <fig2devPdfEpsOptions>        xxx.fig xxx.eps   
fig2dev -L pdftex   <fig2devGenOptions> <fig2devPdfEpsOptions>        xxx.fig xxx.pdf   
fig2dev -L pdftex_t <fig2devGenOptions> <fig2devPtxOptions>    -p xxx xxx.fig xxx.ptx
\end{Verbatim}
%
and ``inputting'' \texttt{xxx.ptx} works for both \LaTeX{} output formats. 

Table~\ref{tab:xfigSuffixes} relates the language specified with the \texttt{-L} option 
with the suffix of the output file chosen canonically, the suffix we choose 
and the actual file format. 
In contrast to \texttt{fig2dev}, 
we choose the actual file format, except if this is TEX\@. 
We opted for the quite unusual suffix \texttt{.ptx} 
instead of \texttt{.tex} 
to avoid that TEX-files may be both, 
source files and created files, 
but this is not compulsory, 
since the same holds and is accepted for pdf-files. 

\begin{longtable}{|l|lll|}
\toprule
Output format (language) & xfig suffix & our suffix & format \\
\midrule
\midrule
\endfirsthead%
\bottomrule%
\caption{\label{tab:xfigSuffixes} Language, suffixes and file format }
\endlastfoot%
pstex                    & pstex       & eps        & eps \\
pstex\_t                 & pstex\_t    & ptx        & tex \\
pdftex                   & pdf         & pdf        & pdf \\
pdftex\_t                & pdf\_t      & pdf        & pdf \\
\end{longtable}


Maybe xfig is intended to export from within the export dialog 
and not directly via a script like \texttt{fig2dev}. 
This may be the reason 
why the magnification must be set in the export dialog, 
but it is stored in the fig-file nevertheless. 

Figure~\ref{fig:fig2dev} shows the transformation 
of figures with \texttt{fig2dev} 
and the inclusion of the eps-file and of the pdf-file in the ptx-file. 
Note that the \texttt{fig2dev}-command is configurable 
via the parameter \texttt{fig2devCommand}, 
but there will be hardly any command with the same command line interface 
performing exactly the transformations given in Figure~\ref{fig:fig2dev}, 
except \texttt{fig2dev} itself. 

At the same time, Figure~\ref{fig:fig2dev} is an example 
for a \LaTeX-file \texttt{xxx.ptx} created from a fig-file 
and embedded in this \LaTeX-file 
with the \cmd{input}-command. 
More than that, 
Figure~\ref{fig:fig2dev} describes the way it has been created. 
Note that all text labels are specified with set ``special''-flag, 
and are thus included as \LaTeX-text, 
except the text \textbf{\tiny postscript} 
which is typeset with a postscript font to make the difference visible. 


\begin{figure}[htb]
\centering
\IfPackageLoadedTF{tex4ht}{%
should be a picture 
}{
\import{}{F4_01fig2dev.ptx}
}
\caption{\label{fig:fig2dev}Conversion of a fig-file 
into pdf-, eps- and ptx-files with inclusions}
\end{figure}


\section{Conversion of gnuplot-files}\label{sec:gnuplot2dev}

The term ``gnuplot'' refers to a file format
and to a program \texttt{gnuplot}
which can read this format, both described in~\cite{GnuPlot6_0}. 

Note that there seems no official file extension 
to identify gnuplot files. 
From the most common extensions \texttt{.plt}, \texttt{.gpi} and \texttt{.gp} 
we have chosen the one with the least collision 
and supported by Emacs, vscode and by my file browser: \texttt{.gp}. 
% FIXME\@: should be configurable.
% FIXME\@: look at https://www.file-extensions.org/
% FIXME\@: have a look in general on extensions for vscode 
% and include them into this manual 

The gnuplot format is a textual command language you can even program with 
and may thus be created with any editor but
for sake of reproducibility it is recommended to use only files
created by \texttt{gnuplot}.
To ensure that a handwritten gnuplot file \texttt{xxx.gp},
e.g.\@ with a single line like
%
\begin{verbatim}
plot [-10:10] sin(x), atan(x), cos(atan(x))
\end{verbatim}
%
really works
with the current \texttt{gnuplot} and to see how it is interpreted,
it is recommended to convert it via
%
\begin{lstlisting}
gnuplot -persist -e "load 'xxx.gp'; save 'xxx.gp'"
\end{lstlisting}
%
If you have a look inside the resulting file 
\href{\urlSite fromTex/F4_03someGnuplot.gp}{\texttt{F4\_03someGnuplot.gp}}, 
you can see, that in a comment line
the current version of \texttt{gnuplot} is documented
and also all the settings implicitly used.
The original line is the last but one. 
Pasting the into VS Code, one can see the highlighting, 
of course provided the extensions 
described in Section~\ref{subsec:instExtVsCode} are installed. 

Also, if a gnuplot file is created with an old version of \texttt{gnuplot},
it is recommended to update version with the same command.
Note that \texttt{gnuplot} does not offer full backward compatibility. 


This software supports including 
figures stored in \texttt{.gp}-files created by \texttt{gnuplot}.
To export a file \texttt{xxx.gp} into several external formats, 
it uses \texttt{gnuplot} itself. 
According to the manual~\cite{GnuPlot6_0}, Part IV, 
\texttt{gnuplot} supports output formats through so-called \emph{terminals}. 
Among those are several ones intended for inclusion into \LaTeX-files, 
like \texttt{Cairolatex}, \texttt{Epscairo}, \texttt{Epslatex}, 
\texttt{Latex}, \texttt{Lua (tikz)}, \texttt{Postscript}, \texttt{Ps(la)tex}, % chktex 36
\texttt{Pstricks}, \texttt{Texdraw} and \texttt{Tikz} 
which is in fact equivalent with \texttt{Lua (tikz)}. 
Comparison with the manual~\cite{GnuPlot5_0} for older versions of \texttt{gnuplot} 
shows that support of \texttt{Eepic}, \texttt{Mp} and \texttt{Tpic} ended. 
Note that also export into the fig-format via the terminal \texttt{Fig} 
is supported which in turn may be included in latex 
as described in Section~\ref{sec:fig2dev}. 
Also, \texttt{gnuplot} pictures may be exported in MetaPost format 
which in turn may be included in latex 
as described in Section~\ref{sec:metapost}. 
% TBD: clarify maybe tikz is superior. see backlog. 

This software supports the export of a file \texttt{xxx.gp} 
only via the terminal \texttt{Cairolatex} 
which offers export to mixed PDF and \LaTeX\@: 
graphics in PDF and text in \LaTeX{}
which yields the fonts typical for \LaTeX. 
This is as described for fig-files in Section~\ref{sec:fig2dev}, 
except that text is generally converted in \LaTeX{}-format, 
and not selectively those text marked with special flag. 

Accordingly, the export yields two files \texttt{xxx.ptx} and
\texttt{xxx.pdf}, both in the directory \texttt{yyy} 
in which \texttt{xxx.gp} resides. 
The file \texttt{xxx.ptx} must be imported via 
%
\begin{lstlisting}[language=TeX]
\import{yyy}{xxx.ptx}
\end{lstlisting}
%
It contains the texts and includes \texttt{xxx.pdf} 
via \cmd{includegraphics\{xxx\}}{} without specifying a suffix. 

Unlike for fig-files, 
\texttt{xxx.ptx} and \texttt{xxx.pdf} are created with a single command: 
%
\begin{verbatim}
gnuplot -e "set terminal cairolatex pdf <gnuplotOptions>;
            set output 'xxx.ptx';
            load 'xxx.gp'"
\end{verbatim}

Accordingly, 
\texttt{xxx.ptx} and \texttt{xxx.eps} are created with a single command: 
%
\begin{verbatim}
gnuplot -e "set terminal cairolatex eps <gnuplotOptions>;
            set output 'xxx.ptx';
            load 'xxx.gp'"
\end{verbatim}
%
Note that this writes another but identical file \texttt{xxx.ptx} 
as no file endings are written 
and so \texttt{xxx.ptx} can include both, \texttt{pdf} and \texttt{eps}. 
When creating both performance is not optimal, 
but \texttt{gnuplot} offers no way to avoid this. 
If being strict, 
\texttt{xxx.ptx} is perfectly correct only for output \texttt{eps},
% TBC: in which respect? 
if comments and error messages are taken into account 
but as long as no error occurs, 
the result is perfectly ok also for \texttt{pdf}. 

As for inclusion of fig-files, 
packages \pkg{graphicx} and \texttt{color} are needed. 


%FIXME\@: 
%Here further options are missing: 
%   // set terminal pdf {monochrome|color|colour}
%    //                      {{no}enhanced}
%    //                      {fname "<font>"} {fsize <fontsize>}
%    //                      {font "<fontname>{,<fontsize>}"}
%    //                      {linewidth <lw>} {rounded|butt}
%    //                      {solid|dashed} {dl <dashlength>}}
%    //                      {size <XX>{unit},<YY>{unit}}

Figure~\ref{fig:gp2pdf} shows the transformation of the plots 
and the inclusion of graphic files. 
In addition, Figure~\ref{fig:gnuplot} shows an example of a \LaTeX-file 
created from a gnuplot file 
and embedded in this \LaTeX-file. 
%Note that the \texttt{gnuplot}-command is configurable 
%via the parameter \texttt{gnuplotCommand}, 
%but there will be hardly any command with the same command line interface 
%performing exactly the transformations given in Figure~\ref{fig:gp2pdf}, 
%except \texttt{gnuplot} itself. 

\begin{figure}[htb]
\centering
\IfPackageLoadedTF{tex4ht}{%
should be a picture 
}{
\import{}{F4_02gp2pdf.ptx}
}
\caption{\label{fig:gp2pdf}Conversion of a gnuplot-file 
into pdf-, eps- and ptx-files with inclusions}
\end{figure}

\begin{figure}[htb]
\centering
\IfPackageLoadedTF{tex4ht}{%
should be a picture 
}{
\import{}{F4_03someGnuplot.ptx}
}
\caption{\label{fig:gnuplot}
Converted sample gnuplot-file into ptx and pdf files }
\end{figure}


\section{Inclusion of MetaPost files}\label{sec:metapost}

A vector graphic format, very native to TeX is \texttt{MetaPost}, 
a derivative of \texttt{Metafont} originally used to describe shape of fonts. 
Although seemingly supported by \TeX{} only, 
\texttt{MetaPost} is interesting in its own right, 
as it is a graphical programming language, 
Turing complete, much like postscript, and allows also declarative programming. 
The manual describing the language is~\cite{MPost24}, 
seemingly complete, but it is not. 
Thus, one can be thankful for~\cite{MPostGuid} 
which offers some introduction and for the really helpful tutorial~\cite{MPostTut}. 

Files containing \texttt{MetaPost} have the ending \texttt{.mp}. 
Note that there are other graphic formats 
like monochrome pictures in TIFF-format 
which are identified with the same extension 
but the MetaPost format has nothing to do with this. 

Since \texttt{MetaPost} is a programming language, 
MetaPost files are created with an editor. 
Since \texttt{MetaPost} is very versatile, 
it is impossible to give an impression by a single example. 
We decided to choose an example 
using a MetaPost library, \texttt{MetaUML}, 
described in~\cite{MetaUml} for some reasons apparent later. 
The example file is given in Listing~\ref{lst:metapost} 
and also on the web as 
\href{\urlSite fromTex/F4_05someMetapost.mp}{\texttt{F4\_05someMetapost.mp}}. 
It is the source file of Figure~\ref{fig:metapost}. 
Pasting the into VS Code, one can see the highlighting, 
of course provided the extensions 
described in Section~\ref{subsec:instExtVsCode} are installed. 


\lstinputlisting[language=MetaPost,
basicstyle=\footnotesize,
float,
showstringspaces={false},
numbers={left}, stepnumber={5}, 
captionpos=b, label={lst:metapost},
caption={An example file in MetaPost}]
{F4_05someMetapost.mp}

Listing~\ref{lst:metapost} illustrates some structure of MetaPost. 
As in \TeX, comments start with \texttt{\%} and end with the line or with the file. 
The proper figures are enclosed between \texttt{beginfig(n)} and \texttt{endfig}, %chktex 36
where $n$ is the number of the figure, the so called \texttt{charcode}\footnote%
{This is a relict from Metafont, where each figure showed a character}, 
and the file ends with \texttt{end}. 
This software relies on specifying a single figure per file; the \texttt{charcode} is irrelevant. 

Code outside figures is possible, but does not belong to a figure 
and is thus not displayed. 
In our example, besides \texttt{end} commands outside the figure 
are just \texttt{input xxx}, where \texttt{xxx} names a so-called library 
defined by the file \texttt{xxx.mp} 
and a sequence of settings of internal variables 
of the MetaPost compiler 
controlling how the following figure is compiled. 
Most of them even in comments. 

The compiler for \texttt{MetaPost} 
is given by the parameter \texttt{metapostCommand} 
which defaults to \texttt{mpost}, occasionally just \texttt{mp}. 

Each internal variable which can be set in the MP file 
can also be set when invoking \texttt{mpost} 
using the option \texttt{-s \textlangle{}variable\textrangle=\textlangle{}value\textrangle} 
as described in~\cite{MPost24},~Section~B.2.1. 
There it is stated that the option is read just before the file is read, 
which implies that the setting in the file overrides the command line setting. 
Caution: in the manual, the variable is referred to as ``key''. 

The most basic setting is \texttt{outputformat:="eps"} %chktex 18
which is the only setting appropriate for latex. 
So don't change\footnote%
{Note that \texttt{metapostCommand} may also besides \gls{eps} 
output \gls{svg} and \gls{png}, 
just by setting \texttt{outputformat:="svg"} or that like. %chktex 18
Caution: case-sensitive, assuming silently \texttt{eps} if the format is not recognized. 
Whereas \gls{svg} is a vector format as MetaPost ifself, \gls{png} is a raster format}. 
Note the strange default setting for the names of the output files, \texttt{outputtemplate}, 
which reflects the \texttt{charcode} of the individual figures as file ending. 
For inclusion in latex, the file ending \texttt{mps} is required 
and so frequently \texttt{outputtemplate} is set to reflect the ending. 
It seems more appropriate to make the setting in the command line 
which yields the following invocation
%
\begin{verbatim}
  mpost -s 'outputtemplate="%{jobname}%{charcode}.mps"' xxx.mp
\end{verbatim}

As we agreed that a MetaPost file shall contain a single figure only, 
we also ignore the \texttt{charcode} which unifies MetaPost with other formats supported. 
This yields 
%
\begin{verbatim}
  mpost -s 'outputtemplate="%{jobname}.mps"' xxx.mp
\end{verbatim}
%
The MetaPost file shall not overwrite the command line settings. 

The setting of \texttt{prologues} controls where fonts come from and 
becomes relevant when using \TeX{} for typesetting. 
Listing~\ref{lst:metapost}, line 21 includes a label via a note implicitly, 
and for the material between \texttt{btex} and \texttt{etex} uses \TeX{}. 
The manual~\cite{MPost24}, Section 8.1 is on typesetting labels 
and specifies the meaning of \texttt{prologues}. 
If we stick to including in \LaTeX{} and creating PDF out of that only, 
the default setting $0$ is appropriate always 
but since this software uses DVI as intermediate format, e.g.\@ to create HTML, 
or because for debugging one wants to view the MPS files standalone in a viewer 
things are not so easy. 
For details see~\cite{MPost24}, Section 14.2. 
Setting \texttt{prologues:=1} is deprecated. 
The only save way to get the correct display is to include fonts in the MPS file, 
setting \texttt{prologues:=3}, but this makes the MPS file quite big. 
So a good compromise is to set \texttt{prologues:=2} as a command line option 
resulting in 
%
\begin{verbatim}
  mpost -s prologues=2 -s 'outputtemplate="%{jobname}.mps"' xxx.mp
\end{verbatim}
%
and overwriting by need as in Listing~\ref{lst:metapost}, line 2. 

As mentioned above, \texttt{input xxx} includes a library 
making the program dependent on a file \texttt{xxx.mp}. 
As for latex processors, also \texttt{mpost} records dependencies recursively 
in an FLS file if invoked with option \texttt{-recorder}. 
Also like latex processors, an error shall not cause break or interaction 
so adding the option \texttt{-interaction=nonstopmode}. 
Thus, we arrive finally at the default invocation 
%
\begin{verbatim}
  mpost -interaction=nonstopmode -recorder \
    -s prologues=2 -s 'outputtemplate="%{jobname}.mps"' xxx.mp
\end{verbatim}






% TBD: treat mpxerr.tex, mpxerr.log
% this must be done both in software and in this manual 





Figure~\ref{fig:mp2mps} illustrates 
how \texttt{mpost} converts an \gls{mp}-file \texttt{xxx.mp} 
with the given settings into various result files: 
%
\begin{itemize}
\item
an \gls{mps}-file or with setting 
\begin{verbatim}outputtemplate="\%{jobname}\%{charcode}.mps"\end{verbatim}
more \gls{mps}-files \texttt{xxx1.mps}\dots \texttt{xxxn.mps}, 
\item
a log-file \texttt{xxx.log} and a fls-file \texttt{xxx.fls} much like \LaTeX{} does 
\item
and an \gls{mpx}-file \texttt{xxx.mpx} containing the \LaTeX{} text of the figure; 
this is not created if there is no such text. 
\end{itemize}



\begin{figure}[htb]
\centering
\IfPackageLoadedTF{tex4ht}{%
should be a picture 
}{
\import{}{F4_04mp2mps.ptx}
}
\caption{\label{fig:mp2mps}Conversion of a MetaPost-file into an mps-file}
\end{figure}


Figure~\ref{fig:metapost} gives an example of a MetaPost file 
included in this \LaTeX-file as ab mps-file 
created from the MetaPost file 
and embedded in this \LaTeX-file 
with the \cmd{includegraphics}-command. 
Normally, \cmd{includegraphics} is invoked with the filename without extension, 
but for mps-files, the extension is needed. 
As for inclusion of fig-files, the package \pkg{graphicx} is needed. 

\begin{figure}[htb]
\centering
\IfPackageLoadedTF{tex4ht}{%
should be a picture 
}{ 
%\includegraphics{F4_05someMetapost1.mps}
\includegraphics{F4_05someMetapost.mps}
}
\caption{\label{fig:metapost}
Converted sample MetaPost-file included as mps-file  }
\end{figure}

One of the descendants of MetaPost is TikZ (see introductory text~\cite{TikzIntro})
and one of the deficiencies resolved is that it allows passing information 
from the main document to the proper figure. 

With \lualatex{} this can be reached for MetaPost also using package \pkg{luamplib}. 
The package itself provides an environment \texttt{mplibcode}. 
Essentially, \lualatex{} interprets all code 
enclosed in the \texttt{mplibcode} environment as MetaPost. 
As described in Section~\ref{subsec:header}, 
this software can inject a header which loads the header 
and enhances it providing the additional command \cmd{inputmpcode} 
which allows also load MetaPost from a file. 
The latter is preferred to direct inclusion with the \texttt{mplibcode} environment, 
e.g.\@ for sake of proper code highlighting. 
Note that the package declaration is enclosed in an if construct, 
ensuring that the package is loaded only if lualatex or that like is run. 

That this allows better integration within the enclosing latex document 
is illustrated by redefining the letter $\aleph$ as $\alpha$ which is really related. 
%
\begin{lstlisting}[language=TeX]
  {% make redefine local 
  \renewcommand{\aleph}{\alpha}
  \inputmpcode{F4_05someMetapost}
  }% to recover from redefine {manualC4graphics.tex}
\end{lstlisting}

Figure~\ref{fig:metapostLua} Documents, 
that the redefinition really influences rendering in the MetaPost file. 

\begin{figure}[htb]
  \centering
  \IfPackageLoadedTF{tex4ht}{%
  should be a picture 
  }{
  \ifluatex%
  {% make redefine local 
  \renewcommand{\aleph}{\alpha}
  \inputmpcode{F4_05someMetapost}
  }% to recover from redefine 
  \else
  \includegraphics{F4_05someMetapost.mps}
  \fi
  }%texFhtLoaded
  \caption{\label{fig:metapostLua}
  Sample MetaPost-file included via \texttt{luamplib} for lua(hb)tex } % chktex 36
\end{figure}

  
  %Question: how to center, how to scale? 
  %Interesting: interaction between MetaPost code and enclosing tex: 
  
  



% I cannot detect an mps file. 
  
% This example is simple but not very illustrative. 
% In the long run we shall use a state diagram based on~\cite{MetaUml}. 

% There must be also a section on installation of packages: 
% in suse linux it is ready to go, but in general, it is not. 
% kpsewhich metauml.mp
% /usr/share/texmf/metapost/metauml/metauml.mp
% shows the location. 
% Put https://github.com/ogheorghies/MetaUML/ there 
% and mktexlsr and then check again with kpsewhich. 
% docs is in 
% /usr/share/texmf/doc/metapost/metauml/README

% What is interesting here also is, that mp can be included directly in lualatex. 
% There shall be some illustration for this. 

% Similarly one can install latex packages. 
% Example: tikz-uml: 
% kpsewhich tikz-uml.sty shows 
% /usr/share/texmf/tex/latex/tikz-uml/tikz-uml.sty

% ok download, move accordingly and also move the docs 
% /usr/share/texmf/doc/amstex/base/README
% /usr/share/texmf/doc/latex/geometry/README.md



\section{Inclusion of SVG-files}\label{sec:picSvg}

Comparable with the xfig-format described in Section~\ref{sec:fig2dev} 
but much more elaborate and widely used is the \gls{svg}-format. 
There is a huge up-to-date official SVG 1.1 specification,~\cite{Svg11} 
and a specification~\cite{Svg12Tiny} for SVG Tiny 1.2, 
which is itself quite short and more readable 
and gives also a good overview on ``SVG Big''. 
For a tutorial, see~\cite{SvgTut}. 
As stated in~\cite{Svg12Tiny}, Section~1.1, 
SVG-files may contain vector graphics, raster images and text. 
It may also contain video and audio elements 
and may be interactive and dynamic, 
which goes beyond what can be included in \LaTeX-files. 

Figure~\ref{fig:svgWithText} shows a picture in \gls{svg}-format. 
As PDF-files are included directly 
via the \cmd{includegraphics}-command, 
using the \LaTeX-packages \pkg{xcolor} and \pkg{graphicx}, 
virtually, 
\texttt{xxx.svg} can be included directly via 
% FIXME\@: done differently 
\begin{lstlisting}[language=TeX]
%\includesvg[width=0.5\textwidth]{xxx}%
\end{lstlisting}
%
using the \LaTeX-packages \pkg{svg} described in~\cite{SvgP}. 
Note that the suffix of the file name shall be omitted. 

A closer look shows, that graphic preprocessing is done behind the scenes 
in the course of a \LaTeX-run 
creating files \texttt{xxx.pdf} and \texttt{xxx.pdf\_tex}. 
As described for fig-files in Section~\ref{sec:fig2dev} 
and for gnuplot-files in Section~\ref{sec:gnuplot2dev}: 
The latter is a \LaTeX-file containing text 
and including the former. 
To include \texttt{xxx.pdf} 
of course the \LaTeX-packages \pkg{xcolor} and \pkg{graphicx} 
are required. 
Moreover, it may happen that the \LaTeX-package \pkg{transparent} 
is required also, depending on the features used in \texttt{xxx.svg}. 

As indicated in~\cite{SvgP}, Section~1, 
the \pkg{svg}-package delegates the transformation 
of \texttt{xxx.svg} \texttt{xxx.pdf} and \texttt{xxx.pdf\_tex} 
to \texttt{inkscape}. 
This is a graphical editor with export functions 
which can be invoked in batch-mode also. 
Of course using the \pkg{svg}-package has the advantage 
that no explicit preprocessing is required, 
the created files updated by need. 
It is worth thinking about whether it is worthwhile 
writing according packages \texttt{fig} and \texttt{gnuplot}. 

On the other hand, 
this breaks the workflow this software normally applies to graphic files. 
In particular, the package creates \LaTeX{} main files 
which are not removed after the latex run 
if parametrized accordingly or if something goes wrong. 
Also, the \pkg{svg}-package does not provide the full flexibility 
of a standard solution. 
Since this software is still under construction 
and more than that, is in an experimental phase, 
we provide explicit preprocessing of SVG-files using \texttt{inkscape}. 
Another problem with the \pkg{svg}-package is, 
that according to~\cite{SvgP}, Section~1, 
it does not work on Windows platforms. 


Some research shows,
that \texttt{inkscape} in the version current at time of this writing
exports mixed PDF and latex: If invoked as 
%
\begin{Verbatim}[fontsize=\footnotesize]
inkscape --export-filename=xxx.pdf --export-area-drawing --export-latex xxx.svg 
\end{Verbatim}
%
\texttt{inkscape} creates a file \texttt{xxx.pdf}
containing all graphics but text and another file \texttt{xxx.pdf\_tex}
containing text and including \texttt{xxx.pdf}.
The file \texttt{xxx.pdf\_tex}
can be integrated into the latex document as
%
\begin{lstlisting}[language=TeX]
\def\svgwidth{0.5\textwidth}
\import{yyy}{xxx.pdf\_tex}%
\end{lstlisting}
%
Unlike \texttt{fig2dev} and \texttt{gnuplot}, 
specifying the files with their full path, 
has no effect, i.e.~inclusion uses the file name only. 
Thus, \cmd{import}{} cannot be replaced by \cmd{input}{} 
and so the \LaTeX-package \pkg{import} is required. 
 

This is essentially the same technique as applied for fig-files 
and for gnuplot-files as described 
in Sections~\ref{sec:fig2dev}~and~\ref{sec:gnuplot2dev}. 

Analogously,
%
\begin{Verbatim}[fontsize=\footnotesize]
inkscape --export-filename=xxx.eps --export-area-drawing --export-latex xxx.svg 
\end{Verbatim}
%
exports files \texttt{xxx.eps\_tex} and \texttt{xxx.eps}.

In older versions of \texttt{inkscape},
there was a configuration allowing \texttt{xxx.eps\_tex}
to include uniformly both \texttt{xxx.pdf} and \texttt{xxx.eps}.
Thus, \texttt{xxx.pdf\_tex} could be deleted
and \texttt{xxx.eps\_tex} moved to \texttt{xxx.ptx}
which in turn could be included into the main document.

As shown in Figure~\ref{fig:svg2pdf},
for the current version of \texttt{inkscape},
this software filters \texttt{xxx.eps\_tex} into \texttt{xxx.ptx}
``manually'' so that both \texttt{xxx.pdf} and \texttt{xxx.eps}
are included in \texttt{xxx.ptx}.
Then it deletes the original files
\texttt{xxx.pdf\_tex} and \texttt{xxx.eps\_tex}.

The author has filed a bug report to the inkscape team,
to avoid this workaround in the future. 

\begin{figure}[htb]
\centering
\IfPackageLoadedTF{tex4ht}{%
should be a picture 
}{
\import{}{F4_06svg2pdf.ptx}%
}
\caption{\label{fig:svg2pdf}Conversion of an SVG-file 
into pdf-, eps- and ptx-files with inclusions}
\end{figure}


\begin{figure}[htb]
\centering
%\def\svgwidth{0.5\textwidth}
\IfPackageLoadedTF{tex4ht}{%
should be a picture 
}{
  \import{}{F4_07someSvg.ptx}% this should be the standard
}
\caption{\label{fig:svgWithText}Some svg-picture with text FIXME\@: uniformity  }
\end{figure}

In contrast to the FIG format, 
SVG pictures can be created by several programs. 
Among those, is also \texttt{inkscape} 
which can be used like \texttt{xfig}  
as a graphical editor with export functionality. 
In contrast to FIG format, 
SVG is essentially human-readable, 
in fact an XML derivative. 
The author calls it ``essentially'', 
referring to the fact, that the format is quite wordy 
as is illustrated by the source code 
\href{\urlSite fromTex/F4_07someSvg.svg}{\texttt{F4\_07someSvg.svg}} 
for the above picture. 
Nevertheless, it can be an advantage 
to go into internals and manipulate with a text editor. 
Pasting the into VS Code, one can see the highlighting and a preview, 
of course provided the extensions 
described in Section~\ref{subsec:instExtVsCode} are installed. 


\section{Pictures which are not transformed}\label{sec:picAsIs}

Figure~\ref{fig:asIsJpg} shows some picture included as JPG\@. 
This is done as usual with the command \cmd{includegraphics}{}
provided by the package \pkg{graphicx}. 
According to the documentation~\cite{GraX}, page 13, 
the bounding box must be provided somehow. 

This may be done via the package \pkg{bmpsize} 
but alternatively also using the command \texttt{ebb}. 
There is some hint, that \pkg{bmpsize} does not work with \texttt{xelatex}. 
So maybe \texttt{ebb} is the better alternative. 
Note that both techniques are available in distribution \texlive, 
but not in \miktex. 

Research shows, that inclusion is seamlessly if PDF files are created. 
So the problem addressed is specific for creating DVI files. 
Also, at time of this writing, it seems that also in DVI mode, no problems occur. 
Nevertheless, the author experienced errors on missing bounding box 
and to be safe, provides a way to invoke \texttt{ebb} on the file \texttt{xxx.jpg}. 

With parameter \texttt{-m}, this creates a file \texttt{xxx.bb} 
containing the bounding box for \texttt{dvipdfm}, 
and with parameter \texttt{-x} a file \texttt{xxx.xbb} 
containing an extended bounding box for \texttt{dvipdfmx}. 
The current implementation seems not to make any difference, 
whether the bounding boxes are created or not. 

Sizes seem to differ in DVI/XDV output after conversion to PDF, 
depending on whether \texttt{dvipdfm} or \texttt{dvipdfmx} is used. 
Only the latter yields the same size as direct conversion to PDF creates. 

Since bounding boxes seem superfluous, 
we control their creation with a parameter \texttt{createBoundingBoxes} 
whether to invoke \texttt{ebb}, which is false by default. 
Nevertheless, if we invoke, then we do twice, 
creating bounded boxes and extended bounding boxes. 


FIXME\@: further research and further documentation is required. 

Note that both for \pdflatex{} and siblings creating PDF-output 
and for \texttt{htlatex} in conjunction with \texttt{dvipdfmx} 
files in the format \gls{pdf}, \gls{png}, \gls{jpg} are supported. 
% **** what about \gls{gif}? 
This list may be incomplete. 

\begin{figure}[htb]
\centering
\IfPackageLoadedTF{tex4ht}{%
should be a picture 
}{
\setkeys{Gin}{width=0.9\textwidth}
\includegraphics%[width=0.9\textwidth]
{F4_08someJpgOboeBaroqueDennerMIR370}%.jpg
}
\caption{\label{fig:asIsJpg}Some JPG-picture, directly included }
\end{figure}

As an example, Figure~\ref{fig:asIsPng} shows the same picture 
as PNG-file. 

FIXME\@: At the moment, \texttt{htlatex} does not work with pictures at all. 

\begin{figure}[htb]
\centering
\IfPackageLoadedTF{tex4ht}{%
should be a picture 
}{
\includegraphics[width=0.9\textwidth]
{F4_09somePngOboeBaroqueDennerMIR370}%.png
}
\caption{\label{fig:asIsPng}Some PNG-picture, directly included }
\end{figure}

Note that in DVI/XDV mode all usual \LaTeX{} engines 
can include BMP-pictures, whereas in PDF mode only \texttt{xelatex} 
can do that, maybe because it creates XDV internally in any case. 
In contrast, \texttt{lualatex} and \texttt{pdflatex} can not. 
% The following works for lualatex but not for pdflatex 
% And still the same picture, Figure~\ref{fig:asIsPng} shows the oboe 
% as bmp-file. 

% \begin{figure}[htb]
%   \centering
%   \IfPackageLoadedTF{tex4ht}{%
%   should be a picture 
%   }{
%   \includegraphics[width=0.9\textwidth]
%   {F4_10someBmpOboeBaroqueDennerMIR370.bmp}%.bmp
%   }
%   \caption{\label{fig:asIsBmp}Some bmp-picture, directly included }
%   \end{figure}
  
