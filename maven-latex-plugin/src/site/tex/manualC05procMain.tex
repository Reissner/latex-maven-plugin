
\chapter{Processing of \LaTeX{} Main Files}\label{chap:latexMainConversions}

Given graphics in formats includable in TEX files, 
which may require preprocessing described in 
Chapter~\ref{chap:GraphConversions}, 
this section describes the conversions of latex main files 
into target files in detail. 
The most important target file format is \gls{pdf}. 
Conversion into this format is described in Section~\ref{sec:tex2pdf}. 
Note that \gls{pdf} also occurs as source format 
for included pictures and as intermediate files. 
Specific for \LaTeX{} is the \gls{dvi} format, 
which is supported mainly for historical reasons. 
% FIXME\@: nowhere described. 

Almost independent of the format created, 
Inclusion of bibliographies, indices and glossaries 
requires additional conversions 
done by several auxiliary programs. 
Bibliographies are described in Section~\ref{sec:bibtex}, 
indices in Section~\ref{sec:indices} 
and glossaries in Section~\ref{sec:glossaries}. 
Only at the first sight different 
but behind the scenes quite analogous 
is inclusion of results of code evaluations, 
code in python and other languages described in Section~\ref{sec:pythontex}. 
Here, an auxiliary program essentially invokes the language interpreter. 

Sections~\ref{sec:rerunMakeIndexGlossaries} 
and~\ref{sec:rerunLatex} 
are special in that they describe rerunning auxiliary programs 
and the \LaTeX-to-pdf converter, 
which may be necessary even if certain lists are present 
like table of contents list of figures or list of tables. 

Section~\ref{sec:chkReprod} is special in that it is not related with conversion 
but with checking reproducibility. 

Besides the output formats traditional for \LaTeX, 
\gls{pdf} and \gls{dvi} describing e.g.~books, 
Section~\ref{sec:tex2html} describes creation of 
\gls{html}, Section~\ref{sec:tex2odt} the creation of \gls{odt} and 
Section~\ref{sec:tex2doc} creation of MS Word formats like \gls{docx}. 
Finally, also pure text can be generated 
as described in Section~\ref{sec:tex2txt}. 

\newpage


\section{Transforming \LaTeX{} files into PDF files}\label{sec:tex2pdf}

The next step is to create a PDF file from the TEX files. 
\LaTeX{} distinguishes master TEX files from TEX files intended to be inputted
from elsewhere. 
Not taking comments and that like into account, 
master TEX files roughly have the form 
%
%\lstset{language=tex, basicstyle=\small}
\begin{lstlisting}[language=tex, basicstyle=\small]
\RequirePackage[l2tabu, orthodox]{nag} % optional 
\documentclass{...}

\begin{document}
...
\end{document}
\end{lstlisting}

The core of conversion of a TEX file into a PDF file 
is running a \LaTeX{}-to-pdf converter \texttt{latex2pdf} 
to a master TEX file \texttt{xxx.tex}.
The \LaTeX-to-pdf converter \texttt{latex2pdf} 
is configurable via the parameter \texttt{latex2pdfCommand}. 
Possible values are \lualatex{}, \xelatex{} and \pdflatex, 
where the first is the default for which this software is also tested. 
It is also possible to pass parameters to the \LaTeX{}-to-pdf converter. 
Besides conversion into \gls{pdf} format, 
all converters offer conversion to the older \gls{dvi} format 
via option \texttt{-{}-output-format} as \lualatex{} and \pdflatex, 
or the alternative \gls{xdv} generalizing \gls{dvi} 
as \xelatex{} does with the option \texttt{-{}-no-pdf}. 

In fact, the converter \texttt{latex2pdf} 
does much more than converting TEX files to PDF files. 
Figure~\ref{fig:tex2pdf} shows for \texttt{latex2pdf} set e.g.~to \lualatex{}, 
that besides the PDF file also a LOG file and an AUX file is created. 
The LOG file contains logging information on the run of the conversion 
and the aux-file transports information from one run to the next, 
writing in one run and reading in the next run. 
Thus, conversion goes without it, but it is read if present. 
This is why it is depicted at input side in dashed lines. 

Optionally, an FLS file is created containing paths to the files 
the converted \LaTeX{} file depends on 
and a file with ending \texttt{synctex.gz} 
with information for mapping locations at the created PDF file 
to the according input files. 
This is to support backward search, meaning click on a place in the PDF viewer 
opens an editor in the source file. 

What is in fact in the aux-file depends on the package. 
Among other information, 
also citations and the location of the bibliography file with ending bib 
are present. 
This cannot be used directly in the next \texttt{latex2pdf} run 
to create the bibliography, 
because the entries referenced in the document must be extracted from the bib-file 
and sorted. 
This is done by invoking \texttt{bibtex} between two \texttt{latex2pdf} runs. 
Based on the aux-file, \texttt{bibtex} creates a bbl-file 
containing the bibliography, which is read in the next \texttt{latex2pdf} run. 
For details see Section~\ref{sec:bibtex}. 

Alternatively to \texttt{bibtex} a bibliography can be created 
with the package \pkg{biblatex} in conjunction with the auxiliary program \texttt{biber}. 
Running a \LaTeX-to-pdf converter with package \pkg{biblatex} loaded 
creates a \gls{bcf}-file read by \texttt{biber}. 
This software does not support that option. 
Nevertheless, for sake of completeness we added this data path to Figure~\ref{fig:tex2pdf}. 

If an index is demanded, 
in addition \texttt{latex2pdf} creates a \gls{idx}-file. 
As the citations, it cannot be used directly to create an index in
the next \texttt{latex2pdf} run, 
because the index entries must be collected and sorted before. 
This is done by invoking \texttt{makeindex} 
between the two \texttt{latex2pdf} runs. 
Based on the \gls{idx}-file, \texttt{makeindex} creates a \gls{ind}-file 
containing the index, which is read in the next \texttt{latex2pdf} run. 
For details see Section~\ref{sec:indices}. 

If more than one index is demanded, 
we suggest using \texttt{splitindex} instead of \texttt{makeindex} 
which creates one \gls{ind}-file per index. 

If a glossary is demanded, this can be read off the \gls{aux}-file 
and a \gls{glo}-file containing the index entries 
is created and a file with style information. 
Depending on the configuration, 
this may be a \gls{ist}-file or a \gls{xdy}-file. 
As for the index the \gls{idx}-file, 
the \gls{glo}-file cannot be used directly to create a glossary in
the next \texttt{latex2pdf} run, 
because the glossary entries must be collected and sorted before. 
This is done by invoking \texttt{makeglossaries} 
between the two \texttt{latex2pdf} runs. 
Based on the \gls{glo}-file, \texttt{makeglossaries} creates a \gls{gls}-file 
containing the glossary, which is read in the next \texttt{latex2pdf} run. 
For details see Section~\ref{sec:glossaries}. 

The package \pkg{pythontex} allows including python code or related 
in the \gls{tex}-file and to evaluate it. 
The first \texttt{latex2pdf} run creates a \gls{pytxcode}-file 
which contains essentially the code parts of the \LaTeX-file. 
Invoking \texttt{pythontex} creates by default 
a folder \texttt{pythontex-files-xxx} 
with material where code is already evaluated. 
In the next \texttt{latex2pdf} run, this material is included in the document. 
The \pkg{pythontex} comes with a second command line utility, 
\texttt{depythontex}, eliminating all python code from the original tex-file. 
Optionally, \texttt{latex2pdf} also creates a \gls{depytx}-file 
with all information to replace python code in the original tex-file 
with evaluated material from \texttt{pythontex-files-xxx}. 
Replacement is done by \texttt{depythontex} 
which by default, sends the result to stdout, 
but there is an option to write into another \LaTeX-file. 
Converting this new \LaTeX-file 
yields the same result as converting the original one. 
Depythonization is a feature needed e.g.~for papers 
when the publisher does not accept included code. 
For details see Section~\ref{sec:pythontex}. 

In addition, if
a table of contents, a list of figures, a list of tables 
or a list of listings is required, 
also a toc-file, a lof-file, a lot-file and a lol-file is created,
respectively, 
collecting the according information. 
Also, if hyper-references are built, an \gls{out}-file 
containing bookmarks is created. 
If such a file is present, it is read in and is used
to create a table of contents, a list of figures, of tables and of listings 
or bookmarks in the second run of \texttt{latex2pdf}. 

To summarize, 
if a table of contents, a list of figures, a list of tables, a list of listings or 
a bibliography, an index or a glossary is present, 
or if code must be replaced by their evaluation, 
a second \LaTeX{} run is required to make that material appear in the pdf-output. 

If a table of contents and at the same time 
a bibliography, an index or a glossary is present, 
even two further \LaTeX{} runs are required: 
After the first one, the bibliography, the index or the glossary 
occurs in the PDF file but not yet in the table of contents. 
This happens after the second additional \LaTeX{} run. 
As described in Sections~\ref{sec:rerunMakeIndexGlossaries} and~\ref{sec:rerunLatex}, 
further runs of \texttt{makeindex}, resp,~\texttt{splitindex}, 
\texttt{makeglossaries} 
and of the \LaTeX-processor \texttt{latex2pdf} may be necessary. 

\begin{figure}[htb]
\centering
\IfPackageLoadedTF{tex4ht}{%
should be a picture 
}{
\import{}{F5_01tex2pdf.ptx}
}
\caption{\label{fig:tex2pdf}Conversion of a TEX file into a PDF, DVI, XDV file }
\end{figure}

\section{Bibliographies}\label{sec:bibtex}

In case the \LaTeX{}-to-pdf converter writes bibliographic information, 
into its aux-file, a bibliography must be generated. 
For each occurrence of a \cmd{cite}-command in the TEX file, 
\texttt{latex2pdf} writes an according entry \cmd{citation} into the aux-file. 
Moreover, a \cmd{bibliography}-command in the TEX file 
writes a link to the bib-file containing the bibliography data 
into the aux-file as \cmd{bibdata}. 
Optionally, a \cmd{bibliographystyle}-command in the TEX file 
writes a link to the bibliography style file 
into the aux-file as \cmd{bibstyle}. 

To create a bibliography, 
a \texttt{bibtexCommand} must be run after the \LaTeX{} run. 
The default command is the traditional \texttt{bibtex}, 
but there are more modern alternatives 
like \texttt{bibtexu} and \texttt{bibtex8} supporting utf8 encoding 
and others. 

% mention Biber and biblatex and also mlbibtex 

Essentially, \texttt{bibtex} extracts the citations in the aux-file, 
unifies them, i.e.~a citation is listed once even if it is used more than once, 
retrieves the according entries from the bib-file, sorts these, 
performs formatting and writes all into a bbl-file 
which can be included in the next run of \texttt{latex2pdf}. 

Note that after a \texttt{bibtex}-run, 
two \LaTeX{} runs are required: 
The first one just puts the bibliography found in the bbl-file 
into the PDF file and the labels of the citations into the aux-file 
as \cmd{bibcite}-commands. 
The second run places the labels of the citations found in the aux-file 
at the citations given by \cmd{cite}. 
The package \pkg{tocbibind} described in~\cite{TocBibIndP}, 
then writes the headline of the bibliography into the table of contents.\index{table of contents}
The package \pkg{rerunfilecheck} described in~\cite{RerunFChkP}, 
ensures that \texttt{latex2pdf} is rerun if needed, 
provided loaded with option \texttt{aux}. 


This software presupposes, that \texttt{bibtex} reads the aux-file 
and creates a bbl-file and also an blg-file with logging output 
as illustrated by Figure~\ref{fig:aux2bbl}. 
From the blg-file this software may determine 
whether \texttt{bibtex} emitted an error or warnings. 


\begin{figure}[htb]
\centering
\IfPackageLoadedTF{tex4ht}{%
should be a picture 
}{
\import{}{F5_02aux2bbl.ptx}
}
\caption{\label{fig:aux2bbl}
Conversion of an aux-file into a bbl-file using a bibliography}
\end{figure}

Vital information on \texttt{bibtex} can be found in~\cite{BibPat} 
and in~\cite{BibMar}. 
Also, Chapter 10 in~\cite{Gra} gives vital information on \texttt{bibtex}. 


\section{Indices}\label{sec:indices}

In case the \LaTeX{}-to-pdf converter writes index information, 
into its \gls{idx}-file, at least one index must be generated. 
Since the \gls{idx}-file contains nothing but index information, 
an index is created if and only if the \gls{idx}-file is created. 
% Well, this is not really the truth: it is needed only if ...
Essentially, 
the command \cmd{makeindex}{} tells \texttt{latex2pdf} 
to open the \gls{idx}-file for writing. 
Then for each occurrence of the \cmd{index}-command 
or similar (details see below) in the TEX file, 
an entry is written sequentially to the \gls{idx}-file as 
\cmd{indexentry}{} comprising the keyword given by the \cmd{index}-command 
and the page number where the \cmd{index}-command occured. 
For example \cmd{index\{ant-task\}}{} creates an entry 
%
\begin{lstlisting}[language=TeX]
\indexentry{ant-task|hyperpage}{3}
\end{lstlisting}
%
in \texttt{xxx.idx}. 

Then the \texttt{makeindex}-command is applied to the \gls{idx}-file 
which sorts keywords and for each keyword collects the according page numbers, 
sorts it and and writes the result into a \gls{ind}-file. 
In the next run of \texttt{latex2pdf}, 
the \cmd{prindindex}-command includes the index as a separate section; 
typically at the end of the PDF file. 
The most basic package to provide this command 
is \pkg{makeidx} described in~\cite{MkidxShIdxP}. 
In addition, \pkg{makeidx} provides the command \cmd{see}{} 
which is for cross-reference within an index. 
The package \pkg{tocbibind} described in~\cite{TocBibIndP}, 
then writes the headline of the index into the table of contents.\index{table of contents}
The package \pkg{rerunfilecheck} described in~\cite{RerunFChkP}, 
ensures that \texttt{latex2pdf} is rerun if needed, 
provided loaded with option \texttt{aux}. 

The same document,~\cite{MkidxShIdxP} 
also describes the package \pkg{showidx} 
which prints index entries at the margin of the document. 
This is for debugging only. 
\medskip


The main restriction of the package \pkg{makeidx} is, 
that only a single index can be created. 
The reason is that, \texttt{latex2pdf} creates a single \gls{idx}-file 
and, as illustrated in Figure~\ref{fig:idx2ind}, 
\texttt{makeindex} creates a single ind-file from that, 
representing a single index. 

To overcome this restriction, 
replace package \pkg{makeidx} and \texttt{makeindex} 
with package \pkg{splitidx} and \texttt{splitindex} 
both described in~\cite{SplitidxP}. 

The package \pkg{splitidx} is used 
in conjunction with the program \texttt{splitindex}. 
It must be possible to create a single index 
without using \pkg{splitidx} and \texttt{splitindex}. **** 

Package option \texttt{split} makes \texttt{latex2pdf} 
creating \gls{idx}-files \texttt{xxx-y.idx} directly. 
Here \texttt{y} represents the identifier of an individual index. 
These \gls{idx}-files can be transformed individually with \texttt{makeindex} 
into ind-files as illustrated in Figure~\ref{fig:idx2indMult}. 
Since \texttt{latex2pdf} can keep open only up to 16 output streams, 
not all of which can be used to create a file \texttt{xxx-y.idx}, 
this approach allows a limited number of indices 
and is thus not recommended and not supported. 
Another reason is, that this approach undermines 
the package \pkg{rerunfilecheck} described in~\cite{RerunFChkP}, 
and so it is not guaranteed that \texttt{latex2pdf} is rerun if needed. 
This explains why option \texttt{split} is not allowed. 
% **** check? 

Instead, without option \texttt{split}, 
\texttt{latex2pdf} creates a single \gls{idx}-file. 
The program \texttt{splitindex} splits it up into several \gls{idx}-files 
and applies \texttt{makeindex} to each of them separately 
as illustrated in Figure~\ref{fig:idx2indSplit}. 

For usage of further packages supporting multiple indices 
which are not intended to be used with this software, 
see Chapter~\ref{chap:gaps}. 

This software presupposes, that \texttt{makeindex} converts the \gls{idx}-file 
into an ind-file containing the index 
and creating also an ilg-file with logging output 
as shown in Figure~\ref{fig:idx2ind}. 
From the ilg-file this software may determine 
whether \texttt{makeindex} emitted an error or warnings. 

\begin{figure}[htb]
\centering
\IfPackageLoadedTF{tex4ht}{%
should be a picture 
}{
\import{}{F5_03idx2ind.ptx}
}
\caption{\label{fig:idx2ind}Conversion of an \gls{idx}-file into an ind-file}
\end{figure}

\begin{figure}[htb]
\centering
\IfPackageLoadedTF{tex4ht}{%
should be a picture 
}{
\import{}{F5_04idx2indMult.ptx}
}
\caption{\label{fig:idx2indMult}
Not supported: Conversion of \gls{idx}-files into ind-files}
\end{figure}

\begin{figure}[htb]
\centering
\IfPackageLoadedTF{tex4ht}{%
should be a picture 
}{
\import{}{F5_05idx2indSplit.ptx}
}
\caption{\label{fig:idx2indSplit}Conversion of an \gls{idx}-file into ind-files}
\end{figure}

It is possible to configure the makeindex-command 
and to pass arbitrary options. 
CAUTION\@: For the usual \texttt{makeindex}-command, 
the options \texttt{-o} specifying an output file 
and \texttt{-t} (transcript) specifying the logging file are not allowed, 
because this breaks the expectation to find the sorted index 
in file \texttt{xxx.ind} 
and bypasses the detection of errors and warnings of this software, 
respectively. 
Also specifying a style file via option \texttt{-s} 
is not recommended because this is used to create a glossary 
and so breaks glossary creation 
as described in Section~\ref{sec:glossaries}. 

Information on the makeindex program can be found in~\cite{MkIdxMoe} 
and in~\cite{MkIdxLam}. 
Also, there is a site~\cite{MakeIdxOpts} 
describing all available options for \texttt{makeindex}. 

As indicated above, the program \texttt{splitindex} 
invokes \texttt{makeindex}. 
Its options are described in~\cite{SplitidxP}, Section~3.10. 
Since the long option names are not understood in all environments, 
only the short options are recommended. 

Since \texttt{splitindex} must satisfy the interface 
given by Figure~\ref{fig:idx2indSplit}, 
the option \texttt{--help} and its shortcut \texttt{-h} are not allowed. 
Likewise for option \texttt{--version} and its shortcut \texttt{-V}. 
The option \texttt{--makeindex <makeindex>}, resp.~\texttt{-m <makeindex>}, 
is used with the \texttt{makeindex} command used for single indices. 
Thus, this may not be given explicitly but is specified implicitly. 
Also, the option \texttt{--identify <regex>}, resp.~\texttt{-i <regex>} 
must be set implicitly because it must be the same expression 
as used to ***** 
Then splitindex.tlu is not allowed, 
because this has another expression. 

Only allowable seems \texttt{-V}, the shortcut for \texttt{--verbose}. 

Then comes the name of the index file to be processed 
without suffix. 

The program \texttt{splitindex} invokes \texttt{makeindex}. 
The option option \texttt{--} coming after the filename, 
indicates that all following options are passed to \texttt{makeindex} 



\section{Glossaries}\label{sec:glossaries}

CAUTION\@: The method described here, 
has at least two severe bugs: 
The number of reruns of the \LaTeX-to-pdf converter and also of \texttt{makeglossaries} 
is not guaranteed as a consequence of a bug in \pkg{rerunfilecheck} 
and the fact, that it does not fit current versions of \pkg{makeglossaries}. 
In addition, entries of the glossaries not mentioned directly in the document 
but must be included because they are used in the explanation of entries to be included 
are not treated properly. 

As a consequence, this document, or to be more precise its glossary, 
could not always be reproduced and so the author excluded the glossary until the problem is fixed. 

In addition, it is a conceptual weakness that a glossary data base 
shall be centralized and shall thus not be included in a \LaTeX{} document 
and not even be written in \LaTeX. 
All weaknesses, bugs and conceptual shortcomings are overcome 
by the package \pkg{glossaries-extra} in conjunction with the auxiliary program \texttt{bib2gls} 
which will replace \pkg{glossaries} and \texttt{makeglossaries}. 
For the time being, use glossaries with caution. 
\medskip


Creating glossaries 
requires the package \pkg{glossaries} described in~\cite{GloP}. 
Note that despite the headline of this section, 
and despite \pkg{glossaries} itself supports multiple glossaries, 
this software supports only a single glossary 
and also sorting and unifying is done 
either via \texttt{makeindex} as for indices or via \texttt{xindy}, 
whereas the option to do without external programs 
offered also by package \pkg{glossaries} 
is not supported by this software. 

For generalizations see Chapter~\ref{chap:gaps}. 

As for creating indices there is a \LaTeX-command \cmd{makeindex}, 
to create a glossary there is a \LaTeX-command \cmd{makeglossaries}{}, 
but the latter is not built-in as \cmd{makeindex} 
but provided by the package \pkg{glossaries}. 
If \texttt{xxx.tex} is the \LaTeX{} main file, 
\cmd{makeglossaries}{} opens the glo-file \texttt{xxx.glo} 
containing glossary entries for writing. 
As the built-in command \cmd{index}{} 
writes entries into the \gls{idx}-file defining the index, 
the command \cmd{gls}{} defined by the package \pkg{glossaries} 
writes an entry into the glo-file. 
Note that \texttt{xxx.glo} typically contains entries more than once 
and that the entries are not sorted. 

To perform sorting, formatting and typically also unification, 
the package \pkg{glossaries} allows three mechanisms. 
This software supports two of them: 
via the shell command \texttt{makeindex}, which is also used for indices, 
and via the shell command \texttt{xindy}. 
Using \texttt{makeindex} is the default but can also be activated through 
\cmd{usepackage[makeindex]\{glossaries\}}. 
Using \texttt{xindy} instead of \texttt{makeindex} is triggered through 
\cmd{usepackage[xindy]\{glossaries\}}. 
Accordingly, for option \texttt{makeindex} the aux-file receives lines 
%
\begin{lstlisting}[language=TeX]
\providecommand\@istfilename[1]{}
\@istfilename{manualLMP.ist}
\end{lstlisting}
%
whereas for option \texttt{xindy}, there are lines 
%
\begin{lstlisting}[language=TeX]
\providecommand\@istfilename[1]{}
\@istfilename{manualLMP.xdy}
...
\providecommand\@xdylanguage[2]{}
\@xdylanguage{main}{english}
\providecommand\@gls@codepage[2]{}
\@gls@codepage{main}{}
\end{lstlisting}



This software neither invokes \texttt{makeindex} nor \texttt{xindy} directly. 
Instead, it invokes the shell command \texttt{makeglossaries}
invoked without file ending  
which determines from the aux-file 
whether to invoke \texttt{makeindex} nor \texttt{xindy}. 
Accordingly, it writes the style definition 
by creating an ist-file \texttt{xxx.ist} or an xdy-file \texttt{xxx.xdy} 
if \texttt{makeindex} or \texttt{xindy} is specified as package option, 
respectively. 

Seemingly, \texttt{makeglossaries} relies on the aux-file 
to determine whether to invoke \texttt{makeindex} or \texttt{xindy} 
for sorting and unification. 
Then it invokes the according command and writes a LOG file 
with ending \texttt{glg}, 
redirecting the logging output of \texttt{makeindex} or \texttt{xindy} 
adding own output so that a glg-file may be written, 
even if e.g.~\texttt{makeindex} is invoked and does not. 
In any case, if the glg-file is written, 
\texttt{makeglossaries} writes text matching 
%
\begin{verbatim}
(^\*\*\* unable to execute: )
\end{verbatim}
%
in the glg-file if an error occurs, 
no matter whether \texttt{makeindex} or \texttt{xindy} is invoked. 
Possibly, there are cases where an error causes no glg-file to be written. 
If no error occurs, a glg-file is written 
and if warnings are emitted, 
they either come from \texttt{makeindex} or from \texttt{xindy}. 
Thus warnings may be detected with the patterns 
defined by \texttt{makeindex} and by \texttt{xindy}. 

The style \texttt{list} (which is the default) is set in the form 
%
\begin{lstlisting}[language=TeX]
\usepackage[style=list]{glossaries}
\end{lstlisting}
%
where~\cite{GloP}, Section~15 lists predefined styles. 
So, the style determines the content of the style definition, 
whereas the options \texttt{makeindex} and \texttt{xindy} 
specify the form in which the style is encoded 
and thus the ending of the style file, 
which is either \texttt{ist} or \texttt{xdy}. 

Sorting the glo-file, as said above, 
currently is only supported using the command \texttt{makeglossaries}. 
The allowed options are essentially those 
making sense for \texttt{makeindex} and those making sense for \texttt{xindy}. 
If the shell command \texttt{makeglossaries} 
invokes \texttt{makeindex} of course only the according options 
are passed supplemented by additional options 
\texttt{-s}, \texttt{-t}, \texttt{-o}, to specify the
ist-file, the glg-file (the transcript-file) and the gls-file,
respectively, 
which is the result of sorting, the output file, 
and contains the entries of the glo-file 
just sorted, formatted and unified.
So for a tex main file \texttt{xxx.tex} the program 
\texttt{makeglossaries} invokes
%
\begin{verbatim}
makeindex  -s "xxx.ist" -t "xxx.glg" -o "xxx.gls" "xxx.glo"
\end{verbatim}
%
Accordingly, if the shell command \texttt{makeglossaries} 
invokes \texttt{xindy} of course only the according options 
are passed supplemented by additional options 
\texttt{-M}, \texttt{-t}, \texttt{-o}. 
This is illustrated in Figure~\ref{fig:glo2gls}. 


\begin{figure}[htb]
\centering
\IfPackageLoadedTF{tex4ht}{%
should be a picture 
}{
\import{}{F5_06glo2gls.ptx}
}
\caption{\label{fig:glo2gls}Conversion of a glo-file into a gls-file 
using \texttt{makeglossaries}}
\end{figure}


\section{Including code via \texttt{pythontex}}\label{sec:pythontex}

The package \pkg{pythontex}, described in~\cite{PythonTexP} 
originally allowed including Python code into a latex document. 
Later on, further languages were added, most notably octave or Matlab, 
and the user can easily extended it to further languages 
as sketched in~\cite{PythonTexP}. Section 7. 
Of course, to that end, the interpreter for the desired language must be installed.\index{pythontex}
The meaning of the term ``including'' used above 
ranges from mere listing to pure execution and comprises also inserting results of execution. 
A field of application is also creating figures. 

Note that like the package \pkg{splitindex}, also \pkg{pythonindex} 
comes with an according auxiliary program, 
in this case, besides \texttt{pythontex} also \texttt{depythontex}. 
Consequently,~\cite{PythonTexP} is not only on the package 
but also on the corresponding command line tools. 
Since~\cite{PythonTexP} is quite detailed, 
there is an introduction~\cite{PythonTexQ} and a gallery~\cite{PythonTexG}. 
For background on the intentions of package \pkg{pythontex}, consult~\cite{PythonTexRepr}. 
Information required to integrate pythontex into this software 
partially goes much beyond the official documentation and is collected in~\cite{PyTexInOut}. 
It could also be interesting for the user for debugging. 

Running the \LaTeX-to-pdf converter on a file \texttt{xxx.tex} 
with package \pkg{pythontex} loaded 
yields a file \texttt{xxx.pytxcode} 
and if the package is loaded with option \texttt{depythontex} 
also a file \texttt{xxx.depytx}.
If the file \texttt{xxx.pytxcode} is present, 
this software invokes the command line tool \texttt{pythontex} 
(same name as the according package) 
to \texttt{xxx.pytxcode} (without ending) 
which converts this into a variety of output files, 
which are, without further configuration, 
all in the folder \texttt{pythontex-files-xxx}
as shown in Figure~\ref{fig:py2dir}, 
which is described in more detail in~\cite{PyTexInOut}, Section~3. 
Note that this software uses the wrapper \texttt{pythontexW}\index{pythontexW} 
of \texttt{pythontex} described in Section~\ref{subsec:pythontexW}, 
instead of \texttt{pythontex} itself. 
The figure reflects this. 

% TBD: later on, if \texttt{xxx.depytx} after \texttt{pythontex} also \texttt{pythontex} is invoked, 
% creating a tex file without python code. 
% TBD: clarify what if input files or include files are present. 

Running the \LaTeX-to-pdf converter again, 
includes all the output files \texttt{*.stdout} 
in the PDF file or whatever output file created. 

% Note that if depythontex is invoked, it is immaterial whether the subsequent run of LaTeX-to-pdf converter 
% is on the original tex file or on the tex file created by depythontex. 

An important remark is that \lualatex{} is the preferred converter, 
because files \texttt{*.stdout} can impose heavy memory usage 
and currently \lualatex{} is the only converter allocating memory dynamically. 

As one can see, \texttt{pythontex} cooperates with \lualatex{} in a way 
also \texttt{bibtex} or the other auxiliary programs do. 
Although \texttt{pythontex}, at time of this writing in version 0.18, 
is quite mature, it refrains from writing a log file and indicates errors and warnings 
just on standard output or error output. 
This is unlike all the other auxiliary programs in a line with \texttt{pythontex}. 
As a consequence, in particular warnings are difficult to detect 
and cannot be detected in a uniform way. 
Thus, the author wrote a little wrapper, called \texttt{pythontexW} 
and place it where it can be found, e.g.~in the folder of \texttt{pythontex}. 

Accordingly, \texttt{depythontex} behaves in a non-standard way: 
Firstly, by default, it does not output a result file but outputs on standard output. 
This can be changed using the option \texttt{--output} or \texttt{-o} for short. 
Also, \texttt{depythontex} changes into interactive mode 
if the output file is already present. 
To avoid this, the option \texttt{--overwrite} is required. 
Overwriting without asking is the standard behavior of all other auxiliary programs. 
As \texttt{pythontex} also \texttt{depythontex} does not write a log file 
but just prints its errors and warnings. 
Thus, the author wrote a little wrapper, 
called \texttt{depythontexW} and described in Section~\ref{subsec:pythontexW}, 
and place it where it can be found, e.g.~in the folder of \texttt{depythontex}\index{depythontexW}. 
% TBD: evaluate: maybe it is a better solution to enable this software itself to write the output to a log file. 
\medskip


The package \pkg{pythontex} and the according auxiliary programs are highly configurable, 
more than this software allows. 

In particular, in the \LaTeX{} document, 
the commands \cmd{setpythontexoutputdir} setting the output directory 
and \cmd{setpythontexworkingdir} setting the working directory shall not be used, 
because this software assumes the default, that the working directory is the directory 
containing the \LaTeX{} main file \texttt{xxx.tex}
and the output directory is in the working directory 
and its name is \texttt{pythontex-files-xxx}. 

Further, the package \pkg{pythontex} can be configured with package options when loading the package. 
Since this software is designed for reproducibility, 
most appropriate would be to specify \texttt{runall=true} meaning that even if no python code is modified 
the auxiliary program \texttt{pythontex} executes the python code in the document. 
Also, it is appropriate to specify \texttt{rerun=always}. 
Note that the defaults are \texttt{runall=false} and \texttt{rerun=errors}. 
This behavior makes sense to speed up creation of the document, 
but it differs from the behavior of all other auxiliary programs 
and causes the check for update of output files to fail. 
Moreover, reproducibility is not as easily shown. 

The package documentation~\cite{PythonTexP} suggests, 
that this makes a difference between \texttt{runall=true/false} 
and \texttt{rerun=always/errors} if external sources are modified, 
but as is proved in~\cite{PyTexInOut}, Section 2.1, 
the package translates package option \texttt{runall=true/false} into key value pair \texttt{rerun=always/errors} 
and this is the only information \texttt{pythontex} obtains from the package, 
so there is no difference. 

Also, the auxiliary program \texttt{pythontex} itself can be configured via command line arguments. 
For the package options \texttt{runall} and \texttt{rerun}, 
there are according command line options \texttt{-{}-runall} and \texttt{-{}-rerun} with the same scope. 
Whereas the package merges options \texttt{runall} and \texttt{rerun} silently, 
the auxiliary program \texttt{pythontex} emits an error, if both are combined. 
Essentially one can forget about \texttt{runall} and stick to \texttt{rerun}. 

Strange enough, according to~\cite{PythonTexP}, Section 4.1, package options overwrite command line options. 
This software shall invoke \texttt{pythontex} 
with the option \texttt{-{}-rerun=always} which is thus specified as the default. 
To force unconditional update, this is not sufficient. 
Instead, this software relies on an undocumented feature of auxiliary program \texttt{pythontex} 
which is likely not to change: 
If one of the expected output files is missing, it recreates all output files, independent of command line options and package options. 
Thus, this software deletes one output file if present, before executing \texttt{pythontex}. 

When this software invokes \texttt{pythontex} 
the exit codes may not be changed via \texttt{-{}-error-exit-code}, 
i.e.~if specified then with value \texttt{true}. 
Neither the options \texttt{-{}-interactive}, \texttt{-h}, \texttt{-{}-help} or \texttt{-{}-version} are allowed. 
Currently, this software does not check for options which are not allowed. 
Fortunately, the latter two command line options have no counterpart in the package configuration. 




If we place some code, e.g.~python code as inline code using \cmd{pyc}
%
\begin{lstlisting}[basicstyle=\footnotesize]
  \usepackage[depythontex]{pythontex}
  ...
  \pyc|print(rf'Python inside latex says: "Hello World; 1+1={1+1}"')|
\end{lstlisting}
%
the code is really evaluated and the string result is included at proper place 
as illustrated by the following text which is created by python: 
%
\begin{quote}
  \texttt{\pyc|print(rf'Python inside latex says: "Hello World; 1+1={1+1}"')|}. % chktex 26  chktex 18 chktex 36
\end{quote}
%
Note that the typewriter font is not created by python, 
it is explicitly set to highlight the string created by python, 
but it is python which evaluates the little computation 
and which prints the string. 

Since \texttt{pythontex} is written in python, 
including python code in the \LaTeX{} document 
uses the python interpreter already installed, as a prerequisite of \texttt{pythontex}. 
To use another language, the according interpreter must be installed in addition to python. 



% TBD: create a figure 


% TBD: depythontex


\begin{figure}[!htb]
  \centering
  \IfPackageLoadedTF{tex4ht}{%
  should be a picture 
  }{
  \import{}{F5_07py2dir.ptx}
  }
  \caption{\label{fig:py2dir}Conversion of a \texttt{pytxcode}-file using \texttt{pythontex}}
  \end{figure}


  Figure~\ref{fig:depy2out} shows the files converted by \texttt{depythontex}. 
  As for \texttt{depythontex}, this software uses the wrapper \texttt{depythontexW} 
  of \texttt{depythontex} instead of \texttt{depythontex} itself. 
  This is reflected in the figure. 


  \begin{figure}[!htb]
    \centering
    \IfPackageLoadedTF{tex4ht}{%
    should be a picture 
    }{
    \import{}{F5_08depy2out.ptx}
    }
    \caption{\label{fig:depy2out}Conversion of a \texttt{depytx}-file using \texttt{depythontex}}
    \end{figure}
  
\section{Rerunning the index- and glossary processor}%
\label{sec:rerunMakeIndexGlossaries}

As described in Section~\ref{sec:tex2pdf}, 
running a \LaTeX-to-pdf converter as \texttt{latex2pdf} 
may detect the presence of a bibliography, an index and/or of a glossary 
and writes raw files to describe them. 
After that, an intermediate step is required, 
sorting, unifying and formatting the entries. 
This is always done by an external program. 

In the next step the \LaTeX{} processor must read in the unified entries again. 
Whereas a \LaTeX-run affects neither the bibliography nor python code, 
it may well invalidate the page numbers 
of the entries of the index or of the glossary. 
%Also within a glossary entry, further glossary entries may be included. 
Thus, the sorted index and glossary must be rebuilt 
before the next \LaTeX{} run makes them visible in the PDF file. 

To that end, we use the package \pkg{rerunfilecheck}. 
%
\begin{lstlisting}[language=TeX]
\usepackage[index, glossary]{rerunfilecheck}
\end{lstlisting}
%
must be put before 
%
\begin{lstlisting}[language=TeX]
\usepackage{makeidx}
\makeindex
\usepackage[toc, xindy]{glossaries}
%, xindy or even [xindy={language=english,codepage=utf8}]
% mainly for index and glossaries 
\makeglossaries
\end{lstlisting}
%
in particular before \cmd{makeindex}{} and \cmd{makeglossaries}. 

Note that the package \pkg{hyperref} already loads \pkg{rerunfilecheck} 
but with the wrong options. 
Thus, the above declaration must come before 
%
\begin{lstlisting}[language=TeX]
\usepackage[...]{hyperref}
\end{lstlisting}
%
to avoid error 
%
\begin{verbatim}
Option clash for package rerunfilecheck
\end{verbatim}

Package \pkg{rerunfilecheck} detects almost safely 
changes of the raw index file writing an according message 
into the LOG file. 
That way, it can be determined whether it is necessary 
to rerun \texttt{makeindex} and \texttt{makeglossaries}. 
After that, of course \LaTeX{} must be rerun at least once. 
Note that Section~\ref{sec:rerunLatex} describes 
when to rerun \LaTeX{} without prior running *****

\section{Rerunning the \LaTeX{} processor}\label{sec:rerunLatex}

FIXME\@: a word on change in toc, lof, lot and lol. 

As indicated in the previous sections, 
\texttt{latex2pdf} must be rerun, 
if \texttt{bibtex} or \texttt{makeindex} \texttt{splitindex} 
or \texttt{makeglossaries} 
had been run 
to read in the bibliography created by \texttt{bibtex} 
or the index created by \texttt{makeindex} 
or the glossary created by \texttt{makeglossaries}. 
Likewise, if a toc-file, a lof-file, a lot-file or a lol-file
had been created in the first \texttt{latex2pdf} run, 
another run is needed to read in these files 
to create a table of contents, a list of figures or a list of tables, 
respectively. 
Note that for all these cases, 
the LOG file does not allow to detect that \texttt{latex2pdf} has to be rerun, 
by matching a fixed pattern. 

After the second run of \texttt{latex2pdf}, 
the table of contents,
the list of figures, the list of tables and the list of listings 
are included and a section with the bibliography, 
the index and the glossary are inserted. 
It takes a third run of \texttt{latex2pdf} 
to include the bibliography the index and the glossary 
into the table of contents. 
Also, it takes that third run to replace the citations 
with the proper labels given in the bibliography. 

Inserting the table of contents,
the list of figures, the list of tables and the list of listings 
may shift the subsequent text 
which may require another run of \texttt{latex2pdf} 
to get the page numbers right. 
As described in Section~\ref{sec:rerunMakeIndexGlossaries} 
intermediate runs of \texttt{makeindex} and \texttt{makeglossaries} 
may be required 
and these also require another run of \texttt{latex2pdf} 
also to get the page numbers right. 
The package \pkg{rerunfilecheck} allows to detect this need to rerun 
by pattern matching on the LOG file almost for sure: 
Still there is some chance 
that the lengths and the md5-sum of all relevant files 
remain the same, although there is a relevant change. 
In this case, this software fails to update 
triggering another \texttt{latex2pdf} run. 


Note that there are several packages which require additional runs, 
such as the package \pkg{longtable}, 
which may vary dimensions of tables. 
This software presupposes, that all these reruns 
may be detected by matching a fixed pattern in the LOG file. 
Since packages are frequently changed and new packages are written, 
also the pattern cannot be fixed. 
Thus, it is configurable. 
 
Note that, if a package requires running other programs 
between two runs of \texttt{latex2pdf}, 
this would require a change in this software. 

\section{Checking reproducibility }\label{sec:chkReprod}

There are use cases, where it is extremely important 
that the according artifacts are really reproducible. 
One is when we have to deliver the sources 
and the receiver has to reconstruct the artifacts. 
Another obvious use case is integration test for this software 
by ensuring that each artifact created 
is equivalent with a confirmed version, 
although this software changed. 
Details are given in Section~\ref{chap:tests}. 


Currently, reproducibility checks are supported for PDF files only. 
% TBD: change that 
The problem with PDF files is, that besides visible contents 
they contain also metadata (see~\cite{Pdf17} or~\cite{Pdf20}, each Section 14.3), 
which depends on the run of the conversion. 
For example the timestamp and the timezone of conversion goes into 
and derived from these other values. 

There are two strategies to deal with the problem: 
%
\begin{itemize}
  \item 
  Make the build process reproducible. 
  The advantage of this approach is that diffing is quite simple, 
  fast and reproducible: it is byte by byte. 
  This is easily done with a fixed installation 
  but tends to break with update of tools. 
  % TBD: allow own version check. 
  % Also, at time of this writing, 
  % the different latex engines cannot be treated uniformly. 
  % TBD: a feature request to hyperref is already posted. 
  % let us keep an eye on that. 
  \item 
  Use diff tools implementing a weaker notion of equivalence, 
  in a sense visibility equivalence of some degree. 
  One approach is the script \texttt{vmdiff} 
  described in Section~\ref{subsec:pythontexW} 
  which combines visibility equivalence 
  with equivalence of part of metadata. 
\end{itemize}%

Since the first one works very well, it is the one we describe here, 
but it is always possible to configure a diff tool with a weaker equivalence check. 

The first question is, whether reproducibility is requested. 
It is, if there is according magic comment in the latex main file requires this 
as described in Section~\ref{subsubsec:openingMagComm}. 
If there is no such magic comment is present, if the setting \texttt{chkDiff} specifies so. 
If in this section settings are given without explicit reference, 
they are described in Table~\ref{tab:paramDiffPdf} on page~\pageref{tab:paramDiffPdf} 
in Section~\ref{sec:paramRepro}. 

Since date and time both visible and in the metadata of a PDF document 
is given relative to a timezone, 
for reproducible builds compilers must run with a fixed timezone 
and, as reproducibility shall not break if changing a timezone 
or if the country running the build changes between daylight saving time and standard time, 
we chose a uniform timezone namely UTC. 

If a latex main file is already under reproducibility control, 
then there is an according original PDF file in \texttt{diffDirectory} or in a subfolder 
to be compared with a newly created PDF file 
which occurs in a subfolder of the TEX source directory \texttt{texSrcDirectory} 
described in Table~\ref{tab:paramGen} on page~\pageref{tab:paramGen}. 
The PDF file for comparison has the same path relative to \texttt{diffDirectory} 
as the created PDF file relative to \texttt{texSrcDirectory}. 

First \texttt{pdfMetainfoCommand} is used 
to extract metadata \texttt{CreationTime} from the original PDF file. 
This comprises time and timezone which is UTC. 

The compilation to create the new PDF file is run in an environment 
with that timezone and with that creation time. 
In addition, there is an environment variable forcing 
that the timestamp does not only affect metadata but also visual data of the PDF file 
to be created, 
as e.g. typically the date at the front page. 
Note that if the PDF file is created from TEX files via DVI/XDV files, 
both converters need the appropriate environment. 

After creating the new PDF file with this environment, 
coincidence with the original PDF file is checked 
using the tool given by setting \texttt{diffPdfCommand} described in Table~\ref{tab:paramDiffPdf}. 
If the actual artifact does not coincide with predefined one 
according to the chosen diff tool, 
a build exception is thrown as specified in Table~\ref{tab:TLP}. 
\medskip


If a latex main file is not already under reproducibility control, 
then no original PDF file exists. 
In this case, the environment for compilation only ensures the timezone UTC. 
Then the created PDF file is copies at proper place into \texttt{diffDirectory} 
-- that's all for setting a document under reproducibility control. 

Finally, if a latex main f8ile file is under reproducibility control 
but is to be changed in a way that also the according PDF file is affected, 
then before compilation just the original PDF file is deleted, 
and the workflow is as setting under reproducibility control. 
\medskip


Reproducibility is affected or even supported by various injections 
as defined in Section~\ref{sec:injFiles}. 
First, the generic header described in Section~\ref{subsec:header} 
affects metadata, above all because it loads the package \pkg{hyperref}. 
Part of this metadata is overwritten by another header 
described in Section~\ref{subsec:headerSuppressMetaPDF}, 
to improve security and privacy, 
but enough metadata remains to keep up reproducibility. 
Reproducibility is guaranteed with the full set of metadata 
or with somehow reduced metadata. 
The only piece of information needed for reproducibility is \texttt{CreationDate} 
and this is preserved by the headers. 
Removing this also has severe consequences 
so that we can assume it is preserved. 
On the other hand, removing metadata may stabilize reproducibility 
as this is true for the banner which identifies the latex compiler and its version 
and consequently breaks reproducibility in any version change. 
Details to reproducibility with a focus on metadata are given in ~\cite{LatexGen}, Section 4. 

Obviously, reproducibility checks cause work 
when putting a document under check, 
i.e. in the end phase of document development 
as defined in Section~\ref{sec:devel}
or if the source document changes, i.e. if document development is entered again, 
or if the output PDF changes unintended 
normally, although the sources did not change in an obvious way, 
which triggers again document development searching the cause of the change in the sources. 

This \LaTeX{} builder is not the tool for document development. 
Instead, Section~\ref{subsec:develLatexmk} suggests to use \tool{latexmk} for, 
and describes how \tool{latexmk} is integrated in this \LaTeX{} builder: 
This builder writes a config file \texttt{.latexmkrc} 
reflecting the settings of this software, at least to some extent. 
The config file \texttt{.latexmkrc} is again written as an injection 
and is described in Section~\ref{subsec:latChkRc}. 
It supports reproducibility checks even reading magic comments, 
checking existence of original PDF file 
and reading its timestamp if the PDF file is present. 
Creation of the new PDF file takes timestamp and timezone into account. 

Two further injections may be helpful in the context of reproducibility checks, 
both described in Section~\ref{subsec:ntlatexVmdiff}: 
\tool{ntlatex} to create a PDF file and \tool{vmdiff} 
realizing a weaker variant of diffing tool as described above: 
It checks for visual equality and equality of metadata. 
% TBD: maybe not true: only in trailer directory of the PDF. 
\medskip


For updating metadata only, we suggest the following technique: 
Keep the original PDF file in \texttt{diffDirectory} 
and check with \tool{vmdiff} that visually, the PDf file remains the same 
and that the correct metadata is updated. 
Of course, a new timestamp is wanted. 
So in a second step, the original PDF file is deleted, 
compilation is repeated, e.g. by \tool{ntlatex} and copied into \texttt{diffDirectory}. 
\medskip


There are rare occasions where the timestamp shall be set explicitly. 
This is not possible directly as it is read off from the original PDF file. 
We suggest to use \texttt{exiftool} to modify the \texttt{CreationDate} 
of the original PDF file in \texttt{diffDirectory} before compilation. 
This is done by something like
%
\begin{verbatim}
  exiftool -PDF:CreateDate=2020-01-01T00:01:02Z xxx.pdf 
\end{verbatim}
%
Here, the option \texttt{PDF:CreateDate} is in fact the name of the tag to be written. 
Note that the timezone must be UTC represented by the \texttt Z 
signifying zero time offset compared to UTC. 
The attentive reader may wonder why the option is \texttt{PDF:CreateDate} instead of \texttt{CreationDate}. 
One may check with \texttt{pdfinfo}, that really \texttt{CreationDate} is modified. 
Note that \tool{exiftool} writes the original PDF file into \texttt{xxx.pdf_original}

Two important details are not so obvious: 
%
\begin{itemize}
\item
Not only the given metadata is changed but also all metadata depending on it, 
in this case the trailer ID. 
This is to keep the PDF file consistent. 
\item 
The metadata is not really overwritten, but it is hidden by new metadata. 
In fact, \texttt{exiftool} uses incremental update specified for the PDF format, 
adding a layer describing the modification. 
All modifications done can also be undone by 
%
\begin{verbatim}
  exiftool -PDF-update:All= xxx.pdf
\end{verbatim}
%
unless the PDF file has been linearized. 
\LaTeX{} to PDF compilers always create linearized PDF files and never update incrementally. 
\end{itemize}


To know that changing metadata is done by incremental update is important, 
insofar as a PDF file with modified timestamp and timezone  
differs from a PDF file compiled directly with the given timestamp and timezone; 
it is shorter. 
So, updating the timestamp of the PDF file in \texttt{diffDirectory} 
does not yield a PDF file which is reproduced. 
Compilation leads to another PDF file and only the updated timestamp is reproduced. 
This compiled PDF file is reproduced, so 
copying it the into \texttt{diffDirectory} solves the problem: 
Next compilation yields a PDF file with the correct timestamp and timezone, 
and it coincides with the PDF file in \texttt{diffDirectory}. 

When subjecting a document under reproduction control with a predefined timestamp, 
then initially there is no original PDF file. 
One could place any PDF file in \texttt{diffDirectory}, 
overwrite the timestamp and timezone by \tool{exiftool}. 
Is content is immaterial. 




\section{Alternative build process with \protect\tool{latexmk}}\label{sec:latexmk}%\protect\tool{latexmk}
\index{latexmk}

This section is on running the build process of \LaTeX{} main files 
with \tool{latexmk} or equivalent. 
Although the functionality is readily explained, 
the intention is not so obvious: 
In Section~\ref{subsec:develLatexmk} 
describes the role of \tool{latexmk} as a build tool 
in the course of document development, 
whereas this \LaTeX{} builder is for final, quality checked build. 
So the two tools seem to be complementary. 
Section~\ref{subsec:latChkRc} describes that this \LaTeX{} builder 
can write its own configuration as 
a config file \texttt{.latexmkrc} for \tool{latexmk} 
so that builds with \tool{latexmk} are in line 
with final builds by this \LaTeX{} builder itself. 

So running \tool{latexmk} from within this \LaTeX{} builder 
seems superfluous at first sight. 
A closer look onto \texttt{.latexmkrc} unveils that this is just a Perl script 
which is very flexible realizing new or special functionality, 
whereas this \LaTeX{} builder is tied to a quite rigid configuration in the pom. 
So, for example if for building a document tools are needed 
which are not supported by this \LaTeX{} builder, 
their invocation can be implemented directly in \texttt{.latexmkrc}. 
Since this \LaTeX{} builder writes a single \texttt{.latexmkrc} 
in the root directory \texttt{texSrcDirectory}, 
which must be made available in each subfolder by adding a link, 
the config \texttt{.latexmkrc} by his \LaTeX{} builder 
may be replaced by a hand-crafted config file for each folder separately. 

Having explained this, the question arises 
why this \LaTeX{} builder does not in general rely on \tool{latexmk} 
and invokes \LaTeX{} compilers and other converters directly. 
One reason is that \LaTeX{} builder does not only invoke converters, 
it also checks return values and, depending on the converter, 
log files emitting errors and warnings if appropriate. 
So, delegating to \tool{latexmk} 
the user can no longer check that the build process passed without warning or error. 
A second aspect is, that the build algorithms differ: 
\tool{latexmk} runs the \LaTeX{} main file then detecting which files are missing 
and then tries to build these based on rules. 
The basic idea behind is ``backward discovery'' of dependencies, 
whereas this \LaTeX{} builder first builds the graphic files globally 
(\tool{latexmk} detects last) 
before for each \LaTeX{} main file is compiled. 
So this \LaTeX{} builder combines ``forward discovery'' and backwards discovery. 
Pure  backward discovery is more elegant 
but as the \LaTeX{} compiler stops at each graphic file not present 
before creating it and rerunning the \LaTeX{} main file, 
it may result in excessive reruns if there are many created graphics in the document. 

So there are strong reasons to avoid \tool{latexmk}, 
but there are also reasons to allow in special cases. 
The parameter \texttt{\$latexmkUsage} described in Table~\ref{tab:paramGen} 
on page~\pageref{tab:paramGen} allows gradually use of \tool{latexmk}, 
not at all, fully or as backend where \tool{latexmk} is invoked 
after graphic files have been created with an internal process. 
As a rule, \tool{latexmk} shall be used as much as required and as little as possible. 

This shows also that it is a good thing 
to be able to activate \tool{latexmk} in individual \LaTeX{} main files 
which is realized with magic comments. % TBD: reference 
XXXXXX Reference to according section required. XXXXXXXX 
Whereas \texttt{\$latexmkUsage} allows three levels of usage, 
the magic comment can choose to use \tool{latexmk} or not. 

The graphic files are preprocessed except if \texttt{\$latexmkUsage} is set to \texttt{Fully}. 
The magic comment results in a warning 
if it does not change whether \tool{latexmk} 
is invoked or not compared to when it is absent. 
% TBD: this must change: magic comments overwrite the setings, this is a general rule. 

% further motivations for integrating latexmk in this builder, in particular for individual files: 
% there are cases where the build process of latexmk works, 
% but not the internal build process of this builder. 
% Integrating latexmk offers the strengths of latexmk. 
% Note that there are also cases where the builtin build process of this builder is mightier than that of latexmk. 
% Another reason for integrating latexmk here is the use case of source distribution: 
% The document(s) may be passed to someone as the source, not as a target, like PDF. 
% It is not clear that the 'customer' uses this latex builder, but maybe (s)he uses latexmk. 
% In this case it makes sense to check, whether the document can be built with latexmk alone. 

% special: latexmk does not update pdf regularly provoking EEX03 false positive 
% We must find a way to prevent this. 
% Also: this is another disadvantage of build with latexmk: 
% It can be checked for update of files only if they were not present before. 
% One could remove pdf before running, but that way an advantage gets lost: 
% that latexmk does not compile anything if all is up to date. %
% touch --date=@1714338620 twoPagerArt.pdf where the number after @ is the epoch time in seconds 
% date -d @1234567890 convert epoch time in seconds into human readable format 
% date -d "Apr 29 2024" +%s convert human readable date into epoch time 
% stat -c%Y file returns epoch time of file in seconds 
% cp -p copy keeping the modification time 
% latexmk -e '$chkDiffMagic="chkDiff"' manualLMP to compile with diff 
% latexmk -e '$programMagic="pdflatex"' manualLMP to compile with pdflatex 
% the latter two can be combined also 



% this is only for goal pdf 
% and maybe even only if pdfViaDvi is not set: LatexProcessor.processLatex2pdf
% changes for latexmk must be done in LatexProcessor.processLatex2dev
% bypassing processLatex2devCore... not clear whether also bypassing warnings. 
% maybe better not doing so: at least part of warnings. 
% on the other hand: misleading. 
% maybe warning that warnings are bypassed. 
% This could be done globally if \texttt{\$latexmkUsage} is not NotAtAll. 
% Also specifically if triggered by a magic comment. 

% currently, preprocessing of graphic files is done in any case, 
% although specified differently. 

% make a note that still the return codes of latexmk are recorded resulting in EEX01 
% and also the presence of a pdf is monitored triggering EEC03 if not updated. 
% also the manual shows how to convert warnings into errors. 

% make a note: special for latexmk: may not touch pdf file although all ok: 
% return code is still 0 

% latexmk is never used for cleanup with latexmk -c or latexmk -C 

% on magic comments: 
% - target: currently, latexmk is applied only to pdf creation. 
%           the target is determined on a higher level, so it is taken into account 
% - program: taken into account. 
% - latexmk: this magic comment shall be added, 
%            because in many cases latexmk is only wanted for individual latex main files. 
%            It is only to switch latexmk usage on
%            There is some interference with latexmkUsage: 
%            - no effect for Fully and AsBackend 
%            - For NotAtAll, graphics are created already. 



% preferred usage: 
% latexmkUsage=NotAtAll, but activation by magic comment shall be preferred usage. 
% this brings problems finding out whether really latexmk has been used. 
% an idea would be to include the fdb_latexmk file as this is specific for latexmk 
% But this has to be done in a way, that compilation still works if not present. 
% Only the diff to the original fails then. 
% In fact, even two latex main files are required: 
% - one for which latexmk is activated although latexmkUsage is set to the default NotAtAll
% - another one for which latexmk is not requested by magic comment. 
% This can be any document which is already present. 

% Maybe check that latexmk respects also the compiler given by magic comment. 
% Decision: no: is checked for one document once and this is suficient although later it is compiled generically. 
% This is decided because wrong compiler is detected quickly. 

% latexmk can be invoked as 
% SOURCE_DATE_EPOCH=0 FORCE_SOURCE_DATE=1 latexmk xxx.tex 
% as well. 
% That way all converters are invoked with the same environment variables. 
% The deceicive ones are xxxlatex and dvipdf 
% It may be a good idea to provide ntlatexmk analogous to ntlatex as an injection. 
% Adapt documentation accordinly. 




% baglock: chkDiff as a magic comment. 
% this may be better in many cases than a global setting chkDiff (although this has applications also)
% extended patternLatexMainFile
% TBD: adapt manual. 
% This is part of preferred usage: both chkDiff setting in pom 
% and according magic comment. 
% unlike for latexmk, it is not possible to check both preferred usages: 
% for this project the setting chkDiff=true holds 
% so the magic comment must be checked outside the project. 

% Idea: for reproduction check do not use epoque time 0 in general 
% but use timestamp of original. 
% This requires a more sophisiticated script ntlatex 
% and some additional code on java side. 




\section{Creating hypertext}\label{sec:tex2html}

To create HTML and XHTML from TEX files (more precise from \LaTeX{} files), 
a \texttt{tex4htCommand}-command is used 
Together with its parameters, 
it is described in Section!\ref{sec:settingsLatex2Html}. 
This may be \texttt{htlatex}, the default based on \texttt{latex} 
and \texttt{htxelatex} based on \xelatex. 

Figure~\ref{fig:tex2xml} shows the steps \texttt{htlatex} performs: 
From the input \LaTeX{} file \texttt{xxx.tex} 
another \LaTeX{} file \texttt{yyy.tex} is created 
which arises from \texttt{xxx.tex} by adding 
%FIXME\@: maybe instead: \RequirePackage which may be placed before documentclass
\begin{lstlisting}[language=TeX]
\usepackage[...]{tex4ht}. 
\end{lstlisting}
%
Then \texttt{htlatex} runs \texttt{latex} on \texttt{yyy.tex} 
which results in \texttt{yyy.dvi}. 
Note that this is in contrast to \lualatex{} 
which would create some \texttt{yyy.pdf} unless otherwise specified. 

Then comes the converter \pkg{tex4ht} into the game 
which creates several html-files among those also \texttt{xxx.html}. 
The other files, \texttt{yyy.idv} and \texttt{yyy.lg}, 
are further processed by \texttt{t4ht} 
creating the stylesheet \texttt{xxx.css} and graphic files. 
\medskip


Let us make this more precise. 
The output of latex is a standard \gls{dvi}-file 
interleaved with special instructions 
for the post-processor \pkg{tex4ht} to use. 
Note that \pkg{tex4ht} is the name both of the post-processor 
and of the \LaTeX-package. 
The special instructions come from implicit and explicit requests 
made in the source file through commands for TeX4ht. 

The utility \pkg{tex4ht} translates the dvi-code into standard text, 
while obeying the requests it gets from the special instructions. 
The special instructions may request the creation of files, 
insertion of html code, filtering of pictures, and so forth. 
In the extreme case that the source code contains no commands of TeX4ht, 
\pkg{tex4ht} gets pure dvi-code and it outputs (almost) plain text 
with no hypertext elements in it.

The special (\cmd{special}) 
instructions seeded in the dvi-code 
are not understood by dvi processors other than those of TeX4ht.

\texttt{t4ht}
This is an interpreter 
for executing the requests made in the \texttt{xxx.lg} script.

\texttt{xxx.idv}
This is a dvi-file extracted from \texttt{xxx.dvi}, 
and it contains the pictures needed in the html files.

\texttt{xxx.lg}
This is a log file listing the pictures of \texttt{xxx.idv}, 
the \gls{png} files that should be created, CSS information, 
and user directives introduced 
through the ``\cmd{Needs\{\ldots\}}'' command.

\raggedbottom{}


\begin{figure}[!htb]
\centering
\IfPackageLoadedTF{tex4ht}{%
should be a picture 
}{
\import{}{F5_09tex2xml.ptx}
}
\caption{\label{fig:tex2xml}Conversion of a TEX file into an xml-file}
\end{figure}

% LTeX: enabled=false
\begin{Verbatim}[fontsize=\tiny]
(/usr/local/texlive/2014/texmf-dist/tex/generic/tex4ht/tex4ht.4ht
version 2009-01-07-07:11
--------------------------------------
Note --- for additional information, use the command line option `info'
--------------------------------------

(/usr/local/texlive/2014/texmf-dist/tex/generic/tex4ht/html4.4ht

Note: to remove the <?xml version=...?> processing instruction 
use the command line option `no-VERSION'

Note: to remove the DOCTYPE declaration 
use the command line option `no-DOCTYPE'
)

--------------------------------------
Note: for marking of the base font, use the command line option `fonts+'
Note: for non active _, use the command line option `no_'
Note: for _ of catcode 13, use the command line option `_13'
Note: for non active ^, use the command line option `no^'
Note: for ^ of catcode 13, use the command line option `^13'
--------------------------------------

(/usr/local/texlive/2014/texmf-dist/tex/generic/tex4ht/html4.4ht
--------------------------------------
Note: For section filenames that reflect on their titles 
use the command line option `sec-filename'

Note: for alternative charset, use the command line option `charset=...'

Note: to ignore CSS font decoration, use the `NoFonts' command line option

Note: for jpg bitmaps of pictures, 
use the `jpg' command line option. 
(Character bitmaps are controled only by `g' 
records of tex4ht.env and `-g' switches of tex4ht.c) 

Note: for gif bitmaps of pictures, use the `gif' command line option. 
(Character bitmaps are controled only by `g' 
records of tex4ht.env and `-g' switches of tex4ht.c) 

Note: for content and toc in 2 frames, 
use the command line option `frames'

Note: for content, toc, and footnotes in 3 frames, 
use the command line option `frames-fn'

Note --- for file extension name xht, use the command line option `xht'
--------------------------------------
TeX4ht package options: xhtml,uni-html4,2,pic-tabular,html
--------------------------------------
Note: to ignore CSS code, use the command line option `-css

Note: for inline CSS code, use the command line option `css-in'

Note: for pop ups on mouse over, use the command line option `mouseover'

Note: for addressing images in a subdirectory, 
use the command line option `imgdir:.../'
)

Note --- for back links to toc, use the command line option `sections+'

Note --- for linear crosslinks of pages, use the command line option `next'

(/usr/local/texlive/2014/texmf-dist/tex/generic/tex4ht/latex.4ht
version 2009-05-21-09:32
--------------------------------------
Note --- for links into captions, instead of float heads, use the command l
ine option `refcaption'
--------------------------------------

(/usr/local/texlive/2014/texmf-dist/tex/generic/tex4ht/html4.4ht
--------------------------------------
Note --- For mini tocs immediately aftter the header 
use the command line option `minitoc<'

Note --- for enumerated list elements with valued data, 
use the command line option `enumerate+'

Note --- for enumerated list elements li's with value attributes, use the c
ommand line option `enumerate-'

Note --- for CSS2 code, use the command line option `css2'

Note --- for bitmap fbox'es, use the command line option `pic-fbox'

Note --- for bitmap framebox'es, use the command line option `pic-framebox'

Note --- for inline footnotes use command line option `fn-in'

Note --- for tracing of latex font commands, 
use the command line option `fonts'
--------------------------------------
--------------------------------------
Note --- for width specifications of tabular p entries, 
use the `p-width' command line option 
or a configuration similar to 
\Configure{HColWidth}{\HCode{style="width:\HColWidth"}}
--------------------------------------
)
(/usr/local/texlive/2014/texmf-dist/tex/generic/tex4ht/html4-math.4ht
version 2009-05-18-23:01
--------------------------------------
Note --- for pictorial eqnarray, use the command line option `pic-eqnarray'

Note --- for pictorial array, use the command line option `pic-array'

Note --- for pictorial $...$ environments, 
use the command line option `pic-m' (not recommended!!)

Note --- for pictorial $...$ and $$...$$ environments with latex alt, 
use the command line option `pic-m+' (not safe!!)

Note --- for pictorial array, use the command line option `pic-array'
)
(/usr/local/texlive/2014/texmf-dist/tex/generic/tex4ht/unicode.4ht
version 2010-12-18-17:40
)
(/usr/local/texlive/2014/texmf-dist/tex/generic/tex4ht/html4-uni.4ht))


(/usr/local/texlive/2014/texmf-dist/tex/generic/tex4ht/html4.4ht
--------------------------------------
Note --- for tocs without * entries, use command line option `notoc*'

Note --- for tocs without * entries, use command line option `notoc*'

Note --- to eliminate mini tables of contents, 
use the command line option `nominitoc'

Note --- for frames-like object-based table of contents, 
use the command line option `obj-toc'

Note --- for files named derived from section titles, 
use the command line option `sec-filename'

Note --- for i-columns index, 
use the command line option `index=i' (e.g., index=2)
--------------------------------------
)

(/usr/local/texlive/2014/texmf-dist/tex/generic/tex4ht/html4.4ht

Note --- if included graphics are of degraded quality, 
try the command line options `graphics-num' or `graphics-'. 
The `num' should provide the density of pixels in the bitmaps (e.g., 110). 

Note --- for key dimensions try the option `Gin-dim'; 
for key dimensions when bounding box is unavailable 
try `Gin-dim+'; neither is recommended
)

(/usr/local/texlive/2014/texmf-dist/tex/generic/tex4ht/html4.4ht
Note --- for URL encoding within href use the command line option `url-enc'
)

(/usr/local/texlive/2014/texmf-dist/tex/generic/tex4ht/html4.4ht

Note --- for pictorial longtable, 
use the command line option `pic-longtable'
)

(/usr/local/texlive/2014/texmf-dist/tex/generic/tex4ht/html4.4ht

Note --- to ensure proper alignments use fixed size fonts (see listings.dtx
)
)
\end{Verbatim}
% LTeX: enabled=true

\pkg{tex4ht} yields 

\begin{Verbatim}[fontsize=\scriptsize]
----------------------------
tex4ht.c (2012-07-25-19:36 kpathsea)
tex4ht 
--- error --- improper command line
tex4ht [-f<path-separator-ch>]in-file[.dvi]
   [-.<ext>]            replacement to default file extension name .dvi
   [-c<tag name>]       choose named segment in env file
   [-e<env-file>]
   [-f<path-separator-ch>]        remove path from the file name
   [-F<ch-code>]        replacement for missing font characters; 0--255; default 0
   [-g<bitmap-file-ext>]
   [-h(e|f|F|g|s|v|V)]  trace: e-errors/warnings, f-htf, F-htf search
                            g-groups, s-specials, v-env, V-env search
   [-i<htf-font-dir>]
   [-l<bookkeeping-file>]
   [-P(*|<filter>)]     permission for system calls: *-always, filter
   [-S<image-script>]
   [-s<css-file-ext>]   default: -s4cs; multiple entries allowed
   [-t<tfm-font-dir>]
   [-u10]               base 10 for unicode characters
   [-utf8]              utf-8 encoding for unicode characters
   [-v<idv version>]    replacement for the given dvi version
   [-xs]           ms-dos file names for automatically generated gifs
\end{Verbatim}


\texttt{t4ht} yields 

\begin{Verbatim}[fontsize=\footnotesize]
--------------------------------------------------------------------
t4ht [-f<dir char>]filename ...
  -b     ignore -d -m -M for bitmaps
  -c...  choose named segment in env file
  -d...  directory for output files       (default:  current)
  -e...  location of tex4ht.env
  -i     debugging info
  -g     ignore errors in system calls
  -m...  chmod ... of new output files (reused bitmaps excluded)
  -p     don't convert pictures           (default:  convert)
  -r     replace bitmaps of all glyphs    (default:  reuse old ones)
  -M...  chmod ... of all output files
  -Q     quit, if tex4ht.c had problems
  -S...  permission for system calls: *-always, filter
  -X...  content for field %%3 in X scripts
  -....  content for field %%2 in . scripts

Example: 
   t4ht name -d/WWW/temp/ -etex4ht-32.env -m644
--------------------------------------------------------------------
\end{Verbatim}

\flushbottom

\section{Creating odt-files}\label{sec:tex2odt}

\section{Creating MS word files}\label{sec:tex2doc}

The best way to convert \LaTeX{} files into MS word files is via ODT files. 
Conversion from \LaTeX{} to odt 
is already described in Section~\ref{sec:tex2odt}. 
The last step can be done by \texttt{odt2doc} which can create both 
doc-format and docx-format and many others 
which is illustrated in Figure~\ref{fig:tex2doc}. 


\begin{figure}[htb]
\centering
\IfPackageLoadedTF{tex4ht}{%
should be a picture 
}{
\import{}{F5_10tex2doc.ptx}
}
\caption{\label{fig:tex2doc}Conversion of a TEX file into a docx-file}
\end{figure}



\section{Creating plain text files}\label{sec:tex2txt}

Why should one create plain text from \LaTeX{} files? 
Maybe this is the minimal format the receiver can work with. 
Another common application is word-count, 
in particular if writing a paper for a journal. 

Plain text files can be created from \LaTeX{} files 
just by stripping off the tex-commands. 
The disadvantage is, 
that references, bibliography, index, glossary, 
table of contents, list of figures, list of tables, \dots 
and symbols get lost. 
Thus, the first step we take is complete creation of a PDF file 
except display of warnings like bad boxes 
as described in Section~\ref{sec:tex2pdf}. 
This creates an appropriate pdf-file, 
with correct numbering and links, 
possibly with overfull boxes and that like. 
As a final step, we convert the pdf-file into a text file 
using, as a default \texttt{pdftotext} with ending \texttt{txt}. 
Figure~\ref{fig:tex2txt} illustrates the translation process. 

\begin{figure}[htb]
\centering
\IfPackageLoadedTF{tex4ht}{%
should be a picture 
}{
\import{}{F5_11tex2txt.ptx}
}
\caption{\label{fig:tex2txt}Conversion of a TEX file into a txt-file}
\end{figure}

Note that \texttt{pdftotext} produces a text file with page numbers 
and signifies the end of a page 
(to see how, just have a look at the end of the file), 
so that one can identify page numbers as such. 
Thus references, index, glossary, table of contents and that like 
referring to page numbers carry valuable information. 
Also symbols available in utf8 encoding are preserved. 
In contrast, heavily stacked formulae become unreadable, 
because \texttt{pdftotext} displays them line by line 
and drops fraction bars completely. 
Also, formulae with complex subformulae in a root operator  
become unreadable because the root operator becomes just a root symbol. 
Likewise for integrals and that like. 

Aspects of figures kept are the captions of course but also the \LaTeX-texts. 
This is displayed line-wise. 
What gets lost is the postscript/pdf-parts, i.e.~the plain graphics. 

\raggedbottom{}

