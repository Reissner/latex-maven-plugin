
\chapter{Usage of Plugin and Task}\label{chap:usage}

% TBD: in the long run: plugin, task and standalone, or as a dependency
This software offers both, a maven plugin and an according ant task,
but the emphasis is on the maven plugin.
Thus, the sections of this chapter are either general
or apply to the maven plugin;
only Section~\ref{sec:usageAntTask} specifically refers to the ant task. 
Usage presupposes installation as described in Chapter~\ref{chap:install}
including settings in \texttt{pom.xml} 
as described in Section~\ref{sec:xmlPom} for the maven plugin 
and the settings in \texttt{build.xml} 
as described in Section~\ref{sec:xmlBuild} for the ant task.

This plugin may be used both if the \LaTeX-sources are ready 
to create ``final'' output from them 
and also to support development of the \LaTeX{} sources. 
Accordingly, this chapter has Section~\ref{sec:sources}
devoted to the form of the sources, including directory structure,
\LaTeX-files and others, mainly graphic files included
and a Section~\ref{sec:outputFormats} on exporting into various formats.

There is a very special usage, called development of documents,
which means while the document is under construction.
The features and goals tied to this phase
are collected in Section~\ref{sec:devel}.

In contrast, Section~\ref{sec:usageLifecycle}
is on usage of the maven plugin within the lifecycles.
This can be used during development of documents
but is more appropriate for small changes
or when development finished at a stage. 

%TBD: add links to all sections 
% also mention where the individual goals are treated. 

\section{The source files and their directories}\label{sec:sources}

Source files are files contributing to creating documentation 
from \LaTeX-files in the build process 
which are not themselves created in the build process. 
They are searched in the \emph{\TeX{} source directory} and subdirectories recursively. 
\index{\TeX{} source directory}% chktex 24
By default, this is \texttt{./src/site/tex}, 
where ``\texttt{.}'' is the \emph{base directory} of this maven/ant-project. 
This structure complies with conventions in maven-projects. 
\index{base directory}% chktex 24

Note that, against the convention of maven-projects, 
the \TeX{} source directory may contain also files created during the build process. 
By default, after the build process is finished, they are removed again. 
For some background on this see Section~\ref{subsec:sourceCreated}. 

Source files may be TEX files treated in Section~\ref{subsec:sourcesLatex} 
and various kinds of graphic files described in Section~\ref{subsec:sourcesGraphic}, 
but may include also 
%
\begin{itemize}
\item 
verbatim text embedded into TEX files with \pkg{verbatim}, 
\item 
BIB files typically describing a bibliography,  or, %TBD: support of biblatex 
not yet supported, a glossary or that like, % TBD: check when supported, mention glossaries-extra
\item 
program files, either included as a listing by package \pkg{listings} 
or executed via the package \pkg{pythontex}\index{pythontex}. % chktex 24
\end{itemize}


\subsection{\texorpdfstring{\glspl{gls:lmf}}{LaTeX main files} and other latex files}%
\label{subsec:sourcesLatex}

The TEX files are special in that only part of them is processed explicitly 
invoking a compiler like \lualatex{} on them, 
part is just included via \cmd{input} or \cmd{include}. 
The \LaTeX-files to be compiled top level, 
are called \emph{\LaTeX{} main files}. 
As an example, 
in the \emph{TEX source directory} of this software, 
\texttt{manualLMP.tex} is a \LaTeX{} main file, 
whereas the file \texttt{header.tex} is not, although also a \LaTeX-file: 
it is intended to be input in another TEX file, in this case \texttt{manualLMP.tex}. 


% identification of the \LaTeX{} main file?
\LaTeX{} main files are detected automatically 
\index{\LaTeX{} main file}% chktex 24
by fitting the regular expression \texttt{patternLatexMainFile} 
described in detail in Table~\ref{tab:paramGen} on page~\pageref{tab:paramGen}, 
and in the reference given there, 
whereas the description here is quite high level. 

As a first approximation, a \LaTeX{} main file is one invoking 
the command \cmd{documentclass} 
or the outdated \cmd{documentstyle}, both specifying the document class. 
It must be excluded that the pattern matches a textual occurrence of \cmd{documentclass}, 
which just occurs because the document is on \LaTeX{} 
and mentions the command \cmd{documentclass}. 
This is quite easy, since there is little allowed in TEX files preceding these commands. 

Consequently, the pattern matches the region from the start of the file 
to and including the \cmd{documentclass} or \cmd{documentstyle} command. 
This starting segment of a \LaTeX{} main file 
is called the \textit{opening}\index{opening}. 

Here a word of warning is at place: if a TEX file does not fit the pattern, 
it is not interpreted as a \LaTeX{} main file without further warning. 
So check whether the file under consideration is processed if 
%
\begin{itemize}
  \item 
  it is built for the first time or 
  \item 
  its opening  is changed or
  \item 
  the parameter \texttt{patternLatexMainFile} is changed 
\end{itemize}
%
If you distrust the recognition mechanism via pattern matching altogether, 
you can explicitly specify each \LaTeX{} main file 
in the parameter \texttt{mainFilesIncluded} 
described in Table~\ref{tab:paramGen} on page~\pageref{tab:paramGen}. 
This is safe because if a file specified in \texttt{mainFilesIncluded} 
that like is not a \LaTeX{} main file according to the pattern \texttt{patternLatexMainFile}, 
a warning is emitted. 
That way one can check whether the pattern is matched. 
We could have decided that these files are compiled with or without warning, 
but this would lead to a technique is that it is inconvenient and not well maintainable. 


\subsubsection{On openings neglecting (magic) comments}\label{subsubsec:openingNoMagComm}

This section contains material both specific for supported document classes 
and general information but magic comments are deferred to Section~\ref{subsubsec:openingMagComm}. 

This software is tested for document classes 
%
\begin{itemize}
  \item \texttt{article} and \texttt{book} which are built-in, 
  \item \texttt{beamer} for presentations as described in~\cite{Beamer}, 
  \item \texttt{leaflet} creating leaflets as explained in~\cite{Leaflet} and in~\cite{LeafletMan}, 
  \item \texttt{scrlttr2} for letters described in~\cite{KomaAnl23}, Part1, Chapter 4, 
  \item and \texttt{minimal} for quite special uses (did not find real documentation). 
\end{itemize}


Note that \texttt{scrlttr2} replaces the built-in \texttt{letter} which is not recommended. 
In fact, is the only KOMA\index{KOMA} class this software is tested for. 
The attentive reader may realize 
that the built-in document class \texttt{report} is not mentioned. 
It shall work but is currently not tested. 

Nevertheless, the pattern \texttt{patternLatexMainFile} 
matches all possible document classes. 
Typically, the document class is loaded with options. 
The most frequent class may be \texttt{article} 
followed by \texttt{report} and \texttt{book}. 
For these, we suggest something like
%
\begin{Verbatim}
\documentclass[a4paper,12pt,english]{article}
\end{Verbatim}
%
where \texttt{a4paper} is a setting, typical for Europe or, not the US\@. 
The default font size is \texttt{10pt} and sometimes it makes sense to increase this. 
For documents which are solely in English no language setting is required, 
except if loading the package \pkg{babel}, else the hyphenation patterns get lost. 
Since we recommend inputting the header file \texttt{header.tex} 
described in Section~\ref{subsec:header} which in turn loads \pkg{babel} and 
by the way also \pkg{csquotes}, a language setting is mandatory. 
It is possible specifying the language when loading \pkg{babel} as an option 
like so \cmd{usepackage[english]\{babel\}}, 
but it is recommended to specify the language as an option of the document class, 
in order to make it available for various packages, 
besides \pkg{babel} also for \pkg{csquotes}. 
If a document has more than one language, specify all of them, 
the last the one the document starts with, 
but~\cite{BabelP24} Sections~1.7 and 1.8 show how to change language, 
here temporarily into German, 
which requires specifying \texttt{german} in front of \texttt{english} 
in the document class. 
Note the quotes and the correct hyphenation in the following paragraph. 

%   \begin{center}
%     \begin{foreignlanguage}{german}
%       Liberté, Egalité, Fraternité
% \end{foreignlanguage}
% \end{center}
\begin{quote}
\begin{otherlanguage}{german}
  Sie las den Artikel \enquote{Chancen für eine diplomatische Lösung} 
  in der "`Wochenpost"', während er es sich nicht nehmen ließ, % chktex 18
  sich Thomas Manns Novelle "`{Der Tod in Venedig}"' zu Gemüte zu führen. % chktex 18
\end{otherlanguage}
\end{quote}

To obtain correct quotes in the above paragraph, 
package \pkg{csquotes} must be loaded with option \texttt{autostyle}. 
Since \pkg{csquotes} is loaded in \texttt{header.tex} 
given by an injection as described in Section~\ref{sec:injFiles}, 
this option must be passed to \pkg{csquotes} 
before loading the document class, e.g.\@ via 
%
\begin{Verbatim}[fontsize=\small]
  \PassOptionsToPackage{autostyle}{csquotes}
\end{Verbatim}
%
This is the technique to pass options to packages in general. 

For documents of class \texttt{minimal} there are no requirements imposed. 
No checks and no PDF-info. 
For all other document classes, it is recommended to load \pkg{nag} 
before \cmd{documentclass} by 
%
\begin{Verbatim}[fontsize=\small]
  \RequirePackage[l2tabu, orthodox]{nag}
\end{Verbatim}
%
Thus the pattern \texttt{patternLatexMainFile} 
allows \cmd{RequirePackage} with its options preceding \cmd{documentclass}. 

There are cases, where one and the same document 
comes in two flavors both of which must be built. 
As an example, consider a document with a confidential variant 
and with a non-confidential variant. 
To define these, the declaration of the document class 
must be preceded by the following kind of code: 
%
\begin{Verbatim}[fontsize=\small]
\RequirePackage{etoolbox}
\newbool{isConfidential}
\setbool{isConfidential}{true}
\end{Verbatim}

The new thing is defining and setting a boolean 
via \cmd{newbool} and \cmd{setbool}, respectively. 
It is a good idea to set a watermark via 
%
\begin{Verbatim}[fontsize=\small]
\ifbool{isConfidential}{%
  \usepackage{draftwatermark}
  \SetWatermarkText{Confidential}%
}{%
% no watermark text
}
\end{Verbatim}

The same technique differentiating between confidential and public 
may be used to define a lecture with and without solution 
or any other kind of variant. 

Documents of class \texttt{leaflet} resemble \texttt{article}s, 
except special options like \texttt{notumble}. 

\begin{Verbatim}[fontsize=\footnotesize]
  \documentclass[a4paper,notumble,10pt,english]{leaflet}% 12pt,notumble
\end{Verbatim}

The same is true for letters of type \texttt{scrlttr2}, 
except for the special specification of font size and versioning of the class: 

\begin{Verbatim}[fontsize=\footnotesize]
\documentclass[english,german,a4paper,fontsize=10pt,version=last]{scrlttr2}
...

%\listfiles
\synctex=1% maybe security issue: draft only? 

% for buildParams to check empty: \ifdefempty
%\usepackage{etoolbox}
% for buildParams: \verbdef 
%\usepackage{newverbs}

% provdies \ifPDFTeX, \ifXeTeX and \ifLuaTeX. 
% iftutex test is true for XeTeX and LuaTeX, 
% and an ifpdf test is provided to test the PDF or DVI output mode.
\usepackage{iftex}

% provides \newboolean, \setboolean 
% is used to integrate html production with tex4ht and pdf production
% used to define texFhtLoaded and beamerLoaded 
% maybe this is not really absolute necessary 

\usepackage{ifthen}
\newboolean{texFhtLoaded}
\setboolean{texFhtLoaded}{false}

% only with pdflatex, warnings for xelatex and for lualatex 
% ifxetex, ifluatex, ifpdf
\ifpdf%
  %\usepackage{mlmodern}
\else
  \makeatletter
  \@ifpackageloaded{tex4ht}{%
    \setboolean{texFhtLoaded}{true}
  }{%
  }% tex4ht not loaded 
  \makeatother
\fi


\newboolean{beamerLoaded}
\makeatletter
\@ifclassloaded{beamer}{%
  \setboolean{beamerLoaded}{true}
}{
  \setboolean{beamerLoaded}{false}
}
\makeatother



\iftutex%
  \usepackage{fontspec}
\else
  % this seems to work with beamer also 
  \usepackage[utf8]{inputenc}
  \usepackage[T1]{fontenc}
\fi
%\usepackage{textalpha}


% absolutely necessary. 
% for document development add certain options. 
% Then remove headline and prevent this plugin from overwriting. 

\ifthenelse{\boolean{beamerLoaded}}{
  % here nothing to do. 
  % beamer loads geometry itself. 
  % The option a4paper does not make sense; 
  % one may set aspectratio in \documentclass
}{
  \usepackage[a4paper]{geometry}% option , showframe, showcrop 
}
%\usepackage{showframe} as an alternative 
\usepackage{microtype}
%\usepackage[indent,skip=0]{parskip}% used by pandoc but not good 
% special characters
\usepackage{textcomp}
\usepackage{anyfontsize}% important e.g. for beamer class 
%\usepackage{cleveref}


% used by hyperref and also to update index and glossary 
% to avoid clash because of loading with different options: 
% declare first 
% Note that without options the check is the most strict one 
\usepackage{rerunfilecheck}

% graphics 

\ifpdf%
  % for accessability with luatex
  %\usepackage{luatex85}
  % compiles for xelatex only 
  %\usepackage[tagged, highstructure]{accessibility}
  \usepackage{xcolor}  % [pdftex]  
  \usepackage{graphicx}% [pdftex] 
  % driver [hpdftex] is autodetected 
  \usepackage[destlabel]{hyperref}
  % sometimes comes in with svg import 
  \usepackage{transparent}
  % warning transparent package: 
  % loading aborted if not pdf-mode 
  % strange: according to documentation not for xelatex; 
  % seems to work anyway 
  % can be extended using l3opacity
\else
  % No PDF, includes dvi/xdv and HTML,... via package tex4ht 
  \usepackage[dvipdfmx]{xcolor}
  \usepackage[dvipdfmx]{graphicx}
  \ifthenelse{\boolean{texFhtLoaded}}{%
    \usepackage[tex4ht, destlabel]{hyperref}
  }{%
    \ifxetex%
      \usepackage[destlabel]{hyperref}
    \else
      \usepackage[dvipdfmx, destlabel]{hyperref}%[dvipdfmx]
      % lualatex: without [dvipdfmx] option did not find 
      % converter dvi to pdf or to ps
      % pdflatex: without [dvipdfmx] option dvips still works, 
      % but no converter for pdf
    \fi
  }% tex4ht not loaded 
  %\usepackage{bmpsize}% not for xelatex 
\fi%ifpdf

\ifluatex%
  \usepackage{luamplib}
  \newcommand*\inputmpcode[1]{\begin{mplibcode}input #1\end{mplibcode}}
\else
\fi

% \@ifpackageloaded{tex4ht}{%
% \usepackage[dvipdfmx]{xcolor}
% \usepackage[dvipdfmx]{graphicx}
% \usepackage[tex4ht]{hyperref}
% \usepackage{bmpsize}
% }{%
% \usepackage{xcolor}  % [pdftex]  
% \usepackage{graphicx}% [pdftex] 
% \usepackage{hyperref}% driver [hpdftex] is autodetected 
% }


%\usepackage[clear,pdf,eps]{svg}

\usepackage{import}
\usepackage{amsmath}

% synchronization between tex and pdf 
%\usepackage[active]{srcltx}
\usepackage{longtable}
\usepackage{listings}
% this is a workaround for including listings with latexmk.. 
% This can be fixed 
% - as shown below 
% - patch in package listings 
% - patch in latexmk 
% I would prefer the latter. 
\usepackage{xpatch}
\makeatletter
\newcommand*{\NewLine}{^^J}%
\xpatchcmd{\lst@MissingFileError}
{Package Listings Error: File `#1(.#2)' not found.}
{LaTeX Error: File `#1.#2' not found.\NewLine}{%
  \typeout{File ending patch for \string\lst@MissingFileError\space done.}%
}{%
  \typeout{File ending patch for \string\lst@MissingFileError\space failed.}%
}
\makeatother

\usepackage{fancyvrb}


% index and glossary
\ifthenelse{\boolean{texFhtLoaded}}{%
  \newcommand{\pkg}[1]{\texttt{#1}}% without indexing 
}{
  \usepackage{splitidx}%[split]
%  \usepackage{makeidx}
%  \usepackage{showidx}
  \makeindex
  \usepackage[toc]{glossaries}%,automake
  % , xindy or even [xindy={language=english,codepage=utf8}]
  % mainly for index and glossaries 
  %\makeglossaries% TBD: activate later
  \newcommand{\pkg}[1]{\texttt{#1}\sindex[pkg]{#1}} % TBD: this must be extracted 
  }

% high quality tables 
\usepackage{booktabs}
\aboverulesep=0ex
\belowrulesep=0ex

\usepackage{xurl}

%\makeglossary% for rerunfilecheck 

%\usepackage{etexcmds} %still later 
\ifthenelse{\boolean{beamerLoaded}}{
  % TBD: clarify this case. 
  % maybe beamer does not support indices or glossaries. 
  % 
}{
  \usepackage[nottoc, numindex, numbib]{tocbibind}
}

%\usepackage{latex-bnf}




\LoadLetterOption{DIN}
\end{Verbatim}

Observe that after inputting \texttt{header.tex} which loads \pkg{geometry}, 
various pseudo lengths have to be re-adjusted. 
This is done by loading the letter option. 
Without it may happen, that the text does not reach until the bottom of the frame. 


What is special for beamer presentations is, 
that in general two documents with the same identifier, e.g.\@ title are created, 
the proper presentation, e.g.\@ 
% TBD: eliminate change of font size: long line must be breakable 
\begin{Verbatim}[fontsize=\scriptsize]
\RequirePackage[l2tabu, orthodox]{nag}
\PassOptionsToPackage{colorlinks,linkcolor=blue,urlcolor=blue,citecolor=blue,destlabel}{hyperref}

\documentclass[10pt,english]{beamer}
\mode<presentation>%



%\listfiles
\synctex=1% maybe security issue: draft only? 

% for buildParams to check empty: \ifdefempty
%\usepackage{etoolbox}
% for buildParams: \verbdef 
%\usepackage{newverbs}

% provdies \ifPDFTeX, \ifXeTeX and \ifLuaTeX. 
% iftutex test is true for XeTeX and LuaTeX, 
% and an ifpdf test is provided to test the PDF or DVI output mode.
\usepackage{iftex}

% provides \newboolean, \setboolean 
% is used to integrate html production with tex4ht and pdf production
% used to define texFhtLoaded and beamerLoaded 
% maybe this is not really absolute necessary 

\usepackage{ifthen}
\newboolean{texFhtLoaded}
\setboolean{texFhtLoaded}{false}

% only with pdflatex, warnings for xelatex and for lualatex 
% ifxetex, ifluatex, ifpdf
\ifpdf%
  %\usepackage{mlmodern}
\else
  \makeatletter
  \@ifpackageloaded{tex4ht}{%
    \setboolean{texFhtLoaded}{true}
  }{%
  }% tex4ht not loaded 
  \makeatother
\fi


\newboolean{beamerLoaded}
\makeatletter
\@ifclassloaded{beamer}{%
  \setboolean{beamerLoaded}{true}
}{
  \setboolean{beamerLoaded}{false}
}
\makeatother



\iftutex%
  \usepackage{fontspec}
\else
  % this seems to work with beamer also 
  \usepackage[utf8]{inputenc}
  \usepackage[T1]{fontenc}
\fi
%\usepackage{textalpha}


% absolutely necessary. 
% for document development add certain options. 
% Then remove headline and prevent this plugin from overwriting. 

\ifthenelse{\boolean{beamerLoaded}}{
  % here nothing to do. 
  % beamer loads geometry itself. 
  % The option a4paper does not make sense; 
  % one may set aspectratio in \documentclass
}{
  \usepackage[a4paper]{geometry}% option , showframe, showcrop 
}
%\usepackage{showframe} as an alternative 
\usepackage{microtype}
%\usepackage[indent,skip=0]{parskip}% used by pandoc but not good 
% special characters
\usepackage{textcomp}
\usepackage{anyfontsize}% important e.g. for beamer class 
%\usepackage{cleveref}


% used by hyperref and also to update index and glossary 
% to avoid clash because of loading with different options: 
% declare first 
% Note that without options the check is the most strict one 
\usepackage{rerunfilecheck}

% graphics 

\ifpdf%
  % for accessability with luatex
  %\usepackage{luatex85}
  % compiles for xelatex only 
  %\usepackage[tagged, highstructure]{accessibility}
  \usepackage{xcolor}  % [pdftex]  
  \usepackage{graphicx}% [pdftex] 
  % driver [hpdftex] is autodetected 
  \usepackage[destlabel]{hyperref}
  % sometimes comes in with svg import 
  \usepackage{transparent}
  % warning transparent package: 
  % loading aborted if not pdf-mode 
  % strange: according to documentation not for xelatex; 
  % seems to work anyway 
  % can be extended using l3opacity
\else
  % No PDF, includes dvi/xdv and HTML,... via package tex4ht 
  \usepackage[dvipdfmx]{xcolor}
  \usepackage[dvipdfmx]{graphicx}
  \ifthenelse{\boolean{texFhtLoaded}}{%
    \usepackage[tex4ht, destlabel]{hyperref}
  }{%
    \ifxetex%
      \usepackage[destlabel]{hyperref}
    \else
      \usepackage[dvipdfmx, destlabel]{hyperref}%[dvipdfmx]
      % lualatex: without [dvipdfmx] option did not find 
      % converter dvi to pdf or to ps
      % pdflatex: without [dvipdfmx] option dvips still works, 
      % but no converter for pdf
    \fi
  }% tex4ht not loaded 
  %\usepackage{bmpsize}% not for xelatex 
\fi%ifpdf

\ifluatex%
  \usepackage{luamplib}
  \newcommand*\inputmpcode[1]{\begin{mplibcode}input #1\end{mplibcode}}
\else
\fi

% \@ifpackageloaded{tex4ht}{%
% \usepackage[dvipdfmx]{xcolor}
% \usepackage[dvipdfmx]{graphicx}
% \usepackage[tex4ht]{hyperref}
% \usepackage{bmpsize}
% }{%
% \usepackage{xcolor}  % [pdftex]  
% \usepackage{graphicx}% [pdftex] 
% \usepackage{hyperref}% driver [hpdftex] is autodetected 
% }


%\usepackage[clear,pdf,eps]{svg}

\usepackage{import}
\usepackage{amsmath}

% synchronization between tex and pdf 
%\usepackage[active]{srcltx}
\usepackage{longtable}
\usepackage{listings}
% this is a workaround for including listings with latexmk.. 
% This can be fixed 
% - as shown below 
% - patch in package listings 
% - patch in latexmk 
% I would prefer the latter. 
\usepackage{xpatch}
\makeatletter
\newcommand*{\NewLine}{^^J}%
\xpatchcmd{\lst@MissingFileError}
{Package Listings Error: File `#1(.#2)' not found.}
{LaTeX Error: File `#1.#2' not found.\NewLine}{%
  \typeout{File ending patch for \string\lst@MissingFileError\space done.}%
}{%
  \typeout{File ending patch for \string\lst@MissingFileError\space failed.}%
}
\makeatother

\usepackage{fancyvrb}


% index and glossary
\ifthenelse{\boolean{texFhtLoaded}}{%
  \newcommand{\pkg}[1]{\texttt{#1}}% without indexing 
}{
  \usepackage{splitidx}%[split]
%  \usepackage{makeidx}
%  \usepackage{showidx}
  \makeindex
  \usepackage[toc]{glossaries}%,automake
  % , xindy or even [xindy={language=english,codepage=utf8}]
  % mainly for index and glossaries 
  %\makeglossaries% TBD: activate later
  \newcommand{\pkg}[1]{\texttt{#1}\sindex[pkg]{#1}} % TBD: this must be extracted 
  }

% high quality tables 
\usepackage{booktabs}
\aboverulesep=0ex
\belowrulesep=0ex

\usepackage{xurl}

%\makeglossary% for rerunfilecheck 

%\usepackage{etexcmds} %still later 
\ifthenelse{\boolean{beamerLoaded}}{
  % TBD: clarify this case. 
  % maybe beamer does not support indices or glossaries. 
  % 
}{
  \usepackage[nottoc, numindex, numbib]{tocbibind}
}

%\usepackage{latex-bnf}




../../tex/headerSuppressMetaPDF.tex

\hypersetup{
  pdfinfo={
    Author      ={Ernst Reissner},
    Title       ={Presentation with/of the latex-maven-plugin },
    Subject     ={presentations with beamer},
    Keywords    ={LaTeX;beamer}
  }
}

% TBD: clarify what to do if no pdf is created. 
% As soon as this question comes up in the course of new development, 
% this will be compiled as html by accident. 
% The we can see, how well this works. 

\usetheme{Berlin}
\title{Presentation with/of the \texttt{latex-maven-plugin}}
\author{E. Reissner}
\date{ernst.reissner@simuline.eu}


\hypersetup{colorlinks,linkcolor=,urlcolor=blue,citecolor=yellow}%
%\hypersetup{frenchlinks}
\begin{document}
 
\mode<article>{\maketitle}




\begin{frame}
  \titlepage%
\end{frame}

\section*{Outline}

\begin{frame}
  \tableofcontents
\end{frame}

\section{Introduction and Purpose}

This is some additional text 

\begin{frame}
  \frametitle{Purpose of these documents }
  The purpose of this document is twofold:
  %
  \begin{itemize}
    \item Give an overview over the plugin. 
    \alert{But: }
    %
    \begin{itemize}
      \item The single official description of the plugin is the manual~\cite{LatexPlugin}.
      \item The 
      \href{http://simuline.eu/LatexMavenPlugin/index.html}{project site} 
      gives already an overview. 
      \item 
      The content of this presentation is updated only by need. 
    \end{itemize}
    
    \item Demonstrate that the plugin can compile a presentation using the \texttt{beamer} class 
    and a handout written as \texttt{beamerarticle}. 

    \begin{itemize}
      \item In fact, when citing this presentation as~\cite{PresBeamer}, 
      this refers to both. 
      \item The presentation is about the plugin 
      \item The handout adds information on how to write a beamer presentation. 
      \item Both documents turn \texttt{beamer} class and package \texttt{beamerarticle} into preferred usage 
      ensuring tests. 
    \end{itemize}
  \end{itemize}
  
\end{frame}


\section{Features}

\subsection{Realized Features}

\begin{frame}
  \frametitle{Ant and Maven}
  Automatically creates documents from LaTeX sources during the Maven \texttt{site} phase 
  and in an ant run. 

  This comprises running basic tools like \texttt{lualatex} and \texttt{bibtex} 
  and rerunning them by need. 

\end{frame}

\begin{frame}
  \frametitle{Supported IO}
  Supports 
  \begin{itemize}
    \item
    many output formats like PDF, DVI, HTML, DOCX, RTF, TXT and others
    \item
    many graphical input formats like PNG, MP, FIG, gnuplot; 
    also provides a separate goal creating them, \texttt{grp} 
    \item
    bibliography, index, glossary and embedded code; in particular split index
  \end{itemize}
\end{frame}


\begin{frame}
  \frametitle{Checks}
  Check 
  %
  \begin{itemize}
    \item
    sources with \texttt{chktex} and logs the results in target and goal \texttt{chk} 
    \item
    versions of used tools via goal \texttt{vrs} 
    \item
    log files detecting errors and warnings 
    \item
    whether a document could have been reproduced, by demand 
  \end{itemize}
\end{frame}

\begin{frame}
  \frametitle{Orchestration and document development}
  Orchestration of various tools detecting need for execution e.g.
  of \texttt{bibtex} including \texttt{rerunfilecheck} for \texttt{lualatex} and friends.

  Supports document development, mainly by cooperating with editor, viewer and with other tools in the build chain.
  \begin{itemize}
    \item Offers installation script for extensions of VS Code.
    \item Offers configuration file \texttt{.chktexrc} for \texttt{chktex}.
    \item Can create configuration file \texttt{.latexmkrc} 
    for \texttt{latexmk} synchronized with the configuration.
    \item Offers a common header file \texttt{header.tex} 
    to unify packages loaded by latex main files.
  \end{itemize}
\end{frame}


\subsection{Planned Features}

\begin{frame}
  \frametitle{Planned Features}
  \begin{itemize}
  \item Support \texttt{biber} replacing \texttt{bibtex} as preferred tool
  \item Support \texttt{xindy} replacing \texttt{makeindex} as preferred tool
  \item Support \texttt{bib2gls} replacing \texttt{makeglossary} as preferred tool
  \item Execute \texttt{glosstex} if needed
  \item Usage of the \texttt{multibib} macros
  \item \dots
  \end{itemize}
\end{frame}

% \section{Installation}

\section{Goals}

\begin{frame}
  \frametitle{Goals}

  An overview is on the 
  \href{http://simuline.eu/LatexMavenPlugin/plugin-info.html}{goals page}. 
  
  \setbeamercolor{alerted text}{fg=blue}
  Besides goals referring to creating output of a given type 
  like \alert{\texttt{pdf}}, \alert{\texttt{odt}} and \alert{\texttt{html}}, 
  there is a goal \alert{\texttt{cfg}}, 
  which allows \setbeamercolor{alerted text}{fg=red}
  \alert con\alert fi\alert guring various output formats as \texttt{targets}. % chktex 1
  
  \setbeamercolor{alerted text}{fg=blue}
  One further goal, \alert{\texttt{chk}} to perform a check with \texttt{chktex} 
  is allowed among the \texttt{targets}. 

  The other goals cannot serve as targets: 
  %
  \begin{description}
    \item[\texttt{vrs}] to check versions of tools, 
    \item[\texttt{grp}] to create graphic files speeding up use of \texttt{latexmk} 
    and 
    \item[\texttt{inj}] to inject useful files like headers or config files 
    like \texttt{.latexmkrc}. 
  \end{description}
  

\end{frame}

% \section{Usage}

% \section{Examples}

\mode<presentation>
\section{References}% if with star not in toc 


\begin{frame}[allowframebreaks]
  \frametitle{References}
  \bibliographystyle{alpha}
  \bibliography{../lit}{}
\end{frame}

\mode<article>
\bibliographystyle{alpha}
\bibliography{../lit}{}

\end{document}
\end{Verbatim}
%
and the corresponding handout 
%
\begin{Verbatim}
  \RequirePackage[l2tabu, orthodox]{nag}

  \documentclass[a4paper]{article}
  \usepackage{beamerarticle}
  

%\listfiles
\synctex=1% maybe security issue: draft only? 

% for buildParams to check empty: \ifdefempty
%\usepackage{etoolbox}
% for buildParams: \verbdef 
%\usepackage{newverbs}

% provdies \ifPDFTeX, \ifXeTeX and \ifLuaTeX. 
% iftutex test is true for XeTeX and LuaTeX, 
% and an ifpdf test is provided to test the PDF or DVI output mode.
\usepackage{iftex}

% provides \newboolean, \setboolean 
% is used to integrate html production with tex4ht and pdf production
% used to define texFhtLoaded and beamerLoaded 
% maybe this is not really absolute necessary 

\usepackage{ifthen}
\newboolean{texFhtLoaded}
\setboolean{texFhtLoaded}{false}

% only with pdflatex, warnings for xelatex and for lualatex 
% ifxetex, ifluatex, ifpdf
\ifpdf%
  %\usepackage{mlmodern}
\else
  \makeatletter
  \@ifpackageloaded{tex4ht}{%
    \setboolean{texFhtLoaded}{true}
  }{%
  }% tex4ht not loaded 
  \makeatother
\fi


\newboolean{beamerLoaded}
\makeatletter
\@ifclassloaded{beamer}{%
  \setboolean{beamerLoaded}{true}
}{
  \setboolean{beamerLoaded}{false}
}
\makeatother



\iftutex%
  \usepackage{fontspec}
\else
  % this seems to work with beamer also 
  \usepackage[utf8]{inputenc}
  \usepackage[T1]{fontenc}
\fi
%\usepackage{textalpha}


% absolutely necessary. 
% for document development add certain options. 
% Then remove headline and prevent this plugin from overwriting. 

\ifthenelse{\boolean{beamerLoaded}}{
  % here nothing to do. 
  % beamer loads geometry itself. 
  % The option a4paper does not make sense; 
  % one may set aspectratio in \documentclass
}{
  \usepackage[a4paper]{geometry}% option , showframe, showcrop 
}
%\usepackage{showframe} as an alternative 
\usepackage{microtype}
%\usepackage[indent,skip=0]{parskip}% used by pandoc but not good 
% special characters
\usepackage{textcomp}
\usepackage{anyfontsize}% important e.g. for beamer class 
%\usepackage{cleveref}


% used by hyperref and also to update index and glossary 
% to avoid clash because of loading with different options: 
% declare first 
% Note that without options the check is the most strict one 
\usepackage{rerunfilecheck}

% graphics 

\ifpdf%
  % for accessability with luatex
  %\usepackage{luatex85}
  % compiles for xelatex only 
  %\usepackage[tagged, highstructure]{accessibility}
  \usepackage{xcolor}  % [pdftex]  
  \usepackage{graphicx}% [pdftex] 
  % driver [hpdftex] is autodetected 
  \usepackage[destlabel]{hyperref}
  % sometimes comes in with svg import 
  \usepackage{transparent}
  % warning transparent package: 
  % loading aborted if not pdf-mode 
  % strange: according to documentation not for xelatex; 
  % seems to work anyway 
  % can be extended using l3opacity
\else
  % No PDF, includes dvi/xdv and HTML,... via package tex4ht 
  \usepackage[dvipdfmx]{xcolor}
  \usepackage[dvipdfmx]{graphicx}
  \ifthenelse{\boolean{texFhtLoaded}}{%
    \usepackage[tex4ht, destlabel]{hyperref}
  }{%
    \ifxetex%
      \usepackage[destlabel]{hyperref}
    \else
      \usepackage[dvipdfmx, destlabel]{hyperref}%[dvipdfmx]
      % lualatex: without [dvipdfmx] option did not find 
      % converter dvi to pdf or to ps
      % pdflatex: without [dvipdfmx] option dvips still works, 
      % but no converter for pdf
    \fi
  }% tex4ht not loaded 
  %\usepackage{bmpsize}% not for xelatex 
\fi%ifpdf

\ifluatex%
  \usepackage{luamplib}
  \newcommand*\inputmpcode[1]{\begin{mplibcode}input #1\end{mplibcode}}
\else
\fi

% \@ifpackageloaded{tex4ht}{%
% \usepackage[dvipdfmx]{xcolor}
% \usepackage[dvipdfmx]{graphicx}
% \usepackage[tex4ht]{hyperref}
% \usepackage{bmpsize}
% }{%
% \usepackage{xcolor}  % [pdftex]  
% \usepackage{graphicx}% [pdftex] 
% \usepackage{hyperref}% driver [hpdftex] is autodetected 
% }


%\usepackage[clear,pdf,eps]{svg}

\usepackage{import}
\usepackage{amsmath}

% synchronization between tex and pdf 
%\usepackage[active]{srcltx}
\usepackage{longtable}
\usepackage{listings}
% this is a workaround for including listings with latexmk.. 
% This can be fixed 
% - as shown below 
% - patch in package listings 
% - patch in latexmk 
% I would prefer the latter. 
\usepackage{xpatch}
\makeatletter
\newcommand*{\NewLine}{^^J}%
\xpatchcmd{\lst@MissingFileError}
{Package Listings Error: File `#1(.#2)' not found.}
{LaTeX Error: File `#1.#2' not found.\NewLine}{%
  \typeout{File ending patch for \string\lst@MissingFileError\space done.}%
}{%
  \typeout{File ending patch for \string\lst@MissingFileError\space failed.}%
}
\makeatother

\usepackage{fancyvrb}


% index and glossary
\ifthenelse{\boolean{texFhtLoaded}}{%
  \newcommand{\pkg}[1]{\texttt{#1}}% without indexing 
}{
  \usepackage{splitidx}%[split]
%  \usepackage{makeidx}
%  \usepackage{showidx}
  \makeindex
  \usepackage[toc]{glossaries}%,automake
  % , xindy or even [xindy={language=english,codepage=utf8}]
  % mainly for index and glossaries 
  %\makeglossaries% TBD: activate later
  \newcommand{\pkg}[1]{\texttt{#1}\sindex[pkg]{#1}} % TBD: this must be extracted 
  }

% high quality tables 
\usepackage{booktabs}
\aboverulesep=0ex
\belowrulesep=0ex

\usepackage{xurl}

%\makeglossary% for rerunfilecheck 

%\usepackage{etexcmds} %still later 
\ifthenelse{\boolean{beamerLoaded}}{
  % TBD: clarify this case. 
  % maybe beamer does not support indices or glossaries. 
  % 
}{
  \usepackage[nottoc, numindex, numbib]{tocbibind}
}

%\usepackage{latex-bnf}




../../tex/headerSuppressMetaPDF.tex

\hypersetup{
  pdfinfo={
    Author      ={Ernst Reissner},
    Title       ={Presentation with/of the latex-maven-plugin },
    Subject     ={presentations with beamer},
    Keywords    ={LaTeX;beamer}
  }
}

% TBD: clarify what to do if no pdf is created. 
% As soon as this question comes up in the course of new development, 
% this will be compiled as html by accident. 
% The we can see, how well this works. 

\usetheme{Berlin}
\title{Presentation with/of the \texttt{latex-maven-plugin}}
\author{E. Reissner}
\date{ernst.reissner@simuline.eu}


\hypersetup{colorlinks,linkcolor=,urlcolor=blue,citecolor=yellow}%
%\hypersetup{frenchlinks}
\begin{document}
 
\mode<article>{\maketitle}




\begin{frame}
  \titlepage%
\end{frame}

\section*{Outline}

\begin{frame}
  \tableofcontents
\end{frame}

\section{Introduction and Purpose}

This is some additional text 

\begin{frame}
  \frametitle{Purpose of these documents }
  The purpose of this document is twofold:
  %
  \begin{itemize}
    \item Give an overview over the plugin. 
    \alert{But: }
    %
    \begin{itemize}
      \item The single official description of the plugin is the manual~\cite{LatexPlugin}.
      \item The 
      \href{http://simuline.eu/LatexMavenPlugin/index.html}{project site} 
      gives already an overview. 
      \item 
      The content of this presentation is updated only by need. 
    \end{itemize}
    
    \item Demonstrate that the plugin can compile a presentation using the \texttt{beamer} class 
    and a handout written as \texttt{beamerarticle}. 

    \begin{itemize}
      \item In fact, when citing this presentation as~\cite{PresBeamer}, 
      this refers to both. 
      \item The presentation is about the plugin 
      \item The handout adds information on how to write a beamer presentation. 
      \item Both documents turn \texttt{beamer} class and package \texttt{beamerarticle} into preferred usage 
      ensuring tests. 
    \end{itemize}
  \end{itemize}
  
\end{frame}


\section{Features}

\subsection{Realized Features}

\begin{frame}
  \frametitle{Ant and Maven}
  Automatically creates documents from LaTeX sources during the Maven \texttt{site} phase 
  and in an ant run. 

  This comprises running basic tools like \texttt{lualatex} and \texttt{bibtex} 
  and rerunning them by need. 

\end{frame}

\begin{frame}
  \frametitle{Supported IO}
  Supports 
  \begin{itemize}
    \item
    many output formats like PDF, DVI, HTML, DOCX, RTF, TXT and others
    \item
    many graphical input formats like PNG, MP, FIG, gnuplot; 
    also provides a separate goal creating them, \texttt{grp} 
    \item
    bibliography, index, glossary and embedded code; in particular split index
  \end{itemize}
\end{frame}


\begin{frame}
  \frametitle{Checks}
  Check 
  %
  \begin{itemize}
    \item
    sources with \texttt{chktex} and logs the results in target and goal \texttt{chk} 
    \item
    versions of used tools via goal \texttt{vrs} 
    \item
    log files detecting errors and warnings 
    \item
    whether a document could have been reproduced, by demand 
  \end{itemize}
\end{frame}

\begin{frame}
  \frametitle{Orchestration and document development}
  Orchestration of various tools detecting need for execution e.g.
  of \texttt{bibtex} including \texttt{rerunfilecheck} for \texttt{lualatex} and friends.

  Supports document development, mainly by cooperating with editor, viewer and with other tools in the build chain.
  \begin{itemize}
    \item Offers installation script for extensions of VS Code.
    \item Offers configuration file \texttt{.chktexrc} for \texttt{chktex}.
    \item Can create configuration file \texttt{.latexmkrc} 
    for \texttt{latexmk} synchronized with the configuration.
    \item Offers a common header file \texttt{header.tex} 
    to unify packages loaded by latex main files.
  \end{itemize}
\end{frame}


\subsection{Planned Features}

\begin{frame}
  \frametitle{Planned Features}
  \begin{itemize}
  \item Support \texttt{biber} replacing \texttt{bibtex} as preferred tool
  \item Support \texttt{xindy} replacing \texttt{makeindex} as preferred tool
  \item Support \texttt{bib2gls} replacing \texttt{makeglossary} as preferred tool
  \item Execute \texttt{glosstex} if needed
  \item Usage of the \texttt{multibib} macros
  \item \dots
  \end{itemize}
\end{frame}

% \section{Installation}

\section{Goals}

\begin{frame}
  \frametitle{Goals}

  An overview is on the 
  \href{http://simuline.eu/LatexMavenPlugin/plugin-info.html}{goals page}. 
  
  \setbeamercolor{alerted text}{fg=blue}
  Besides goals referring to creating output of a given type 
  like \alert{\texttt{pdf}}, \alert{\texttt{odt}} and \alert{\texttt{html}}, 
  there is a goal \alert{\texttt{cfg}}, 
  which allows \setbeamercolor{alerted text}{fg=red}
  \alert con\alert fi\alert guring various output formats as \texttt{targets}. % chktex 1
  
  \setbeamercolor{alerted text}{fg=blue}
  One further goal, \alert{\texttt{chk}} to perform a check with \texttt{chktex} 
  is allowed among the \texttt{targets}. 

  The other goals cannot serve as targets: 
  %
  \begin{description}
    \item[\texttt{vrs}] to check versions of tools, 
    \item[\texttt{grp}] to create graphic files speeding up use of \texttt{latexmk} 
    and 
    \item[\texttt{inj}] to inject useful files like headers or config files 
    like \texttt{.latexmkrc}. 
  \end{description}
  

\end{frame}

% \section{Usage}

% \section{Examples}

\mode<presentation>
\section{References}% if with star not in toc 


\begin{frame}[allowframebreaks]
  \frametitle{References}
  \bibliographystyle{alpha}
  \bibliography{../lit}{}
\end{frame}

\mode<article>
\bibliographystyle{alpha}
\bibliography{../lit}{}

\end{document}
\end{Verbatim}
%
both including the same piece of code which is included from file \texttt{useBeamer.tex}. 
The author recommends to stick to this convention. 
As an example document may serve~\cite{PresBeamer} 
which is a presentation of this software including the handout 
and illustrates the use of the beamer class. 
Observe, that both documents use \texttt{header.tex} injected 
as described in Section~\ref{subsec:header} loading various packages. 
The beamer class is special in that it loads the \pkg{hyperref} package itself. 
To avoid option clash with \texttt{header.tex}, 
for document class \texttt{beamer} option \texttt{destlabel} 
must be passed to the package. 
% originally in \documentclass[hyperref=destlabel] 
Maybe it is a matter of taste, but \texttt{beamer} tends to make links invisible. 
To force loading options specifying colors for links and \texttt{destlabel}, 
use \cmd{PassOptionsToPackage} as shown above. 

All documents but beamer documents 
must specify the paper size globally via \cmd{documentclass}. 
Beamer documents may specify accordingly \texttt{aspectratio}. 
All this must be allowed for \LaTeX{} main files. 


\subsubsection{Magic comments}\label{subsubsec:openingMagComm}

It also makes sense to allow comments also in openings, 
i.e.\@ text from unescaped \texttt{\%} to the end of the line, 
and also magic comments. 
A magic comment, as all comments, is ignored by the \LaTeX{} compilers 
but give hints to more high level tools 
like IDEs or build tools like this \LaTeX{} builder. 
It is the mechanism to treat a document in a specific way 
so magic comments override the general settings. 

Typically, a magic comment comprises a whole line and starts with \texttt{\%~!}, 
maybe followed by an identifier of the tool it refers to 
or by an identifier referring to TEX files in general. 
For example latex workshop and \TeX{}shop support the magic comment \texttt{\%~!TEX root} 
and this must be essentially in the first line. 
The magic comments specific for this tool may be preceded by general magic comments 
and start with \texttt{\%~!LMP} which is short for ``latex maven plugin''. 
This is not fully correct but easy to remember. 

This \LaTeX{} builder is designed to cooperate with other tools. 
The magic comments of the other tools 
as described in various places in Section~\ref{sec:devel} 
on document development and in particular in Section~\ref{subsec:develLatexmk}. 
Thus, if appropriate, also magic comments of other tools are read, 
except those of \auctex, because \auctex{} places magic comments at the end of file, 
forcing this software to read all the file if it accessed \auctex{} magic comments also. 
All other tools including latex workshop for VS Code 
support a subset of what is 
\href{http://transit.iut2.upmf-grenoble.fr/doc/texstudio/html/usermanual_en.html#TEXCOM}%
{defined by \TeX{}studio}. 
From all magic comments in the context of signifying \LaTeX{} main files 
only \texttt{program} and \texttt{root} are relevant. 
If a root is given, then the file is no \LaTeX{} main file 
and, provided the feature is used, also the converse is true. 
Since this software shall not rely on further tooling, 
it does not use \texttt{root}. 
All in all, among the general magic comments only 
\texttt{\%~!TEX program=\dots} is read. 
It can occur more than once, but the first occurrence is what counts; 
the others are ignored silently. % TBD: really good? 
Note that the magic comment \texttt{\%~!TEX program=\dots} overrides 
the setting \texttt{latex2pdfCommand} for creating PDF files and related, 
specified in Table~\ref{tab:paramLatex2pdf} on page~\pageref{tab:paramLatex2pdf}, 
but not the \texttt{tex4htCommand} from Table~\ref{tab:paramLatex2Html} 
on page~\pageref{tab:paramLatex2Html}. 

After the general magic comments of the form \texttt{\%~!TEX \dots} 
come the ones specific for this \LaTeX{} builder. 
They take the form \texttt{\%~!LMP \dots}. 
Like the general magic comments, the specific ones are all optional, 
but in contrast, they come in a fixed order without repetition. 

What follows is a full range of magic comments: 
%
\begin{Verbatim}
% !TEX program=lualatex
% !LMP chkDiff
% !LMP latexmk
% !LMP targets=chk,pdf,html

\documentclass[a4paper]{article}
\end{Verbatim}


Section~\ref{subsec:patternLatexMainFile} 
describes the meaning of the individual comments 
in the course of explaining the pattern \texttt{patternLatexMainFile}. 
Note that there the names of the magic comments is given, whereas the above listing refers to the content, 
but it is easy to identify the according magic comments. 
The relation of the magic comments is described in the following. 

The magic comments may come only in the ordering given in the above listing, 
but each of them is optional. 
They can be freely combined, 
but note that \texttt{chkDiff} and \texttt{latexmkMagic} 
apply to creation and check PDF files only. 
So, for \texttt{targets=pdf,html}, these magic comments apply to target \texttt{pdf}, 
but not to \texttt{html}. 
For \texttt{targets=chk,html} it even takes no effect at all 
without issuing a warning. 
As explained above, \texttt{program} affects 
only the targets \texttt{pdf} and \texttt{dvi} including also XDV files. 


% Since the pattern \texttt{patternLatexMainFile} may never be appropriate for all situations 
% to match the document class in \cmd{documentclass} 
% and not all users are familiar with regular expressions to adapt by need, 
% there need to be a workaround to adapting the pattern. 
% One is to use \cmd{input} to inlude material which does not fit. 

% In future, it is conceivable to use also magic comments for this task. 
% In fact, any of our magic comments signifies already a \LaTeX{} main file. 
% The idea is instead of 
% The only problem is, that if the pattern 
% marking a \LaTeX{} main file by specifying its document class like in 
% \texttt{\%~!LMP class=\dots} is an easy workaround. 

% As will be explained in detail in Section~\ref{sec:outputFormats}, 
% the explicit document class is only used 
% to restrict the targets for which to compile a \LaTeX{} main file. 
% The targets can be specified for each \LaTeX{} main file individually 
% using \texttt{\%~!LMP targets=\dots}. 
% In fact specifying targets marks a TEX file as a \LaTeX{} main file also 
% making specification of the class as \texttt{\%~!LMP class=\dots} superfluous. 
% Nevertheless, \texttt{\%~!LMP class=\dots\ targets=\dots} is allowed. 



% In contrast, magic comments like \texttt{\%~!TeX program=\dots} 
% specifying the \LaTeX{} converter signify a \LaTeX{} main file in any case. 
% It overwrites also the converter 
% given in the general setting \texttt{latex2pdfCommand} 
% specified in Table~\ref{tab:paramLatex2pdf} on page~\pageref{tab:paramLatex2pdf}. 
% A magic comment specifying the converter does not specify, 
% neither the target directly nor via the document class. 


% The pattern \texttt{patternLatexMainFile} matches comments and in particular magic comments, 
% but it is written also to extract the parameters in magic comments described above 
% as it extracts the document class from the \cmd{documentclass} command. 
% In fact more than that: 
% It matches if the magic comment indicates 
% that the file under consideration is a \LaTeX{} main file 
% or if the presence of a \cmd{documentclass} command signifies this. 

% If the magic comment of this tool specifies the document class or the target set, 
% it is considered a \LaTeX{} main file, even if no \texttt{documentclass} is matched. 
% In fact, it does not make sense to specify both in the magic comment, 
% the document class and the target set, 
% because the document class is used only to determine the target set. 
% Typically, the document class is given as a magic comment, 
% if it is not easy to get the pattern for the specific document right, 
% maybe because the user is not so familiar with the technique, so it is a fallback. 
% In contrast, the target set is given in the magic comment 
% if it shall be specified individually for the given \LaTeX{} main file. 


Note that documents of the classes \texttt{beamer}, \texttt{leaflet} and \texttt{scrlttr2} 
can essentially only be compiled into a PDF, 
and maybe further to a TXT file. 
In addition, to targets and goals \texttt{pdf} and \texttt{txt}, 
it can be checked with target or goal \texttt{chk}. 
Other targets are skipped, and a message is displayed. 
The relations are configurable 
through settings \texttt{targets} and \texttt{docClassesToTargets} 
both in Table~\ref{tab:paramGen} on page~\pageref{tab:paramGen}. 
Also, if a document class occurs, 
which is not registered in \texttt{docClassesToTargets}, 
a warning \texttt{WLP09} described in Table~\ref{tab:WarnLPP} 
on page~\pageref{tab:WarnLPP} is displayed. 
% TBD: clarify whether this is not an exception 






\subsection{Source graphic files}\label{subsec:sourcesGraphic}

The great bulk of file types occurring as sources, 
are graphic files in various formats. 
Note that this section is not about intermediate file types like PDF or MPS 
used to include the original file types into the target. 

As regards the way the according files are included in \LaTeX-files, 
there are the following kinds of graphic formats, 
all included in the TEX source directory. 
%
\begin{enumerate}
\item
The first can be included into \LaTeX-files directly via \cmd{input}. 
These formats are essentially \LaTeX{}
and are defined in an according package. 
Examples are \pkg{eepic} described in~\cite{EEpic}
and above all \pkg{tikz} described in~\cite{TikzPGF23}. 
\item
The second one via the command \cmd{includegraphics}{} 
defined by the package \pkg{graphicx} 
which is described in~\cite{GraX}. 
Chapter 2 therein mentions the supported drivers, 
among these are also \texttt{dvipdfm} and \texttt{dvipdfmx}, the latter is the default. 
It is not the package but the driver 
which decides on the support of graphic formats. 
The dvipdfmx user manual,~\cite{DviPdfMx}, Section 3.1.1 lists the allowed formats 
MetaPost (i.e.~\gls{mps}), postscript (i.e.~\gls{eps}), 
\gls{pdf}, \gls{jpg} including jpeg2000 and \gls{png}. 
\item\label{it:transExp}
The third one must be transformed into a graphics format 
of one of the former two kinds using an external tool for transformation. 
Here, of course, only a limited support is possible, 
because there is a broad variety of formats. 
We have chosen
%
\begin{itemize}
\item
  the \gls{fig}-format described in~\cite{XFigF}
  because of its simplicity, 
\item
  the gnuplot format, described in~\cite{GnuPlot6_0}, 
  because it allows computation of function plots, 
\item
  scalable vector graphics \gls{svg}-format specified in~\cite{Svg11}\footnote%
  { As the specification is hard to digest,
  we refer to the tutorial~\cite{SvgTut}. } 
  as it is important for construction and the counterpart of pixel oriented
  formats,
\item
  likewise, metapost (\gls{mp}-format),
  described in~\cite{MPost24} because it is native to \LaTeX{} 
  and quite versatile 
\end{itemize}
\item\label{it:transImp}
The fourth kind of graphics formats 
has to be transformed into one of the kinds one or two 
but unlike in type three, this is not done explicitly 
by an external tool but by a latex-package during the \LaTeX-run. 
Note that, although not required to be explicitly transformed, 
those graphics files induce additional files 
by running \LaTeX.
Essentially, each of the abovementioned type of format
can be included that way but currently,
this is done for the \gls{svg}-format only
included by the package \pkg{svg} (see~\cite{SvgP}).
The author personally refrains from using packages like that
because of the lack of flexibility and further drawbacks. 
\item 
Finally, there is a way to include graphics which is not really a graphic format: 
In the course of running code, e.g.~by package \pkg{pythontex} in Python, 
as described in Section~\ref{sec:pythontex}, 
it is also possible to create computed graphics. 
It may be advisable to separate code into special files to be included via \cmd{input}, 
but it is not strictly required. 
In the long run it seems a good idea, to extend \pkg{pythontex} 
to read in code files, e.g.~in python directly. 
\end{enumerate}

% The \LaTeX-files and the graphic files belonging to a \LaTeX{} main file 
% are assumed to be in one single folder. 
% If one file is included by two different main files, 
% a link shall be used.
%TBC: True???



\subsection{Created files in the \TeX{} source directory}\label{subsec:sourceCreated}

Note that against maven convention and unlike former versions of this software, 
the current version does not create a working directory 
by cloning the TEX source directory. 
Instead, it operates directly on the TEX source directory 
also creating intermediate files, 
deleting them again by default after the build process.
The advantage of processing that way is,
that this allows cooperation between this software
and other tool chains which are better suited for developing latex files.
Details are described in Section~\ref{sec:devel} an in particular in Section~\ref{subsec:develLatexmk}.

The downside is that a file residing in the TEX source directory 
risks being overwritten or deleted by this software, 
if it does not stick to the rules. 
The rules are simple: 
%
\begin{itemize}
\item
For each graphic file being transformed, 
i.e.~of types~\ref{it:transExp} or~\ref{it:transImp} above, 
additional files are created with the same name up to the suffix. 
Thus, for these graphic files no file with the same name 
up to the ending is allowed. 
\item
For \LaTeX{} main files more general files are created, 
but they all must match those in pattern \texttt{patternCreatedFromLatexMain} 
described in Table~\ref{tab:paramGen} on page~\pageref{tab:paramGen}. 
So it is save to add files not matching this pattern. 
\end{itemize}


Note Section~\ref{sec:injFiles}  which is on goal \texttt{inj} 
injecting diles in the \TeX{} source directory and 
Section~\ref{subsec:develGraph} on goal \texttt{grp} 
processing graphic files which creates intermediate files therein also. 

To get rid of intermediate files, there is a separate goal \texttt{clr} 
described in Section~\ref{subsec:develClear}. 




\section{Exporting in various formats and checking sources}\label{sec:outputFormats}

% TBD: magic comment to specify converter name and its options. 


After having added the configuration of the plugin to the \texttt{pom.xml},
minimally the one given in Listing~\ref{lst:coords},
it can be used directly invoking maven through 
\texttt{mvn latex:cfg}. 
Here \texttt{latex} is the (short) name of the plugin 
and \texttt{cfg} is the goal. 
It can also be interpreted as \texttt{mvn $<$source$>$:$<$targets$>$}: 
The source files are in \texttt{latex}-format and the \texttt{targets} 
are read from the \emph{configuration} in the pom 
(\emph{configuration} is what \texttt{cfg} stands for) 
which is illustrated in Listing~\ref{lst:coordsConfig}. 
For a detailed description of the setting \texttt{targets} 
see Table~\ref{tab:paramGen} on page~\pageref{tab:paramGen}. 
Here only an overview is given. 
% TBD: clarify what about the ant task 

By default, the targets configured are \texttt{chk}, \texttt{pdf} and \texttt{html}. 
The following Listing~\ref{lst:targetsAll} shows a configuration 
with the full range of output formats including in addition 
the OpenOffice document format \texttt{odt}, 
the MS word-formats \texttt{doc(x)} and \texttt{rtf} % chktex 36
and also plain text format \texttt{txt} in utf8 encoding. 

Note that the target \texttt{docx} converts by default into \gls{docx} 
but may also be configured to produce the old-fashioned \gls{doc} format. 

Be aware that the target \texttt{dvi} creates output in DVI format 
only for \LaTeX{} engines \lualatex{} and \pdflatex{}, 
whereas \xelatex{} creates the XDV (extended DVI) format for target \texttt{dvi}. 

%\lstset{language=xml, basicstyle=\small}
\begin{lstlisting}[language=xml, basicstyle=\small,
escapechar=|,
float, captionpos=b, label={lst:targetsAll}, 
caption={Configuration with full range output formats}]
<!-- create html and pdf and other formats from latex -->
<plugin>
  <groupId>|\groupId|/groupId>
  <artifactId>|\artifactId|</artifactId>
  <version>|\strippedVersionID|</version>
	
  <configuration>
    <settings>
      <targets>chk,pdf,dvi,html,odt,docx,rtf,txt</targets>
    </settings>
  </configuration>
</plugin>
\end{lstlisting}

Somehow special is the target \texttt{chk} 
which is mere checking by invoking \texttt{chktex} 
without resulting output file. 
It just displays a warning if a rule is violated. 

The resulting files in the given output formats 
are copied to the site directory, 
which is \texttt{./target/site} in a default maven project. 

Sometimes it is more convenient 
to specify the output formats not via the pom 
but directly as a goal on the command line. 
In particular, one may write \texttt{mvn latex:pdf} to create documentation 
in PDF-format only.
Likewise, command \texttt{mvn latex:dvi} to get good old dvi/xdv files
or even \texttt{mvn latex:txt} for plain text, just as examples. 
Accordingly, \texttt{mvn latex:chk} performs a pure check. 
This occurs preferably in the context of documentation development. 
In particular, checking is treated separately in Section~\ref{subsec:develCheck}. 

Note that the \texttt{-X} switch activates debugging 
which results in a more verbose output. 
Example: \texttt{mvn -X latex:cfg}. 

Although the possible targets can be configured globally 
via the setting \texttt{targets}, 
the possible targets may depend on the document class of the \LaTeX{} main file. 
At time of this writing, 
all document classes in preferred usage as defined in Chapter~\ref{chap:tests} 
support all targets with obvious exceptions: 
Besides checking (target \texttt{chk}) for obvious reasons 
the classes \texttt{beamer}, \texttt{leaflet} and the letter class \texttt{scrlttr2} 
directly support only target \texttt{pdf} 
and because texts are created from PDF files, also target \texttt{txt}. 
The mapping from document classes to allowed targets 
is given in setting \texttt{docClassesToTargets} given in 
Table~\ref{tab:paramGen} on page~\pageref{tab:paramGen}. 
This parameter restricts the targets given in parameter \texttt{targets}. 
As explained in detail in Section~\ref{subsubsec:openingMagComm}, 
if a document class cannot be identified by the command \texttt{documentclass} 
or the outdated \texttt{documentstyle}, 
it can be specified by a magic comment directly. 

Finally, the targets can be specified individually for each \LaTeX{} main file 
using a magic comment as described in Section~\ref{subsubsec:openingMagComm}. 
A target specification in a magic comment overwrites all settings in 
\texttt{targets} and in \texttt{docClassesToTargets}. 
If a magic comment specifies the targets directly, 
the document class need not be known. 
In particular, a magic comment only specifying targets identifies already a \LaTeX{} main file 
as specified in Section~\ref{subsubsec:openingMagComm}. 

As a magic comment can be used 
to specify the target formats for a \LaTeX{} main file individually, 
Section~\ref{subsubsec:openingMagComm} shows 
how to specify the \LaTeX{} engine to be used for this file 
overwriting the general setting \texttt{latex2pdfCommand} in the pom 
given in Table~\ref{tab:paramLatex2pdf} on page~\pageref{tab:paramLatex2pdf}. 

% The code shows that allowed is only what is defined in enum Target. 
% Thus chk, which is a goal, is not output format. 
% This makes sense because Target specifies the files to be copied to the target filder. 

% in a sense this section mixes up output formats and goals. 
% For goal latex:cfg the output formats are as listed under config target. 
% Other goals, like latex:pdf specify a single output format. 

% Still other goals have no output format. These are 
% - latex:chk, although a check file is created. 
%   TBD: currently, there is no config to make chk run on \LaTeX{} main files. 
% - latex:grp, although graphic files are created but they are not exported. 
%   This may be done if source distributions are defined. 
% - latex:vrs, creating a version output or emitting warning if a version does not fit. 
% - latex:clr, for cleanup. 

% TBD: latex:grp does not fit: it is not an output format. 
% maybe it becomes output format if introducing source distributions 
% 
% Same is true for target clr. 



In a standard maven project, 
the above minimal configuration should be sufficient. 
Only if the folder structure deviates from the standard 
or if the \LaTeX{} sources require special configuration, 
parameters have to be given explicitly, 
because they deviate from the default values. 
Chapter~\ref{chap:settings} summarizes all available parameters, 
giving the default value and a description. 


For sake of uniformity, 
the name of the ant-task is \texttt{latex:cfg}, 
and it can be invoked via \texttt{ant latex:cfg}. 
Unlike the maven-plugin, the ant-task 
does not allow to specify a target on the command line. 
The \texttt{-d} switch activates debugging 
which results in a more verbose output. 
Example: \texttt{ant -d latex:cfg}. 

Whereas by default the target directory and in particular 
the target site directory with all output of this plugin is deleted 
in maven's \texttt{clean} life-cycle, 
the tools invoked by this software also create intermediate files 
in the source directory. 
By default, i.e.\@ for setting \texttt{<cleanUp>true</cleanUp>}, 
all files created in the source directory in the last run are cleaned. 
Nevertheless, for document development intermediate files are vital 
and so cleanup is frequently set to false. 
In this case, cleanup must be done in a separate goal, 
described in Section~\ref{subsec:develClear}. 

% By default, the goal \texttt{clr} 
% is also executed in maven's \texttt{clear} life-cycle. 

% There is an according ant-task \texttt{latex:cfg} 
% which can be invoked via \texttt{ant -d latex:cfg}. 
% FIXME\@: ant  \texttt{latex:clr} has duplicate parameters. 
% This can be fixed only by properties. 
% Another problem is, to provide a complete subset of parameters 
% which apply to \texttt{latex:cfg} and to \texttt{latex:cfg}, respectively. 

\section{Checking versions of converters}\label{sec:chkVersions}

The goal \texttt{vrs} is to display meta information, above all version information:
% 
\begin{Verbatim}
mvn latex:vrs
\end{Verbatim}
%
displays something like what is displayed in Listing~\ref{lst:vrsOut}. 
Besides information on this software including version and even git commits, 
there are information on so-called registered converters, 
i.e.\@ converters intended to be invoked by this software. 

The goal yields a full list of registered converters, 
signifying which of them are excluded 
according to parameter \texttt{convertersExcluded}, 
which are not installed, 
and for each of the rest, the actual version, the allowed range 
and a warning if the actual version is out of range. 

The parameter \texttt{convertersExcluded} 
is described in Table~\ref{tab:paramGen} on page~\pageref{tab:paramGen}. 
Excluded converters are prevented from being used: 
if tried, Exception TSS07 
described in Table~\ref{tab:TSS} on page~\pageref{tab:TSS} is thrown. 
If a converter is not installed, but tired to be used, 
this kind of failure is obvious. 
Only if a converter is used with an unintended version bears some risk. 
Note that also unregistered converters can be used; 
but then the user is responsible to provide an appropriate version. 
An example for an unregistered converter 
is given in Table~\ref{tab:paramPythontex} on page~\pageref{tab:paramPythontex}: 
\texttt{pythontexW:pythontex} 
indicating the converter \texttt{pythontexW} with category \texttt{pythontex}. 


As one can see, a warning WMI02 indicates 
that the version of a converter is out of the intended range, 
provided, the converter is installed, and it is not excluded 
according to the configuration \texttt{convertersExcluded}.

Note that in the given version and in the installation of the author,
of course, all converters are installed and are up-to-date
to be able to check validity.
The according messages are forced for illustration only. 
For a user of this software which does no development, 
of course only converters need to be installed which are really needed. 
% TBC: what is an interface of a converter?
% Note that \texttt{makeindex} is not in the list.
% This is because it is not possible
% to regularly check the version of that application.
% It may be an option to use \texttt{upmendex} instead,
% although in beta state at time of this writing
% and not completely compatible.
% According to \cite{UpMendex}, \cite{MkIdxMoe}
% and further research,
% \texttt{upmendex} does not support options \texttt{-T} and \texttt{-L}
% and does interprete \texttt{-g} differently,
% namely as japanese instead of german. 


% TBD: make this listing dynamic 
% TBD: this is almost duplicate of lst:version.properites
% maybe one shall be eliminated. 
\begin{lstlisting}[basicstyle=\tiny,
float, captionpos=b, label={lst:vrsOut}, 
caption={Output of goal \texttt{latex:vrs}}]
[INFO] --- latex:2.0-SNAPSHOT:vrs (default-cli) @ latex-maven-plugin ---
[INFO] Manifest properties: 
[INFO] MANIFEST: (1.0)
[INFO]        Implementation-Version: '2.0-SNAPSHOT'
[INFO] PackageImplementation-Version: '2.0-SNAPSHOT'
[INFO] pom properties:
[INFO] coordinate.groupId:    'eu.simuline.m2latex'
[INFO] coordinate.artifactId: 'latex-maven-plugin'
[INFO] coordinate.version:    '2.0-SNAPSHOT'
[INFO] git properties: 
[INFO] build version:  '2.0-SNAPSHOT'
[INFO] commit id desc: 'latex-maven-plugin-1.8-209-g5ac27b7-dirty'
[INFO] buildTime:      '2023-06-25T23:31:20+0200'
[INFO] tool versions: 
[INFO] ?warn?    command             'actual version'(not)in[expected version interval]
[INFO]           mvn:                '3.9.4'in[3.9.1;3.9.4]
[INFO]           ant:                '1.10.14'in[1.10.12;1.10.14]
[INFO]           java:               '17.0.9'in[17.0.9]
[INFO]           python:             '3.11.6'in[3.11.6]
[INFO]           perl:               '5.38.2'in[5.38.2]
[INFO]           pdflatex:           '1.40.25'in[1.40.21;1.40.25]
[INFO]           lualatex:           '1.17.0'in[1.12.0;1.17.0]
[INFO]           xelatex:            '0.999995'in[0.999992;0.999995]
[INFO]           latex2rtf:          '2.3.18 r1267'in[2.3.16 r1254;2.3.18 r1267]
[INFO]           odt2doc:            '0.9.0'in[0.9.0]
[INFO]           pdftotext:          '23.11.0'in[21.04.0;23.11.0]
[INFO]           dvips:              '2023.1'in[2020.1;2023.1]
[INFO]           dvipdfm:            '20220710'in[20210318;20220710]
[INFO]           dvipdfmx:           '20220710'in[20200315;20220710]
[INFO]           xdvipdfmx:          '20220710'in[20200315;20220710]
[INFO]           dvipdft:            '20090604.0046'in[20090604.0046]
[INFO]           gs:                 '9.56.1'in[9.52.0;9.56.1]
[INFO]           chktex:             '1.7.8'in[1.7.8]
[INFO]           diff-pdf-visually:  '1.7.0'in[1.6.4;1.7.0]
[INFO]           diff-pdf:           '300'in[300]
[INFO]           diff:               '3.10'in[3.8;3.10]
[INFO]           pdfinfo:            '23.11.0'in[22.01.0;23.11.0]
[INFO]           exiftool:           '12.71'in[12.39;12.71]
[INFO]           bibtex:             '0.99d'in[0.99d]
[INFO]           bibtexu:            '4.00'in[4.00;4.00]
[INFO]           bibtex8:            '4.00'in[4.00;4.00]
[WARNING] WMI02: makeindex:          '2.17'not in[2.15;2.16]
[INFO]           splitindex:         '0.1'in[0.1]
[INFO]           makeglossaries:     '4.51'in[4.45;4.51]
[INFO]           pythontex:          '0.18'in[0.17;0.18]
[INFO]           depythontex:        '0.18'in[0.17;0.18]
[INFO]           mpost:              '2.02'in[2.00;2.02]
[INFO]           ebb:                '20220710'in[20200315;20220710]
[INFO]           gnuplot:            '5.4 patchlevel 10'in[5.4 patchlevel 0;5.4 patchlevel 10]
[INFO]           inkscape:           '1.3.2'in[1.0.2;1.3.2]
[INFO]           fig2dev:            '3.2.9'in[3.2.7b;3.2.9]
[INFO] tools excluded: 
[INFO] upmendex, xindy
[INFO] tools not found: 
[INFO] latexmk
[INFO] ------------------------------------------------------------------------
\end{lstlisting}
%[INFO]           upmendex:           '1.00'in[0.54;1.00]
%[INFO]           xindy:              '2.5.1'in[2.5.1]


\section{Logging of errors and warnings}\label{sec:chkErrWarn}

It is a design goal to notify the user of formal deficiencies of the created documents 
by tight logging of errors and warnings 
to give high confidence in the formal quality of the result in the absence of errors and warnings. 
Logging is devoted all of Chapter~\ref{chap:exceptionLogging}. 
Of particular intestest is Table~\ref{tab:WarnALP} on failures running a tool 
on page~\pageref{tab:WarnALP}. 
We highlight 
%
\begin{itemize}
  \item
  Check of return codes 
  \item
  absent/outdated target files or log files
  \item
  log files containing errors and warnings (provided by many but not all tools). 
  Among those notification of bad boxes. 
\end{itemize}


\section{Injection of files}\label{sec:injFiles}

The goal \texttt{inj} is to inject files 
into the working directory \texttt{texSrcDirectory}, 
by default in maven lifecycle phase \texttt{valiate} 
or from command line in the root directory. 
The injected files are in general adapted to the current configuration of the plugin. 

Note that each of these files is written only 
if it is guaranteed that only files written by this plugin are overwritten. 
This is the case, if no file is overwritten at all 
or if the file to be overwritten is recognized to start with a comment 
indicating that this file is written by this plugin. 
Of course the guarantee holds only if the headline does not tell a lie. 

If the headline cannot be read or in some other exotic conditions, 
it cannot be ensured that the files are written by this software, 
and so they are not overwritten by goal \texttt{inj} 
and by the way not erased by goal 
\texttt{clr} as described in Section~\ref{subsec:develClear}. 
In case of such a doubt, a warning is displayed. 

That way, injected files written by the plugin can be updated each run, 
which is necessary to keep them synchronized with the configuration of this plugin, 
but according files written e.g.\@ by the user are protected. 

A first description of the goal \texttt{inj} is given by 
%
\begin{verbatim}
mvn latex:help -Ddetail -Dgoal=inj
\end{verbatim}
%
which yields a list of files which can be injected. 
Note the distinction between the injection, which is the act of injecting 
and the according file which is injected. 

The set of injections can be divided into the following categories 
according to the function of the files injected: 
%
\begin{itemize}
  \item
  Configuration files for \tool{latexmk} and \texttt{chktex}. 
  These are hidden files and form the default. 
  In particular, the configuration file of \tool{latexmk} 
  is adapted to the configuration of this plugin, 
  to ensure that the results are the same whether created by \tool{latexmk} 
  or by this plugin. 
  \item
  Header files are intended to be included in TEX files. 
  They load packages and provide commands. 
  In general, header files are designed to run on all usual \LaTeX{} compilers, 
  with various document classes  
  and take creation of PDF into account but also of other formats like HTML 
  and also of DVI/XDV which is an important intermediate format. 

  The packages are loaded with minimum options, 
  but these can be modified outside the headers by \cmd{PassOptionsToPackage} 
  as described in Section~\ref{subsubsec:openingNoMagComm}. 
  \item
  Script files which are intended to run by the user 
  supporting the automatic build process ``from outside''
  above all in the course of document development. 
  Thus, usually, their injection is triggered selectively from the command line 
  as described below. 
  In contrast to the files in the other categories, these are executable. 
\end{itemize}

Table~\ref{tab:injections} shows the possible injections 
and the ones really to be performed are given in the configuration \texttt{injections}. 
This configuration is described in Section~\ref{sec:settingsGoalVrsInj} 
on page~\pageref{sec:settingsGoalVrsInj}. 
It is a comma separated list and the default is \texttt{latexmkrc,chktexrc}, 
representing the configuration files. 


\begin{longtable}{l|ll}
  \toprule
  Name & File & explanation \\
  \midrule
  \midrule
  \endfirsthead%
  \bottomrule
  \caption{\label{tab:injections} Overview over all injections }
  \endlastfoot%
  \multicolumn{2}{l}{configuration files} \\
  \midrule
  latexmkrc             & \texttt{\footnotesize .latexmkrc}        & config file for \tool{latexmk} \\% chktex 26
  chktexrc              & \texttt{\footnotesize .chktexrc}         & config file for \texttt{chktex}  \\% chktex 26
  \midrule
  \multicolumn{2}{l}{header files} \\
  \midrule
  header                & \texttt{\footnotesize header.tex}        & fundamental     \\
  headerGrp             & \texttt{\footnotesize headerGrp.tex}     & for graphics      \\
 {\tiny headerSuppressMetaPDF} & \texttt{\tiny headerSuppressMetaPDF.tex} & to control PDF meta-info      \\
  \midrule
  \multicolumn{2}{l}{shell scripts} \\
  \midrule
  vscodeExt             & \texttt{\footnotesize instVScode4tex.sh} & installs VS Code extensions \\
  ntlatex               & \texttt{\footnotesize ntlatex}           & timeless \LaTeX{} compiler \\
  vmdiff                & \texttt{\footnotesize vmdiff}            & special diff tool for PDF files \\
  pythontexW            & \texttt{\footnotesize pythontexW}        & surrogate for \texttt{pythontex} \\
  depythontexW          & \texttt{\footnotesize depythontexW}      & surrogate for \texttt{depythontex} \\
  \end{longtable}


As described in Section~\ref{subsec:xmlPomExecutions}, 
by default the goal \texttt{inj} is tied to the maven phase \texttt{validate}, 
an early phase preparing the proper build process, 
because the injected files are a prerequisite for building. 
Then the files are injected in the TEX root directory \texttt{texSrcDirectory}. 

On the other hand, injections can be also invoked by command line via \texttt{mvn latex:inj} 
with the default injections or, with given list of injections, 
e.g. 
%
\begin{Verbatim}
  mvn latex:inj -Dlatex.injections=vscodeExt,ntlatex,vmdiff
\end{Verbatim}
%
In fact, injection from command line is typically used for scripts, 
whereas the others files are injected during the build process in phase \texttt{validate}. 
Of course maven is invoked from the project root 
and there also the prescribed files are injected. 

Note that the folder where cleanup of injections with \texttt{mvn clean} is done, 
depends also on whether \texttt{-Dlatex.injections=\dots} is specified, 
but the value is irrelevant as long as it is valid. 

In the sequel, all these injections are described in detail separately, 
but in fact they are all related. 
For example, \texttt{header.tex} handles the possible configurations 
reflected in \texttt{.latexmkrc}. 
It provides packages used in \texttt{headerGrp.tex} 
and provides commands to exclude checking by \texttt{chktex} 
controlled by \texttt{.chktexrc}. 
The header \texttt{header.tex} is very crucial 
for example controlling and guaranteeing rerun of the \LaTeX{} engine 
by including package \pkg{rerunfilecheck}. 
Its presence makes the results uniform and is a cornerstone for quality guarantees. 
As said above, the default for injections is \texttt{latexmkrc,chktexrc}, 
but it is advisable to use \texttt{latexmkrc,chktexrc,header}. 





Now let us treat the injections individually. 

% TBD: for injections allowing filtering, 
% one must think carefully, what is wanted to be shown in this documentation: 
% the filtered files are mere examples, whereas the raw ones are more informative 
% but more difficult to understand. 
% Many of the links of this manual do not point on the optimal kind of injection file. 
% linking fromTex means filtered. This is appropriate if a file is not filtered at all. 
% for headerSuppressMetPdf.tex, and for .chktexrc this may change. 
% For .latexmkrc it is inappropriate to link to fromTex only. 


\subsection{The configuration files \texttt{.latexmkrc} and \texttt{.chktexrc}}\label{subsec:latChkRc}

For document development the tool \tool{latexmk} is a valuable build tool. 
Also, a linter like \texttt{chktex} is helpful 
both for end control and for document development. 

The file \href{\urlSite fromTex/.latexmkrc}{.latexmkrc} tied to the injection \texttt{latexmkrc} 
is the configuration file for the build tool \tool{latexmk} 
and likewise \href{\urlSite fromTex/.chktexrc}{.chktexrc} tied to the injection \texttt{chktexrc} 
is the configuration file for the style check tool \texttt{chktex}. 
The configuration files determine the behavior of the two tools without further options. 
The user is kindly asked to help to improve these files, in particular \texttt{.chktexrc}. 

Ideally, the injected \texttt{.latexmkrc} is adapted to the current settings of this plugin 
and so invoking \tool{latexmk} invoked with its configuration file 
behaves like this latex plugin. 
Currently, not all possible settings of this plugin 
are taken into account in the \texttt{.latexmkrc}, 
but the magic comments in the source files described in Section~\ref{subsubsec:openingMagComm} 
are read and taken into account as far as this makes sense. 

For default settings, maybe partially overwritten by magic comments, 
this maven plugin and \tool{latexmk} create the same target files. 
This is true even for this manual. 
In particular, the graphic formats 
described in Chapter~\ref{chap:GraphConversions} are supported. 
So are bibliographies, indices and glossaries and also material computed by \tool{pythontex} 
as described in Chapter~\ref{chap:latexMainConversions}, 
but without reflecting all options and patters to supervise log files. 
Also, reproducibility check is supported including magic comments and all parameters. 
% TBD: what about chk? 
% what about dvi/xdv? read the manual of latexmk

At time of this writing, \texttt{.latexmkrc} works for various \LaTeX{} generators 
but supports target \texttt{pdf} only, 
although in the long run also \texttt{chk} and \texttt{dvi} could be useful. 
Still, creation of PDF files is supported in both variants, via DVI/XDV or directly. 
Compilation via \tool{latexmk} is based on code in \texttt{.latexmkrc} 
and this mimics a wide range of functionality offered by this tool. 
Among these also build for reproducibility checks as described in Section~\ref{sec:chkReprod}

A sensible config file \texttt{.chktexrc} mainly depends on the packages loaded. 
In Section~\ref{subsec:header} we suggest injecting also a header file \texttt{header.tex} 
loading packages. 
The config file \texttt{.chktexrc} is adapted to the header file \texttt{header.tex}. 

Observe, that due to an incompatibility between tool \tool{latexmk} and package \pkg{listings}, 
this manual can only be compiled with \tool{latexmk} 
if \pkg{listings} is not only loaded but also patched 
as done by \texttt{header.tex} and described in Section~\ref{subsec:header}. 
With that patch, this software yields the same resulting PDF file 
as compilation with \tool{latexmk}. 

Currently, \texttt{.chktexrc} serves only to suppress warnings, 
mainly on material which is the argument of commands or the content of an environment. 
What is really needed depends on the packages loaded 
and on the commands and environments defined in addition. 

Since basic packages are loaded and basic commands are defined 
in the injected \texttt{header.tex} 
which is described in Section~\ref{subsec:header}, 
it makes sense, to synchronize \texttt{.chktex} and \texttt{header.tex}. 
As an example, \texttt{header.tex} loads package \pkg{listings} 
which provides the environment \texttt{lstlisting} and the command \cmd{listinputlisting}. 
The content of both shall not be subject to checks via \texttt{chktex} 
and must thus be excluded in \texttt{.chktexrc}. 


All this together illustrates why it is recommended to inject 
besides the default \texttt{.chktexrc} and \texttt{.latexmkrc} 
also \texttt{header.tex}. 
\medskip


As described in Sections~\ref{subsec:develLatexmk} and~\ref{subsec:develCheck}, 
both tools \texttt{chktex} and \tool{latexmk} 
are invoked directly by the user in the course of document development, 
but they may be invoked by this \LaTeX{} builder in the course of a regular build, 
i.e.\@ for maven goal \texttt{cfg} also. 
% TBD: and ant task with the same name ... well, this will not work 
% maybe when the ant task is adapted accordingly, including injections. 
So their respective configuration files must be injected 
in the maven build process before the \LaTeX{} build tools are invoked, 
i.e.\@ prior to the phase \texttt{site}. 
Thus, goal \texttt{inj} has default phase \texttt{validate}. 


As described in~\cite{LatexMk23}, 
Section~``CONFIGURATION/INITIALIZATION (RC) FILES'', 
there are various configuration files \texttt{latexmkrc} or \texttt{.latexmkrc}, 
among these a global one, a local one referring to the enclosing folder, 
and finally one specified by the command line option \texttt{-r} 
which is described in~\cite{LatexMk23}, 
Section~``LATEXMK OPTIONS AND ARGUMENTS ON COMMAND LINE''. 

Likewise,~\cite{ChkTeX22}, Section 6.1.3, shows that also \texttt{chktexrc} 
has a global configuration file \texttt{chktexrc} 
and a local one \texttt{.chktexrc} or \texttt{chktexrc}, 
depending on the operating system. 
Finally, a configuration file can be specified with the option \texttt{-l}, 
according to~\cite{ChkTeX22}, Section 6.1.1. 
Unfortunately,~\cite{ChkTeX22} does not tell about the ordering 
in which the configuration file given by the option \texttt{-l} is read in. 

For sake of reproducibility, we recommend restricting to the global configuration file 
which is tied to the installation and to a local file, specific to the latex source directory 
which shall be valid for all \LaTeX{} main files in that directory. 

Goal \texttt{inj} injects the configuration files \texttt{.latexmkrc}, \texttt{.chktexrc} 
and further files, 
all in the latex source directory. 
It is natural to use each as the local configuration file. 

Caution: According to~\cite{ChkTeX22}, Section 6.1.3, 
as described, the local configuration file fits only for UNIX-like operating systems. 
For Windows and that like, \texttt{chktexrc} is expected instead of \texttt{.chktexrc}. 
Uniformity with respect to the operating systems can be realized 
with a link to \texttt{chktexrc} named \texttt{.chktexrc}. 
That way independent of the operating system, 
the configuration files \texttt{.latexmkrc} and \texttt{.chktexrc} are sufficient. 

It is important that there is a unique central configuration file 
applying to all \LaTeX{} main files. 
There is the choice between at least two mechanisms to ensure this: 
Either \tool{latexmk} and \texttt{chktex} 
are invoked with options \texttt{-r} and \texttt{-l}, respectively, 
specifying the configuration file explicitly 
or for each folder containing a \LaTeX{} main file 
there must be a link named \texttt{.latexmkrc} and \texttt{.chktexrc}, respectively, 
to the according central configuration file. 

We recommend using links because then \tool{latexmk} and \texttt{chktex} 
can be used on the command line without further options. 
This is convenient for the user when invoking the tools directly 
which is the typical usage for document development. 


\subsection{A generic header file \texttt{header.tex}}\label{subsec:header}

It is observed that the headers of various \LaTeX{} files are quite similar. 
In particular the packages loaded 
have a huge overlap and at the same time, although rare, 
exotic packages tend to be loaded which may be replaced by standard ones. 
This hurts single source principle 
and at the same times makes it almost impossible 
for a build tool as this one, 
to make guarantees that it works still with the unexpected packages. 
This is, e.g.\@ because a package may write warnings 
in an unexpected format into some log file. 

The injection \texttt{header} 
is tied to the file \href{\urlSite fromTex/header.tex}{header.tex}, 
which is intended to be included in each \LaTeX{} main file. 
Essentially it includes packages always needed. 
% and packages which are patched. 
It is inspired by the packages \texttt{pandoc} includes by default 
according to \url{https://pandoc.org/MANUAL.html#creating-a-pdf}. 

Some loaded packages are also patched. 
The patch for package \pkg{listings} is given in Listing~\ref{lst:fileIfExists}. 
It applies only if \pkg{listings} is loaded prior input of \texttt{header.tex}. 
One modification is, redefinition of \cmd{lstlistoflistings} 
to make the list of listings occur in the table of contents 
and to rename the title so that it fits other lists as the list of figures.  
The other point is modifying \pkg{listings}' output to make it digestible for \tool{latexmk}. 
For details see Section~\ref{subsec:develLatexmk}. 

\lstinputlisting[linerange={119-156},
language=tex, basicstyle=\scriptsize,
float, captionpos=b, label={lst:fileIfExists}, 
caption={A patch of the \texttt{listings} package }]%
{./header.tex}% listing 


The other package patched is \pkg{luamplib}; the patch is given in Listing~\ref{lst:luamplib}. 
It applies only if \pkg{luamplib} is loaded prior input of \texttt{header.tex}. 
As discussed in Section~\ref{sec:metapost}, 
\pkg{luamplib} is available for \lualatex{} only. 
It provides an environment \texttt{mplibcode} to enclose literal MetaPost. 
The enhancement is an additional command \cmd{inputmpcode} 
which allows including MetaPost files. 
This functionality is analogous to package \pkg{listings} 
which allows both literal listings and loading listings from files. 
MetaPost code within a \LaTeX{} document typically disturbs syntax highlighting 
of both, enclosing code and included code. 

%
\lstinputlisting[linerange={103-105},
language=tex, basicstyle=\scriptsize, 
float, captionpos=b, label={lst:luamplib},
caption={A patch of the \texttt{luamplib} package }]
{./header.tex}% listing 

Besides loading packages it also sets \texttt{synctex} 
which is crucial for synchronizing TEX files and according PDF files 
via forward search and backward search 
as described in Section~\ref{subsec:editViewLatex}. 

% TBD \listfiles to check loaded packages 

It also provides the command \cmd{setMinorVersionPdf} to set the minor version of the PDF file created. 
This is mostly needed because some graphic tool creates PDF files with a newer PDF version 
than the \LaTeX{} distribution does. 
Setting the version high enough, avoids an according warning \texttt{WAP03} 
listed in Table~\ref{tab:WarnALP} on page~\pageref{tab:WarnALP}. 
The warning pattern is described in Section~\ref{subsec:patternWarnLatex} in detail. 
As described in~\cite{DocMetaDataSuppCode}, Section 2 
the recommended way to set major and minor version of the PDF output 
like in \cmd{DocumentMetadata\{pdfversion=1.7\}}, 
but due to a bug, any invocation of \cmd{DocumentMetadata} corrupts reproducibility of the created PDFs. 
After this is fixed, the \cmd{setMinorVersionPdf} shall be removed again. 
% TBD: check from time to time. 

Another class of commands provided is represented by \cmd{textttNoChk} 
which sets the argument in typewriter font 
just like \cmd{texttt} but for which checks by \texttt{chktex} are suppressed. 
Another example for this kind of command is \cmd{inputNoChk} 
which may be used to input either generated material like TikZ,
or text which is no \LaTeX{} at all. 
For details see Sections~\ref{subsec:latChkRc} and~\ref{subsec:develCheck}. 
\medskip


As the configuration files described above, 
\texttt{header.tex} is intended to be injected in phase \texttt{validate}. 

Note that \texttt{header.tex} is written for use of 
%
\begin{itemize}
  \item various \LaTeX{} enginess, \lualatex, \xelatex{} and \pdflatex{} 
  \item various document classes, 
  specifically \texttt{article}, \texttt{book}, \texttt{beamer} for presentations, 
  \texttt{leaflet} and the letter class \texttt{scrlttr2}. 
  \item for creating PDF files, but futher formats created with \pkg{tex4ht} as well
  \item direct creation of PDF and via intermediate DVI/XDV 
\end{itemize}

Which packages are loaded at all and if loaded their options, 
depend on the \LaTeX{} compiler, the output/intermediate format, 
the document class, packages loaded before and maybe on other criteria. 

This is realized with a bunch of if-constructs. 
In the long run, \texttt{header.tex} could be adapted 
to the configuration as \texttt{.latexmkrc}, 
but currently it detects the use case 
as the \LaTeX{} engine or the target format 
and loads the according packages. 
It is also conceivable to create different headers, one for each document class. 


\subsection{A header file for graphics via package \texttt{graphicx} }%
\label{subsec:headerGrp}

Chapter~\ref{chap:GraphConversions} lists various techniques 
to include graphics into a \LaTeX{} document. 
Most are based on the package \pkg{graphicx} and related packages. 
The injection \texttt{headerGrp} 
is tied to the file \href{\urlSite fromTex/headerGrp.tex}{headerGrp.tex}, 
which is intended to be included in the \LaTeX{} main file 
after \texttt{header.tex} described in Section~\ref{subsec:header} 
and which loads the packages required for that kind of graphics 
by need and with the appropriate options, 
depending on the \LaTeX{} compiler, the output/intermediate format, 
packages loaded before and maybe on other criteria. 


\subsection{A header file to suppress meta-info for PDF files }\label{subsec:headerSuppressMetaPDF}

Whereas the header file described in Section~\ref{subsec:header} is intended to be used 
in merely any \LaTeX{} main file, 
the one described here, is optional. 

It refers to created PDF files only and does not influence the optical appearance 
but suppresses writing certain meta-data. 
The main motivation is security, i.e.\@ privacy, 
but it can also be used to turn the resulting PDF reproducible. 

The injection \texttt{headerSuppressMetaPDF} 
is tied to \href{\urlSite fromTex/headerSuppressMetaPDF.tex}{headerSuppressMetaPDF.tex}. 
Above all, it suppresses information on creation and modification time, on the tool chain used 
and the trailer identifier. 
By intention the latter changes in each build run even if the sources are the same. 
Typically, this is implemented merging the current time into the build process. 
The trailer identifier is fixed by the header file 
and so created PDF files created from the same sources are the same, 
except if date and time are included manually, as e.g.\@ by the command \cmd{today}, 
except for \xelatex, which uses the system time to create further hash codes. 
So, including \texttt{headerSuppressMetaPDF.tex} may serve to create reproducible PDF files. 
As described in Section~\ref{sec:chkReprod}, 
the mainstream technique to reach reproducibility 
is via manipulating the system time, 
but if an environment does not support this, 
including \texttt{headerSuppressMetaPDF.tex} is a fallback strategy, 
if not using \xelatex. 

The extent to which meta info is suppressed is inspired by reproducibility 
but above all, it is subjective. 
It is planned to make it configurable, 
i.e.\@ the file \texttt{headerSuppressMetaPDF.tex} 
is created according to security settings of this maven plugin. 

For further information on meta info in PDF files 
related with security and reproducibility 
see~\cite{LatexGen}, Section 4 
and how this software treats the handles the issues 
see Section~\ref{sec:chkReprod}. 


\subsection{An installation script for VS Code Extensions }\label{subsec:instExtVsCode}

Calling from project root 
%
\begin{Verbatim}
  mvn latex:inj -Dlatex.injections=vscodeExt,latexmkrc
\end{Verbatim}
%
injects the according files \href{\urlSite fromMain/instVScode4tex}{instVScode4tex.sh} and \texttt{.latexmkrc}. 

If the editor VS Code is already installed, 
the script \texttt{instVScode4tex.sh}, 
installs and updates all extensions of VS Code the author used to write \LaTeX-code. 
Project \url{https://github.com/Reissner/QMngMnt} 
uses the script for automation of installation and update. 
It is the only injected file which is executable. 

Pasting \href{\urlSite fromTex/.latexmkrc}{.latexmkrc}, 
which is just Perl code, 
into VS Code, one can see the highlighting, 
of course provided the extensions given by \texttt{instVScode4tex.sh} are installed; 
The configuration file \texttt{.latexmkrc} 
for the development tool \tool{latexmk} is in fact a Perl script. 


\subsection{Scripts in conjunction with reproducibility }\label{subsec:ntlatexVmdiff}

Calling from project root 
%
\begin{Verbatim}
  mvn latex:inj -Dlatex.injections=ntlatex,vmdiff
\end{Verbatim}
%
injects the according files \texttt{ntlatex} and \texttt{vmdiff} in the root directory. 

The injection \texttt{ntlatex} 
injects the file \href{\urlSite fromMain/ntlatex}{ntlatex} 
which runs the \LaTeX{} compiler specified in the pom, or in the magic comments if present, 
to create a PDF file. 
As usual, magic comments override configuration in the pom. 
Also, it takes into account whether the PDF file is created via intermediate DVI/XDV files or not, 
depending on the configuration. 

This invocation takes also processing time and the timezone into account 
to guarantee reproducibility if so configured. 
As \tool{latexmk} is, also \tool{ntlatex} shall be independent 
of the configuration given by the pom. 
This is realized in the same way, 
namely by encoding the configuration in the injection \texttt{.latexmkrc}. 
The downside is, that \tool{ntlatex} like \tool{latexmk} requires Perl to work. 
For details see Section~\ref{sec:chkReprod} on reproducibility. 

But if \tool{ntlatex} so close to \tool{latexmk}, 
why is it needed in addition? 
It is because \tool{latexmk} won't recompile, if the expected PDF file exist already 
and no sources changed. 
So \tool{ntlatex} is needed to force recompilation. 
\medskip

Complementary to this \texttt{vmdiff} is a diff tool for PDF files 
combining visual equality checked with \texttt{diff-pdf-visually} 
with equality of metadata checked via \texttt{pdfinfo} 
if the files are visually the same. 
It is realized as a bash script \href{\urlSite fromMain/ntlatex}{vmdiff} and requires no installation 
except \texttt{diff-pdf-visually} and \texttt{pdfinfo}. 
% TBD: clarify: elsewhere we also use exiftool for similar tasks. 
% but the according output must be filtered a bit. 
% There is a category for tools reading metadata, 
% but it is not clear whether pdfinfo and exiftool belong to the same category in the long run. 

\subsection{Script \texttt{(de)pythontexW} patching \texttt{(de)pythontex} }% chktex 36
\label{subsec:pythontexW}

Calling from project root 
%
\begin{Verbatim}
  mvn latex:inj -Dlatex.injections=pythontexW,depythontexW
\end{Verbatim}
%
injects the according files 
\href{\urlSite fromMain/pythontexW}{pythontexW} and 
\href{\urlSite fromMain/depythontexW}{depythontexW}
which just invokes \texttt{(de)pythontex} %chktex 36
but does not simply output feedback on \texttt{stdout}
but besides doing so writes it in a log file. 
This is needed to provide an interface usual in the \TeX{} ecosystem. 

Note that all this is specific for unix -like operating systems 
but can be easily adapted to windows. 

\section{Development of documents}\label{sec:devel}

The term ``development of documents'' is coined by the author 
and reflects that writing a document 
resembles developing software 
in that it is an iterative process consisting in producing pieces of information, 
checking, modifying, correcting, erasing it, checking again\dots. 
After initial creation, is like a dialog between the author and its work. 

This is true of course independent of the tools used, 
but some tools support this process better than others. 
For document development the ideal are WYSIWYG (``what you see is what you get'') editors, 
which should maybe be better called WYRIWYR (``what you write is what you read''), 
or, taking also drawings into account, IllO (``input looks like output''). %TBD: maybe better glossary 
For software development the ideal languages are prototyping languages, interpreted at least. 

From that point of view, \LaTeX{} and friends is the worst conceivable choice: 
%
\begin{itemize}
  \item
  You write in an editor, but you read off from a viewer. 
  So you must permanently switch your attention. 
  \item
  You write a sequence of commands, but you read text, formulae, drawings. 
  In a sense you program the appearance of a page or site. 

  This discrepancy becomes particularly apparent when creating a drawing in \LaTeX, 
  e.g.\@ with TikZ, because even drawings are described or programmed quite formally. 
  \item
  You cannot just see instantly the result of your work; 
  first you have to trigger a compilation process and wait some time. 
  So, besides an editor and a viewer you also need some kind of console. 
  It is even worse: 
  Typically, based on the console output you must either rerun the \LaTeX{} engine
  or run some auxiliary program, even more of them 
  and then again the compiler, maybe several times. 
  The decision whether the viewer shows the final result already, 
  or whether another command has to be issued and if so which one, 
  is based on the console output\footnote%
  {What is worse, there are cases where the console output fails 
  to contain a hint to rerun some program. 
  % pythontex z.B. aber auch makeglossaries.
  }. 
  So part of your attention must be on the console also. 
  The console is also used to issue the next command. 
  \item 
  The compilation process may go wrong or be in a sense deficient, 
  so what you need is observing logs, either on the console or in a log file. 
  Even if the input is accepted by build tools even without warning, 
  still there may be something wrong. 
  The \LaTeX{} tools do not include any spell checking or grammar checking. 
  Since \LaTeX{} documents are in a sense programmed, 
  an additional burden is the need for a kind of linting, 
  which is done, e.g.\@ by \texttt{chktex}. 
  This must be invoked manually and yields another log file, 
  although no output. 
\end{itemize}

The situation is visualized in Figure~\ref{fig:docDevelBase}. 
It is no UML diagram although using elements of UML\@. 
The developer of the document (it may or may not be the author) 
is visualized as a stick figure 
and the tools used for development are the boxes surrounding it,
resembling instances in a UML class diagram. 
Besides the tool under consideration, the according files are shown. 
The console is to invoke conversion commands like \lualatex. 
This shows already, that the user does not face a single counterpart, 
but has to juggle with a bunch of tools at once. 
The arrows represent data flows. 
If this data comprises commands the lines are solid, else they are dashed. 

This explains the need for tools and techniques to mitigate the situation. 

At first sight, this \LaTeX-builder is not to contribute to document development, 
because it is used after the end of the development process, 
automating the compilation process. 
Since the \LaTeX-builder is also a checker tool, 
supervising even warnings, e.g.\@ on bad boxes, and by default invoking \texttt{chktex} 
and monitoring its log file, and since compilation may always fail, 
the \LaTeX-builder may initiate another loop in the development process. 

Before describing the contribution of this \LaTeX-builder 
to the process of document development, 
let us describe the process of document development in more detail, 
in particular the other tools supporting document development and their interaction. 
With this background in mind, it is easier to describe the role of the \LaTeX-builder 
in the team of development tools. 

\begin{figure}
  \centering
  \IfPackageLoadedTF{tex4ht}{%
should be a picture 
}{ 
%\includegraphics{F4_05someMetapost1.mps}
\includegraphics{F3_01texUsagePlain.mps}
}
  \caption{\label{fig:docDevelBase}Document development with base tools}
\end{figure}

The minimum needed to develop a document in \LaTeX{} 
are an editor, an according viewer and the \LaTeX{} tools for build and check 
as described in Section~\ref{subsec:editViewLatex}. 
As described above, using this basic tools directly 
distracts much of the attention of the author/developer 
from the content. 
Thus, it is a good idea to use a tool to orchestrate the \LaTeX{} tools. 
The author of this software prefers the orchestration tool \tool{latexmk} 
which is described in Section~\ref{subsec:develLatexmk}. 

The check tool \texttt{chktex} and the according goal \texttt{chk} 
are already described in Section~\ref{sec:outputFormats}. 
Nevertheless, the aspects of checking 
in the context of document development is treated separately 
in Section~\ref{subsec:develCheck}. 

The goals \texttt{grp} and \texttt{clr} 
described in Sections~\ref{subsec:develGraph} and~\ref{subsec:develClear} 
make sense only in the context of document development. 
For details see these sections. 

Finally, Section~\ref{subsec:develConfig} 
is on installing extensions for document development on the editor VS Code. 
To that end, this software provides an installation script. 


\subsection{Editors, viewers and \LaTeX}\label{subsec:editViewLatex}

Although there are alternatives like Emacs with extension \auctex, 
the author recommends using VS Code in conjunction with extensions 
to write and build \LaTeX{} documents 
and to view the results on \texttt{okular}. 
The recommended extensions are those 
installed by the script \texttt{instVScode4tex.sh} 
described in Section~\ref{subsec:develConfig}. 

Most of the recommended extensions of VS Code 
are to highlight the code of the various file types, 
one, \ltex{} is a spell and grammar checker, 
but the central extension is \texttt{james-yu.latex-workshop} 
which also provides build functionality. 
Among the build ``recipies'' is \texttt{latexmk (latexmkrc)} 
which is recommended because it integrates well with build tool \tool{latexmk} 
described in Section~\ref{subsec:develLatexmk} 
in a way which integrates \tool{latexmk} well with this \LaTeX{} builder. 
Note that \LaTeX{} Workshop also offers a command ``clean up'', 
corresponding with goal \texttt{clr} of this software 
which is described in more detail in Section~\ref{subsec:develClear}

As a PDF viewer, we use \texttt{okular} 
with settings given by the menu ``settings'' 
and submenu ``configure okular''. 
To make \texttt{okular} update as soon as the PDF changes, 
in tab \texttt{General} 
  %
\begin{itemize}
  \item deselect show backend selection dialog and 
  \item select reload document on file change. 
\end{itemize}

To enable backward search described below, 
in tab \texttt{Editor} choose ``custom editor'' and type 
%
\begin{Verbatim}
  code -r --goto %f:%l
\end{Verbatim}

Together the ``general'' settings make \texttt{okular} update automatically 
when a new PDF occurs. 

As an HTML viewer any of the usual browsers is usable; 
they all update as soon as the rendered HMTL file changes. 

Still it is a problem to synchronize editor and viewer. 
As far as the author knows, synchronization is possible only for PDF viewers. 
Synchronization means at least that for a position on the editor, 
the according position on the viewer must easily be found 
and vice versa. 
Even better would be if moving in the editor selects the according site at the viewer 
and the other way round. 
These two features are called \emph{forward search} and \emph{backward search}, respectively. 
If the \LaTeX{} main file has a setting \texttt{synchtex=1} or \texttt{synchtex=-1}, 
then the created PDF has the according information. 
Then for VS Code with \LaTeX{} Workshop offers forward search: 
the keystroke \texttt{ctrl-alt-j} 
makes the viewer move to the site corresponding with the cursor position. 
For the viewer \texttt{okular}, backward search is configured in tab ``editor'' 
as described above, 
and it works with the browse tool just hovering over the location of interest 
and pressing shift plus mouse left key. 

This software supports forward and backward search 
in that it offers injection of a header file \texttt{header.tex} 
which sets \texttt{synchtex=1} 
and offers injection of an installation script \texttt{instVScode4tex.sh} 
which allows installation of the relevant extensions of VS Code. 
\medskip


Besides the separation of editor and viewer, 
the time delay between writing and reading 
disturbs document development. 
\LaTeX{} has a way to speed up compilation: 
compiling only parts of a longer document 
which are under construction and which may thus change. 
These parts must be in separate TEX files 
and must be included with \cmd{include} 
not just input using \cmd{input}. 
With the command \cmd{includeonly} 
one can specify the files to be recompiled. 
This works particularly well for document class \texttt{book} 
when including chapters because each chapter starts with a new page, 
so, page breaks are the same whether 
compiling a chapter with \cmd{includeonly} 
or compiling the whole document. 

This plugin supports partial compilation 
insofar as the goal \texttt{clr} described in Section~\ref{subsec:develClear} 
eliminates additional AUX files tied to included sections. 


\subsection{The build tool \tool{latexmk}}\label{subsec:develLatexmk}

Essentially, it is possible to compile latex files 
only with editor, viewer and a console. 
Let us collect the challenges. 
The document may contain graphic files 
which must be precompiled by further tools 
which must be invoked a priori on the console. 
For FIG files this is \texttt{fig2dev}. 
This invocation must be repeated as soon as a FIG file changes\footnote%
{Typically, this triggers a sequence of invocations of converters 
along files one depending on the other. }. 

Then a \LaTeX{} engine like \lualatex{} must be invoked. 
Typically, the \LaTeX{} engine must be invoked more than once 
and besides the \LaTeX{} engine 
some further auxiliary programs like \texttt{makeindex} must be run. 
The console displays indications to the user what action to be taken next. 
Normally after invocation of an auxiliary program, 
the \LaTeX{} engine must be rerun at least once. 
Each of the programs may fail. 
Most of the programs write success messages and more detailed information 
containing error messages, warnings or just information messages 
on the console and in their respective log files. 
Potentially, these influence the actions the user must take next. 
\smallskip

What is needed, is a tool for orchestration\index{orchestration} 
of the basic tools: 
Orchestration means invoking more basic tools in a reasonable order 
and supervising the results, at least success and to react appropriately. 
This frees the user from deciding 
which of the many auxiliary programs are to be invoked next 
and whether the \LaTeX{} engine is to be invoked once more 
to get final correct output. 
Also, an orchestration tool detects if a build fails 
or ideally even if a warning indicates 
that the result is not correct or maybe only not ideal. 

There is a tool doing this work, \tool{latexmk}, 
except that it does not care about warnings. 

If something goes wrong, and it is not clear what, 
it is typically a good idea to fall back to the more basic tools. 
A great point with \tool{latexmk} is, 
that this is possible without any problem, 
and it is as simple to switch back from basic tools to \tool{latexmk}. 
This is what we mean saying that \tool{latexmk} 
\emph{integrates} the basic tools. 

The best way to invoke \tool{latexmk} for document development is 
%
\begin{Verbatim}
  latexmk -pvc latexFile
\end{Verbatim}

According to~\cite{LatexMk23}, Section ``DESCRIPTION'', 
the option \texttt{-pvc} is shorthand for ``preview continuously'', 
a kind of nonstop mode: 
The PDF file, or whatsoever is created, 
then a viewer is opened in the background 
if not yet open 
and then \tool{latexmk} monitors changes of dependencies 
and triggers a rebuild each time a change is detected 
performing a proper sequence of invocations of \LaTeX{} engines and auxiliary tools. 
Note that \tool{latexmk} does not stop after finishing a compilation, 
whether successful or not. 
Instead, it awaits a change of a source file which triggers a new run of some basic tool 
until interrupted by the user. 
The option \texttt{-pvc} is described in more detail 
in~\cite{LatexMk23}, Section ``LATEXMK OPTIONS AND ARGUMENTS ON COMMAND LINE''. 
One detail to be added, mentioned in~\cite{LatexMk23}, Section ``DESCRIPTION'', is, 
that \tool{latexmk} detects dependencies 
based on the FLS file written by the \LaTeX{} engine 
when invoked with the \texttt{-recorder} option. 

A small fallback step advisable if something goes wrong 
is to interrupt continuous viewing and to invoke \tool{latexmk} without options. 
Then \tool{latexmk} performs a single build and finishes; no viewer is opened. 
This may help in understanding the problem, 
but in general, it is advisable to go back to basic tools like \lualatex. 
To understand the build process from scratch, 
erase all created files by \texttt{latexmk -C} 
or all intermediate files by \texttt{latexmk -c}, which does not erase the resulting PDF file, 
before using the basic tools. 
\medskip

\subsubsection{Differences of \tool{latexmk} with this \LaTeX{} builder}%
\label{subsubsec:latexmkDiff}


Let us discuss the differences between \tool{latexmk} and this latex plugin: 
First, the plugin runs within a maven process which introduces a lot of overhead. 
So this cannot be as fast as \tool{latexmk} is. 
In addition, a maven plugin cannot open a viewer. 
Moreover, the plugin is designed to build all \LaTeX{} main files %in a project 
and not to focus on a single one. 
In many cases, more than one output format shall be created. 
The latter properties which are disadvantages in the context of document development, 
can be overcome, by specifying a single target in the setting \texttt{targets} 
or by invoking goals with a single target, e.g.\@ by \texttt{mvn latex:pdf} 
and to restrict to building a subset of files and if needed a single \LaTeX{} main file
with the settings \texttt{mainFilesExcluded} or \texttt{mainFilesIncluded} 
described in Table~\ref{tab:paramGen} on page~\pageref{tab:paramGen}. 

Another difference is, that by default, the plugin cleans up the folder with the TEX sources, 
and only the resulting file, e.g.\@ PDF is copied to the target folder before cleanup. 
To be more precise, only the files present before the build are kept, possibly updated, 
all the others are removed. 
This is appropriate for a maven plugin but destroys log files containing vital information 
if the build goes wrong. 
Still if a file is interesting it may be created by touch or by some basic latex tool 
as \lualatex{} or \texttt{makeindex} 
and then a build done by this plugin will pertain the file updated by the build process. 
For document development, 
the parameter \texttt{cleanUp}, also described in Table~\ref{tab:paramGen}, 
which is \texttt{true} by default, can be set to \texttt{false} 
so that no file in the latex directory is deleted. 

So, it is clear that this plugin is for final global build with a lot of supervision 
sensitive to detecting caveats. 
To overcome these, further development of the document is necessary, 
which is better done individually on the problematic document with \tool{latexmk}. 
In a sense \tool{latexmk} is the fallback to this maven plugin 
as much as \lualatex{} is the fallback to \tool{latexmk}. 

To make this work, this plugin must integrate \tool{latexmk} 
as \tool{latexmk} integrates \lualatex{}. 
This is guaranteed, if this plugin can write a config file \texttt{.latexmkrc} 
which causes \tool{latexmk} to behave like this plugin. 
This is exactly what injection of \texttt{.latexmkrc} is intended to do 
according to Section~\ref{subsec:latChkRc}. 
% TBD: in fact, this is not fully reached. 
Note that this feature is just offered, 
but the user may also use his/her own file \texttt{.latexmkrc}. 

Based on injection of \texttt{.latexmkrc}, this plugin may even use \tool{latexmk} 
as a means to build bypassing its internal build rules. 
For motivation of this feature and for details in implementation see Section~\ref{sec:latexmk}. 

There is a difference in the build processes (except if this plugin uses \tool{latexmk}) 
concerning mostly graphic files: 
\tool{latexmk} detects dependencies via the \texttt{-recorder} option of the latex generator 
and creates or recreated what is new or changed. 
This is more elegant than the idea of this plugin 
which is creates a fixed set of graphic files first 
and is from that point on 
based on detecting hard coded set of files and tracing log files. 
In other words, \tool{latexmk} has no graphic preprocessing 
as Chapter~\ref{chap:GraphConversions} describes for this build tool. 
This offers the advantage, that \tool{latexmk} never creates graphic files 
which are later not needed for inclusion. 
Nevertheless, 
to deal with graphic files which are to be created in the course of the build, 
\tool{latexmk} runs the \LaTeX{} engine in \texttt{nonstopmode} mode. 
Still, the run of the engine is interrupted, 
a single graphic file is created according to some rule 
and then the engine is rerun. 
For a document with 10 graphic files to be created in the course of the build, 
for \tool{latexmk} only the 10th runs of the \LaTeX{} engine is completed. 
In contrast, this plugin requires a single run, 
so performance is significantly better. 
The use of \cmd{IfFileExists} is not really elegant 
but prevents \tool{latexmk} from frequent reruns 
and in some cases is a technique to make the build process with \tool{latexmk} work. 
One of these cases, 
related with using \pkg{listings} is discussed below in this section. 

More general, there are cases, 
where this latex builder succeeds but \tool{latexmk} does not. 
As this latex builder may invoke \tool{latexmk} either in general or for selected files, 
this latex builder is mightier than \tool{latexmk}. 
If both approaches succeed, 
the results shall be the same for this plugin and for \tool{latexmk}. 

It is possible to combine this plugin with \tool{latexmk} to speed up \tool{latexmk}: 
%
\begin{Verbatim}
  mvn clean validate latex:grp
\end{Verbatim}
%
cleans like \texttt{latexmk -C} and in \texttt{validate} 
invokes goal \texttt{inj} injecting \texttt{.latexmkrc} to configure \tool{latexmk} 
and maybe \texttt{header.tex} necessary to compile the latex files at all. 
Finally, goal \texttt{grp} creates the graphic files which speeds up \tool{latexmk}. 
Of course the above maven invocation is also a good initialization 
for building with the basic tools without \tool{latexmk}. 

Finally, goal \texttt{clr} tied to phase \texttt{clean} erases all intermediate files 
and thus makes the next build independent of the previous one. 


For further reading on goal \texttt{grp} creating graphics files 
see Section~\ref{subsec:develGraph}, 
Section~\ref{sec:injFiles} is on file injection and 
goal \texttt{clr} to clear created files is 
described in Section~\ref{subsec:develClear}. 
\medskip

\subsubsection{How \tool{latexmk} is integrated}\label{subsubsec:latexmkIntegration}

Finally, we show how this plugin may support \tool{latexmk} where. 
To understand in which sense, one must dive very deep. 
In short, injection of a header patches package \pkg{listings} 
in a way that saves performance of \tool{latexmk}. 
Let us elaborate. 

This software and \tool{latexmk} follow a different philosophy in finding dependencies: 
Whereas this software creates image files in advance before invoking a \LaTeX{} engine, 
\tool{latexmk} first calls the \LaTeX{} engine in \texttt{nonstopmode} 
to avoid a stop because of a missing file. 
Then the file is created using the appropriate rule (hopefully unique) 
and the emgome is run again, 
this time passing the inclusion of the first created files 
failing at the next one. 
To find out that another rule is needed, \tool{latexmk} parses the LOG file of the latex compiler. 
As the packages write log messages in their own style, 
this is the point where the solution is no longer generic 
and so it is no wonder that there is at least one kind of inclusion which does not work that way: 
inclusion with \cmd{lstinputlisting} provided by \pkg{listings}. 
In fact, the author has an email from J.~Hoffmann, author of \pkg{listings} 
telling that there are more packages with the same problem. 
To be checked: \pkg{fancyvrb} and \pkg{moreverb}. 
Nevertheless, all other ways of inclusion \emph{used by this manual} 
like the one with \cmd{import} seem to work fine. 

The current workaround for the second problem 
is by patching \pkg{listings} as described in Section~\ref{subsec:header}.

The suggested workaround for the first problem is 
creating graphic files using goal \texttt{grp} as described in Section~\ref{subsec:develGraph} 
before invoking \tool{latexmk}. 

Still some generalization in \tool{latexmk} could spare this modification. 

Another point is, that currently for each file \tool{latexmk} creates with a separate rule, 
another run of the \LaTeX{} engine is required: 
The initial run is interrupted with the first missing file. 
Then that file is created by an appropriate rule and the \LaTeX{} engine is rerun 
failing with the next missing file. 
That way the process goes on until the last file is created with a rule. 
Of course this procedure is quite time-consuming, so an alternative is required. 



\subsection{Checks in the context of document development}\label{subsec:develCheck}

The target \texttt{chk} just invoke the tool \texttt{chktex} and logs finding in a CLG file. 
It is invoked as the final quality check for the documents created from latex sources. 
But if this check fails, there is a transition to document development. 
As said in Section~\ref{subsec:develLatexmk} on running this plugin on a single file with a single target 
applies here also. 
But here again, this plugin is not the first choice: Better is to invoke \texttt{chktex} directly 
and to eliminate the warnings iteratively. 
Since the file \texttt{.chktexrc} injected by this plugin as described in Section~\ref{sec:injFiles} 
configures \texttt{chktex} whether \texttt{chktex} is invoked directly by the user 
or via the plugin in goal \texttt{chk}, the results are the same. 
In the wording coined in Section~\ref{subsec:develLatexmk}, 
this plugin integrates \texttt{chktex} very much the same way as it integrates \tool{latexmk} 
namely by injection of a config file. 

The config file \texttt{.chktexrc} in turn is adapted to the header \texttt{header.tex} 
which is also injected. 
In general, \texttt{.chktexrc} excludes content of environments 
and of arguments of commands defined in packages loaded by \texttt{header.tex} 
or defined therein directly. 
A nice example of another kind of synergy is the command \cmd{textttNoChk} 
defined in \texttt{header.tex}. 
Functionally, it is just \cmd{texttt} which sets the argument in typewriter font, 
but in \texttt{.chktexrc} it is listed among the commands 
the arguments of which shall not be checked by \texttt{chktex}. 
\medskip


After eliminating warnings until direct invocation of \texttt{chktex} displays no warnings, 
one can be sure that also check with goal \texttt{chk} of this plugin does not yield warnings. 


\subsection{Goal Graphics \texttt{grp}}\label{subsec:develGraph}

In the context of document development, 
typically compilation is done by basic tools like \lualatex{} 
or by an orchestration tool like \tool{latexmk}. 
Nevertheless, since separation of builds is desirable, 
intermediate files like graphic files are not present. 
Maybe they are removed by cleaning. 

The 

TBD\@: check whether this is really needed: is also described in section on latexmk. 
Maybe we need a section on this plugin describing grp and clr uniformly. 
Maybe also first write on chktex and its relation to this plugin. 

Hint to relation with latexmk. 
needs mvn validate \& mvn latex:grp. 

For creating the graphic files in the TEX source directory, 
there is a goal \emph{graphics}, invoked by \texttt{mvn latex:grp}. 
This goal does not create any output in the site directory. 
Instead, it populates the source directories 
with graphic files which can be directly included into the \LaTeX-file 
and so it allows to run the \LaTeX-compiler on the \LaTeX{} main files 
from within a development environment. 
Thus, the goal \emph{graphics} is thus a vital feature 
for development of documents. 


Note that in general \texttt{mvn clean validate latex:grp} 
creates all files necessary to compile with a \LaTeX{} engine like \lualatex{} 
and also to compile smoothly with \tool{latexmk}. 



\subsection{Goal Clear \texttt{clr}}\label{subsec:develClear}

When invoking this plugin as a final build, 
\texttt{cleanUp} is set to its default \texttt{true}. 
Thus, all files not present at the beginning of the build process are removed. 
As a consequence, there is no need for a separate goal \texttt{clr}. 
This comes into the game only in the context of document development. 
Either \texttt{cleanUp} was set to \texttt{false} 
or other more basic tools created intermediate files which must be deleted by \texttt{clr}. 

Cleaning is vital because it makes the next build independent of the previous one. 
Deletion is driven by a regular expression \texttt{patternCreatedFromLatexMain} 
described in Table~\ref{tab:paramGen} on page~\pageref{tab:paramGen}. 
Completeness can be guaranteed only 
if the set of loaded packages is limited. 
Of course, only created files shall be deleted. 
For packages introduced in the injected header \texttt{header.tex} 
described in Section~\ref{subsec:header}, this shall be the case. 
The author's criterion for a correct regular expression is, 
that after deletion exactly the files under version control remain. 

The goal \texttt{clr} corresponds with \texttt{latexmk -C} and is tied to phase \texttt{clean}. 

Clearing comprises files created by the goal \texttt{grp} and by any other goals. 
Note that AUX files are deleted if they belong to a \LaTeX{} main file or to an included file. 

The most interesting files are those created by injection, 
i.e.\@ by goal \texttt{inj} like \texttt{.latexmkrc}: 
As pointed out in Section~\ref{sec:injFiles}, 
each of the files in question is deleted only 
if they were definitively written by this plugin. 
If this is proved to be false or a proof is not possible, 
the configuration files are not deleted. 
As for goal \texttt{inj}, in case of a doubt, a warning is displayed. 



\subsection{Installation and Configuration}\label{subsec:develConfig}

TBD\@: rework: maybe better describe the goal \texttt{inj}. 
The goal \texttt{inj} is to create a set of files, 
partially adapted to the current configuration. 



By default, it is tied to lifecycle phase \texttt{validate} 
and comprises the set of injections \texttt{latexmkrc,chktexrc}. 

The first we treat is injection \texttt{vscodeExt} 
injecting a file \texttt{instVScode4tex.sh} in the TEX source directory. 
Typically, this is not injected during a lifecycle, 
but when installing or updating extensions for VS Code 
used during document development. 
Thus, typically it is invoked in the form 
%
\begin{verbatim}
  mvn latex:inj -Dlatex.injections=vscodeExt
\end{verbatim}

\noindent
In the default configuration, this creates an executable file 
%
\begin{verbatim}
  src/site/tex/instVScode4tex.sh
\end{verbatim}
%
using bash shell. 
The extensions are those described 



Install script for installing extensions for VS Code 
helping in developing \LaTeX{} documents. 


In addition, configuration scripts for \tool{latexmk} and \texttt{chktex}. 
Also describe how to use. 



\subsection{Miscellaneous}
% TBD: This is to be (re-)moved in the long run. 



During development, it is comfortable, 
to have the log-file in the same directory as the \LaTeX{} main file. 
Also, if PDF- and TEX-files are synchronized, 
% FIXME: reference to package 
also the PDF-file should be in the same directory. 
Likewise, files in graphic formats 
which cannot be included into a \LaTeX-file without conversion, 
that converted file shall be in the same directory as the original one. 
So, all files, manually created files 
and files arising from automatic conversions 
shall be in the same folder, at least during development. 
Also, typically, one wants to mix creation by this maven-plugin or ant-task 
with at least partial creation through external tools. 
For example, if writing \LaTeX-files with Emacs, 
it is much more convenient, to compile the \LaTeX{} main file 
via \pdflatex{} from within Emacs 
or to create a PDF-file from a \gls{fig}-file 
through \texttt{xfig}'s export dialog, 
than using this maven-plugin or this ant-task. 
Also, these tools work best, if all is in one folder. 

On the other hand, 
conventionally, in a maven project, 
sources are held in folder \texttt{src}, 
whereas created files occur in the folder \texttt{target}. 
Likewise for ant. 
The compromise, this maven-plugin and this ant-task take, 
is, that at the end of a run, 
at most the files present at the beginning of the run 
may be present in the source directory. 
So, this software builds in the following steps: 
%
\begin{itemize}
\item
Store a list of all files present at the beginning of a run.
\item
Process all graphics files of the formats requiring preprocessing.
\item
Determine the \LaTeX{} main files.
\item
Run the \LaTeX{} engine, e.g.~the one creating PDF-output or DOCX-output.
This may include running auxiliary programs like \texttt{bibtex} or \texttt{pythontex} 
and also rerunning the \LaTeX{} engine several times. 
\item
Copy the result files (if any) into the target folder.
\item
Remove all files not present at the beginning of a run, by default. 
% FIXME: maybe different for goal chk. 
\end{itemize}

To keep e.g.~the resulting PDF, 
just create it via compilation through Emacs, 
even if not all graphic files to be included are present 
or just by a \texttt{touch}-command. 
Then in the next run of this plugin, 
this PDF will be re-created, 
that time complete with the graphics output. 
That way, synchronization between \LaTeX- and PDF-files is possible. 
Likewise, to keep the log-file or the aux-file, just touch it. 
This technique is really valuable for debugging. 

To keep all created files after a run of this maven-plugin, 
set the parameter \texttt{cleanUp} in the pom 
to \texttt{false} as illustrated in Listing~\ref{lst:noCleanup}. 
For the ant-task likewise. 

%\lstset{language=xml, basicstyle=\small}
\begin{lstlisting}[language=xml, basicstyle=\small,
escapechar=|,
float=b, captionpos=b, label={lst:noCleanup},
caption={Configuration without cleanup}]
<!-- create html and pdf and other formats from latex -->
<plugin>
  <groupId>|\groupId|</groupId>
  <artifactId>|\artifactId|</artifactId>
  <version>|\strippedVersionID|</version>
	
  <configuration>
    <settings>
      <targets>pdf</targets>
      <cleanUp>false</cleanUp>
    </settings>
  </configuration>
</plugin>
\end{lstlisting}


But how can one get rid of all these newly created files? 
That is what is the goal \texttt{latex:clr} is for: 
% 
\texttt{mvn latex:clr}
%
removes all created graphic files 
and for each \LaTeX{} main file, it removes all files with ``similar'' names
including log files, index files and that like.
Typically, this suffices, to remove all files created. 
If not, 
try to modify parameter \texttt{\$patternCreatedFromLatexMain} 
in the pom accordingly. 
If this does not help either, please inform the developer of this software. 
Of course, if further software is used which creates additional files, 
like Emacs creates a folder \texttt{auto}, 
these files cannot be removed by this maven-plugin or this ant-task.
Note that \texttt{latex:clr}
also removes exported files as listed in Section~\ref{sec:outputFormats}
from the target folder. 

During development of a \LaTeX-main file, 
it is often more convenient to compile from within an editor like Emacs. 
The problem is, that compilation fails if the graphic files are missing. 
This is what the goal \emph{graphics} accessible via 
% 
\begin{Verbatim}
mvn latex:grp
\end{Verbatim}
%
is for: 
It creates all graphic files required to compile the \LaTeX-main files. 

Still this does not create a bibliography, an index or a glossary. 
With \auctex\index{auctex}, an Emacs-package for editing \LaTeX, 
bibliography and index are well-supported. 
To create a glossary, \auctex{} has to be modified a little. 

%FIXME\@: include this into auctex. 

That way also the log-files required are created: 
In case of this manual, 
the files \texttt{manualLMP.xxx} are created 
where \texttt{xxx} is 
%
\begin{itemize}
\item
\texttt{log} for \LaTeX, 
\item
\texttt{blg} for \texttt{BibTeX}, 
\item
\texttt{glg} for \texttt{makeglossaries} and 
\item
\texttt{ilg} for \texttt{makeindex}. 
\end{itemize}

The last goal regularly used for development of documentation is \emph{check}. 
It is invoked via 
% 
\begin{Verbatim}
mvn latex:chk
\end{Verbatim}
%
and runs \texttt{chktex}, described in~\cite{ChkTeX22}, 
on each \LaTeX{} main file 
after having created graphic files as for goal \emph{graphics}. 
As a result, a log-file with suffix \texttt{.clg} is created 
but not copied to the target folder. 
If the log-file contains an entry, 
an according message is logged. 
% FIXME: there is a lot more to do here. 
Note that, with default configuration, 
\texttt{chktex} requires the \LaTeX-package \pkg{booktabs} 
described in~\cite{BooktP}. 

Besides the basic configuration packaged with \texttt{chktex}, 
there can be an additional configuration file \texttt{.chktexrc} 
which partially overwrites variables set by the basic configuration file, 
partially, for list-valued variables, adds entries. 
Section~\ref{sec:xmlPom} describes how to access the \texttt{.chktexrc} 
with which this manual is checked and 
details to the form of \texttt{.chktexrc} can be found in~\cite{ChkTeX22},~Section 6.1.5.  


% rsvg-convert -f pdf -o t.pdf t.svg
% inkscape t.svg --export-pdf=t.pdf
% convert file.svgz file.pdf 
% rasterizer -m application/pdf file.svgz -d file.pdf
% cairosvg in.svg -o out.pdf
% yyy



Another aspect of document development is integration with other tools. 

Document development starts with the editor. 
Above the Emacs editor enhanced with \auctex{} was mentioned. 
We recommend VS Code in conjunction with several extensions. 
If VS Code itself is already installed 
the script \href{\urlSite fromMain/instVScode4tex}{instVScode4tex.sh} 
installs and updates all extensions 
the author used to develop this manual. 
The core extension is \texttt{latex workshop}, 
the others are mainly used for editing graphic files. 
For details see Section~\ref{sec:xmlPom}. 







% TBD: link? 

\section{Goals in the maven lifecycle}\label{sec:usageLifecycle}

The goal \texttt{latex:cfg} exporting in the formats configured
is tied to the lifecycle phase \texttt{site} so is invoked
when commanding
%
\begin{Verbatim}[fontsize=\scriptsize]
mvn site
\end{Verbatim}
%
or subsequent phase.

Also, the goal \texttt{latex:clr} cleaning created files
both from source directory and from target directory
is tied to phase \texttt{clean} so is invoked
when commanding
%
\begin{Verbatim}[fontsize=\scriptsize]
mvn clean
\end{Verbatim}

Finally, the goal \texttt{latex:vrs} displaying versions of converters 
and the goal \texttt{latex:inj} injecting a set of files 
depending on the configuration 
are tied to the phase \texttt{validate}. 
Thus, both goals are invoked when commanding
%
\begin{Verbatim}[fontsize=\scriptsize]
mvn validate
\end{Verbatim}
%
which is invoked not only in installation, but also by the site plugin.
This ensures, that the converters are checked for correct version
before being used. 
Note that by default, \texttt{mvn latex:vrs} displays complete version info,
whereas \texttt{mvn validate} only displays warnings if appropriate. 
This is, because in the first case the plugin runs with the default \texttt{versionsWarnOnly=true} 
whereas in the second case, is configured with \texttt{versionsWarnOnly=false} 
as in Listing~\ref{lst:executions}. 
Also Listing~\ref{lst:executions} shows a recommended configuration 
for the goal \texttt{latex:inj} which determines injected the files. 




\section{The ant-tasks}\label{sec:usageAntTask}


Section~\ref{sec:outputFormats} treats goal \texttt{cfg} 
to create output from one source in various formats 
and also check which is without output. 
The target formats and also the checks are specified in the parameter \texttt{targets}. 

There is an according ant task \texttt{cfg} 
doing the same also based on parameter \texttt{targets}. 
Whereas the maven plugin provides separate goals for each target, 
the ant-task has no such convenience feature. 
Section~\ref{sec:outputFormats} briefly mentions goal \texttt{clr} 
used for cleanup. 
There is an according ant-task relying on according parameters. 
Note that the ant task does not support very much of document development, 
but it is likely, that the user performs document development 
and runs other programs than the ant task on the sources. 
In this case, the \texttt{clr} task is vital. 


If this ant-task is used in an ant project 
with folder structure conforming with a maven project 
and if the \LaTeX{} sources do not require a special configuration, 
the above configuration is sufficient. 
Otherwise, parameters have to be given explicitly 
overwriting the default values. 

