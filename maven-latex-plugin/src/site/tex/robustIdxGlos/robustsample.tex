% robustsample.tex
\RequirePackage[l2tabu, orthodox]{nag}
\documentclass{book}

\synctex=1% maybe security issue: draft only? 
\usepackage{iftex}

\iftutex%
  \usepackage{fontspec}
\else
  \usepackage[utf8]{inputenc}
  \usepackage[T1]{fontenc}
\fi
%\usepackage{textalpha}

\usepackage{ifthen}
\newboolean{texFhtLoaded}
\setboolean{texFhtLoaded}{false}

% only with pdflatex, warnings for xelatex and for lualatex 
% ifxetex, ifluatex, ifpdf
\ifpdf%
  %\usepackage{mlmodern}
\else
  \makeatletter
  \@ifpackageloaded{tex4ht}{%
    \setboolean{texFhtLoaded}{true}
  }{%
  }% tex4ht not loaded 
  \makeatother
\fi

% for document development add certain options. 
% Then remove headline and prevent this plugin from overwriting. 
\usepackage[a4paper]{geometry}% option , showframe, showcrop 
%\usepackage{showframe} as an alternative 
\usepackage{microtype}
% special characters
\usepackage{textcomp}
%\usepackage{anyfontsize}
%\usepackage{cleveref}




\usepackage{makeidx}
\usepackage{robustindex}
\usepackage{robustglossary}


\ifpdf%
  \usepackage[destlabel, plainpages=false, hyperindex=false]{hyperref} %
\else
  \IfPackageLoadedTF{tex4ht}{%
    \usepackage[tex4ht,     destlabel, plainpages=false, hyperindex=false]{hyperref}
  }{%
    \ifXeTeX%
      \usepackage[          destlabel, plainpages=false, hyperindex=false]{hyperref}%[plainpages=false,hyperindex=false]
    \else%[plainpages=false,hyperindex=false]
      \usepackage[dvipdfmx, destlabel, plainpages=false, hyperindex=false]{hyperref}
      % lualatex: without [dvipdfmx] option did not find converter dvi to pdf or to ps
      % pdflatex: without [dvipdfmx] option dvips still works, but no converter for pdf
      % with [dvipdfmx] all works fine 
    \fi%
  }
\fi
../../tex/headerSuppressMetaPDF.tex

\hypersetup{
  pdfinfo={
    Author      ={Wilberd van der Kallen},
    Title       ={A latex file with robust index},
    Subject     ={robust index in latex},
    Keywords    ={LaTeX;index;makeidx;robustindex;robustglossary}
  }
}


\title{A latex file with robust index}

\author{Wilberd van der Kallen\index{Wilberd}}
\date{}

\makeindex    % obligatory with robustindex
%\makeglossary % obligatory with robustglossary

\begin{document}
\renewcommand\thepage{\roman{page}}

\maketitle

\tableofcontents
\addcontentsline{toc}{chapter}{Contents}

\chapter*{Preface}
\addcontentsline{toc}{chapter}{Preface}
\markboth{Preface}{Preface}
We will recall the purpose of \verb|robustindex.sty| in Appendix~\ref{purpose}.
Although this is not a book, we have chosen the book style for illustration.
If you want to see how \verb|robustindex.sty| may be used, compare the file
\verb|robustsample.tex| or Appendix~\ref{appendix} with the output.
If you are interested in our usage of \verb|robustglossary.sty|, 
you must read the file \verb|robustsample.tex|.
This preface has an \emph{indexed term}.\index{indexed term}

% Let us pretend this page explains $\mathrm{Ext}^i_B(M,N)$.
%\glossary{$\mathrm{Ext}^i_B(M,N)$&Ext group in the category $\mathcal C_B$}

\chapter{Introduction}\sloppy
\renewcommand\thepage{\arabic{page}}
\addtocounter{page}{-2}% Just to get overlap in roman and arabic page numbers.
This introduction has another indexed term\index{other term!cited twice on same page}
as well as the old indexed term.\index{indexed term}
The page number is again 5, but now in Arabic.
If you index the same term twice on the same page, then one simply gets
a repeat in the index.\index{other term!cited twice on same page}
That is inherent in our method.
Indeed, we had to disable the implicit page range formation of makeindex.
% Let us pretend this page explains $H^0(X,\mathcal L)$.
%\glossary{$H^0(X,\mathcal L)$&the sections of the vector bundle
%$\mathcal L|_X$ over $X$}

Of course the index file \textit{filename}.\texttt{ind}
may be edited to remedy such flaws. Such editing will actually be the
recommended way to achieve more complicated effects like the page ranges on
page~\pageref{range}, which also use several fonts in their page numbers.

Traditionally page ranges and fonts for page numbers could be done with the
encapsulating construction of makeindex, but that construction is broken by our
package.
(The encapsulating construction uses the encapsulating character $|$ in the argument of
\verb|\index|.)
We have restored it to some extent, sufficient for common practice,
but our rules are more naive than those of makeindex.
See the file \verb|robustsample.tex| for details.

Sub-entries are still supported. We illustrate this in
\verb|robustsample.tex| and Appendix~\ref{appendix} with alpha, beta and gamma.
This also illustrates the use of \verb|\gobblepageref|.%
\index{alpha!see beta}%
\index{alpha!see also gamma\gobblepageref}

After running \LaTeX\ on \textit{filename} 
one must run makeindex on \textit{filename} to get the index entries in \textit{filename}.\texttt{ind}.
Before this there may be error messages about labels or names.
Do not exclude any relevant files. 
For instance, do not use \verb|\includeonly|.

Say we wish the index to occur in the table of contents.
We then give the command \verb|\indexincontents| before the index.

The package \verb|hyperref| may break our code. 
Thus, one must use the option
\verb|hyperindex=false|, as in
\verb|\usepackage[hyperindex=false]{hyperref}|.
Note that \verb|robustindex| already puts
hyperlinks in the index. Alternatively one may go against the advice of
\verb|hyperref| and put \verb|robustindex| later in the preamble.
In the file \verb|robustsample.tex| we actually used
\verb|\usepackage[plainpages=false,hyperindex=false]{hyperref}|
because of the roman page numbers in the beginning.

% For special effects you could simply edit the filename.ind file.
% If you insist on using encapsulation, you can experiment with constructs
% like those below.
%
% Let us first describe some rules.
% In \index{arg} the argument "arg" may contain at most one "|".
% If it contains "|", then the part after "|" is for encapsulating.
% It may be of one of the following two types.
%
% 1. A sequence of letters like "texttt" with the property that if "\"
% is prepended, as in "\texttt", the result is a valid command that takes
% one argument. This command will be applied to the page number.
%
% 2. One of the brackets "(" or ")", or one of those brackets followed by
% a sequence as in type 1.
% Any "(" must be matched by its ")", and this ")" must occur in a later
% \index command that shares the part before the "|". No intervening \index
% command may have this same part before the "|".
% So here we deviate from makeindex.
%
% Makeindex also attaches special meanings to @, !, ". These meanings stay,
% but " can no longer be used to escape |. One must simply replace | with
% \vert, when | is not intended to start encapsulation. Here is an example:
%
\index{page range!risky@this $\vert$is$\vert$ complicated|(textbf}% chktex 24
% The "(" means that this starts a page range, and "textbf" means that
% one wants to apply \textbf to the page number that starts the range.
% The name "risky" of the subentry is used for sorting by makeindex, and will
% not be printed.
%
\index{page range|(textbf}%
% This belongs to the entry "page range". The previous
% \index command started a subentry of this entry.
%
\index{simpler!page range|(}%
\index{simplest page range|(}%
% encapsulating to be continued below

\appendix
\chapter{Structure of the \LaTeX\ file}\label{appendix}
\begin{verbatim}
\documentclass{book}

\usepackage{makeidx}
\usepackage{robustindex}
...
...\index{Wilberd}}                              % on page i
...
\makeindex    % obligatory with robustindex

\begin{document}
...
\tableofcontents
...\index{indexed term}                          % on page v
...\index{other term!cited twice on same page}   % on page 5
...\index{indexed term}                          % on page 5
...\index{other term!cited twice on same page}   % on page 5
...
\index{alpha!see beta}                           % on page 5
\index{alpha!see also gamma\gobblepageref}       % on page 5
...
% For special effects you could simply edit filename.ind
% If you insist on using encapsulation,...
\index{...|...}...                               % on page 5
\index{...|...}...                               % on page 7
\index{...|...}...                               % on page 9
....
\indexincontents % always put this before the index.
\printindex

\end{document}
\end{verbatim}

% continuation of encapsulating
%
\index{page range!risky@this $\vert$is$\vert$ complicated|)textit}% chktex 24
% encapsulating to be continued below

\chapter{Purpose of \texttt{robustindex.sty}}\label{purpose}
The package \texttt{robustindex} tries to remedy two problems.
\begin{itemize}
\item Amazingly often a third party changes the page numbers without rerunning
makeindex. It is much nicer to have an index that adapts itself to all
changes that do not involve
\verb|\index{|\textit{entry}\verb|}| commands.

The package achieves this
by invoking the \verb|\pageref| mechanism on automatically generated labels
of the form \verb|ind.2|, \verb|ind.4|, \dots.

Only after adding/deleting/modifying an \verb|\index{|\textit{entry}\verb|}|
command, or after changing the order of \verb|\index{|\textit{entry}\verb|}|
commands, does one have to run makeindex. But then it is important that you
do run makeindex, even when no warning is issued. The robustness has its
limits.
\item The \texttt{theindex} environment lacks a destination for hyperlinks.
That is a nuisance when one wants to add the index to
the bookmarks or to the table of contents in a PDF file: When one clicks on the
link that displays the word Index, one wants to end up in the index.
\end{itemize}

% continuation of encapsulating
%
\index{page range|)textit}% chktex 24
\index{page range!risky@this $\vert$is$\vert$ complicated|texttt}%
%
\index{page range!with risky label\label{range}}%
\index{simpler!page range|)}%
\index{simplest page range|)}%
% end of encapsulating


%%%%%%%%%%%%%%%%%%%%%%%%%%%%%%%%%%%%%%%%%%%%%%%%%%%%%%%%%%%%%%%%%%%%%%%
%
%    Our glossary entries are given in the format
%    \glossaryentry{$B$&Borel subgroup}{page number}
%    which would be generated by \glossary{$B$&Borel subgroup}
%    and written into the filename.glo file.
%    So the & separates the formula from its explanation.
%    We assume the explanation fits on the line.
%
%    Our implementation:

\def\itemamp#1&#2{%
   \mbox{%
      \hbox to 10em{% chktex 1
         #1%        the formula
         \hss%
      }%
   }%
     #2%        its explanation
}%

\def\glossaryentry#1#2{%
   {\itemamp #1{\unskip, #2}% chktex 1
   }\par
}%

\newenvironment{theglossary}{\parindent0pt\parskip0pt}{}

% \chapter*{Glossary of Notations}

% \begin{theglossary}\addcontentsline{toc}{chapter}{Glossary of Notations}

% % Now we copy from filename.glo, making changes as appropriate.
% % In this example we just changed the order of the entries.

% \glossaryentry{$H^0(X,\mathcal L)$&the sections of the vector bundle $\mathcal
% L
% |_X$ over $X$}{\glopageref{4}}

% \glossaryentry{$\mathrm{Ext}^i_B(M,N)$&Ext group in the category $\mathcal
% C_B$}
% {\glopageref{2}}

% \end{theglossary}

\indexincontents% always put this before the index.
\printindex
\end{document}

