\RequirePackage[l2tabu, orthodox]{nag}
\documentclass[10pt, a4paper]{article}
\synctex=1
\usepackage{iftex}

\iftutex%
\usepackage{fontspec}
\else
\usepackage[utf8]{inputenc}
\usepackage[T1]{fontenc}
\fi
\usepackage{fancyvrb}


%%\pdfinfoomitdate1
%\pdfsuppressptexinfo-1
%\pdftrailerid{}


\usepackage{ifthen}
\newboolean{texFhtLoaded}
\setboolean{texFhtLoaded}{false}

\usepackage{booktabs}
\aboverulesep=0ex
\belowrulesep=0ex

\usepackage[aux]{rerunfilecheck}

% graphics 
\makeatletter

\ifpdf%
\usepackage{xcolor}  % [pdftex]
\usepackage{graphicx}% [pdftex] 
\usepackage{hyperref}% driver [hpdftex] is autodetected 

\ifLuaTeX%
% for lualatex
\pdfvariable minorversion=7% chktex 1
% omit CreationDate and ModDate keys.
\pdfvariable suppressoptionalinfo 767% chktex 1
% no adding to the trailer dictionary.
\pdfvariable trailerid{}% chktex 1
\pdfvariable suppressoptionalinfo -1% chktex 1
\else
\ifXeTeX%
% for xelatex
\special{pdf:minorversion 7}
% TBD: find a way to express pdfinfoomitdate: necessary? 
\special{pdf:trailerid []}
\else
\ifPDFTeX%
\pdfminorversion=7         % for pdflatex
% omit CreationDate and ModDate keys.
% not before pdfTeX 3.14159265-2.6-1.40.17
\pdfinfoomitdate=1                   % for pdflatex
% no adding to the trailer dictionary.
%\pdftrailer=0                        % for pdflatex
\pdftrailerid{}                       % for pdflatex
\pdfsuppressptexinfo=-1               % for pdflatex
\else
% Here, the tex processor is unknown. 
\fi%pdftex
\fi%xetex
\fi%luatex



\hypersetup{
  pdfinfo={
    Author      ={Ernst Reissner},
    Title       ={The xfig file format for xfig 3.2},
    CreationDate={unknown},
    ModDate     ={unknown},
    Producer    ={unknown},
    Subject     ={xfig files},
    Keywords    ={LaTeX;xfig}
  }
}


\else
\usepackage[dvipdfmx]{xcolor}
\usepackage[dvipdfmx]{graphicx}
\@ifpackageloaded{tex4ht}{%
  % \usepackage[tex4ht]{hyperref}
  \setboolean{texFhtLoaded}{true}
}{%
%\usepackage[dvipdfmx]{hyperref}
\ifXeTeX%
\usepackage{hyperref}
\else
\usepackage[dvipdfmx]{hyperref}
% lualatex: without [dvipdfmx] option did not find converter dvi to pdf or to ps
% pdflatex: without [dvipdfmx] option dvips still works, but no converter for pdf
% with [dvipdfmx] all works fine 
\fi%
}
%\usepackage{bmpsize}% not for xelatex; lualatex and newer pdflatex
\fi

\usepackage[nottoc, numindex, numbib]{tocbibind}

%\usepackage{latex-bnf}
\title{The xfig file format for xfig 3.2 }
\author{Ernst Reissner (rei3ner@arcor.de)}
\date{}

\begin{document}

\maketitle
\tableofcontents
%\listoftables

\section{Introduction}

This document is created with \texttt{lualatex} or that like 
with output format 
\ifpdf%
pdf%
\else
dvi%
\fi.
The package \texttt{tex4ht} 
is \ifthenelse{\boolean{texFhtLoaded}}{}{not} loaded. 

This document describes the grammar 
of the native file format of the xfig graphic program version 3.2. 
It is based on xfig.3.2/Doc/FORMAT3.2 in xfig 3.2 source distribution. 
The file extension is \texttt{.fig} and the file-format is human-readable. 

If a line contains \texttt{\#}, 
the rest of the line is ignored as regards the 
Lines starting with \texttt{\#} are comment lines and are ignored. 
Also, xfig does not insert comments, 
except the first line as described in Section~\ref{sec:header}. 
On the other hand, if one writes comments, e.g.~with an editor, 
these are essentially preserved when xfig reads the file and writes it 
(well, if the first character of a comment is not a blank, 
a blank is added, but that is all what is changed). 
Note that xfig accepts lines with trailing comments, 
but does not preserve these kinds of comments. 
Also, comments with leading blanks are not allowed. 
So comments occur line wise. 

Although the format is human-readable, 
it is not designed to make it easy to interpret the content 
and to obtain an overview. 
Thus, comments may be valuable aid to give a notion 
of the intent behind the various elements contained. 

As comments, also empty lines are ignored, 
but xfig does not preserve them when writing back the file. 

%FIXME: what if a file is read and saved again? Comments are preserved?? 
% what about empty lines? 
Each xfig-file starts with a header as described in Section~\ref{sec:header} 
followed by a (typically non-empty) sequence of object descriptions. 
The header specifies general attributes which apply to the whole document, 
whereas the object descriptions refer to individual objects. 

For each type of object, there is a separate grammar, 
described in Section~\ref{sec:obj} and subsections. 
There are elements occurring for more than one type of object. 
To avoid repetition in the description, 
these elements are described separately in Section~\ref{sec:param} 
and referenced from within Section~\ref{sec:obj}. 

Note that the header may be interpreted line wise 
and also both objects and their parameters are described as lines. 
Apart from comment lines, the entries of each line are separated by blanks. 
When reading a fig-file, 
xfig interprets a string of subsequent blanks like one blank, 
except for text strings in text objects (see Section~\ref{subsec:text}). 
Consequently, writing back, each such maximal sequence of blanks 
is replaced by a single blank. 
Trailing blanks are not preserved. 

Leading blanks do not occur in the header. 
Lines with leading blanks in the header are ignored. 
Thus, e.g.~the orientation line 
``\texttt{\ \ \ Landscape}'' is replaced by ``\texttt{Portrait}'', 
which demands the default orientation. 

Whereas each attribute given in the header is described by a single line, 
for various object types a single line is not sufficient. 
Each object description may start with an indefinite number of comment lines. 
The first mandatory line of an object description is called the \emph{headline}. 
The headline does not start with a blank. 
When reading subsequent lines referring to the same object 
xfig accepts a non-empty sequence of blanks, 
but when writing, such lines start with a single tab character, 
in some cases followed by a single blank. 

There is a single exception to this rule: 
Compound objects as described in Section~\ref{subsec:compound} 
which turns objects described by subsequent lines 
into a unique object described by a 
start line and an end line specific for that compound. 
Both, start line and end line does not allow a leading blank. 


\subsection{X-Splines and backwards-compatibility}

Splines are always X-splines 
which allows the user to mix interpolation and approximation 
points in a same curve. 
More precisely, it means that an X-spline curve 
is neither an interpolated spline nor an approximated one, 
it is BOTH (the behavior of each point 
is controlled by one single parameter called ``shape factor''). 
For additional information about X-splines, see~\cite{XSpline}. 

Caveat: Because spline models of previous versions 
(quadratic B-splines and Bezier with hidden points) 
are no longer supported, 
curves that are present in version 3.1 and older files 
are automatically converted to X-splines. 
This translation is only an approximation process. 
It means that the converted curves 
are not exactly the same as the original ones. 
Though the translation usually provides almost identical curves, 
some hand-fitting may be needed in some pathological cases. 


\section{Header}\label{sec:header}

The very first line is comment line containing the name and version: 
%
\begin{verbatim}
\#FIG 3.2  Produced by xfig version 3.2.5c 
\end{verbatim}


The subsequent non-comment lines of the header are the following 
(in the order given): \\
%
\begin{tabular}{lll}
\toprule
type & name & Possible values or explanation \\
\midrule
\midrule
string & Orientation             & `Landscape'|`Portrait' \\
string & Justification           & `Center' or `Flush Left' \\
string & Units                   & `Metric' or `Inches' \\
string & PaperSize               & `Letter'|`Legal'|`Ledger'|`Tabloid'|\\
&&                                 `A'|`B'|`C'|`D'|`E'|\\
&&                                 `A4'|`A3'|`A2'|`A1'|`A0'|`B5' \\
float  & Magnification           & export and print magnification, \% \\
string & MultiplePage           & `Single' or `Multiple' pages \\
int    & TransparentColor       & color number for transparent color \\
&&                                 for GIF export. \\
&&                                -2=None, -1=background, \\
&&                                0--31 for standard colors or \\
&&                                32-- for user colors \\
string & [Comment]              & optional comment \\%FIXME: how? #? 
int    & Resolution coord-system & Fig units/inch and coordinate system:\\
&&                                 1: origin at lower left corner \\
&&                                    (NOT USED)\\
&&                                 2: upper left \\
\bottomrule
\end{tabular}



 Fig `resolution' is the resolution of the figure in the file.
    Xfig will always write the file with a resolution of 1200ppi so it
    will scale the figure upon reading it in if its resolution is different
    from 1200ppi.  Pixels are assumed to be square.

    Xfig will read the orientation string and change the canvas to match
    either the Landscape or Portrait mode of the figure file.

    The specification of the units is self-explanatory.

    The coordinate-system variable is ignored --- the origin is ALWAYS the
    upper-left corner.

    ** Coordinates are given in `fig\_resolution' units.
    ** Line thicknesses are given in 80-ths of an inch (`display units').
    ** dash-lengths/dot-gaps are given in 80-ths of an inch.


%\begin{grammar} 
%\end{grammar}
\begin{Verbatim}[fontsize=\scriptsize]
https://hackage.haskell.org/package/fig-1.4.0/docs/Graphics-Fig-Syntax.html
\end{Verbatim}





Orientation: (`Landscape'|`Portrait')

Justification: (`Center'|`FlushLeft')

Units: (`Metric'|`Inches')





Transparent:
Background	 
None	 
TransparentDefault	 
Transparent ColorSpec	 


CoordinateSystem:
LowerLeft	 
UpperLeft	 
(both are integers)

Commented:
Comment [String] a	 


Color:
color\_number:: Integer
color\_rgb\_values:: String


Documentation

data Fig
Source

Fig	 
figHeader:: Header
figColors:: [Commented Color]
figObjects:: [Commented Object]


\section{Objects}\label{sec:obj}


The rest of the file contains various objects.  
An object can be one of six classes (or types) with according \texttt{ObjectCode}.
%
\begin{enumerate}
\item[0]
Color pseudo-object (see Section~\ref{subsec:colorPseudoObj}). 
\item
Ellipse (see Section~\ref{subsec:ellipse}) which is a generalization of circle.
\item
Polyline (see Section~\ref{subsec:polyline}) which includes polygon and box.
\item
Spline (see Section~\ref{subsec:spline}) 
which may be closed, open approximated, interpolated or an x-spline.
\item
Text (see Section~\ref{subsec:text}).
\item
Arc (see Section~\ref{subsec:arc}).
\item
Compound object (see Section~\ref{subsec:compound}) 
which is composed of one or more objects. 
\end{enumerate}

In the following elaboration on object formats, 
every value of fig output are separated by blank characters 
or new line (`\textbackslash{}n').  
The value of the unused parameters will be -1.

 Some fields are described as `enumeration type' or `bit vector'; the
 values which these fields can take are defined in the header file object.h.
 The pen\_style field is unused.
 These values may be defined in some future version of Fig.


\setcounter{subsection}{-1}
\subsection{Color Pseudo-Object}\label{subsec:colorPseudoObj}

Color Pseudo-objects (user-defined colors) 
define color numbers in terms of rgb-values. 
This is used to define arbitrary colors beyond the 32 standard colors. 
See also Section~\ref{subsec:paramColorsFill}. 
The color objects must be defined before any other Fig objects.

\noindent
Headline: \\
%
\begin{tabular}{lll}
\toprule
type & name & Possible values or explanation \\
\midrule
\midrule
int        & ObjectCode  & always 0 (see Section~\ref{sec:obj}) \\
int        & ColorNumber & color number, from 32--543 (512 total)\\
hex-string & RgbValues   & red, green and blue values (e.g.~\#330099) \\
\bottomrule
\end{tabular}

The \texttt{Headline} ist the sole line defining a Color Pseudo-Object. 


\subsection{Ellipse including Circle}\label{subsec:ellipse}

Defines an ellipse in terms of center and vertex of the bounding box 
or in terms of two diagonal vertices of the bounding box. 
Defines also a circle in terms of center and a point on the circle 
or in terms of two points on the circle. 

\noindent
The \texttt{Headline} following now is the sole line defining an ellipse. \\
%
\begin{tabular}{lll}
\toprule
Type  & name                & Possible values or explanation \\
\midrule
\midrule
int   &  ObjectCode         & always 1 (see Section~\ref{sec:obj}) \\
int   &  SubType            & 1: ellipse defined by ``radiuses'' \\
&&                            2: ellipse defined by ``diameters'' \\
&&                            3: circle defined by radius \\
&&                            4: circle defined by diameter \\
int   &  LineStyle          & see Section~\ref{subsec:styDep} \\
int   &  LineThickness      & multiples of 1/80 inch \\
int   &  PenColor           & see Section~\ref{subsec:paramColorsFill} \\
int   &  FillColor          & see Section~\ref{subsec:paramColorsFill} \\
int   &  Depth              & see Section~\ref{subsec:styDep} \\
int   &  PenStyle           & -1, not used \\
int   &  AreaFill           & see Section~\ref{subsec:paramColorsFill} \\
float &  StyleVal           & multiples of 1/80 inch, 
                              see Section~\ref{subsec:styDep}  \\
int   &  Direction          & always 1 \\
float &  Angle              & radians, the angle of the x-axis \\
int   &  center-x, center-y & Fig units \\
int   &  radius-x, radius-y & Fig units (same for circles) \\
int   &  start-x, start-y   & Fig units; the 1st point entered \\
int   &  end-x, end-y       & Fig units; the last point entered \\
\bottomrule
\end{tabular}


\noindent
For further explanations on start-x/y and end-x/y: \\
%
\begin{tabular}{lll}
\toprule
SubType  & start-x, start-y  & end-x, end-y \\
\midrule
\midrule
1 & the center            & some vertex of the BB \\
2 & a vertex of the BB    & opposite vertex of the BB \\
3 & the center            & a point on the circle \\
4 & a point on the circle & opposite point on the circle \\
\bottomrule
\end{tabular}

Ellipse	 

ellipse\_common:: Common
ellipse\_direction:: Integer
ellipse\_angle:: Double
ellipse\_center\_x:: Integer
ellipse\_center\_y:: Integer
ellipse\_radius\_x:: Integer
ellipse\_radius\_y:: Integer
ellipse\_start\_x:: Integer
ellipse\_start\_y:: Integer
ellipse\_end\_x:: Integer
ellipse\_end\_y:: Integer

\subsection{Polygonal line}\label{subsec:polyline}

Defines various kinds of polygonal lines, both open and closed. 
As special cases also boxes and arc-boxes, i.e.~boxes with rounded vertices. 
As a special case also a box which serves as bounding box for a picture. 
In this case, this polygonal line finally defines the picture itself. 
See \texttt{SubType} in the Headline below. 

\noindent
Headline: \\
%
\begin{tabular}{lll}
\toprule
type   & name          & Possible values or explanation \\
\midrule
\midrule
int    & ObjectCode    & always 2 (see Section~\ref{sec:obj}) \\
int    & SubType       & 1: polyline (open) \\
&&                       2: box \\
&&                       3: polygon (closed, regular or not) \\
&&                       4: arc-box (box with round vertices) \\
&&                       5: imported-picture bounding-box \\
int    & LineStyle     & see Section~\ref{subsec:styDep} \\
int    & LineThickness & multiples of 1/80 inch \\
int    & PenColor      & see Section~\ref{subsec:paramColorsFill} \\
int    & FillColor     & see Section~\ref{subsec:paramColorsFill} \\
int    & Depth         & see Section~\ref{subsec:styDep} \\
int    & PenStyle      & -1, not used \\
int    & AreaFill      & see Section~\ref{subsec:paramColorsFill} \\
float  & StyleVal      & multiples of 1/80 inch, \\
&&                       see Section~\ref{subsec:styDep} \\
int    & JoinStyle     & 0: Miter (the default in xfig 2.1 and earlier) \\
&&                       1: Bevel \\
&&                       2: Round \\
int    & CapStyle      & see Section~\ref{subsec:styDep}, \\
&&                       only used for subtype POLYLINE \\
int    & Radius        & radius of arc-boxes as multiples of 1/80 inch \\
int    & ForwardArrow  & 0: off, 1: on \\
int    & BackwardArrow & 0: off, 1: on \\
int    & NPoints       & number of points in line \\
\bottomrule
\end{tabular}

\noindent
After the Headline follows 
%
\begin{itemize}
\item
the \texttt{ForwardArrowLine}  only if \texttt{ForwardArrow}  is not 0 and 
\item
the \texttt{BackwardArrowLine} only if \texttt{BackwardArrow} is not 0; 
\end{itemize}
%
both described in Section~\ref{subsec:paramArrowLines}. 


Only if \texttt{SubType} in the Headline is 5, 
follows a \texttt{PicLine} with the following form:
%
\begin{tabular}{lll}
\toprule
type    & name        & Possible values or explanation \\
\midrule
\midrule
boolean & Orientation & 0: normal \\
&&                      1: flipped at diagonal \\
string  & File        & name of picture file to import \\
\bottomrule
\end{tabular}

Then follows a \texttt{PointsLine} with a number of coordinates 
given by \texttt{NPoints} in the first line. 
If the \texttt{SubType} is 5, the \texttt{PointsLine} 
defines the boundary of the picture and also rotation. 
The form of the \texttt{PointsLine} is given 
in Section~\ref{subsec:paramPointsLine}. 
This is the same as for Polylines described in Section~\ref{subsec:spline}. 



\subsection{Spline }\label{subsec:spline}

Defines various kinds of splines, 
open or closed, interpolated or approximated or 
something in between as x-spline. 

\noindent
Headline: \\
%
\begin{tabular}{lll}
\toprule
type  & name          & Possible values or explanation \\
\midrule
\midrule
int   & ObjectCode    & always 3 (see Section~\ref{sec:obj}) \\
int   & SubType       & 0: opened approximated spline \\
&&                      1: closed approximated spline \\
&&                      2: opened interpolated spline \\
&&                      3: closed interpolated spline \\
&&                      4: opened x-spline \\
&&                      5: closed x-spline \\
int   & LineStyle     & see Section~\ref{subsec:styDep} \\
int   & LineThickness & multiples of 1/80 inch \\
int   & PenColor      & see Section~\ref{subsec:paramColorsFill} \\
int   & FillColor     & see Section~\ref{subsec:paramColorsFill} \\
int   & Depth         & see Section~\ref{subsec:styDep} \\
int   & PenStyle      & -1, not used \\
int   & AreaFill      & see Section~\ref{subsec:paramColorsFill} \\
float & StyleVal      & multiples of 1/80 inch, see Section~\ref{subsec:styDep} \\
int   & CapStyle      & Section~\ref{subsec:styDep}, only used for open splines \\
int   & ForwardArrow  & 0: off, 1: on \\
int   & BackwardArrow & 0: off, 1: on \\
int   & NPoints       & number of control points in spline \\
\bottomrule
\end{tabular}

\noindent
After the first line follows 
%
\begin{itemize}
\item
the \texttt{ForwardArrowLine}  only if \texttt{ForwardArrow}  is not 0 and 
\item
the \texttt{BackwardArrowLine} only if \texttt{BackwardArrow} is not 0; 
\end{itemize}
%
both described in Section~\ref{subsec:paramArrowLines}. 

Then follows a \texttt{PointsLine} with a number of coordinates 
given by \texttt{NPoints} in the first line. 
The form of the \texttt{PointsLine} is given 
in Section~\ref{subsec:paramPointsLine}. 
This is the same as for Polylines described in Section~\ref{subsec:polyline}. 

Finally, and unlike for Polylines, there comes the \texttt{ControlPointsLine}: 

There is one shape factor for each point in the \texttt{PointsLine}, 
i.e.~\texttt{NPoints} shape factors. 
The value of this factor must be between 
%
\begin{itemize}
\item[-1] which means that the spline is interpolated at this point  and 
\item[+1] which means that the spline is approximated at this point. 
\end{itemize}
%
The spline is always smooth in the neighbourhood of a control point, 
except when the value of the factor is 0 
for which there is a first-order discontinuity (i.e.~angular point).


\subsection{Text}\label{subsec:text}

A Text object, defines a text element 
with text, font, font size and many other attributes. 
Note that horizontal justification is defined, 
but not vertical justification. 

\noindent
Headline: \\
%
\begin{tabular}{lll}
\toprule
type   & name       & Possible values or explanation \\
\midrule
\midrule
int    & ObjectCode & always 4 (see Section~\ref{sec:obj}) \\
int    & SubType    & 0: Left justified \\ %better: Justification 
&&                    1: Center justified \\
&&                    2: Right justified \\
int    & PenColor   & see Section~\ref{subsec:paramColorsFill} \\
int    & Depth      & see Section~\ref{subsec:styDep} \\
int    & PenStyle   & -1, not used \\
int    & Font       & see below \\
float  & FontSize   & font size in points \\
float  & Angle      & radians, the angle of the text \\
int    & FontFlags  & three bit vector (see below) \\
float  & height     & Fig units \\
float  & length     & Fig units \\
int    & x, y       & Fig units, coordinate of the origin of the string.  \\
&&                    If SubType is 0/1/2, it is the \\
&&                    lower left/center/right corner of the string.\\
string & Text       & see below \\
\bottomrule
\end{tabular}

The \texttt{Headline} ist the sole line defining a text object. 

\noindent
The bits of the \texttt{FontFlags} field are defined as follows:
%
\begin{enumerate}
\item[0]
Rigid text: text doesn't scale when scaling compound objects
\item
Special text: interpret test as for \LaTeX-code. 
If this is set, the following flag affects the view on the xfig GUI 
but not external view, e.g.~when printing or exporting. 
\item
PostScript font: interpret \texttt{Font} above 
as Postscript font (otherwise as \LaTeX{}-font) 
Details follow below. 
\item
Hidden text: whether the text is hidden in the xfig GUI\@. 
This does not affect the external view, e.g.~printing and export. 
\end{enumerate}

The \texttt{Font} field is immaterial, if the Special flag above is set. 
Otherwise, its meaning depends on the PostScrpt/\LaTeX-flag 
described above: 

For \texttt{FontFlags} bit 2 = 0, 
\texttt{Font} is interpreted as \LaTeX-font with the following encoding: 
%
\begin{enumerate}
\item[0] \LaTeX{} Default font
\item    Roman
\item    Bold
\item    Italic
\item    Sans Serif
\item    Typewriter
\end{enumerate}

For \texttt{FontFlags} bit 2 = 1, 
\texttt{Font} is interpreted as PostScript-font with the following encoding: 

\begin{enumerate}
\item[-1] PostScript Default font
\item[0]  Times Roman
\item     Times Italic
\item     Times Bold
\item     Times Bold Italic
\item     AvantGarde Book
\item     AvantGarde Book Oblique
\item     AvantGarde Demi
\item     AvantGarde Demi Oblique
\item     Bookman Light
\item     Bookman Light Italic
\item     Bookman Demi
\item     Bookman Demi Italic
\item     Courier
\item     Courier Oblique
\item     Courier Bold
\item     Courier Bold Oblique
\item     Helvetica
\item     Helvetica Oblique
\item     Helvetica Bold
\item     Helvetica Bold Oblique
\item     Helvetica Narrow
\item     Helvetica Narrow Oblique
\item     Helvetica Narrow Bold
\item     Helvetica Narrow Bold Oblique
\item     New Century Schoolbook Roman
\item     New Century Schoolbook Italic
\item     New Century Schoolbook Bold
\item     New Century Schoolbook Bold Italic
\item     Palatino Roman
\item     Palatino Italic
\item     Palatino Bold
\item     Palatino Bold Italic
\item     Symbol
\item     Zapf Chancery Medium Italic
\item     Zapf Dingbats
\end{enumerate}
      
The \texttt{Text} field above is a sequence of ASCII characters. 
It starts after a \emph{single} blank character 
which separates the coordinate field \texttt{x, y} 
from the \texttt{Text} field. 
This allows to specify texts with leading blanks. 
The text field ends before the sequence ``\texttt{\textbackslash{}001}''; 
the latter sequence is not part of the text field. 
Characters above octal 177 are represented by ``\texttt{\textbackslash{}xxx}'' 
where \texttt{xxx} is the octal value. 
This permits fig files to be edited with 7-bit editors 
and sent by e-mail without data loss. 
Note that the \texttt{Text} field may contain ``\texttt{\textbackslash{}n}''. 
To specify a literal backslash write ``\texttt{\textbackslash\textbackslash}''. 
The latter is vital for including \LaTeX{} in fig-figures. 




ColorSpec: see Section~\ref{subsec:paramColorsFill}
Font:
Latex LatexFont	 
Ps PsFont	 


LatexFont:
LatexDefault	 
Roman	 
Bold	 
Italic	 
SansSerif	 
Typewriter	 


FontFlags:
hidden:: Bool
special:: Bool
rigid:: Bool





\subsection{Arc}\label{subsec:arc}

Defines an arc, i.e.~a segment of a circle by a start point, 
an intermediate point and an end point. 

\noindent
Headline: \\
%
\begin{tabular}{lll}
\toprule
type  & name               & Possible values or explanation \\
\midrule
\midrule
int   & ObjectCode         & always 5 (see Section~\ref{sec:obj}) \\
int   & SubType            & 0: closed (pie-wedge) \\
&&                           1: open ended  \\
int   & LineStyle          & see Section~\ref{subsec:styDep} \\
int   & LineThickness      & multiples of 1/80 inch \\
int   & PenColor           & see Section~\ref{subsec:paramColorsFill} \\
int   & FillColor          & see Section~\ref{subsec:paramColorsFill} \\
int   & Depth              & see Section~\ref{subsec:styDep} \\
int   & PenStyle           & -1, not used \\
int   & AreaFill           & see Section~\ref{subsec:paramColorsFill} \\
float & StyleVal           & multiples of 1/80 inch, see Section~\ref{subsec:styDep} \\
int   & CapStyle           & see Section~\ref{subsec:styDep} \\
int   & Direction          & 0: clockwise, 1: counterclockwise \\
int   & ForwardArrow       & 0: off, 1: on \\
int   & BackwardArrow      & 0: off, 1: on \\
float & center-x, center-y & center of the arc \\
int   & x1, y1             & Fig units, the 1st point the user entered \\
int   & x2, y2             & Fig units, the 2nd point \\
int   & x3, y3             & Fig units, the last point \\
\bottomrule
\end{tabular}

\noindent
After the first line follows 
%
\begin{itemize}
\item
the \texttt{ForwardArrowLine}  only if \texttt{ForwardArrow}  is not 0 and 
\item
the \texttt{BackwardArrowLine} only if \texttt{BackwardArrow} is not 0; 
\end{itemize}
%
both described in Section~\ref{subsec:paramArrowLines}. 


Arc ArcLine (Maybe Arrow) (Maybe Arrow)	 
Spline SplineLine (Maybe Arrow) (Maybe Arrow) [(Integer, Integer)] [Double]


SplineLine:
spline\_common:: Common
spline\_cap\_style:: CapStyle

Arrow:
arrowType:: ArrowType
arrowStyle:: ArrowStyle
arrowThickness:: Double
arrowWidth:: Double
arrowHeight:: Double


ArrowStyle:
HollowArrow	 
FilledArrow	 


ArrowType:
Stick	 
Closed	 
Indented	 
Pointed	 


 
\subsection{Compound}\label{subsec:compound}

Defines the composition of other objects, 
which may well be compounds themselves. 

\noindent
Headline: \\
%
\begin{tabular}{lll}
\toprule
type & name & Possible values or explanation \\
\midrule
\midrule
int & ObjectCode          & always 6 (see Section~\ref{sec:obj}) \\
int & UpperRight-corner-x & Fig units \\
int & UpperRight-corner-y & Fig units \\
int & LowerLeft-corner-x  & Fig units \\
int & LowerLeft-corner-y  & Fig units \\
\bottomrule
\end{tabular}
% FIXME: another information: not FigUnits, but integers 

Subsequent lines of the compound except the last one 
define a list of objects in the Compound
These objects may well be Compounds themselves. 
The last line is \texttt{-6}. 



CapStyle: see Section~\ref{subsec:styDep}








Common:
subType:: Integer
lineStyle:: LineStyle
lineThickness:: Integer
penColor:: ColorSpec
fillColor:: ColorSpec
depth:: Integer
penStyle:: Integer
areaFill:: AreaFill
styleVal:: Double

LineStyle:
LineStyleDefault	 
Solid	 
Dashed	 
Dotted	 
DashDotted	 
DashDoubleDotted	 
DashTripleDotted	 



AreaFill: (NoFill|Filled Integer|Pattern Integer)





\section{Parameters}\label{sec:param}

\subsection{Style and Depth}\label{subsec:styDep}

The \texttt{CapStyle} field is defined FOR LINES, OPEN SPLINES and ARCS only 
used in ArcLine, SplineLine, PolylineLine
and is encoded as follows:
%
\begin{enumerate}
\item[0] Butt (the default in xfig 2.1 and earlier)
\item    Round
\item    Projecting
\end{enumerate}

\noindent
The \texttt{LineStyle} field is encoded as follows:
%
\begin{enumerate}
\item[-1] Default
\item[0]  Solid
\item     Dashed
\item     Dotted
\item     Dash-dotted
\item     Dash-double-dotted
\item     Dash-triple-dotted
\end{enumerate}

\noindent
The \texttt{StyleVal} field is defined as 
%
\begin{itemize}
\item
the length, in 1/80 inches, of the on/off dashes for dashed lines, and 
\item
the distance between the dots, in 1/80 inches, for dotted lines.
\end{itemize}

\noindent
The \texttt{Depth} field is defined as follows:

         0 \ldots 999 where larger value means object is deeper than (under)
                   objects with smaller depth

\subsection{Arrow lines}\label{subsec:paramArrowLines}

Forward arrow line (Optional; absent if ForwardArrow is 0): \\
%
\begin{tabular}{lll}
\toprule
type & name & Possible values or explanation \\
\midrule
\midrule
int    & ArrowType      & see below \\
int    & ArrowStyle     & see below \\
float  & ArrowThickness & multiples of 1/80 inch \\
float  & ArrowWidth     & Fig units \\
float  & ArrowHeight    & Fig units \\
\bottomrule
\end{tabular}

Backward arrow line (Optional; absent if backward-arrow is 0): \\
%
\begin{tabular}{lll}
\toprule
type & name & Possible values or explanation \\
\midrule
\midrule
int    & ArrowType      & see below \\
int    & ArrowStyle     & see below \\
float  & ArrowThickness & multiples of 1/80 inch \\
float  & ArrowWidth     & Fig units \\
float  & ArrowHeight    & Fig units \\
\bottomrule
\end{tabular}

The \texttt{ArrowType} field is defined for LINES, ARCS and OPEN SPLINES only as follows:
%
\begin{enumerate}
\item[0] Stick-type (the default in xfig 2.1 and earlier)
\item    Closed triangle:
                % |\
                % |  \
                % |    \
                % |    /
                % |  /
                % |/
 \item   Closed with `indented' butt:
                % |\
                % \  \
                %  \   \
                %   \    \
                %   /    /
                %  /   /
                % /  /
                % |/
\item    Closed with `pointed' butt:
                %    |\
                %   /   \
                %  /      \
                % /         \
                % \         /
                %  \      /
                %   \   /
                %    |/
\end{enumerate}


The \texttt{ArrowStyle} field is defined for LINES, ARCS and OPEN SPLINES only 
as follows:
%
\begin{enumerate}
\item[0] Hollow (actually filled with white)
\item    Filled with \texttt{PenColor} defined in Section~\ref{subsec:paramColorsFill}
\end{enumerate}


\subsection{PointsLine}\label{subsec:paramPointsLine}

\noindent
The form of the \texttt{PointsLine} is as follows: \\
%
\begin{tabular}{lll}
\toprule
type    & name             & Possible values or explanation \\
\midrule
\midrule
int    & x1, y1            & Fig units \\
int    & x2, y2            & Fig units \\
\vdots & \vdots            &  \vdots    \\
int    & xnpoints ynpoints & this will be the same as the 1st point \\
       &                   & if the described line is closed \\
       &                   & and in particular if it is the outline of a picture \\
\bottomrule
\end{tabular}

\subsection{Colors and Filling}\label{subsec:paramColorsFill}

\noindent
Both \texttt{PenColor} and \texttt{FillColor} are encoded as follows: 
%
\begin{enumerate}
\item[-1]    Default
\item[0]     Black
\item        Blue
\item        Green
\item        Cyan
\item        Red
\item        Magenta
\item        Yellow
\item        White
\item[ 8--11] four shades of blue (dark to lighter)
\item[12--14] three shades of green (dark to lighter)
\item[15--17] three shades of cyan (dark to lighter)
\item[18--20] three shades of red (dark to lighter)
\item[21--23] three shades of magenta (dark to lighter)
\item[24--26] three shades of brown (dark to lighter)
\item[27--30] four shades of pink (dark to lighter)
\item[31]    Gold
\item[32--543] (512 total) are user colors and
              are defined in color pseudo-objects 
              described in Section~\ref{subsec:colorPseudoObj}. 
              Only those indices may be used 
              which are previously defined in some color pseudo-object. 
\end{enumerate}

The field \texttt{AreaFill} is defined depending on the value of \texttt{FillColor} 
(which is defined for all kinds of objects 
for which \texttt{AreaFill} is defined). 


For all \texttt{FillColor}s except \texttt{Black} or \texttt{Default} and \texttt{White}, 
the \texttt{AreaFill} pattern is defined as follows: 
%
\begin{enumerate}
\item[-1]    not filled
\item[0]     black
\item[1--19]  `shades' of the \texttt{FillColor}, from darker to lighter.
             A shade is defined as the color mixed with black
\item[20]    full saturation of the color
\item[21--39] `tints' of the color from the color to white.
             A tint is defined as the color mixed with white
\item[40]    white
\item[41]    30 degree left diagonal pattern
\item[42]    30 degree right diagonal pattern
\item[43]    30 degree crosshatch
\item[44]    45 degree left diagonal pattern
\item[45]    45 degree right diagonal pattern
\item[46]    45 degree crosshatch
\item[47]    bricks
\item[48]    circles
\item[49]    horizontal lines
\item[50]    vertical lines
\item[51]    crosshatch
\item[52]    fish scales
\item[53]    small fish scales
\item[54]    octagons
\item[55]    horizontal `tire treads'
\item[56]    vertical `tire treads'
\end{enumerate}

For \texttt{FillColor}s \texttt{Black} and \texttt{Default}, 
there are the following deviations: 
%
\begin{enumerate}
\item[-1]    not filled
\item[0]     white
\item[1--19]  `shades' of gray, from lighter to darker.
\item[20]    (full saturation of) black
\item[21--40] not used. 
\end{enumerate}

If the \texttt{FillColor} is \texttt{White} there are the following deviations 
from standard colors: 
%
\begin{enumerate}
\item[-1]    not filled
\item[0]     black
\item[1--19]  `shades' of gray, from darker to lighter.
\item[20]    (full saturation of) white
\item[21--40] not used. 
\end{enumerate}



used in Text and Common, 

ColorSpec:
ColorSpecDefault	 
Black	 
Blue	 
Green	 
Cyan	 
Red	 
Magenta	 
Yellow	 
White	 
Blue4	 
Blue3	 
Blue2	 
LtBlue	 
Green4	 
Green3	 
Green2	 
Cyan4	 
Cyan3	 
Cyan2	 
Red4	 
Red3	 
Red2	 
Magenta4	 
Magenta3	 
Magenta2	 
Brown4	 
Brown3	 
Brown2	 
Pink4	 
Pink3	 
Pink2	 
Pink	 
Gold	 
UserDefined Integer	 

\bibliographystyle{alpha}
\bibliography{../lit}{}% chktex 11 


\end{document}

%%% Local Variables:
%%% mode: latex
%%% TeX-master: t
%%% End:
