% !TEX root = manualLMP.tex

\chapter{Parameters resp. Settings}\label{chap:settings}

This section describes the parameters 
of both the ant-task and the maven-plugin. 
There are also general aspects, treated in Section~\ref{sec:genOnSettings}. 
% TBD: shall be integrated in this introduction maybe. 
% The advantage would be that so sections and tables are in a one to one relation. 
As this software mainly acts by invoking other converter and checker, 
most parameters refer to commands and options for invocations, 
but there are also parameters which cannot be associated to an individual invocation. 
Parameters referring to the conversion process or to checking 
as a whole are collected in Section~\ref{sec:settingsGen}. 
A special case is Section~\ref{sec:settingsGoalVrsInj} 
which collects the parameters for goals \texttt{vrs} and \texttt{inj}. 
All the other sections refer to one or more converters. 

The parameters are listed 
in Tables~\ref{tab:paramGen}~through~\ref{tab:paramDiffPdf} 
with names, default values and short explanations. 
Note that neither of the parameters is mandatory, 
as there are always valid default values. 

Each of the tables is described in a separate section, 
only the tables~\ref{tab:paramPythontex} and~\ref{tab:paramDePythontex} 
for \texttt{pythontex} and for \texttt{depythontex}, respectively, 
are collected in the single Section~\ref{sec:pythontex}. 

Table~\ref{tab:paramGen} 
shows parameters controlling the general conversion process 
described in detail in Section~\ref{sec:settingsGen}. 
These are directories with names \texttt{xxxDirectory} 
and further parameters not following a naming convention. 
The other tables show parameters after a certain naming scheme: 
Command names have the form \texttt{xxxCommand} 
and the parameter with the according options have the form \texttt{xxxOptions}. 
Here \texttt{xxx} represents a certain converter. 
This is one of 
%
\begin{itemize}
\item[fig2dev]
The converter of fig-files into mixed latex- and PDF-files. 
\item[gnuplot]
The converter of gnuplot-files into mixed latex- and PDF-files. 
\item[metapost]
The converter of MetaPost-files into mixed latex- and PDF-files.
\item[latex2pdf]
The converter of latex-files into PDF-files. 
\item[bibtex]
The creator of a bibliography from an aux-file.
\item[makeindex]
The makeindex utility creating an index. 
\item[makeglossaries]
The makeglossaries utility creating a glossary. 
\item[pythontex]
The utility to invoke python and other languages from within \LaTeX{} 
and to replace the code by its results dynamically. 
\item[depythontex]
The utility to replace code finally after a run of \texttt{pythontex}. 
\item[tex4ht]
The converter of latex into HTML and also into ODT, 
depending on the parameters. 
\item[latex2rtf]
The converter of latex-files into RTF-files. 
\item[odt2doc]
The converter of ODT-files into doc(x)-files. % chktex 36
\item[pdf2txt]
The converter of PDF-files into TXT-files. 
\item[chktex]
A code-checker converting in a sense a latex main file into a log-file 
containing errors, warnings and further messages. 
\item[diffPdf]
A diff tool comparing the PDF file created with an expected PDF file. 
This is relevant only if a PDF file has been created 
and if the comparison is activated, which is not true by default. 
\end{itemize}

It is a little more complicated 
with the parameters in Section~\ref{sec:settingsLatex2Html}. 

The command name and the list of options describes the invocation of the command. 
This \LaTeX{} builder supervises also the return value frequently and the log file is supervised. 

There are some parameters of the form \texttt{patternXxxYyy}, 
referring to a pattern in the log-file of the converter \texttt{Xxx} 
indicating an event \texttt{Yyy} which is one of the following: 
%
\begin{itemize}
\item[ReRun]
indicates that \texttt{Xxx} needs to be rerun.
\item[Err]
indicates that \texttt{Xxx} had an error. 
\item[Warn]
indicates that \texttt{Xxx} had a warning. 
\end{itemize}

Besides the abovementioned patterns, describing events in log files, 
there are further patterns. 
The maybe the most prominent one, \texttt{patternLatexMainFile}, 
is devoted a separate section, Section~\ref{subsec:patternLatexMainFile}. 
All patterns, i.e.\@ all parameters of the form \texttt{patternZZZ} 
are interpreted as regular expressions 
in a variant slightly generalizing the default implementation for java. 
We owe \href{https://github.com/florianingerl/com.florianingerl.util.regex}{description} 
and implementation to Florian Ingerl. 

Essentially, there are two kinds of parameters: 
Most are just passed to the converters invoked by this software. 
The parameters of this software 
are so that the choice of the converter, i.e.\@ the name of the application 
can be configured, 
and also each converter can be almost freely configured. 

Parameters not passed to an application, 
are either really crucial 
or are included to allow also development of latex files. 

\section{Generalities on parameters}\label{sec:genOnSettings}

As pointed out in the introduction of this chapter, 
this software acts mainly by invoking various converters. 
The converters are grouped in so-called \emph{categories}. 
The converters of a category have the same (file-)interface, %chktex 36
means read and write the same files and, 
mostly but not strictly necessarily, have the same options. 
For each category there is an option \texttt{xxxCommand}, 
where \texttt{xxx} is the name of the category in lowercase letters\footnote%
{In fact there are exceptions to this rule: E.g.\@ for category \texttt{LaTeX} 
the command is called \texttt{latex2pdf} referring to the common output format \gls{pdf}, 
although also \gls{dvi} and \gls{xdv} are possible}. 
The value of the option is the command to invoke the converter of the category. 
Also, there is a default converter in each category, 
and sometimes there is just a single converter possible. 
For example, \lualatex{} is the default converter in the category \texttt{latex}. 

This software knows about converters and registers the ones approved for this software. 
Among the advantages are, that it is ensured that the converter is really in that category 
and that this software checks whether a converter used is in the right category, 
and it checks whether it is installed in an approved version. 
On the other hand, there are cases, where the user needs to invoke a custom converter. 
In this case, the command name shall be given in the form 
%
\begin{verbatim}
  <categoryCommand>commandName:category</categoryCommand>
\end{verbatim}
%
to make sure, that the user is really aware that the converter (s)he uses % chktex 36
is in the correct category, i.e.\@ has the required interface. 
Since neither of the registered converters has a \texttt{:} in its name, 
This form is identified by the occurrence of a colon. 
Since the categories neither have colons in their names, 
separation of command name and category is by the last colon occurring. 
That way, command names may contain colons also. 

For most categories of converters, in fact at the time of this writing with a single exception, 
one can specify command line options, specified in the form 
%
\begin{verbatim}
  <categoryOptions></categoryOptions>
\end{verbatim}
%
In fact, only for \texttt{diffPdfCommand} there is no option at all, 
and for some converters with more complex options, the options are split over more than one setting, 
e.g.\@ for converter category \texttt{fig2dev} converting \gls{fig}-files, 
there are general settings given by \texttt{fig2devGenOptions}
and settings specific for the output language: 
\LaTeX{} (\texttt{fig2devPtxOptions}) and \gls{eps} (\texttt{fig2devPdfEpsOptions}. 
In any case, options are trimmed, i.e.\@ leading and trailing white spaces are removed 
before being processed. 
There are cases, where the options as given are not directly passed to the converter 
but is further processed. 
In this case, the processing is documented. 

% log, patterns warn, err. 




\section{General parameters }\label{sec:settingsGen}

This section describes the general parameters given in Table~\ref{tab:paramGen}. 


%TODO\@: do this. 

\begin{longtable}{|ll|}
\toprule
Parameter        & Default  \\
\multicolumn2{|l|}{Explanation }  \\
\midrule
\midrule
\endfirsthead%
\bottomrule
\caption{\label{tab:paramGen} General parameters}
\endlastfoot%

\texttt{texSrcDirectory}  & \texttt {src/site/tex}  \\
\multicolumn2{|l|}{
\begin{minipage}{0.95\linewidth}
The latex source directory as a string 
relative to \texttt{\$baseDirectory}, 
containing \texttt{\$texSrcProcDirectory}. 
This directory determines also the subdirectory of 
\texttt{\$outputDirectory} to lay down the generated artifacts. 
The default value is ``\texttt{src/site/tex}'' on Unix systems. 
\end{minipage}
} \\
\texttt{texSrcProcDirectory}  & \texttt {.}  \\
\multicolumn2{|l|}{
\begin{minipage}{0.95\linewidth}
The latex source processing directory as a string 
relative to \texttt{\$texSrcDirectory}
containing all tex main documents 
and the graphic files to be processed 
and also to be cleaned. 
Whether this is done recursively in sub-folders 
is specified by \texttt{\$readTexSrcProcDirRec}. 
The default value is ``\texttt{.}''. 
\end{minipage}
} \\
\texttt{readTexSrcProcDirRec}  & \texttt{true}  \\
\multicolumn2{|l|}{
\begin{minipage}{0.95\linewidth}
Whether the tex source directory \texttt{\$texSrcProcDirectory} 
shall be read recursively for creation of graphic files, 
i.e.\@ including the sub-directories recursively. 
This is set to \texttt{false} only during development of documentation. 
%The default value is 'true'.
\end{minipage}
} \\
\texttt{outputDirectory}  & \texttt{.}             \\
\multicolumn2{|l|}{
\begin{minipage}{0.95\linewidth}
The generated artifacts will be copied to \texttt{outputDirectory}
relative to \texttt{\$targetSiteDirectory} 
which is by default '\texttt{\$targetDirectory/site}' on Unix systems. 
%The default value is '\texttt{.}'.  
\end{minipage}
} \\
\texttt{targets}          & \texttt{chk, pdf, html}     \\
\multicolumn2{|l|}{
\begin{minipage}{0.95\linewidth}
A comma separated list of targets without blanks to be stored in \texttt{\$targetSet}. 
Allowed values are \texttt{chk}, \texttt{dvi}, \texttt{pdf}, \texttt{html}, 
\texttt{odt}, \texttt{docx}, \texttt{rtf} and \texttt{txt}. 

The targets are mostly related to output formats. 
One exception is \texttt{chk} which represents a check, i.e.\@ linting of the source. 

While in general target \texttt{dvi} represents creation of output in DVI format, 
if \texttt{\$latex2pdfCommand} is set to \xelatex, 
the target \texttt{dvi} yields an output in extended DVI format, i.e.\@ in XDV\@. 
Also target \texttt{html} may represent creation of HTML files and of XHMTL files. 
Analogously \texttt{docx} corresponds with DOCX format by default, 
but can also be configured to mean DOC\@. 

Independent of the order given, the given targets are created 
in an internal ordering. 

CAUTION\@: These targets are the default targets for any latex main file, 
but depending on the document class, there may be further restrictions 
given by setting '\texttt{\$docClassesToTargets}'. 
Currently, only the class \texttt{beamer} used for presentations has restrictions. 
Moreover, these targets may be overwritten for individual \LaTeX{} main files 
using magic comments as described in Section~\ref{subsubsec:openingNoMagComm}. 

%The default value is '\texttt{chk, pdf, html}'. 
\end{minipage}
} \\
\texttt{convertersExcluded}          & empty     \\
\multicolumn2{|l|}{
\begin{minipage}{0.95\linewidth}
A comma separated list of excluded {Converter}s 
given by their command. 
Excluded converters need not be installed, but their names must be known. 
They don't show up in the version check of target \texttt{vrs} 
and of course they are not allowed to be used. 
% By default, this list is empty. 
\end{minipage}
} \\
\texttt{patternLatexMainFile} & see Section~\ref{subsec:patternLatexMainFile}\\
\multicolumn2{|l|}{
\begin{minipage}{0.95\linewidth}
The pattern to be applied to the beginning of the contents of TEX-files 
which identifies a \LaTeX{} main file, i.e.\@ a file to be compiled. 
If the file is really a \LaTeX{} main file, 
the pattern contributes to finding the targets for compilation. 
This may be done either directly via a magic comment 
or via the document class. 

The default value for the pattern is chosen to match quite exactly the start of 
the \LaTeX{} main files. 
Here we assume that the latex main file should contain 
the declaration ``\cmd{documentclass}'' 
or the old-fashioned ``\cmd{documentstyle}'' 
preceded by a few constructs and followed by the document class. 
Among the few constructs are also comments and in particular magic comments. 

Strictly speaking, a tight match is not necessary, 
only separation of \LaTeX{} main files from other files is 
and so is extraction of the document class. 
For a more thorough discussion, 
and for an alternative approach, consult the manual. 

Since the pattern is chosen 
according to documentation collected from the internet, 
one can never be sure whether the pattern is perfect. 

If the current default value is not appropriate, 
please overwrite it in the configuration 
and notify the developer of this plugin of the deficiency. 
In any case, matching of the group named \texttt{class} must be retained 
so that the document class is matched. 
\end{minipage}
} \\
\texttt{docClassesToTargets} & see description below  \\
\multicolumn2{|l|}{
\begin{minipage}{0.95\linewidth}
Assigns to document classes their allowed `\texttt{targets}'. 
The map expression is a list of chunks separated by a single blank. 
Each chunk is divided by a single colon 
in a comma separated list of document classes, 
and a comma separated list of targets. 

A chunk means that all given document classes are compiled for the given targets. 
Thus, the set of document classes may not be empty, 
i.e.\@ the colon may not be at the first place of its chunk. 
In contrast, a colon at the last place of a chunk indicates an empty target set, 
meaning that documents of the given class are not processed at all. 

The document classes of the chunks may not overlap. 
A document of a class is compiled for a target if this is specified so by a chunk. 

As a side effect, compilation of document classes cause warnings if not registered here. 
The default value consists of two chunks: 
\begin{itemize}
  \item
  `\small{\texttt{article,report,book,minimal:chk,dvi,pdf,html,odt,docx,rtf,txt}}' 
  ensures that article and book and others allow all targets. 
  \item
  `\small{\texttt{beamer,leaflet,scrlttr2:chk,pdf,txt}}' beamer 
  allows mainly \texttt{pdf} and derived from that \texttt{txt}.  
  Checking with \texttt{chk} does not depend on the document class. 
  Note that maybe leaflets or letters may work for DVI or XDV also, even for word formats and related, 
  we restrict ourselves to the given output for simplification. 
\end{itemize}

CAUTION\@: Due to a 
\href{https://issues.apache.org/jira/projects/MNG/issues/MNG-7927?filter=reportedbyme}{bug in maven}, 
setting this to the empty string is ignored. 

CAUTION\@: This setting is ignored, if targets are specified for individual \LaTeX{} main files 
using magic comments as described in Section~\ref{subsubsec:openingMagComm}. 
\end{minipage}
} \\
\texttt{mainFilesIncluded}          &  empty string        \\
\multicolumn2{|l|}{
\begin{minipage}{0.95\linewidth}
  The list of names of latex main files 
  without extension \texttt{.tex} separated by whitespace 
  which shall be included for creating targets, 
  except if this is empty in which cases all are included. 
  It is assumed that the names of the latex main files 
  do not contain whitespace. 
  Note that leading and trailing whitespace are trimmed. 
  Currently, names of latex main files 
  should better have pairwise different names, 
  even if in different directories. 
  The empty string is the default, i.e.\@ including all. 
  See parameter \texttt{mainFilesExcluded}. 
\end{minipage}
} \\
\texttt{mainFilesExcluded}          &  empty string        \\
\multicolumn2{|l|}{
\begin{minipage}{0.95\linewidth}
  The list of names of latex main files 
  without extension \texttt{.tex} 
  separated by whitespace 
  which shall be excluded for creating targets. 
  It is assumed that the names of the latex main files 
  do not contain whitespace. 
  Note that leading and trailing whitespace are trimmed. 
  Currently, names of latex main files 
  should better have pairwise different names, 
  even if in different directories. 
  Together with \texttt{mainFilesExcluded}, 
  this is used for document development 
  to build the PDF-files of a subset of documents 
  and e.g.\@ because for a site one needs all documents, 
  but with the software only the manual is shipped. 
  The empty string is the default, i.e.\@ excluding no file. 
  See parameter \texttt{mainFilesIncluded}. 
\end{minipage}
} \\
\texttt{latexmkUsage}          &  \texttt{NotAtAll}        \\
\multicolumn2{|l|}{
\begin{minipage}{0.95\linewidth}
The extent to which \tool{latexmk} or to be more precise, 
the command given by \texttt{latexmkCommand} is used to build. 
The following values for build strategy are allowed: 
%
\begin{description}
  \item[NotAtAll] \tool{latexmk} is not used at all. 
  \item[AsBackend] \tool{latexmk} is used as backend, 
  i.e.\@ graphic files are created as for goal \texttt{grp} 
  as in strategy \texttt{NotAtAll} before \tool{latexmk} is invoked 
  on the individual \LaTeX{} main files. 
  \item[Fully] build is by applying \tool{latexmk} 
  on the individual \LaTeX{} main files without any prior actions. 
\end{description}

This setting can be overwritten for individual latex main files 
by the magic comment \texttt{latexmkMagic} 
described in Section~\ref{subsec:patternLatexMainFile}. 

For a more detailed description of usage of \tool{latexmk} see Section~\ref{sec:latexmk}. 
\end{minipage}
} \\
\texttt{texPath}          &  empty string        \\
\multicolumn2{|l|}{
\begin{minipage}{0.95\linewidth}
Path to the TeX scripts or null. 
In the latter case, the scripts must be on the system path. 
Note that in the pom, \texttt{$<$texPath/$>$} 
and even \texttt{$<$texPath$>$\ \ \ \ $<$/texPath$>$} represent the null-File. 
The default value is null.
\end{minipage}
} \\
\texttt{cleanUp}             & \texttt{true}             \\
\multicolumn2{|l|}{
\begin{minipage}{0.95\linewidth}
Clean up the working directory in the end? 
May be used for debugging when setting \texttt{false}. 
%The default value is '\texttt{true}'. 
\end{minipage}
} \\
\texttt{patternCreatedFromLatexMain} & 
see Section~\ref{subsec:patternCreatedFromLatexMain} \\
\multicolumn2{|l|}{
\begin{minipage}{0.95\linewidth}
This pattern is applied to file names 
and matching shall accept all the files 
which were created from a latex main file `\texttt{xxx.tex}'. 
It is neither applied to directories 
nor to `\texttt{xxx.tex}' itself. 
It shall comprise neither graphic files to be processed 
nor files created from those graphic files. 

This pattern is applied 
in the course of processing graphic files 
to decide which graphic files should be processed 
(those rejected by this pattern) 
and to log warnings if there is a risk, 
that graphic files to be processed 
are skipped or that processing a latex main file overwrites 
the result of graphic preprocessing. 

When clearing the \LaTeX{} source processing directory 
\texttt{\$texSrcProcDirectory}, 
i.e.\@ all generated files should be removed, 
first those created from latex main files. 
As an approximation, 
those are removed which match this pattern. 

The sequence `\texttt{T\$T}' 
is replaced by the prefix `\texttt{xxx}'. 
The sequence `\texttt{T\$T}' must always be replaced: 
The symbol `\texttt{\$}' occurs as end-sign as `\texttt{)\$}' % chktex 9
or as literal symbol as `\texttt{\$}'. 
Thus, `\texttt{T\$T}' is no regular occurrence 
and must always be replaced with `\texttt{xxx}'. 
		 
Spaces and newlines are removed 
from that pattern before matching. 

This pattern may never be ensured to be complete, 
because any package 
may create files with names matching its own patterns 
and so any new package may break completeness. 
Nevertheless, the default value aims completeness 
while be tight enough not to match names of files not created. 
% TBD: link to injected header.tex also backlink: 
% if header.tex is injected, and no further package is loaded, 
% then this guarantees that this pattern is appropriate 

If the current default value is not appropriate, 
please overwrite it in the configuration 
and notify the developer of this plugin of the deficiency. 
\end{minipage}
} \\ % chktex 9 chktex 10 (bug) 
\end{longtable}


\subsection{The parameter \texttt{patternLatexMainFile}}%
\label{subsec:patternLatexMainFile}

Before reading the details given in this section, 
the user is advised to at least skim 
through Section~\ref{subsec:sourcesLatex} and~\ref{sec:outputFormats} 
for intuitive understanding. 

The regular expression pattern \texttt{patternLatexMainFile} 
matches exactly the files to be compiled, 
the so called \LaTeX{} main files, and for \LaTeX{} main files 
it extracts the following pieces of information, 
all by named capturing groups which have the form \texttt{(!<name>pattern)} 
but in the pom \texttt{<} and \texttt{>} must be escaped 
and so capturing groups take the form \texttt{(!\&lt;name\&gt;pattern)}. 
The capturing group named \texttt{docClass} extracts the document class 
from the command \cmd{documentclass}. 
If \texttt{patternLatexMainFile} matches, 
also the capturing group \texttt{docClass} matches, 
so that for latex main files the document class is always known. 
All other capturing groups are defined though magic comments. 
As the document class, they are directives specific to the given file 
on how to compile it. 
They override the general settings given in the pom. 
A magic comment may not match, 
which means that there is no according specific directive 
and so the general setting holds. 

Whether a capturing group must match or not, 
the regular expression pattern \texttt{patternLatexMainFile} 
must contain each of the following named capturing groups, 
because the software asks for it. 
Distinguish between the pattern and the matching strings: 
Whereas in the pattern all groups must be mentioned, 
a string may match without matching the group. 
Whereas \texttt{docClass} is mandatory, i.e.\@ the according group matches, 
the magic comments are all optional, i.e.\@ they need not match any part of the string. 

\begin{description}
  \item[docClass]
  the document class given by the command \cmd{documentclass}. 
  \item[programMagic]
  the \LaTeX{} converter to be used 
  specifically for the according document. 
  This is intended to be specified only 
  if the required converter for the given document deviates 
  from what is specified 
  globally as setting \texttt{latex2pdfCommand} described in 
  Table~\ref{tab:paramLatex2pdf} on page~\pageref{tab:paramLatex2pdf}. 
  \item[targetsMagic]
  the targets to be built. 
  This is intended to be specified only 
  if the targets for the given document deviate from what is specified 
  globally as setting \texttt{targets} and \texttt{docClassesToTargets}, 
  both given in 
  Table~\ref{tab:paramGen} on page~\pageref{tab:paramGen}. 
  \item[latexmkMagic, latexmkMagicVal] 
  whether \emph{for creating PDF files} \tool{latexmk} shall be used. 
  This is intended to run the build process with \tool{latexmk} 
  although the global setting \texttt{latexmkUsage} 
  given in Table~\ref{tab:paramGen} on page~\pageref{tab:paramGen} 
  may specify direct compilation without \tool{latexmk}. 

  The magic comment can take the form \texttt{\% !LMP latexmkMagic=<bool>} % chktex 26
  or \texttt{\% !LMP latexmkMagic} which is just short for \texttt{\% !LMP latexmkMagic=true}. % chktex 26

  \item[chkDiffMagic, chkDiffMagicVal]
  whether the \emph{created PDF file} shall be checked against an original 
  ensuring that it is correctly reproduced. 
  This is intended to control a check specific for this file 
  overwriting the general setting \texttt{chkDiff} 
  given in Table~\ref{tab:paramDiffPdf} on page~\pageref{tab:paramDiffPdf}. 

  The magic comment can take the form \texttt{\% !LMP chkDiff=<bool>} % chktex 26
  or \texttt{\% !LMP chkDiff} which is just short for \texttt{\% !LMP chkDiff=true}. % chktex 26
\end{description}

The default pattern for identifying \LaTeX{} main files 
and to extract the above pieces of information 
is given by Listing~\ref{lst:patternLatexMainFile}. 


\lstinputlisting[language=XML, basicstyle=\scriptsize,
numbers={left}, %stepnumber={5},
linerange={166-181}, % chktex 8 TBD: why is this not covered by .chktexrc? 
breaklines,
%float,
captionpos=b, label={lst:patternLatexMainFile},
caption={The default pattern of the latex main file in a form as in a pom configuration}]
{../../main/resources/rawPoms/pom4pdf.xml}% chktex 11 

% \begin{Verbatim}[fontsize=\scriptsize]
  % \A
  % (%\s*!\s*T[eE]X (TXS|spellcheck|encoding|root).*\R)*
  % (%\s*!\s*T[eE]X program\s*=\s*(?&lt;programMagic&gt;[^} ]+)\R)?
  % (%\s*!\s*T[eE]X .*\R)*
  % (%\s*!\s*LMP (?&lt;chkDiffMagic&gt;chkDiff)(=(?&lt;chkDiffMagicVal&gt;true|false))?\R)?
  % (%\s*!\s*LMP (?&lt;latexmkMagic&gt;latexmk)(=(?&lt;latexmkMagicVal&gt;true|false))?\R)?
  % (%\s*!\s*LMP targets=(?&lt;targetsMagic&gt;(\p{Lower}|,)+)\R)?
  % (\s*(
  % \\RequirePackage\s*(\[(\s|\w|[,=])*\])?\s*\{(\w|-)+\}\s*(\[(\d|[-./])+\])?|
  % \\PassOptionsToPackage\s*\{(\s|\w|[,=])*\}\s*\{(\w|-)+\}|
  % \\newbool\s*\{\w+\}|
  % \\setbool\s*\{\w+\}\{(true|false)\}|
  % \\DocumentMetadata(?&lt;brace&gt;\{(?:[^{}]|(?'brace'))*\})|
  % \\input\s*\{[^{}]*\}
  % )?\s*(%.*)?\R)*
  % \\(documentstyle|documentclass)\s*(\[[^]]*\])?\s*\{(?&lt;docClass&gt;[^} ]+)\}
% \end{Verbatim}

Let us trace through line by line: 
%
\begin{description}
  \item[1]
  The \cmd{A} indicates the start of the file. 
  \item[2--4]
  These lines match magic comments \texttt{\%~!TEX}, 
  which are used by other build tools also. 
  Line 3 extracts \texttt{programMagic} 
  from the first magic comment of the form \texttt{\%~!TEX program=\dots}. 
  This is the behavior of the other tools also. 
  The other lines are to skip information from magic comments \texttt{\%~!TEX} 
  which are not needed. 
  \item[5--7]
  These lines match magic comments of the form \texttt{\%~!LMP\dots} 
  which are specific for this software. 
  Like the above magic comments they are all optional, 
  but their ordering is fixed: 
  %
 \begin{description}
  \item[\texttt{chkDiffMagic}]
  to activate diffing to check reproduction,
  or in conjunction with \texttt{chkDiffMagicVal} to switch reproduction check.
  \item[\texttt{latexmkMagic}]
  to delegate build to \tool{latexmk}.
  \item[\texttt{targetsMagic}] 
  allows to specify a list of targets.
  This is the sole of these magic comments not only applying to creating PDF files.
\end{description}
  %  It defines optionally \texttt{docClassMagic} and 
  \item[8--11]
  This defines material which may precede the command \cmd{documentclass}, 
  except for magic comments and is the only one without magic comments. 
  Besides lines with specific commands, it matches empty lines and comment lines. 
  Also, a line may start with whitespace 
  and may contain a comand and end in a comment. 
  The commands specified there may occur in arbitrary multiplicity and order. 
  This section is likely to be modified by the user. 
  \item[11]
  Matches the command \cmd{documentclass} and extracts \texttt{docClass}. 
\end{description}



Between magic comments and 
\cmd{documentclass} or \cmd{documentstyle} 
only the following material is allowed: 
%
\begin{itemize}
\item
the command \cmd{RequirePackage} specifying packages to be loaded 
before \cmd{documentclass}, 
in contrast to \cmd{usepackage} which is used after, 
\item
the command \cmd{PassOptionsToPackage} 
allowing to pass one or more options to a package, 
although including with \cmd{usepackage} is without options, 
\item 
\cmd{newbool} and \cmd{setbool} to define and set a boolean value 
defining variants (preceeded by \cmd{RequirePackage\{etoolbox\}}), 
\item 
the command \cmd{DocumentMetadata} allowing arbitrarily nested braces, 
\item
the command \cmd{input}, and 
\item
whitespace, empty lines, comment lines even magic comments, 
although for this tool they are ignored. 
\end{itemize}
%
This may be too restrictive and here is the point, 
where the use has freedom to change the pattern. 
On the other hand, \cmd{input} offers a quick workaround 
to add material if a user is not familiar with regular expressions. 

In the long run it must be thought of weakening the pattern: 
It is not necessary, that exactly the correct files are parsed, 
because incorrect files are detected by the \LaTeX{} converter anyway. 
Instead, among the correct files the latex main files shall be detected. 

As a workaround for very special \LaTeX{} main files, 
it is a good idea to let it indicate in a magic comment. 
Then the pattern as a whole must match, even not matching a \cmd{documentclass}. 
From the point of view of this software, 
it makes sense to specify the document type in the magic comment then. 
Thinking one step further, also specifying the target 
or the latex converter in a magic comment 
indicates already a latex main file. 
Whereas the target set makes the document class superfluous, 
this is not the case for the magic comment specifying the latex converter. 


% From the point of view of the \LaTeX{} converter, 
% \cmd{documentclass} is not required in the main file, 
% it could be hidden in some \cmd{input}-file. 
% If we stick to the wise convention to open and to close the same environment 
% in the same file and to have \cmd{documentclass}, 
% \cmd{begin\{document\}} and \cmd{end\{document\}} 
% in the same file also, 
% then a latex main file without \cmd{documentclass} 
% cannot contain very much material. 

% So it is almost no restriction to the user to have \cmd{documentclass} 
% declared in the latex main file. 
% As explained above, this is necessary for this \LaTeX{} builder to find out the document class. 
% % TBD: explain why: link to support of various classes. 
% % Partially, this is not yet documented. 



The latex extension latex workshop for VS Code offers two similar 
\href{https://github.com/James-Yu/LaTeX-Workshop/wiki/Compile#multi-file-projects}%
{alternatives to identify \LaTeX{} main files}: 
Occurrence of \cmd{documentclass} without checking preceding material and 
absence of first line \texttt{\%~!TEX root=} declaring a TEX file as depending on a latex main file 
which must be given explicitly. 
The first alternative risks that a TEX file is recognized as main file, 
just because it deals with document classes, 
whereas the second alternative is inconvenient 
and does not work if a file has two potential latex main files 
as is suggested for \texttt{beamer} presentations in Section~\ref{sec:sources} 
and realized in~\cite{PresBeamer}. 
Although presence of \texttt{\%~!TEX root=} indicates that the according file 
is no \LaTeX{} main file, this software ignores this magic comment. 

Emacs with package \auctex, uses an alternative
to the current technique to determine the latex main files: 
Latex main files are marked with an end section as this file: 
%
\begin{Verbatim}[fontsize=\footnotesize]
%%% Local Variables: 
%%% mode: latex
%%% TeX-command-extra-options: "-recorder -shell-escape" 
%%% TeX-master: t
%%% End: 
\end{Verbatim}
% 
The vital line in this context is \texttt{\%\%\% TeX-master: t}. 
In contrast to this, a non-master file 
either has no end-section at all or has an end section 
declaring the according master file (if it is unique) 
explicitly as the following one from \texttt{header.tex}: 
%
\begin{verbatim}
%%% Local Variables:
%%% mode: latex
%%% TeX-master: "manualLMP"
%%% End:
\end{verbatim}

Unlike the document class to be extracted from \cmd{documentclass} 
and unlike other magic comments to be taken into account, 
those of \auctex{} are at the end of the file. 

Although the author considers this approach charming, 
this software ignores \auctex-style magic comments, 
since otherwise the whole file is to be parsed. 
Sticking to regular expressions, 
the parsing engine must then keep the whole file in memory. 
All this would push down performance. 



\subsection{The parameter \texttt{patternCreatedFromLatexMain}}%
\label{subsec:patternCreatedFromLatexMain}

The files created from a latex main file 
depend strongly on the compiler options 
and on packages used in the latex main file 
and in the TEX-files inputted. 
The default value 
`\texttt{\^{}T\$T\textbackslash.$[$\^{}.$]*$}' 
is appropriate for most parameters and packages: 
Most packages create files with names only 
which coincide with the name of the latex main file, except the suffix. 
This is all sufficient even for programs doing post-processing 
such as \texttt{bibtex}, \texttt{makeindex}, \texttt{xindy} and 
\texttt{makeglossaries}. 

% TBD: take bib2gls into account 
The program \texttt{splitindex} requires in addition 
`\texttt{\^{}T\$T-.+\textbackslash.(idx|ind|ilg)}'. % chktex 36 

The utility \texttt{pythontex} requires 
`\texttt{\^{}T\$T\.depytx(\textbackslash.tex)?}' % chktex 36 
and creates a bunch of further files all in a folder of the form 
`\texttt{pythontex-files-T\$T}' which must also be added to the regular expression. 

Whereas typically \tool{latexmk} creates only `\texttt{\^{}T\$T\.fdb\_latexmk}' 
which is included in the very first expression, 
during its run it creates `\texttt{\^{}(pdf|xe|lua)?latex\textbackslash d+\textbackslash.fls}', % chktex 36 
where the digits represent the process number. 
If interrupting \tool{latexmk}, these files may remain, 
so it is appropriate to add them to the regular expression. 

Package `\pkg{srcltx}' or also \texttt{synctex} requires in addition 
`\texttt{\^{}T\$T\textbackslash.synctex(\textbackslash.gz)?}' depending on the setting % chktex 36 
\texttt{synctex=1} or \texttt{synctex=-1}. 
For long files the synctex may create a busy file 
`\texttt{\^{}T\$T\textbackslash.synctex\textbackslash(busy\textbackslash)?}'. 
Even if the \LaTeX{} process is interrupted regularly, 
at the end the busy file is erased, 
but still if interrupted from outside it may remain, 
so we add also the busy variant to the regular expression. 
Strictly speaking, 
`\texttt{\^{}T\$T\textbackslash.synctex(\textbackslash(busy\textbackslash))?(\textbackslash.gz)?}' % chktex 36 
is not precisely what may occur, but is a good approximation. 

% TBD: clarify out\\.ps 
The class beamer creates a lot of additional files 
but finally in addition to what we already have, 
it needs an additional \texttt{\^{}T\$T\textbackslash.run\textbackslash.xml} 
and at times \texttt{\^{}T\$T\textbackslash.\textbackslash d+\textbackslash.vrb}. 

Finally, package `\pkg{tex4ht}' is for all the rest of the cases, created by packages. 


The pattern is designed 
to match quite exactly the created files, 
not much more and at any case not less. 
In particular, it has to comprise the files matching pattern 
\texttt{\$patternT4htOutputFiles}. 
Nevertheless, since any new package may break the pattern, 
and not every package is well documented, 
this pattern cannot be guaranteed to be final. 
% TBD: This is one of the places where normation of packages comes into the game. 

If the current default value is not appropriate, 
please overwrite it in the configuration 
and notify the developer of this plugin of the deficiency. 

The default value for this pattern is currently: 
%
\begin{verbatim}
^(
T$T                      (\.([^.]*|
                             synctex(\(busy\))?(\.gz)?|
                             out\.ps|run\.xml|\d+\.vrb|
                             depytx(\.tex)?)|
    (-|ch|se|su|ap|li)?\d+\.x?html?|
                      \d+x\.x?bb|
                     \d+x?\.png|
                      -\d+\.svg|
                       -.+\.(idx|ind|ilg))|
      pythontex-files-T$T|
      zzT$T\.e?ps|
      (cmsy)\d+(-c)?-\d+c?\.png|
      (pdf|xe|lua)?latex\d+\.fls)$
\end{verbatim}

\section{Parameters for goals \texttt{vrs} and \texttt{inj} }\label{sec:settingsGoalVrsInj}
 
This section describes the parameters for the goals \texttt{vrs} and \texttt{inj} 
given in Table~\ref{tab:paramVrsInj}. 
As illustrated in Listing~\ref{lst:executions} of the part of the pom 
referring to this plugin, 
in general parameters are configured in a \texttt{settings} element 
contained in a general \texttt{configuration} element. 
In contrast to this, the parameters for the goals \texttt{vrs} and \texttt{inj} 
are given in configurations within executions specific for these goals. 

\begin{longtable}{|ll|}
  \toprule
  Parameter        & Default  \\
  \multicolumn2{|l|}{Explanation }  \\
  \midrule
  \midrule
  \endfirsthead%
  \bottomrule
  \caption{\label{tab:paramVrsInj} Parameters for goals \texttt{vrs} and \texttt{inj} }
  \endlastfoot%
  \texttt{versionsWarnOnly}  & \texttt {false}  \\
  \multicolumn2{|l|}{
  \begin{minipage}{0.95\linewidth}
  Indicates whether the goal \texttt{vrs} displays warnings only 
  or also creates pieces of info. 
  Info refers to the version of this plugin and also on its git commit, 
  but also on the versions of the converters found 
  and lists the converters excluded, 
  i.e.\@ those not used and thus not tested on version.
  
  Warnings are emitted e.g.\@ if the version of a converter does not fit the expectations,
  the version of a converter could not be retrieved,
  e.g.\@ because it is not installed 
  or if the converter specified is unknown altogether. 
  This defaults to \texttt{false} displaying also info.
  
  The latter is appropriate for using in command line 
  \texttt{mvn latex:vrs}, whereas in builds by default 
  the pom overwrites this to have output only 
  in case something goes wrong.
  \end{minipage}
  }\\ % chktex 9 chktex 10 (bug) 
  \texttt{injections}  & \texttt {latexmkrc,chktexrc}  \\
  \multicolumn2{|l|}{
  \begin{minipage}{0.95\linewidth}
Indicates the files injected by the goal \texttt{inj}. 
This is a comma separated list of \texttt{injections} without blanks. 
For further description see Section~\ref{sec:injFiles}. 
\end{minipage}
}\\ % chktex 9 chktex 10 (bug) 
\end{longtable}




\section{Parameters for graphical preprocessing}\label{sec:settingsGraph}

This section describes the parameters for graphical preprocessing 
given in Table~\ref{tab:paramGraphics}. 

TODO\@: do this. 


\begin{longtable}{|ll|}
\toprule
Parameter        & Default  \\
\multicolumn2{|l|}{Explanation }  \\
\midrule
\midrule
\endfirsthead%
\bottomrule
\caption{\label{tab:paramGraphics} Parameters for graphics preprocessing}
\endlastfoot%
\texttt{fig2devCommand}   & \texttt{fig2dev}\index{fig2dev}     \\
\multicolumn2{|l|}{
\begin{minipage}{0.95\linewidth}
The fig2dev command for conversion of fig-files into various formats. 
Currently, only \texttt{PDF} combined with \texttt{ptx} is supported. 
%The default value is '\texttt{fig2dev}'. 
\end{minipage}
} \\
\texttt{fig2devGenOptions}   & \makeEmptyExplicit{\figToDevGenOptions}     \\
\multicolumn2{|l|}{
\begin{minipage}{0.95\linewidth}
The options for the command \texttt{\$fig2devCommand} 
common to both output languages. 
For the options specific for the two output languages 
`\texttt{pdftex}' and `\texttt{pdftex\_t}', 
see the explanation of the parameters 
\texttt{\$fig2devPtxOptions} and \texttt{\$fig2devPdfEpsOptions}, 
respectively. 
% The default value is the empty string. 
\end{minipage}
} \\
\texttt{fig2devPtxOptions}   & \makeEmptyExplicit{\figToDevPtxOptions}     \\
\multicolumn2{|l|}{
\begin{minipage}{0.95\linewidth}
The options for the command \texttt{\$fig2devCommand} 
specific for the output language `\texttt{pdftex\_t}'. 
Note that in addition to these options, 
the option `\texttt{-L pdftex\_t}' specifies the language, 
\texttt{\$fig2devGenOptions} specifies the options 
common for the two output languages 
`\texttt{pdftex}' and `\texttt{pdftex\_t}' 
and `\texttt{-p xxx.pdf}' specifies the PDF-file to be included. 
% The default value for this option is the empty string. 
\end{minipage}
} \\
\texttt{fig2devPdfEpsOptions}   & \makeEmptyExplicit{\figToDevPdfEpsOptions}     \\
\multicolumn2{|l|}{
\begin{minipage}{0.95\linewidth}
The options for the command \texttt{\$fig2devCommand} 
specific for the output language `\texttt{pdftex}'. 
Note that in addition to these option1s, 
the option `\texttt{-L pdftex}' specifies the language and 
\texttt{\$fig2devGenOptions} specifies the options 
common for the two output languages 
`\texttt{pdftex}' and `\texttt{pdftex\_t}'. 
% The default value for this option is the empty string. 
\end{minipage}
} \\
\midrule
\texttt{gnuplotCommand}   & \texttt{gnuplot}\index{gnuplot}     \\
\multicolumn2{|l|}{
\begin{minipage}{0.95\linewidth}
The command for conversion of gnuplot-files 
into various formats. 
Currently, only \texttt{pdf} (graphics) 
combined with \texttt{pdf\_t} (latex-texts) is supported. 
% The default value is 'gnuplot'. 
\end{minipage}
} \\
\texttt{gnuplotOptions}   & \makeEmptyExplicit{\gnuplotOptions}    \\
\multicolumn2{|l|}{
\begin{minipage}{0.95\linewidth}
The options specific for \texttt{\$gnuplotCommand}'s 
output terminal ``cairolatex'', 
used for mixed latex/pdf-creation. 
Note that the option `\texttt{pdf|eps}' 
of the terminal `\texttt{cairolatex}' is not available, 
because it is set internally. 
% The default option string is empty. 
\end{minipage}
} \\
\midrule
\texttt{metapostCommand} & \texttt{mpost}\index{metapost}\index{mpost}     \\
\multicolumn2{|l|}{
\begin{minipage}{0.95\linewidth}
The command for conversion of gnuplot-files 
into metapost's postscript. 
% The default value is 'mpost'. 
\end{minipage}
} \\
\texttt{metapostOptions} & see Section~\ref{subsec:metapostOptions} \\
\multicolumn2{|l|}{
\begin{minipage}{0.95\linewidth}
The options for the command \texttt{\$metapostCommand}.
Leading and trailing blanks are ignored. 
A sequence of at least one blank separate the proper options. 
%The default value is 
%'\texttt{-interaction=nonstopmode -recorder -s prologues=2}'. 
\end{minipage}
} \\
\texttt{patternErrMPost} & \texttt{(\^{}!\ )} \\% chktex 37
\multicolumn2{|l|}{
\begin{minipage}{0.95\linewidth}
The pattern is applied line by line to the log-file of \texttt{\$metapostCommand} 
and matching indicates an error 
emitted by the command \texttt{\$metapostCommand}. 

The default value is chosen to match quite exactly 
the latex errors in the log file, no more no less. 
Since no official documentation was found, 

The default pattern may be incomplete. 
In fact, it presupposes, that \texttt{\$metapostOptions}  
does not contain `\texttt{-file-line-error-style}'.   

If the current default value is not appropriate, 
please overwrite it in the configuration 
and notify the developer of this plugin of the deficiency. 
%The default value is `<code>(^! )</code>' (note the space).  
\end{minipage}
} \\
\texttt{patternWarnMPost} & \texttt{\^{}([Ww]arning: )} \\% chktex 36 chktex 37
\multicolumn2{|l|}{
\begin{minipage}{0.95\linewidth}
The pattern is applied line by line to the log-file of \texttt{\$metapostCommand} 
and matching indicates a warning 
emitted by the command \texttt{\$metapostCommand}. 

This pattern may never be ensured to be complete, 
because any library may indicate a warning 
with its own pattern any new package may break completeness. 
Nevertheless, the default value aims completeness 
while be restrictive enough 
not to indicate a warning where none was emitted. 

If the current default value is not appropriate, 
please overwrite it in the configuration 
and notify the developer of this plugin of the deficiency. 
The default value is given below. 
%The default value is `<code>(^! )</code>' (note the space).  
\end{minipage}
} \\
\midrule
\texttt{svg2devCommand} & \texttt{inkscape} \\
\multicolumn2{|l|}{
\begin{minipage}{0.95\linewidth}
The command for conversion of SVG-files into a mixed format. 
%The default value is 'inkscape'. 
\end{minipage}
} \\
\texttt{svg2devOptions} & \makeEmptyExplicit{\svgToDevOptions} \\
\multicolumn2{|l|}{
\begin{minipage}{0.95\linewidth}
The options for the command \texttt{\$svg2devCommand} 
for exporting SVG-figures into latex compatible files. 
For more details see Section~\ref{subsec:svg2devOptions}. 
\end{minipage}
} \\
\midrule
\texttt{createBoundingBoxes} & \texttt{false} \\
\multicolumn2{|l|}{
\begin{minipage}{0.95\linewidth}
   Whether for pixel formats like JPG and PNG 
   command \texttt{\$ebbCommand} is invoked to determine the bounding box. 
   This is relevant, if at all, only in dvi-mode. 
   Note that the package \texttt{bmpsize} is an alternative 
   to invoking \texttt{ebb}, 
   which seems not to work for \texttt{xelatex}. 
   Moreover, all seems to work fine with neither of these techniques. 
   The \texttt{\$dvi2pdfCommand} given by the default, \texttt{dvipdfmx}, 
   seems the only which yields the picture sizes as in PDF mode 
   which fit well. 
   Note also that \miktex{} offers neither package \pkg{bmpsize} 
   nor \texttt{ebb}. 
   This alone requires to switch off invocation of \texttt{ebb} by default. 
   So the default value is \texttt{false}. 
  \end{minipage}
} \\
\texttt{ebbCommand} & \texttt{ebb} \\
\multicolumn2{|l|}{
\begin{minipage}{0.95\linewidth}
The command to create bounding box information 
from JPG-files and from PNG-files. 
This is run twice: 
once with parameter `\texttt{-m}' 
to create `.bb'-files for driver `\texttt{dvipdfm}' and 
once with parameter `\texttt{-x}' 
to create `.xbb'-files for driver `\texttt{dvipdfmx}'. 
% The default value is 'ebb'. 
\end{minipage}
} \\
\texttt{ebbOptions} & \texttt{-v} \\
\multicolumn2{|l|}{
\begin{minipage}{0.95\linewidth}
The options for the command \texttt{\$ebbCommand} 
except `\texttt{-m}' and `\texttt{-x}' 
which are added automatically. 
% The default value is `-v'. 
\end{minipage}
} \\
\end{longtable}

\subsection{The parameter \texttt{metapostOptions}}%
\label{subsec:metapostOptions}

The options of the (sole standalone) metapost compiler 
are given in the metapost manual~\cite{MPost24}, Appendix B.2.1. 
The current default option line for this software is as follows: 
%
\begin{center}
  {\scriptsize\texttt{\metapostOptions}}
\end{center}
%-interaction=nonstopmode -recorder -s prologues=2 -s outputtemplate="%j.mps"

The details are as follows: 
%
\begin{description}
  \item[-interaction=nonstopmode] To avoid user interaction in case of an error 
  This seems mandatory. 
  \item[-recorder] Strictly speaking not necessary at the current stage, 
    but for later versions of this software, to allow dependencies tracking. 
  \item[-s prologues=2] In general the \texttt{-s} assigns an internal key a value. 
  Here it is the kind of the prologue. 
  The value 2 is a compromise between safe quality of output and length of artifact. 
  As described in detail in~\cite{MPost24}, Section 8.2, 
  a value of 0 is sufficient for PDF output. 
  Also, if no \LaTeX{} is used to typeset labels, the prologue value is irrelevant. 
  The value 1 is deprecated, 2 yields a prologue only slightly longer than with 0, 
  whereas the safest setting 3 yields a huge prologue. 
  So the compromise is 2 and if 3 is needed in individual cases, 
  this setting can be overwritten in the MP file. 
  \item[outputtemplate="\%j.mps"] determines the name of the output file. %chktex 18
  The default given here uses the ``jobname'' and the canonical ending. 
  Unlike the default value of \texttt{mpost}, 
  no number of the figure within the metapost file is given. 
  This comes from the fact that we assume a single figure only 
  and ignore the number of the figure. 
  \end{description}

  % TBD: here one could add the options which are excluded. 
  % -dvitopm, -help, -version, -troff, -halt-on-error, ...

  % better the useful options: 
  % -numbersystem=⟨string⟩ Set arithmetic mode to one of scaled, double, binary, decimal

\subsection{The parameter \texttt{svg2devOptions}}%
\label{subsec:svg2devOptions}

The following options are mandatory: 
%
\begin{description}
\item[--export-area-drawing] 
Export the drawing (not the page). 
\item[--export-latex] 
Export into PDF/PS/EPS format without text. 
Besides the PDF/PS/EPS files, a \LaTeX-file \texttt{latexfile.tex} is exported,
putting the text on top of the PDF/PS/EPS file, 
i.e.\@ including the according pure graphic file. 
Include the result in \LaTeX{} as: \cmd{input\{latexfile.tex\}}. 

Note that the latter option is necessary, 
to create the expected files. 
It is also conceivable to export text as pdf/eps 
\end{description}

\noindent
The following options are prohibited, 
because they are automatically added by the software or interfers with: 
%
\begin{description}
\item[--export-filename=FILENAME] Export document to a file with type given by the extension. 
This is used both to export into PDF and into EPS format. 
The extension is always given explicitly. 
\item[--export-type=TYPE] Overwrites the type given by \texttt{--export-filename}. 
If no extension is given, this is to determine the export type. 
\end{description}
% The default value is the minimal value, `- -export-area-drawing - -export-latex'. 


\section{Parameters for the \LaTeX-to-pdf Conversion}%
\label{sec:settingsLatex2pdf}

This section describes the parameters 
of the \LaTeX-to-pdf converter 
which are given in Table~\ref{tab:paramLatex2pdf}. 

TODO\@: do this. 


\begin{longtable}{|ll|}
\toprule
Parameter        & Default  \\
\multicolumn2{|l|}{Explanation }  \\
\midrule
\midrule
\endfirsthead%
\bottomrule
\caption{\label{tab:paramLatex2pdf} The \LaTeX-to-pdf-converter}
\endlastfoot%
\texttt{latex2pdfCommand}       & \lualatex{}      \\
\multicolumn2{|l|}{
\begin{minipage}{0.95\linewidth}
The \LaTeX{} command to create above all a PDF-file with. 
Further formats are DVI and XDV and also other formats based on these. 
Expected values are \lualatex, \xelatex{} and \pdflatex. 
CAUTION\@: This setting may be overwritten for individual \LaTeX{} main files 
using magic comments as described in Section~\ref{subsubsec:openingMagComm}. 

Note that for \xelatex{} dvi mode 
(creating xdv-files instead of DVI-files) is not supported, 
even not creating PDF or other formats via XDV\@. 
See also the according options \texttt{\$latex2pdfOptions} and \texttt{\$pdfViaDvi}. 
In particular, this maven plugin does not allow goal \texttt{dvi} and related 
for \xelatex. 
Consequently, '\texttt{\$targets}' may not contain any of these goals. 
%The default value is '\lualatex{}'. 
\end{minipage}
} \\
\texttt{latex2pdfOptions}   & see Section~\ref{subsec:latex2pdfOptions} \\
\multicolumn2{|l|}{
\begin{minipage}{0.95\linewidth}
The options for the command \texttt{\$latex2pdfCommand}.
Leading and trailing blanks are ignored. 
A sequence of at least one blank separate the proper options. 
\end{minipage}
} \\
\texttt{patternErrLatex} & \texttt{(\^{}!\ )} \\% chktex 37
\multicolumn2{|l|}{
\begin{minipage}{0.95\linewidth}
The pattern is applied line-wise to the log-file 
and matching indicating an error 
emitted by the command \texttt{\$latex2pdfCommand}. 

The default value is chosen to match quite exactly 
the latex errors in the log file, no more no less. 
Since no official documentation was found, 
the default pattern may be incomplete. 
In fact, it presupposes, that \texttt{\$latex2pdfOptions} 
does not contain ``\texttt{-file-line-error-style}''.   

If the current default value is not appropriate, 
please overwrite it in the configuration 
and notify the developer of this plugin of the deficiency. 
\end{minipage}
} \\
\texttt{patternWarnLatex} & see Section~\ref{subsec:patternWarnLatex} \\
\multicolumn2{|l|}{
\begin{minipage}{0.95\linewidth}
The pattern is applied line-wise to the log-file 
and matching indicates a warning 
emitted by the command \texttt{\$latex2pdfCommand}, 
disregarding warnings on bad boxes 
provided \texttt{\$debugWarnings} is set. 

This pattern may never be ensured to be complete, 
because any package may indicate a warning 
with its own pattern any new package may break completeness. 
Nevertheless, the default value aims completeness 
while be restrictive enough 
not to indicate a warning where none was emitted. 

If the current default value is not appropriate, 
please overwrite it in the configuration 
and notify the developer of this plugin of the deficiency. 
% The default value is given below. 
\end{minipage}
} \\
\texttt{debugBadBoxes}    & \texttt{true}          \\
\multicolumn2{|l|}{
\begin{minipage}{0.95\linewidth}
Whether debugging of overfull/underfull hboxes/vboxes is on: 
If so, a bad box occurs in the last \LaTeX{} run, a warning is displayed. 
For details, set \texttt{\$cleanUp} to false, 
rerun \LaTeX{} and have a look at the log-file.
%The default value is '\texttt{true}'. 
\end{minipage}
} \\
\texttt{debugWarnings}    & \texttt{true}          \\
\multicolumn2{|l|}{
\begin{minipage}{0.95\linewidth}
Whether debugging of warnings is on: 
If so, a warning in the last \LaTeX{} run is displayed. 
For details, set \texttt{\$cleanUp} to false, 
rerun \LaTeX{} and have a look at the log-file. 
%The default value is '\texttt{true}'. 
\end{minipage}
} \\
\texttt{pdfViaDvi}    & \texttt{false}          \\
\multicolumn2{|l|}{
\begin{minipage}{0.95\linewidth}
Whether creation of PDF-files from \LaTeX-files 
goes via dvi-files. 

If \texttt{\$pdfViaDvi} is set 
and the latex processor needs repetitions, 
these are all done creating dvi 
and then pdf is created in a final step 
invoking the command \texttt{\$dvi2pdfCommand}. 
If \texttt{\$pdfViaDvi} is not set, 
latex is directly converted into pdf. 

Currently, not only conversion of \LaTeX-files is affected, 
but also conversion of graphic files 
into graphic formats which allow inclusion in the tex-file. 
If it goes via latex, 
then the formats are more based on (encapsulated) postscript; 
else on pdf. 

In the dvi-file for jpg, png and svg 
only some space is visible and only in the final step 
performed by \texttt{\$dvi2pdfCommand}, 
the pictures are included using the bounding boxes 
given by the \texttt{.bb} or the \texttt{.xbb}-file. 
These are both created by \texttt{\$ebbCommand}. 

Of course, the target dvi is not affected: 
This uses always the dvi-format. 
What is also affected are the tasks 
creating HTML, ODT or docs: 
Although these are based on \texttt{htlatex} which is always dvi-based, 
the preprocessing is done in dvi or in pdf. 
Also the task TXT is affected. 

As indicated in \texttt{\$latex2pdfCommand}, 
the processor \xelatex{} does not create \texttt{dvi} 
but \texttt{xdv} files. 
In a sense, the \texttt{xdv} format is an extension of \texttt{dvi} 
but as for he \texttt{xdv} format there is no viewer, 
no way \texttt{htlatex} or other applications (except the \xelatex-internal \texttt{xdvidpfmx}) 
and also no according mime type, 
we refrained from subsumming this under ``kind of dvi''. 
Thus, with \xelatex{} the flag \texttt{\$pdfViaDvi} may not be set. 

%The default value is '\texttt{false}'. 
\end{minipage}
} \\
\texttt{dvi2pdfCommand}   & \texttt{dvipdfmx}         \\
\multicolumn2{|l|}{
\begin{minipage}{0.95\linewidth}
The driver to convert dvi into PDF-files. 
Note that this must fit the options 
of the packages `\pkg{xcolor}', `\pkg{graphicx}' and, provided no autodetection, \pkg{hyperref}. 
Sensible values are `\texttt{dvipdfm}', `\texttt{dvipdfmx}' and `\texttt{dvipdft}', 
which are all the same in my implementation 
and `\texttt{dvipdft}' 
(which is roughly a wrapper around `\texttt{dvipdfm}' with option \texttt{-t} using `\texttt{gs}'). 
Note that `\texttt{dvipdf}' is just a script around `\texttt{dvips}' using `\texttt{gs}', 
but does not provide proper options; so not allowed. 
%The default value is `dvipdfmx'. 
\end{minipage}
} \\
\texttt{dvi2pdfOptions}   & the empty string          \\
\multicolumn2{|l|}{
\begin{minipage}{0.95\linewidth}
The options for the command \texttt{\$dvi2pdfCommand}. 
The default value is '\texttt{-V1.7}' specifying the PDF version to be created. 
The default version for PDF format for \texttt{\$dvi2pdfCommand} is version 1.5. 
The reason for using version 1.7 is \texttt{\$fig2dev} which creates PDF figures in version 1.7 
and forces \texttt{\$latex2pdfCommand} in DVI mode to include PDF version 1.7 
and finally \texttt{\$dvi2pdfCommand} to use that also to avoid warnings. 

Using \texttt{\$latex2pdfCommand} if used to create PDF directly, 
by default also PDF version 1.5 is created. 
For sake of uniformity, it is advisable to create PDF version 1.7 also. 
In future this will be done uniformly through \texttt{\textbackslash{}DocumentMetadata} command. 
%The default value is '-V1.7' but should be the empty string.
\end{minipage}
} \\
\texttt{patternReRunLatex} &  see Section~\ref{subsec:patternReRunLatex} \\
\multicolumn2{|l|}{
\begin{minipage}{0.95\linewidth}
The pattern is applied line-wise to the log file 
and matching triggers rerunning \texttt{\$latex2pdfCommand} 
if \texttt{\$maxNumReRunsLatex} is not yet reached 
to ensure termination. 

This pattern may never be ensured to be complete, 
because any package 
may indicate the need to rerun \texttt{\$latex2pdfCommand} 
with its own pattern and so any new package may break completeness. 
Nevertheless, the default value aims completeness 
while be tight enough not to trigger a superfluous rerun. 

If the current default value is not appropriate, 
please overwrite it in the configuration 
and notify the developer of this plugin of the deficiency. 
\end{minipage}
} \\
\texttt{maxNumRerunsLatex}        & \texttt{5}               \\
\multicolumn2{|l|}{
\begin{minipage}{0.95\linewidth}
The maximal allowed number of reruns of the \LaTeX{} process. 
This is to avoid endless repetitions. 
%The default value is 5. 
This shall be non-negative 
or -1 which signifies that there is no threshold. 
\end{minipage}
} \\
\end{longtable}


\subsection{The parameter \texttt{latex2pdfOptions}}%
\label{subsec:latex2pdfOptions}

An overview over the options supported by the usual latex engines in distribution \texlive{} 
is given in~\cite{LatexGen}, Section 2. 
In particular, there is a table with the options occurring in any latex converter 
and columns indicating for each option for which converters it is valid. 
Note that unlike the other engines, \lualatex{} defines options starting with \texttt{-{}-}, 
it works on according options starting with single dash also. 
To support all engines with the same parameters, 
the default options are among the ones common to all supported converters. 
Currently, default option line is as follows: 
%
\begin{center}
  \texttt{\latexToPdfOptions}
\end{center}
%
The details are as follows: 
%
\begin{description}
  \item[-interaction=nonstopmode] To avoid user interaction in case of an error 
  This seems mandatory. 
  \item[-synctex=1] to create \texttt{.synctex.gz} files needed for interaction between editor and viewer. 
  \item[-recorder] Strictly speaking not necessary at the current stage, 
    but for later versions of this software, to allows tracking dependencies. 
  \item[-shell-escape] allows the TEX engine to access the shell to execute. 
  This is needed for some reason for driver \texttt{dvipdfmx} 
  which seems to be the sole one supporting 
  PDF-pictures in DVI-mode and PDF-pictures in PDF-mode. 

  An alternative would be \texttt{-shell-restricted}. 
  CAUTION\@: In \miktex{} this is \texttt{--enable-write18} instead. 
\end{description}

% According to~\cite{LatexGen}, Section 3 with the exeption of \texttt{-shell-escape}, 
% these options are also available in MiKTeX distribution, 
% up to the fact that options start with \texttt{-{}-}. 
% Instead of \texttt{-shell-escape}, one has to write \texttt{-{}-enable-write18}. 
% Note that besides \texttt{-{}-enable-write18} and \texttt{-{}-disable-write18}, 
% MiKTeX offers also \texttt{-{}-restrict-write18}. 
% It must be clarified, how MiKTeX treats options starting with single dash. 

Note that part of the default values is mandatory, in particular \texttt{nonstopmode}, 
but there are also options which are not allowed. 
In most of the cases, the problem is that the latex engine does not create an output 
or does not create it in the expected location or in the expected form. 
This may apply to the main artifact, i.e.\@ PDF or DVI or XDV, 
but it may also apply to log files and other files. 

The following list of prohibited options is illustrative but not complete: 
%
\begin{description}
  \item[-draftmode] switch on draft mode (generates no output PDF which causes an error) 
  \item[-output-directory=dir] to specify the output directory 
  \item[-aux-directory=dir] to specify the auxiliary output directory 
  \item[-job-name=name] effectively changes the output file name 
  \item[-quiet] makes the log quiet and 
      so circumvents error and warning detection 
  \item[-fmt=FMTNAME] use FMTNAME instead of program name or a \%\& line 
  \item[--luaonly] run a lua file, then exit 
  \item[-output-format=FORMAT] use FORMAT for job output; FORMAT is `dvi' or `pdf'
                   pdf is the only allowed \dots. 
                   This is not supported by \xelatex. 
  \item[-no-pdf] generate XDV (extended DVI) output rather than PDF\@. 
   This is specific for \xelatex. 
  \item[-progname=STRING] set program (and fmt) name to STRING
    only names also without -progname are possible 
  \item[-help] display this help and exit
  \item[-version] output version information and exit
\end{description}

Note that the default value of \texttt{\$patternErrLatex} 
excludes option \texttt{-file-line-error-style} 
and its synonym \texttt{--file-line-error-style}. 
Nevertheless, these options can be used 
if the pattern \texttt{\$patternErrLatex} is adapted. 

Also option \texttt{-halt-on-error} is not strictly forbidden, but not recommended, 
because it prevents operation as intended for this software. 


Two options deserve particular notification, both specifying the output format: 
%
\begin{description}
  \item[-no-pdf] which is specific to \xelatex, makes \xelatex{} create XDV files 
  which currently cannot be further processed by this software. 
  As soon as this software supports XDV files, this option is set by this software, 
  not by the user. 
  \item[-output-format=FORMAT], which this software uses to set the output format, 
  either to \texttt{dvi} or to \texttt{pdf}. 
  Strictly speaking, this option is supported by all converters, except for \xelatex. 
  For \xelatex, this software only supports \texttt{pdf}, 
  which \xelatex{} creates because \texttt{-no-pdf} is not given. 
  The option \texttt{-output-format=pdf} does no harm, because it is ignored. 
  As soon as this software supports XDV creation, 
  it will no longer pass \texttt{-output-format} to \xelatex. 
\end{description}


In general, there are two forms of options, one starting with double dash, \texttt{-{}-}, 
and the other form with single dash. 
In \texlive, \pdflatex{} and \xelatex{} use single dash, 
whereas \lualatex{} uses double dash according to the help text. 
But using the single dash always is ok, 
because \lualatex{} understands single dash also. 

In \miktex, all options of all converters are double dash. 
It must be clarified, whether they understand single dash. 
If not one has to clarify whether in \texlive{} all converters understand double dash. 
If so all must be changed into double dash. 


\subsection{The parameter \texttt{patternWarnLatex}}%
\label{subsec:patternWarnLatex}

The patterns given below are just by (unwritten) convention. 
As a consequence, the pattern has a comprehensive default value 
covering all warnings known to the author, 
while not detecting a warning, where there is none. 
To that end, the pattern requires 
that the warning text starts with the line of the log file. 
Still the pattern has to be configurable 
to allow the user to overwrite the default value 
not being forced to wait for the developer 
to change it. 

For the current default value, we distinguish 
%
\begin{itemize}
\item
\LaTeX-warnings emitted directly by \LaTeX{} 
starting with \texttt{LaTeX Warning: }, 
\item
\LaTeX-font-warnings related with fonts/font selection 
starting with \texttt{LaTeX Font Warning: }, 
\item
Package warnings emitted by a package. 
By convention, a package emitting a warning identifies itself 
by its name \texttt{<name>}
emitting a warning starting with \texttt{Package <name> Warning: }, 
\item
Class warnings emitted by a package. 
By convention, a class emitting a warning identifies itself 
by its name \texttt{<name>}
emitting a warning starting with \texttt{Class <name> Warning: }, 
\item
pdf\TeX-warning starting with \texttt{pdfTeX warning} 
and being specific for the compiler \pdflatex, 
\item 
Warnings on inclusion of a PDF file, 
e.g.\@ inclusion of PDF files with incompatible version, 
starting with \texttt{warning  (file <filename>) (pdf inclusion)}, 
\item
Font specification warnings starting with \texttt{* fontspec warning}\footnote%
{Please note the leading character ``\texttt{*}''. }, 
\item
Further warnings not identifying themselves as warnings 
as the word ``warning'' does not occur. 
Still they are treated as warning 
because they all indicate some imperfection in the output. 
\end{itemize}

The resulting default pattern is 
%
\begin{Verbatim}
^(LaTeX Warning: |
LaTeX Font Warning: |
(Package|Class) .+ Warning: |
pdfTeX warning( \((\d|\w)+\))?: |
\* fontspec warning: |
Non-PDF special ignored!|
Missing character: There is no .* in font .*!$|
A space is missing\. (No warning)\.)
\end{Verbatim}
%$ for emacs only 


\subsection{The parameter \texttt{patternReRunLatex}}%
\label{subsec:patternReRunLatex}

TODO\@: rework based on comments in class Settings. 

For the package \pkg{rerunfilecheck} an analysis of the code is possible, 
and the warnings emitted by this package indicating the need for rerun 
are taken into account for the pattern. 

Besides this package, also other packages may require rerun, 
but these are not analyzes systematically. 
A first step would be to analyze those given in \texttt{header.tex} 
created by injection. % TBD: add reference and some indication in that section 

As a consequence, the pattern has a comprehensive default value 
covering all warnings known to the author, 
while not detecting a warning, where there is none. 
To that end, the pattern requires 
that the warning text starts with the line of the log file. 
Still the pattern has to be configurable 
to allow the user to overwrite the default value 
not being forced to wait for the developer to change it. 


The resulting default pattern is 
%
\begin{Verbatim}[fontsize=\scriptsize]
^(LaTeX Warning: Label\(s\) may have changed\. Rerun to get cross-references right\.$|
Package \w+ Warning: .*Rerun( .*|\.)$|
Package rerunfilecheck Info: Checksums for |
Package \w+ Warning: .*$^\(\w+\) .*Rerun( .*|\.)$|
LaTeX Warning: Etaremune labels have changed\.$|
\(rerunfilecheck\)                Rerun to get outlines right$|
\(rerunfilecheck\)                Rerun LaTeX)
\end{Verbatim}

There is one \texttt{Info} message in there, also indicating the need for rerun. 
This is inserted because another rerun warning may fail to apply 
because it contains the file name and if this is too long, 
then the required sequence ``\texttt{Rerun.}'' is cut off 
and is not on the current line. 

Still what is good, if such a warning is not recognized 
as a pattern indicating the need for rerun, 
it occurs in the final LOG file and is recognized as a warning. 
So it is merely impossible to get a result with not enough reruns 
and without warning. 

FIXME\@: There is a bug in this pattern. See Section~\ref{chap:bugs}. 


\section{Parameters for creation of the bibliography}\label{sec:settingsBibTeX}

This section describes the parameters 
or creation of the bibliography
which are given in Table~\ref{tab:paramBibTeX}. 

TODO\@: do this. 

\begin{longtable}{|ll|}
\toprule
Parameter        & Default  \\
\multicolumn2{|l|}{Explanation }  \\
\midrule
\midrule
\endfirsthead%
\bottomrule
\caption{\label{tab:paramBibTeX} The BibTeX-utility }
\endlastfoot%
\texttt{bibtexCommand}    & \texttt{bibtex}        \\
\multicolumn2{|l|}{
\begin{minipage}{0.95\linewidth}
The BibTeX command to create a bbl-file 
from an aux-file and a bib-file (using a bst-style file). 
%The default value is '\texttt{bibtex}'. 
\end{minipage}
} \\
\texttt{bibtexOptions}    & empty        \\
\multicolumn2{|l|}{
\begin{minipage}{0.95\linewidth}
The options for the command \texttt{\$bibtexCommand}. 
% The default value is the empty string.
\end{minipage}
} \\
\texttt{patternErrBibtex}    & \texttt{error message}        \\
\multicolumn2{|l|}{
\begin{minipage}{0.95\linewidth}
The pattern is applied line-wise to the blg-file 
and matching indicates that \texttt{\$bibtexCommand} failed. 
The default value is chosen 
according to the `\texttt{bibtex}' documentation. 
\end{minipage}

} \\
\texttt{patternWarnBibtex}    & \texttt{\^{}Warning-{}-}        \\
\multicolumn2{|l|}{
\begin{minipage}{0.95\linewidth}
The pattern is applied line-wise to the blg-file 
and matching indicates a warning \texttt{\$bibtexCommand} emitted. 
The default value is chosen 
according to the `\texttt{bibtex}' documentation. 
\end{minipage}
} \\
\end{longtable}

\section{Parameters for creation of the indices}%
\label{sec:settingsMakeIndex}

This section describes the parameters 
or creation of the indices
which are given in Table~\ref{tab:paramMakeIndex}. 

TODO\@: do this. 


\begin{longtable}{|ll|}
\toprule
Parameter        & Default  \\
\multicolumn2{|l|}{Explanation }  \\
\midrule
\midrule
\endfirsthead%
\bottomrule
\caption{\label{tab:paramMakeIndex} 
The utilities MakeIndex and SplitIndex}
\endlastfoot%
\texttt{makeIndexCommand} & \texttt{makeindex}     \\
\multicolumn2{|l|}{
\begin{minipage}{0.95\linewidth}
The MakeIndex command to create an ind-file from an idx-file 
logging on an ilg-file. 
%The default value is '\texttt{makeindex}'. 
\end{minipage}
} \\
\texttt{makeIndexOptions} & the empty string     \\
\multicolumn2{|l|}{
\begin{minipage}{0.95\linewidth}
The options for the MakeIndex command. 
% The default value is the empty string. 
\end{minipage}
} \\
\texttt{patternErrMakeIndex} 
& \texttt{(!! Input index error\ )} \\% chktex 36 % chktex 37
\multicolumn2{|l|}{
\begin{minipage}{0.95\linewidth}
The pattern is applied line-wise to the ilg-file 
and matching indicates that \texttt{\$makeIndexCommand} failed. 
The default value is chosen 
according to the `\texttt{makeindex}' documentation.
\end{minipage}
} \\
\texttt{patternWarnMakeIndex} & \texttt{(\#\# Warning )} \\% chktex 37
\multicolumn2{|l|}{
\begin{minipage}{0.95\linewidth}
The pattern is applied line-wise to the ilg-file 
and matching indicates a warning \texttt{\$makeIndexCommand} emitted. 
The default value is chosen 
according to the `\texttt{makeindex}' documentation.
\end{minipage}
} \\
\texttt{patternReRunMakeIndex} & see Section~\ref{subsec:patternReRunMakeIndex} \\
\multicolumn2{|l|}{
\begin{minipage}{0.95\linewidth}
The pattern is applied line-wise to the log-file 
and matching triggers rerunning \texttt{\$makeIndexCommand} 
followed by \texttt{\$latex2pdfCommand}. 

This pattern only matches a warning 
emitted by the package `\pkg{rerunfilecheck}' 
e.g.\@ used with option `\texttt{index}'. 
The default value 
is chosen according to the package documentation. 

If the current default value is not appropriate, 
please overwrite it in the configuration 
and notify the developer of this plugin of the bug. 
\end{minipage}
} \\
\texttt{splitIndexCommand} & \texttt{splitindex}     \\
\multicolumn2{|l|}{
\begin{minipage}{0.95\linewidth}
The SplitIndex command to create ind-files 
from an idx-file logging on ilg-files. 
This command invokes \texttt{\$makeIndexCommand}. 
%The default value is `splitindex'. 
\end{minipage}
} \\
\texttt{splitIndexOptions} & \texttt{-V}     \\
\multicolumn2{|l|}{
\begin{minipage}{0.95\linewidth}
The options for \texttt{\$splitIndexCommand}. 
Here, one has to distinguish between the options 
processed by \texttt{\$splitIndexCommand} 
and those passed to \texttt{\$makeIndexCommand}. 
The second category cannot be specified here, 
it is already given by \texttt{\$makeIndexOptions}. 
In the first category is the option `\texttt{-m}' 
to specify the \texttt{\$makeIndexCommand}. 
This is used automatically and cannot be specified here. 
Since \texttt{\$splitIndexCommand} is used 
in conjunction with package `\texttt{splitidx}', 
which hardcodes various parameters 
which are the default values for \texttt{\$splitIndexCommand} 
and because the option may not alter certain interfaces, 
the only option which may be given explicitly 
is `\texttt{-V}', the short cut for `\texttt{--verbose}'. 
Do not use `\texttt{--verbose}' either for sake of portability. 
%The default value is `-V'; it could also be empty. 
\end{minipage}
} \\
\end{longtable}


\subsection{The parameter \texttt{patternReRunMakeIndex}}%
\label{subsec:patternReRunMakeIndex}

As shown in~\cite{RerunFChkP}, Section 2.3, Listing line 166, 
the pattern is 
%
\begin{Verbatim}
(^\(rerunfilecheck\) +Rerun LaTeX/makeindex to get index right\.$)
\end{Verbatim}
%$ for emacs only 


\section{Parameters for creation of the Glossary}%
\label{sec:settingsMakeGlossaries}

This section describes the parameters 
or creation of the glossary
which are given in Table~\ref{tab:paramMakeGlossaries}. 

TODO\@: do this. 



\begin{longtable}{|ll|}
\toprule
Parameter        & Default  \\
\multicolumn2{|l|}{Explanation }  \\
\midrule
\midrule
\endfirsthead%
\bottomrule
\caption{\label{tab:paramMakeGlossaries} The MakeGlossaries-utility }
\endlastfoot%
\texttt{makeGlossariesCommand} & \texttt{makeglossaries}         \\
\multicolumn2{|l|}{
\begin{minipage}{0.95\linewidth}
The MakeGlossaries command to create a gls-file 
from a glo-file (invoked without file ending) 
also taking ist-file or xdy-file 
into account logging on a glg-file. 
\end{minipage}
} \\
\texttt{makeGlossariesOptions} & the empty string          \\
\multicolumn2{|l|}{
\begin{minipage}{0.95\linewidth}
The options for the \texttt{\$makeGlossariesCommand}. 
These are the options for `\texttt{makeindex}' (not for \texttt{\$makeIndexCommand}) 
and for `\texttt{xindy}' (also hardcoded). 
The aux-file decides on whether program is executed 
and consequently which options are used. 
		 
The default value is the empty option string. 
Nevertheless, `\texttt{xindy}' is invoked as `\texttt{xindy -L english  -I xindy -M\ldots}'. 
With option `\texttt{-L german}', this is added. 
Options `\texttt{-M<}' for `\texttt{xindy}' 
`\texttt{-s}' for `\texttt{makeindex}' and `\texttt{-t}' and `\texttt{-o}' for both, 
`\texttt{xindy}' and `\texttt{makeindex}'. 
\end{minipage}
} \\
\texttt{patternErrMakeGlossaries} & 
\texttt{(\^{}\textbackslash*\textbackslash*\textbackslash* unable to execute:\ )} \\% chktex 37
\multicolumn2{|l|}{
\begin{minipage}{0.95\linewidth}
The pattern is applied line-wise to the `glg'-file 
and matching indicates that \texttt{\$makeGlossariesCommand} failed. 
The default value `\texttt{(\^\*\*\* unable to execute:~)}' 
is chosen according to the \texttt{makeindex} documentation. 
If the default value is not appropriate, please modify 
and notify the developer of this plugin. 
\end{minipage}
} \\
\texttt{patternErrXindy} & \texttt{(\^{}ERROR:\ )} \\% chktex 36 chktex 37
\multicolumn2{|l|}{
\begin{minipage}{0.95\linewidth}
The pattern in the \gls{glg}-file 
which indicates an error when running `\texttt{xindy}' 
via \texttt{\$makeGlossariesCommand}. 
If the default value is not appropriate, please modify 
and notify the developer of this plugin. 
\end{minipage}
} \\
\texttt{patternWarnXindy} & \texttt{(\^{}WARNING:\ )}         \\% chktex 37
\multicolumn2{|l|}{
\begin{minipage}{0.95\linewidth}
The pattern is applied line-wise to the `glg'-file 
and matching indicates a warning when running `\texttt{xindy}' 
via \texttt{\$makeGlossariesCommand}. 

The default value `\texttt{(\^{}WARNING:~)}' 
(note the space and the brackets) 
is chosen according to the '\texttt{xindy}' documentation. 

If the current default value is not appropriate, 
please overwrite it in the configuration 
and notify the developer of this plugin of the deficiency. 
\end{minipage}
} \\
\texttt{patternReRunMakeGlossaries}  
& see Section~\ref{subsec:patternReRunMakeGlossaries}   \\
\multicolumn2{|l|}{
\begin{minipage}{0.95\linewidth}
The pattern is applied line-wise to the log file 
and matching triggers rerunning \texttt{\$makeGlossariesCommand} 
followed by \texttt{\$latex2pdfCommand}. 

This pattern only matches a warning 
emitted by the package `\pkg{rerunfilecheck}' 
e.g.\@ used with option `\texttt{glossary}'. 
The default value 
is chosen according to the package documentation. 

If the current default value is not appropriate, 
please overwrite it in the configuration 
and notify the developer of this plugin of the bug. 
\end{minipage}
} \\
\end{longtable}

\subsection{The parameter \texttt{patternReRunMakeGlossaries}}\label{subsec:patternReRunMakeGlossaries}

As shown in~\cite{RerunFChkP}, Section 2.3, Listing line 210, 
the pattern is 
%
\begin{Verbatim}
(^\(rerunfilecheck\) +Rerun LaTeX/makeindex to get index right\.$)
\end{Verbatim}
%$ for emacs only 

This holds even if \texttt{splitindex} is used. 

\section{Parameters for including code via \texttt{pythontex}}%
\label{sec:settingsPythontex}

This section describes the parameters for invoking 
\texttt{pythontex} and parameters for invoking \texttt{depythontex} 
which are given in Table~\ref{tab:paramPythontex} 
and in Table~\ref{tab:paramDePythontex}, 
respectively. 


\begin{longtable}{|ll|}
\toprule
Parameter        & Default  \\
\multicolumn2{|l|}{Explanation }  \\
\midrule
\midrule
\endfirsthead%
\bottomrule
\caption{\label{tab:paramPythontex} Injecting output of code via \texttt{pythontex} }
\endlastfoot%
\texttt{pythontexCommand}       & \texttt{pythontex}  \\
\multicolumn2{|l|}{
\begin{minipage}{0.95\linewidth}
The PythonTeX command which creates a folder \texttt{pythontex-files-xxx} 
with various files inside 
from a \gls{pytxcode}-file (invoked without file ending) 
and logging in a \gls{plg}-file. 
The default value is \texttt{pythontex} 
but as long as this does not write a log file this software really needs, 
we have to configure it with \texttt{pythontexW} 
which is a simple wrapper of \texttt{pythontex} writing a log file. 
CAUTION\@: Since \texttt{pythontexW} is not registered with this software, 
one has to specify it with its category as \texttt{pythontexW:pythontex}. 
\end{minipage}
} \\
\texttt{pythontexOptions}    & \texttt{--rerun=always}  \\
\multicolumn2{|l|}{
\begin{minipage}{0.95\linewidth}
The options for the command \texttt{\$pythontexCommand}. 

For the possibilities see the manual of the pythontex package 
or the help dialog of \texttt{pythontex}. 
CAUTION\@: \texttt{-{}-rerun} and \texttt{-{}-runall} cannot be specified both in one invocation. 
In the context of this software, the option
\texttt{-{}-interactive} is not appropriate. 
CAUTION\@: For many options of the command line tool, 
there is an according package option and the latter overrides the former. 
CAUTION\@: This software overwrites settings \texttt{-{}-rerun} and \texttt{-{}-runall} anyway, 
and forces setting \texttt{-{}-rerun=always}. 
The default value is \texttt{-{}-rerun=always}. 
\end{minipage}
} \\
\texttt{patternErrPyTex}    & see Section~\ref{sec:settingsPythontex} \\
\multicolumn2{|l|}{
\begin{minipage}{0.95\linewidth}
The pattern in the \gls{plg}-file 
indicating that running \texttt{pythontex}, 
resp.\@ \texttt{pythontexW} via \texttt{\$pythontexCommand} failed.
%The default value is essentially
%
%\begin{Verbatim}[showspaces=true, fontsize=\small, commentchar=!]
%!(PythonTeX:  .+ -|    - Current: ) [1-9][0-9]* error\\(s\\), [0-9]+ warning\\(s\\)
%\end{Verbatim}
%
%but due to a bug in \texttt{pythontex} it is slightly more complicated.
The pattern would fit into a single line but because of a bug in \texttt{pythontex}, 
it is a bit more complicated. 
If this is not appropriate, please modify and notify the developer of this plugin. 
\end{minipage}
} \\
\texttt{patternWarnPyTex}    & see Section~\ref{sec:settingsPythontex} \\
\multicolumn2{|l|}{
\begin{minipage}{0.95\linewidth}
The pattern in the \gls{plg}-file 
indicating a warning when running \texttt{pythontex}, 
resp.\@ \texttt{pythontexW} via \texttt{\$mpythontexCommand}.
If this is not appropriate, please modify and notify the developer of this plugin. 
\end{minipage}
} \\
\texttt{prefixPytexOutFolder}    & \texttt{pythontex-files-} \\
\multicolumn2{|l|}{
\begin{minipage}{0.95\linewidth}
The prefix of the name of the folder written by \texttt{\$pythontexCommand}. 
The full name of that folder is this prefix 
followed by the jobname of the latex main file, 
i.e.\@ the filename without ending. 
 
CAUTION\@: This is readonly, 
because in both, the pythontex tool and the according latex package 
this prefix is hardcoded at time of this writing. 
\end{minipage}
} \\\end{longtable}

\begin{longtable}{|ll|}
  \toprule
  Parameter        & Default  \\
  \multicolumn2{|l|}{Explanation }  \\
  \midrule
  \midrule
  \endfirsthead%
  \bottomrule
  \caption{\label{tab:paramDePythontex} Replacing code by its output via \texttt{depythontex} }
  \endlastfoot%
  \texttt{depythontexCommand}       & \texttt{depythontex}  \\
  \multicolumn2{|l|}{
  \begin{minipage}{0.95\linewidth}
    The Depythontex command invoked with no file ending 
    to create a file \texttt{xxx.depytx.tex} filefrom a tex-file, 
    a \gls{depytx}-file taking the output of \texttt{pythontex} into account 
    and logging on a \gls{dplg}-file. 
    The default value is \texttt{depythontex} 
    but as long as this does not write a log file this software really needs, 
    we have to configure it with \texttt{depythontexW} 
    which is a simple wrapper of \texttt{depythontex} writing a log file. 
    CAUTION\@: Since \texttt{depythontexW} is not registered with this software, 
    one has to specify it with its category as \texttt{depythontexW:depythontex}. 
  \end{minipage}
  } \\
  \texttt{depythontexOptions}    & the empty string  \\
  \multicolumn2{|l|}{
  \begin{minipage}{0.95\linewidth}
  \end{minipage}
  } \\
  \end{longtable}

The pattern \texttt{patternErrPyTex} is by default 
%
\begin{Verbatim}[showspaces=true, fontsize=\small]
\* PythonTeX error|...
\end{Verbatim}
substituting the dots by 
%
\begin{Verbatim}[showspaces=true, fontsize=\footnotesize]
(PythonTeX:  .+ -|    - Current: ) [1-9][0-9]* error\(s\), [0-9]+ warning\(s\)
\end{Verbatim}

Accordingly, the pattern texttt{patternWarnPyTex} is by default 
%
\begin{Verbatim}[showspaces=true, fontsize=\footnotesize]
(PythonTeX:  .+ -|    - Current: ) [0-9]+ error\\(s\), [1-9][0-9]* warning\(s\)
\end{Verbatim}
   

\section{Parameters for conversion \LaTeX{} to HTML}%
%\section{Parameters for the \LaTeX-to-html Conversion}
\label{sec:settingsLatex2Html}

This section describes the parameters 
of the \LaTeX-to-html converter 
which are given in Table~\ref{tab:paramLatex2Html}. 

%TODO\@: do this. 



\begin{longtable}{|ll|}
\toprule
Parameter        & Default  \\
\multicolumn2{|l|}{Explanation }  \\
\midrule
\midrule
\endfirsthead%
\bottomrule
\caption{\label{tab:paramLatex2Html} The \LaTeX-to-html-converter }
\endlastfoot%
\texttt{tex4htCommand}       & \texttt{htlatex}  \\
\multicolumn2{|l|}{
\begin{minipage}{0.95\linewidth}
\end{minipage}
} \\
\texttt{tex4htStyOptions}    & \texttt{xhtml,uni-html4,2,svg,pic-tabular}  \\
\multicolumn2{|l|}{
\begin{minipage}{0.95\linewidth}
\end{minipage}
} \\
\texttt{tex4htOptions}       & \texttt{' -cunihtf -utf8'}         \\
\multicolumn2{|l|}{
\begin{minipage}{0.95\linewidth}
\end{minipage}
} \\
\texttt{t4htOptions}         & the empty string              \\%-cvalidate
\multicolumn2{|l|}{
\begin{minipage}{0.95\linewidth}
The options for `\texttt{t4ht}' which converts idv-file and lg-file 
into css-files, tmp-file and, 
by need and if configured accordingly into \gls{png}-files. 
The value `\texttt{-p}' prevents creation of \gls{png}-pictures.
%The default value is the empty string. 
\end{minipage}
} \\
\texttt{patternT4htOutputFiles} & 
see Section~\ref{subsec:patternT4htOutputFiles}\\
\multicolumn2{|l|}{
\begin{minipage}{0.95\linewidth}
The pattern is applied to file names 
and matching shall accept 
exactly the target files of goal `\texttt{html}' 
for a given latex main file `\texttt{xxx.tex}'. 
Matching triggers copying those files to \texttt{\$outputDirectory}. 

The patterns for the other targets 
are hardcoded and take the form 
`\texttt{\^{}T\$T\textbackslash.yyy\$}', where `\texttt{yyy}' 
may be an ending or an alternative of endings. 
This pattern is neither applied to directories 
nor to `\texttt{xxx.tex}' itself. 
 
For an explanation of the pattern `\texttt{T\$T}' 
see \texttt{\$patternCreatedFromLatexMain}. 
Spaces and newlines are removed 
from that pattern before processing. 

The pattern is designed to match quite exactly 
the files to be copied to \texttt{\$targetSiteDirectory}, 
for the goal `\texttt{html}', 
not much more and at any case not less. 
Since \texttt{\$tex2htCommand} is not well documented, 
and still subject to development, 
this pattern cannot be guaranteed to be final. 

If the current default value is not appropriate, 
please overwrite it in the configuration 
and notify the developer of this plugin of the bug. 
\end{minipage}
} \\
\end{longtable}


\subsection{The parameter \texttt{patternT4htOutputFiles}}%
\label{subsec:patternT4htOutputFiles}



The default value has the following components: 
%
\begin{itemize}
\item
`\texttt{\^{}T\$T\textbackslash.x?html?\$}' is the main output file. 
\item
`\texttt{\^{}T\$Tli\textbackslash d+\textbackslash.x?html?\$}' are lists: 
toc, lof, lot, indices, glossaries, NOT the bibliography. 
\item
'\texttt{\^{}T\$T(ch|se|su|ap)\textbackslash d+\textbackslash.x?html?\$}' % chktex 36
are chapters, sections and subsections or below 
and appendices. 
\item
`\texttt{\^{}T\$T\textbackslash d+\textbackslash.x?html?\$}' are footnotes. 
\item
`\texttt{\^{}T\$T\textbackslash.css\$}' are cascaded stylesheets. 
\item
`\texttt{\^{}T\$T-\textbackslash d+\textbackslash.svg\$}' and 
`\texttt{\^{}T\$T\textbackslash d+x\textbackslash.png\$}' 
are svg/png-files representing figures. 
\item
`\texttt{\^{}T\$T\textbackslash d+x\textbackslash.x?bb}' % chktex 15 chktex 9
are the bounding boxes 
(suffix \texttt{.bb}  for \texttt{dvipdfm} and 
 suffix \texttt{.xbb} for \texttt{dvipdfmx}). 
\item
`\texttt{\^{}(cmsy)\textbackslash d+(-c)?-\textbackslash d+c?\textbackslash.png\$}' % chktex 36
represents special symbols. 
\end{itemize}
		 
Note that the patterns for the html-files can be summarized as 
%
\begin{Verbatim}
^T$T((ch|se|su|ap|li)?\d+)?\.x?html?\$
\end{Verbatim}
%$ for emacs 

This altogether constitutes the default value for this pattern: 
%
\begin{verbatim}
^(T$T(((ch|se|su|ap|li)?\d+)?\.x?html?|
\.css|
\d+x\.x?bb|
\d+x\.png|
-\d+\.svg)|
(cmsy)\d+(-c)?-\d+c?\.png)$
\end{verbatim}

The pattern is designed to match quite exactly 
the files to be copied to \texttt{\$targetSiteDirectory}, 
for the goal ``html'', 
not much more and at any case not less. 
since \texttt{\$tex2htCommand} is not well documented, 
and still subject to development, 
this pattern cannot be guaranteed to be final. 


\section{Parameters for further conversions}\label{sec:settingsExt}

This section describes the parameters 
of the converter from and to further formats 
which are given in Table~\ref{tab:paramExt}. 

These converters convert latex into RTF directly, 
they convert ODT into doc-like documents and pdf into pure text. 
A special case is the code-checker 
in a sense converting latex into a log-file. 
For each of them, the name of the command can be specified 
and also the options. 
Since neither of them, except the code checker, write a log-file, 
there are no further parameters necessary. 


\begin{longtable}{|ll|}
\toprule
Parameter        & Default  \\
\multicolumn2{|l|}{Explanation }  \\
\midrule
\midrule
\endfirsthead%
\bottomrule
\caption{\label{tab:paramExt} The parameters of further converters }
\endlastfoot%
\texttt{latex2rtfCommand}    & \texttt{latex2rtf}        \\
\multicolumn2{|l|}{
\begin{minipage}{0.95\linewidth}
The latex2rtf command to create RTF from latex directly. 
%The default value is '\texttt{latex2rtf}'. 
\end{minipage}
} \\
\texttt{latex2rtfOptions}    & the empty string        \\
\multicolumn2{|l|}{
\begin{minipage}{0.95\linewidth}
The options of the command \texttt{\$latex2rtfCommand}. 
% The default value is the empty string. 
\end{minipage}
} \\
\texttt{odt2docCommand}      & \texttt{odt2doc}          \\
\multicolumn2{|l|}{
\begin{minipage}{0.95\linewidth}
The odt2doc command to create MS word-formats from otd-files. 
%The default value is '\texttt{odt2doc}'. 
\end{minipage}
} \\
\texttt{odt2docOptions}      & \texttt{-fdocx}          \\
\multicolumn2{|l|}{
\begin{minipage}{0.95\linewidth}
The options of the command \texttt{\$odt2docCommand}. 
Above all specification of output format 
via the option '-f'. 
Invocation is '\texttt{odt2doc -f$<$format$>$ $<$file$>$.odt}'. 
All output formats are shown by `\texttt{odt2doc --show}' 
but the formats interesting in this context 
are the following: 
\texttt{doc}, \texttt{doc6}, \texttt{doc95}, \texttt{docbook}, \texttt{docx}, 
\texttt{docx7}, \texttt{ooxml} and \texttt{rtf}. 
Interesting also the verbosity options 
`\texttt{-v}', `\texttt{-vv}', `\texttt{-vvv}' 
the timeout `\texttt{-T=secs}' and `\texttt{--preserve}' 
to keep permissions and timestamp of the original document. 
%The default value is '\texttt{-fdocx}'. 
\end{minipage}
} \\
\texttt{pdf2txtCommand}      & \texttt{pdftotext}        \\
\multicolumn2{|l|}{
\begin{minipage}{0.95\linewidth}
The pdf2txt-command for converting PDF-files 
into plain text files. 
%The default value is '\texttt{pdftotext}'. 
\end{minipage}
} \\
\texttt{pdf2txtOptions}      & the empty string  \\
\multicolumn2{|l|}{
\begin{minipage}{0.95\linewidth}
The options of the command \texttt{\$pdf2txtCommand}. 
%The default value is the empty string. 
\end{minipage}
} \\
\end{longtable}





FIXME\@: 
Note that \texttt{pdftotext -h} prints a usage message. 
This is a way to obtain not the specified output. 
It shows that \texttt{pdftotext -q} does not print any messages or errors. 
This indicates that \texttt{pdftotext} normally does display error messages 
on the standard output. 
These may be led to a log file to indicate errors and warnings. 
Here, further research is required. 

The option \texttt{-htmlmeta} seems not approprate. 
The option resolution \texttt{-r} seems sensible only in conjunction 
with the crop area defined by \texttt{-x} and \texttt{-y} 
which does not make sense in our context. 
The same holds for specification of the first and the last page 
via \texttt{-f} and \texttt{-l}. 
What does make sense is specifying the encoding via \texttt{-enc} 
with possible values given by \texttt{pdftotext -listenc}. 
What makes sense most is \texttt{UTF-8}. 




\section{Parameters for the code checker \texttt{chktex}}\label{sec:chkTex}

Among the applications used by this software, 
the codechecker plays a special role: 
it is not really a converter, 
unless we interprete the log file as artifact. 
Like for the most converters also for the codechecker 
we can specify the command ant its options, both given in Table~\ref{tab:paramChkTex}. 

\begin{longtable}{|ll|}
  \toprule
  Parameter        & Default  \\
  \multicolumn2{|l|}{Explanation }  \\
  \midrule
  \midrule
  \endfirsthead%
  \bottomrule
  \caption{\label{tab:paramChkTex} The parameters of the code checker }
  \endlastfoot%
  \texttt{chkTexCommand}      & \texttt{chktex}        \\
  \multicolumn2{|l|}{
  \begin{minipage}{0.95\linewidth}
  The chktex-command for checking latex main files. 
  %The default value is 'chktex'. 
  \end{minipage}
  } \\
  \texttt{chkTexOptions}      & \texttt{-q -b0}  \\
  \multicolumn2{|l|}{
  \begin{minipage}{0.95\linewidth}
  The options of the command \texttt{\$chkTexCommand}, 
  except ``\texttt{-o output-file}'' 
  specifying the output file which is added automatically. 
  For further details see the options below. 
  %The default value is '\texttt{-q -b0}'. 
  \end{minipage}
  } \\
\end{longtable}

The options of \texttt{chktex} 
are described in detail in~\cite{ChkTeX22}, Section 6.1.2. 

Here is a list of options useful in this context. 
The first group of these are muting options: 
%
\begin{itemize}
\item '-w', '-e', '-m', 
Make the message number passed as parameter 
a warning/an error/a message and turns it on. 
Messages are not counted. 
\item '-n'
Turns the warning/error number passed as a parameter off. 
\item '-L'
Turns off suppression of messages on a per line basis. 
\end{itemize}

The next group of interesting options are for output control: 
%
\begin{itemize}
\item['-q']
Shuts up about copyright information.
\item['-o output-file']
Specifies the output file. This is added automatically 
and shall thus not be specified by the user. 
\item['-b0/1']
If you use the -o switch, and the named output-file exists,
it will be renamed to `filename.bak' for option \texttt{-b1} 
and not for \texttt{-b0}.
\item['-f format']
Specifies the format of the output 
via a format similar to ``\texttt{printf()}'. %chktex 36
For details consult the manual~\cite{ChkTeX22}, Section 6.1.2. 
The codes are listed below. 
\item['-vd']
Verbosity level followed by a number `d' 
specifying the format of the output according to the listing below. 
The verbosity number is resolved as a pattern 
as if given by the option `-f format'. 
Thus the option `-v' is ignored 
if the option `-f format' is specified. 
\end{itemize}

The default value \texttt{-q -b0} 
avoids verbose output and backing up the output log-file. 

Code
%
\begin{itemize}
\item[\%b]
String to print between fields (from -s option).
\item[\%c]
Column position of error.
\item[\%d]
Length of error (digit).
\item[\%f]
Current file-name.
\item[\%i]
Turn on inverse printing mode.
\item[\%I]
Turn off inverse printing mode.
\item[\%k]
kind of error (warning, error, message).
\item[\%l]
line number of error.
\item[\%m]
Warning message.
\item[\%n]
Warning number.
\item[\%u]
An underlining line (like the one which appears when using '-v1').
\item[\%r]
Part of line in front of error ('S'-1).
\item[\%s]
Part of line which contains error (string).
\item[\%t]
Part of line after error ('S'+1).
\end{itemize}



FIXME\@: to be inserted. 
See~\cite{ChkTeX22}, Section~6.1.6. 
From \texttt{chktexrc}: 
\begin{verbatim}
OutFormat
{
# -v0; silent mode
%f%b%l%b%c%b%n%b%m!n

# -v1; normal mode
"%k %n in %f line %l: %m!n%r%s%t!n%u!n"

# -v2; fancy mode
"%k %n in %f line %l: %m!n%r%i%s%I%t!n!n"

# -v3; lacheck mode
"!"%f!", line %l: %m!n"

# -v4; verbose lacheck mode
"!"%f!", line %l: %m!n%r%s%t!n%u!n"

# -v5; no line number, ease auto-test
"%k %n in %f: %m!n%r%s%t!n%u!n"

# -v6; emacs compilation mode
"!"%f!", line %l.%c:(#%n) %m!n"
}
\end{verbatim}

Note that ``\texttt{!}'' is to escape quotes and newline. 
More than these can be added to \texttt{chktexrc}. 


This document is checked with options deviating from the default value: 
%
\begin{verbatim}
-q -b0 -v1 -g0 -l ${basedir}/src/site/tex/chktexrc
\end{verbatim}
%
The default is \texttt{-q -b0}, 
option \texttt{-g0} means that the global \texttt{chktexrc} is not used 
and option 
%
\begin{verbatim}
-l ${basedir}/src/site/tex/chktexrc
\end{verbatim}
%
specifies a record file tailored to the needs of this project. 
In particular, the pattern for \texttt{-v1} is slightly modified: 
It is 
%
\begin{verbatim}
# -v1; normal mode
"%k %n in %f line %l: %m!n %r%s%t!n %u!n"
\end{verbatim}
%
which adds a blank to all lines but the headlines. 
That way, the kind of issue (\%k) is easily parsed. 
This could be used for emitting an error instead of a warning 
when processing goal \emph{check}. 

Although the return code of \texttt{chktex} is not documented, 
a bit of reverse engineering shows the following distinction: 
%
\begin{enumerate}\setcounter{enumi}{-1}
  \item Successful execution and found neither an error nor a warning. 
  \item Execution as such did not succeed, 
  e.g.\@ because of an invalid option like \texttt{-exx}. 
  \item An error occurred and in particular execution as such suceeded. 
  \item A warning occurred but no error and in particular execution as such suceeded. 
\end{enumerate}

On this behavior this software bases its failure messages. 

The options of \texttt{chktex} are described in detail 
in~\cite{ChkTeX22}, Section 6.1.2. % chktex 17 bug: rest but in 1674


\section{Parameters for ensuring reproducibility}\label{sec:paramRepro}

For a general description of the reproducibility check 
see Section~\ref{sec:chkReprod}. 
Here we go into the details and identify the parameters 
controlling the check 
and specified in great detail in Table~\ref{tab:paramDiffPdf}. 
As already mentioned in Section~\ref{sec:chkReprod}, 
currently, checks are performed for artifacts in pdf format only; 
more formally, if the target 
(which is in parameter \texttt{target} described in Table~\ref{tab:paramGen}) 
is \texttt{pdf}. 
% TBD: this may change. 

But if so, the parameter \texttt{chkDiff} 
decides whether a check is performed at all. 
Note that checking is off by default. 
Then a diffing tool given by \texttt{diffPdfCommand} 
expects the blueprints in the directory \texttt{diffDirectory}. 
In contrast, the actual artifacts to be checked are in \texttt{outputDirectory}, 
whereas the sources are in \texttt{texSrcDirectory}. 

The location of a source tex file relative to \texttt{texSrcDirectory} 
is the location of the artifact relative to \texttt{outputDirectory}. 
This path relative to \texttt{diffDirectory} is the location of the blueprint. 
With the actual artifact in \texttt{outputDirectory} 
and the blueprint in \texttt{diffDirectory} 
the diff-tool determines whether the both are equivalent. 
If so, equivalence is logged as an info, 
else an exception described in Table~\ref{tab:TLP} is thrown. 

Note that the choiced of the diff tool \texttt{diffPdfCommand} 
determines the notion of equivalence of the pdf artifacts, 
ranging from byte equivalence to some kind of visual equivalence. 


\begin{longtable}{|ll|}
  \toprule
  Parameter        & Default  \\
  \multicolumn2{|l|}{Explanation }  \\
  \midrule
  \midrule
  \endfirsthead%
  \bottomrule
  \caption{\label{tab:paramDiffPdf} The parameters of the pdf differ }
  \endlastfoot%
  \texttt{diffDirectory}      & \texttt{src/main/resources/docsCmp}        \\
  \multicolumn2{|l|}{
  \begin{minipage}{0.95\linewidth}
    Diff directory relative to \texttt{\$baseDirectory} 
    used for diffing actually created artifact against prescribed one inthis directory. 
    This is relevant only if \texttt{\$chkDiff} is set. 
  \end{minipage}
  } \\
  \texttt{chkDiff}      & \texttt{false}  \\
  \multicolumn2{|l|}{
  \begin{minipage}{0.95\linewidth}
    Indicates whether after creating artifacts 
		and copying them to the output directory \texttt{\$outputDirectory} 
		the artifacts are checked by diffing them against preexisting artifacts 
		in \texttt{\$diffDirectory} 
		using the diff command given by \texttt{\$diffPdfCommand}. 
    If this is set, the system time is set to $0$ indicating 1970--01--01. 
		Note that currently, only pdf files are checked. 
    % TBD: correct the information on epoch time 

    This setting can be overwritten for individual latex main files 
    by the magic comment \texttt{chkDiffMagic} 
    described in Section~\ref{subsec:patternLatexMainFile}. 

		This is \texttt{false} by default and is set to \texttt{true} only 
		in the context of tests. 
  \end{minipage}
  } \\
  \texttt{diffPdfCommand}      & \texttt{diff}  \\
  \multicolumn2{|l|}{
  \begin{minipage}{0.95\linewidth}
    The diff-command for diffing PDF-files strictly or just visually 
    to check that the created pdf files are equivalent with prescribed ones. 
    CAUTION\@: There are two philsophies: 
    Either the latex source files are created in a way that they reproduce strictly. 
    Then a strict diff command like \texttt{diff} is appropriate. 
    Else another diff command is required which checks for a kind of visual equality. 
    The default value is a mere \texttt{diff}. 
    Alternatives are \texttt{diff-pdf} and \texttt{diff-pdf-visually} 
    both implementing a visual diff. 
    Note that unlike for other tools, no options can be passed in this case explicitly. 
  \end{minipage}
  } \\
  \texttt{pdfMetainfoCommand}      & \texttt{pdfinfo}  \\
  \multicolumn2{|l|}{
  \begin{minipage}{0.95\linewidth}
    Command to retrieve metainfo from PDF files. 
    Essentially, there are two possibilities, 
    \texttt{exiftool} or \texttt{pdfinfo} 
    but currently this software is restricted to the latter. 
    At time of this writing, only creation time is considered. 
    Note that meta info \texttt{CreationTime} is not identical 
    with creation time in a file system.  
  \end{minipage}
  } \\
  \texttt{pdfMetainfoOptions}      & \texttt{-isodates}  \\
  \multicolumn2{|l|}{
  \begin{minipage}{0.95\linewidth}
    The options for the command \texttt{\$pdfMetainfoCommand} 
    which is currently always \texttt{pdfinfo}. 
    At time of this writing, only creation time is considered. 
    This software has little flexibility in treating various time formats, 
    so it must be decided. 
    Format offered by \texttt{pdfinfo} 
    most commonly known and easily converted to the required epoch time, 
    is really according to ISO 8601. 
    This motivates \texttt{-isodates} to be a mandatory option. 
    Further options do not make sense, 
    as currently only creation time is used. 
    So \texttt{-isodates} is more than a mere default value.
  \end{minipage}
  } \\
\end{longtable}




\section{Parameters for \protect\tool{latexmk} and related}\label{sec:paramLatexmk}

As described in Section~\ref{sec:latexmk}, 
based on the parameter \texttt{\$latexmkUsage} described in Table~\ref{tab:paramGen} 
on page~\pageref{tab:paramGen}, 
the build process can be delegated gradually to \tool{latexmk} 
or an equivalent tool. 
Table~\ref{tab:paramLatexmk} lists the parameters controlling invocation. 
Note that besides the options, which shall be used with care, 
also the config file \texttt{.latexmkrc} goes into. 
The details concerning the config file are described in~\cite{LatexMk23}, 
Section ``CONFIGURATION/INITIALIZATION (RC) FILES''. 
On the other hand, 
as indicated in~\cite{LatexMk23}, 
Section ``DEALING WITH ERRORS, PROBLEMS, ETC'', 
\tool{latexmk} does not write its own log file 
and so there is no parameter in Table~\ref{tab:paramLatexmk} for a pattern 
of warnings or errors. 


\begin{longtable}{|ll|}
  \toprule
  Parameter        & Default  \\
  \multicolumn2{|l|}{Explanation }  \\
  \midrule
  \midrule
  \endfirsthead%
  \bottomrule
    \caption{\label{tab:paramLatexmk} The parameters for \protect\tool{latexmk} and related}
  \endlastfoot%
  \texttt{latexmkCommand}      & \tool{latexmk}        \\
  \multicolumn2{|l|}{
  \begin{minipage}{0.95\linewidth}
    The latexmk command to create a pdf-file from a latex file and other files. 
    Instead of the default value \tool{latexmk} a wrapper is conceivable, 
    a reimplementation seems quite unlikely *smile*. 
  \end{minipage}
  } \\
  \texttt{latexmkOptions}      & empty  \\
  \multicolumn2{|l|}{
  \begin{minipage}{0.95\linewidth}
    The options for the command \texttt{\$latexmkCommand}. 
    Since this command is controlled 
    to a wide extend by the config file \texttt{.latexmkrc}, 
    the options are of minor importance. 
    On the other hand, there are options not allowed for this software 
    because they change behavior in a way not taken into account. 
    So add options with care. 
    The allowed options and their defaults are given in~\cite{LatexMk23}, 
    Section ``LATEXMK OPTIONS AND ARGUMENTS ON COMMAND LINE''. 
  \end{minipage}
  } \\
\end{longtable}